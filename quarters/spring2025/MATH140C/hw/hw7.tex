\documentclass{eeleyes}

\usepackage{fancyhdr}
\pagestyle{fancy}
\fancyhead[lcr]{}
\fancyhead[l]{Eli Griffiths}
\fancyhead[c]{MATH $140$C}
\fancyhead[r]{HW \#$7$}

\usepackage{multicol}

\newcommand\conj[1]{\overline{#1}}
\newcommand\term[1]{\textbf{#1}}
\newcommand\eps{\varepsilon}
\DeclareMathOperator{\interior}{int}

\begin{document}

\section*{Problem 1}
\begin{proof}
    Suppose towards contradiction that $f$ has no critical points and $E$ has interior points. Then there exists a point $a \in E$ such that $\exists r > 0$ where $B_r(a) \subseteq E$. Note that $a + t e_k \in B_r(a)$ for $|t| < r$ since
    \[
        |t| = |te_k| = |(a + te_k) - a| < r
    .\]
    Therefore since $f(p) = 0$ for all $p \in B_r(a)$ and both $a$ and $a + t e_k$ are in $B_r(a)$ for sufficiently small $t$,
    \[
        \partial_{x_k} f(a) = \lim_{t \to 0} \frac{f(a+t e_k) - f(a)}{t} = 0
    .\]
    Since each partial derivative of $f$ at $a$ is $0$, then $\nabla f (a) = 0$. Hence $a$ is a critical point, a contradiction.
\end{proof}

\section*{Problem 2}
\begin{proof}
    Take $x, h \in \R$ with $h > 0$. Since $f \in C^2(\R; \R)$, by the single variable Taylors theorem there exists $\xi \in (x, x + h)$ such that
    \[
        f(x + h) = f(x) + f'(x)h + \frac{1}{2} f''(\xi) h^2
    .\]
    This can be rearranged to get
    \[
        f'(x) = \frac{f(x+h) - f(x)}{h} - \frac{1}{2} f''(\xi) h
    .\]
    Since $|f''(x)| \leq M$ for any $x$, it follows
    \[
        f'(x) = \frac{f(x + h) - f(x)}{h} - \frac{M}{2} \cdot h
    .\]
    Take $\eps > 0$. Since $\lim_{x \to \infty} f(x) = 0$, then there exists $N \in \N$ such that for $x \geq N$, $|f(x)| \leq \eps$. Thus if $h = \sqrt{\eps}$, then for $x \geq N$
    \[
        |f'(x)| = \qty|\frac{f(x+h) - f(x)}{h}| + \frac{M}{2} \cdot h \leq \frac{2\eps}{\sqrt{\eps}} + \frac{M}{2} \cdot \sqrt{\eps} = \qty(2 + \frac{M}{2}) \sqrt{\eps}
    .\]
    Therefore since $2 + \frac{M}{2}$ is a constant, it follows that for $x \to \infty$ that $f'(x) = 0$.
\end{proof}

\section*{Problem 3}
\begin{proof}
    Suppose towards contradiction that $f(x) = f(x')$ for some $x, x' \in U$ where $x \neq x'$. Then there exists some $h \in \R^n \setminus\qty{0}$ such that $x' = x + h$. Let $g : [0,1] \to \R$ where $g(t) = h \cdot f(x + th)$. Since $U$ is convex, $x + th \in U$ for all $t \in [0,1]$ and thus $g$ is well defined on $U$. Thus by the chain rule
    \[
        g'(t) = h \cdot f'(x + th) h
    .\]
    Since $h \cdot f'(x) h > 0$ for all $x \in U$, $g'(t) > 0$ for all $t \in [0,1]$. Therefore $g$ is strictly increasing meaning $g(0) \neq g(1) \implies h \cdot f(x) \neq h \cdot f(x')$. But since $f(x) = f(x')$, $h \cdot f(x) = h \cdot f(x')$, a contradiction. Therefore $f(x) = f(x')$ only when $x = x'$, hence $f$ is one-to-one.
\end{proof}

\section*{Problem 4}
\begin{proof}
    First note that for $c \in \R \setminus \qty{0}$ that
    \[
        \lim_{x \to 0} \frac{\arctan(cx)}{x} = c \lim_{h \to 0} \frac{\arctan(0 + ch) - \arctan(0)}{ch} = c \arctan'(0) = \frac{c}{1 + 0^2} = c
    \]
    and that
    \[
        \lim_{x \to 0} x \arctan(\frac{c}{x}) = 0
    \]
    since $\arctan(\frac{c}{x})$ is bounded. Consider $\partial^2_{x_2 x_1} f(0,0)$. Note that for $x_2 \neq 0$ that
    \begin{align*}
        \partial_{x_1} f(0, x_2) &= \lim_{h \to 0} \frac{f(h, x_2) - f(0, x_2)}{h} \\
        &= \lim_{h \to 0} \frac{f(h,x_2)}{h}  \\
        &= \lim_{h \to 0} h \arctan\qty(\frac{x_2}{h}) - x_2^2 \frac{\arctan(\frac{h}{x_2})}{h} \\
        &= 0 - x_2^2 \cdot \frac{1}{x_2} = -x_2
    \end{align*}
    Since $\partial_{x_1} f(0,0) = 0$ which is the same as above when $x_2 = 0$, $\partial^2_{x_2 x_1} f(0,0) = -1$. Now consider $\partial^2_{x_1 x_2} f(0,0)$. Note that for $x_1 \neq 0$ that
    \begin{align*}
        \partial_{x_2} f(x_1, 0) &= \lim_{h \to 0} \frac{f(x_1, h) - f(x_1, 0)}{h} \\
        &= \lim_{h \to 0} \frac{f(x_1, h)}{h} \\
        &= \lim_{h \to 0} x_1^2 \frac{\arctan\qty(\frac{h}{x_1})}{h} - h \arctan(\frac{x_1}{h}) \\
        &= x_1^2 \cdot \frac{1}{x_1} - 0 = x_1
    \end{align*}
    Since $\partial_{x_2} f(0,0) = 0$ which is the same as above when $x_1 = 0$, $\partial^2_{x_1 x_2} f(0,0) = 1 \neq -1 = \partial^2_{x_2 x_1} f(0,0)$.
\end{proof}

\section*{Problem 5}
\begin{proof}
    Suppose $f$ has a local extremum at $a$. Take $h \in \R^n \setminus \qty{0}$ and note that $f$ has a Taylor expansion about $a$ where
    \[
        f(a + h) = f(a) + L(h) + \frac{1}{2} Q(h) + R(h)
    .\]
    Since $a$ is a local extremum, $L(h) = \nabla f(h) = 0$. Therefore
    \[
        f(a + h) - f(a) = \frac{1}{2} Q(h) + R(h)
    .\]
    Suppose towards contradiction that $a$ is a local minimum and $Q(h)$ is not positive semidefinite. Therefore there exists $v \in \R^n \setminus \qty{0}$ such that $Q(v) < 0$. WLOG $|v| = 1$ since $Q\bigl(\frac{v}{|v|}\bigr) = \frac{1}{|v|^2} Q(v)$ and thus both have the same sign. Set $h = tv$ for $t$ small enough that $a+h$ is in the open ball around $a$ where $a$ is local minimum. Therefore $f(a + h) \geq f(a)$. Note then that 
    \[
        f(a + tv) - f(a) = \frac{1}{2} Q(tv) + R(tv) = \frac{t^2}{2} Q(v) + R(h)
    .\]
    Thus dividing through by $t^2$ gives
    \[
        \frac{f(a+tv) - f(a)}{t^2} = \frac{1}{2} Q(v) + \frac{R(h)}{t^2} = \frac{1}{2} Q(v) + \frac{R(h)}{|h|^2}
    .\]
    Let $\frac{1}{2} Q(v) = -c$ for some $c > 0$. Since $\lim_{h \to 0} \frac{R(h)}{|h|^2} = 0$, then for $t$ sufficiently small (take it to be smaller than the previous $t$ selected)
    \[
        \qty|\frac{R(tv)}{t^2}| < \frac{c}{2}
    .\]
    Therefore
    \[
        \frac{f(a + tv) - f(a)}{t^2} < -c + \frac{c}{2} = -\frac{c}{2} < 0 \implies f(a + h) - f(a) < 0
    \]
    meaning $f(a + tv) < f(a)$. Note that since $t$ can be made arbitrarily small, in every open neighborhood of $a$, $t$ can be chosen such that $a+tv$ is in this nbhd and $f(a+tv) < f(a)$, a contradiction. Therefore $Q$ must be positive semidefinite.
    Proving the case in which $a$ is a local maximum and $Q$ is negative semidefinite is identical to above except
    \begin{enumerate}
        \item $Q$ is not negative semidefinite so there exists $v$ where $Q(v) > 0$.
        \item $\frac{1}{2} Q(v) = c$ for some $c > 0$.
        \item The final inequality is instead
            \[
                \frac{f(a+tv) - f(a)}{t^2} > c - \frac{c}{2} = \frac{c}{2} > 0
            \]
            giving $f(a + tv) - f(a) > 0 \implies f(a + tv) > f(a)$.
        \item Since $t$ can be chosen small enough such that $a + tv$ is in every open neighborhood of $a$, $a$ cannot be a local maximum. Thus $Q$ is negative semidefinite.
    \end{enumerate}
\end{proof}

\section*{Problem 6}
To find possible critical points, consider the points where $\nabla f(x,y) = 0$. Note that
\[
    \nabla f(x,y) = \qty[\frac{4y}{1 + (xy)^2} - 2x, \frac{4x}{1 + (xy)^2} - 2y]
.\]
Note that then $\nabla f(x,y) = 0$ iff
\[
    4y - 2x(1 + (xy)^2) = 0 \implies 2y = x(1 + (xy)^2)
\]
and
\[
    4x - 2y(1 + (xy)^2) = 0 \implies 2x = y(1+ (xy)^2)
.\]
Subtracting these from each other gives
\[
    2(x-y) = (y - x)(1 + (xy)^2)
.\]
Suppose $x \neq y$. Then $x - y \neq 0$ and so
\[
    2(x-y) = -(x-y) (1 + (xy)^2) \implies -2 = 1 + (xy)^2
.\]
But $1 + (xy)^2 \geq 1$ thus there is no real solution in this case. Thus $x = y$ and substituting gives
\[
    2x = x (1+x^4) \implies 2x = x + x^4 \implies x = x^4
.\]
The solutions to this are $x \in \qty{0, -1, 1}$. Therefore the critical points of $f$ are $(0,0), (-1, -1)$ and $(1,1)$. Since $f \in C^2(R^2; \R)$, $f$ has a Taylor expansion at each of these critical points $(x,y)$ whose $Q(h) = Hf(x,y) h \cdot h$ where
\[
    Hf(x,y) = \mqty[
        -\frac{8xy^3}{(1 + (xy)^2)^2} - 2 & \frac{4}{1 + (xy)^2} - \frac{8(xy)^2}{(1+(xy)^2)^2} \\
        \frac{4}{1 + (xy)^2} - \frac{8(xy)^2}{(1+(xy)^2)^2} & -\frac{8x^3y}{(1 + (xy)^2)^2} - 2
    ]
.\]
Therefore
\begin{itemize}
    \item $Hf(0,0) = \mqty[-2 & 4 \\ 4 & -2]$. Take $h^{+} = (-1, -1)$ and $h^{-} = (1,-1)$. Note that
        \[
            Hf(0,0) h^{+} \cdot h^{+} = (-1)(-2) + (-1)(-2) = 4 > 0
        \]
        and
        \[
            Hf(0,0) h^{-} \cdot h^{-} = (1)(-6) + (-1)(6) = -12 < 0
        .\]
        Therefore $Q(h)$ at $(0,0)$ is indefinite and thus $(0,0)$ is neither a local maximum or minimum.
    \item $Hf(1,1) = \mqty[-4 & 0 \\ 0 & -4]$. Note then that for any $h = (h_1, h_2) \neq 0$ that 
        \[
            Hf(1,1) h \cdot h = -4h_1^2 - 4h_2^2 < 0
        .\]
        Therefore $Q(h)$ at $(1,1)$ is negative definite and thus $(1,1)$ is a local maximum.
    \item $Hf(-1,-1) = \mqty[-4 & 0 \\ 0 & -4]$, which by the same logic as above means $(-1, -1)$ is a local maximum.
\end{itemize}

\end{document}
