\documentclass[../main.tex]{subfiles}

\begin{document}

\begin{theorem}[Uniqueness of Taylor's Formula]
    \label{thm:taylorformulaunique}
    Let $D \subseteq \R^n$ be open with $a \in D$ and $f \in C^2(D; \R)$. Suppose for $h$ small that
    \[
        f(a + h) = C + \tilde{L}(h) + \frac{1}{2} \tilde{Q}(h) + \tilde{R}(h)
    \]
    where $C$ is a constant, $\tilde{L}$ is a linear form, $\tilde{Q}$ is a quadratic form, and
    \[
        \lim_{h \to 0} \frac{\tilde{R}(h)}{|h|^2} = 0
    .\]
    Then $C = f(a)$, $\tilde{L} = L$, $\tilde{Q} = Q$ as in from \Cref{thm:taylorformula}
\end{theorem}

\section{Quadratic Forms}

Let $f \in C^2(D; \R)$ for $D \subseteq \R^n$ open. If $a \in D$ is a critical point of $f$, then $\nabla f(a) = 0$, and by \nameref{thm:taylorformula}
\[
    f(a+h) = f(a) + \frac{1}{2} Q(h) + R(h)
.\]
Note that this means the behavior of $f$ around small neighborhoods of $a$ is described mostly by $Q(h)$ since $f(a)$ is a constant and $R(h)$ will drop off. Thus it is of interest to understand $Q(h)$ to understand the behavior of $f$ at $a$.

\begin{definition}[Definiteness]
    Let $Q$ be a quadratic form on $\R^n$. Then $Q$ is
    \begin{itemize}
        \item \term{positive semidefinite} if $Q(h) \geq 0$ for all $h \in \R^n$
        \item \term{positive definite} if $Q(h) > 0$ for all $h \in R^n \setminus \qty{0}$
        \item \term{negative semidefinite} if $Q(h) \leq 0$ for all $h \in \R^n$
        \item \term{negative definite} if $Q(h) < 0$ for all $h \in \R^n \setminus \qty{0}$
    \end{itemize}
    If $Q$ is neither positive or negative semidefinite, then $Q$ is \term{indefinite}.
\end{definition}

\begin{remark}
    $Q$ is indefinite iff there exists $h^{+}, h^{-} \in \R^n$ such that $Q(h^{+}) > 0$ and $Q(h^{-}) < 0$.
\end{remark}

An important property of quadratic forms is that they always have an associated matrix form. That is for any quadratic form $Q$ on $\R^n$, there exists a symmetric $n \times n$ matrix $A$ such that
\[
    Q(h) = Ah \cdot h
.\]
The properties of $Q$ can be then ascertained by the properties of its associated matrix $A$, as outlined below.

\begin{theorem}
    Let $Q$ be a quadratic form. Then for all eigenvalues $\lambda_i$ of the associated matrix $A$
    \begin{itemize}
        \item $Q$ is positive definite $\Leftrightarrow$ $\lambda_i > 0$
        \item $Q$ is positive semidefinite $\Leftrightarrow$ $\lambda_i \geq 0$
        \item $Q$ is negative definite $\Leftrightarrow$ $\lambda_i < 0$
        \item $Q$ is negative semidefinite $\Leftrightarrow$ $\lambda_i \leq 0$
        \item $Q$ is indefinite $\Leftrightarrow$ there exists $\lambda_j > 0$ and $\lambda_k < 0$
    \end{itemize}
\end{theorem}

\end{document}
