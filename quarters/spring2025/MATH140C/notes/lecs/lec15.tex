\documentclass[../main.tex]{subfiles}

\begin{document}

\section[Derivatives I]{Multidimensional Derivative}

\begin{definition}[Differentiability]
    \label{def:differentiability}
    Let $D \subseteq \R^n$ be open and $f : D \to \R^p$. Then $f$ is \term{differentiable} at $a \in D$ if there exists $T \in \mathcal{L}(\R^n, \R^p)$ such that
    \[
        \lim_{h \to 0} \frac{\norm{f(a + h) - f(a) - Th}_{\R^p}}{\norm{h}_{\R_n}} = 0
    \]
    where $h \in \R^n \setminus{0}$. Furthermore $f$ is differentiable on $D$ if $f$ is differentiable at every $a \in D$.
\end{definition}

\begin{theorem}[Uniqueness of Derivative]
    \label{thm:uniquederivative}
    If $f : D \to \R^p$ is differentiable at $a$, the corresponding linear transformation $T \in \mathcal{L}(\R^n, \R^p)$ is unique.
\end{theorem}

\begin{proof}
    Consider any $h \in \R^n \setminus \qty{0}$. Replacing $h$ with $th$ where $t \in \R$ and $t \neq 0$ in \Cref{def:differentiability} gives
    \[
        \rho(t) \coloneq \frac{f(a+th) - f(a) - tTh}{\qty|t| \cdot \norm{h}}
    \]
    and $\rho(t) \to 0$ as $t \to 0$. Thus
    \begin{align*}
        f(a+th) - f(a) - tTh &= \rho(t) \qty|t| \norm{h}  \\
        Th &= \frac{f(a+th) - f(a)}{t} - \frac{|t|}{t} |h| \rho(t)
    \end{align*}
    Therefore
    \[
        Th = \lim_{t \to 0} \frac{f(a+th) - f(a)}{t} \quad h \in \R^n, h \neq 0
    \]
    meaning $T$ is completely determined by $f$ and is thus unique.
\end{proof}

\begin{definition}[Differential]
    If $f$ is differentiable at $a$, then the \term{differential} or \term{total derivative} of $f$ at $a$ is denoted as $f'(a) = \mathcal{D} f(a) = T$.
\end{definition}

\begin{example}
    \hfill
    \begin{enumerate}
        \item Consider $T \in \mathcal{L}(\R^n, \R^p)$. Note that
            \[
                T(a + h) - T(a) = Ta + Th - Ta = Th
            .\]
            Therefore
            \[
                \lim_{h \to 0} \frac{\norm{T(a+h) - T(a) - Th}}{\norm{h}} = \lim_{h \to 0} \frac{0}{\norm{h}} = 0
            \]
            meaning $T$ is differentiable at every $a \in \R^n$ and is its own differential.
        \item Since the identity map $I : \R^n \to \R^n : x \mapsto x$ is a linear transformation, then $I' = I$.
        \item Consider $\rho : \R^n \times \R^n \to \R^n$ where $\rho(x,y) = x \cdot y$. Then $\rho$ is differentiable and $\rho'(a,b)$ is the mapping satisfying
            \[
                \rho'(a,b) (h,k) = a \cdot k + b \cdot h
            .\]
            To see this, note that
            \begin{align*}
                \rho(a+h, b+k) - \rho(a,b) &= (a+h) \cdot (b+k) - a \cdot b \\
                &= a \cdot b + h \cdot b + a \cdot k + h \cdot k - a \cdot b \\
                &= a \cdot k + b \cdot h + h \cdot k
            \end{align*}
            Therefore
            \begin{align*}
                \frac{\norm{\rho(a+h, b+k) - \rho(a,b) - \rho'(a,b)(h,k)}}{\norm{(h,k)}} &= \frac{|h \cdot k|}{\norm{(h,k)}} \\
                &\leq \frac{\norm{h} \norm{k}}{\norm{(h,k)}} \\
                &\leq \norm{k}
            \end{align*}
            meaning the quantity goes to $0$ as $h \to 0$. Thus the proposed differential is indeed correct.
    \end{enumerate}
\end{example}

\begin{theorem}
    Let $f : D \to \R^p$ with $D \subseteq \R^n$ open. If $f$ is differentiable at $a \in D$, then $f$ is continuous at $a$.
\end{theorem}

\begin{theorem}
    Let $f : D \to \R^p$ with $D \subseteq \R^n$ open. Then $f$ is differentiable at $a \in D$ iff $f_j$ is differentiable at $a_j$ for $1 \leq j \leq n$.
\end{theorem}

\section[Derivatives II]{Directional and Partial Derivatives}

Consider some $f : D \to \R^p$ with $D \subseteq \R^n$ open that is differentiable at $a \in D$. Then for $h \in \R^n$ and $t \in \R$ with $h, t \neq 0$ it follows from \Cref{thm:uniquederivative} that
\[
    f'(a) h = Th = \lim_{t \to 0} \frac{f(a + th) - f(a)}{t}
.\]
Therefore the function $t \mapsto f(a+th)$ is differentiable at $0$ with derivative
\[
    \dv{t} \eval_{t=0} f(a+th) = f'(a) h
.\]

\begin{definition}[Directional Derivative]
    \label{def:directional_derivative}
    Let $f : D \to \R^p$ with $D \subseteq \R^n$ open. Then $f$ has a \term{directional derivative} at $a \in D$ in the direction $v \in \R^n \setminus\qty{0}$ if the function $t \to f(a + tv)$ is differentiable at $0$. In this case,
    \[
        f'(a) v = \dv{t} \eval_{t=0} f(a+tv) = \lim_{t \to 0} \frac{f(a+tv) - f(a)}{t} \eqcolon \mathcal{D}_{v} f(a) \eqcolon f'_{v}(a)
    \]
    is called the directional derivative of $f$ at $a$ in the direction of $v$.
\end{definition}

\end{document}
