\documentclass[12pt,titlepage]{extarticle}
% Document Layout and Font
\usepackage{subfiles}
\usepackage[margin=2cm, headheight=15pt]{geometry}
\usepackage{fancyhdr}
\usepackage{enumitem}	
\usepackage{wrapfig}
\usepackage{multicol}
\usepackage{caption, subcaption}

\usepackage[p,osf]{scholax}

\renewcommand*\contentsname{Table of Contents}
\renewcommand{\headrulewidth}{0pt}
\pagestyle{fancy}
\fancyhf{}
\fancyfoot[R]{$\thepage$}
\setlength{\parindent}{0cm}
\setlength{\headheight}{17pt}
\hfuzz=9pt

% Utility Management
\usepackage{color}
\usepackage{colortbl}
\usepackage{xcolor}
\usepackage{xpatch}
\usepackage{xparse}

\definecolor{links}{HTML}{1c73a5}
\definecolor{bar}{HTML}{584AA8}

% Math Packages
\usepackage{mathtools, amsmath, amsthm, thmtools, amssymb, physics}
\usepackage[scaled=1.075,ncf,vvarbb]{newtxmath}

\newcommand\B{\mathbb{B}}
\newcommand\C{\mathbb{C}}
\newcommand\R{\mathbb{R}}
\newcommand\Q{\mathbb{Q}}
\newcommand\N{\mathbb{N}}
\newcommand\Z{\mathbb{Z}}

\newcommand\Prob[1]{\mathbb{P}\qty(#1)}
\newcommand\Var[1]{\text{Var}\qty(#1)}
\newcommand\Exp[1]{\mathbb{E}\qty[#1]}
\newcommand\ball[1]{\B\qty(#1)}
\newcommand\res[1]{\underset{#1}{\operatorname{Res}}\;}
\renewcommand\pv{\mathrm{p.v.}}

\newcommand\conj[1]{\overline{#1}}
\DeclareMathOperator{\Arg}{Arg}
\DeclareMathOperator{\Log}{Log}
\DeclareMathOperator{\cis}{cis}

\DeclareMathOperator{\dom}{dom}
\DeclareMathOperator{\spann}{span}
\DeclareMathOperator{\nullity}{nullity}

\newcommand\st{\text{ s.t. }}

% TIKZ
\usepackage{tikz}
\usepackage{pgfplots}
\usetikzlibrary{arrows.meta}
\usetikzlibrary{math}
\usetikzlibrary{cd}
\usetikzlibrary{patterns}
\usetikzlibrary{decorations.markings}
\usetikzlibrary{calc}

% Boxes and Theorems
\usepackage[most]{tcolorbox}
\tcbuselibrary{skins}
\tcbuselibrary{breakable}
\tcbuselibrary{theorems}

\newtheoremstyle{default}{0pt}{0pt}{}{}{\bfseries}{\normalfont.}{0.5em}{}
\theoremstyle{default}

\renewcommand*{\proofname}{\textit{\textbf{Proof.}}}
\renewcommand*{\qedsymbol}{$\blacksquare$}
\tcolorboxenvironment{proof}{
	breakable,
	coltitle = black,
	colback = white,
	frame hidden,
	boxrule = 0pt,
	boxsep = 0pt,
	borderline west={3pt}{0pt}{bar},
	sharp corners = all,
	enhanced,
}

\newtheorem{theorem}{Theorem}[section]{\bfseries}{}
\tcolorboxenvironment{theorem}{
	breakable,
	enhanced,
	boxrule = 0pt,
	frame hidden,
	coltitle = black,
	colback = blue!7,
	left = 0.5em,
	sharp corners = all,
}

\newtheorem{corollary}{Corollary}[section]{\bfseries}{}
\tcolorboxenvironment{corollary}{
	breakable,
	enhanced,
	boxrule = 0pt,
	frame hidden,
	coltitle = black,
	colback = white!0,
	left = 0.5em,
	sharp corners = all,
}

\newtheorem{lemma}{Lemma}[section]{\bfseries}{}
\tcolorboxenvironment{lemma}{
	breakable,
	enhanced,
	boxrule = 0pt,
	frame hidden,
	coltitle = black,
	colback = green!7,
	left = 0.5em,
	sharp corners = all,
}

\newtheorem{definition}{Definition}[section]{\bfseries}{}
\tcolorboxenvironment{definition}{
	breakable,
	coltitle = black,
	colback = white,
	frame hidden,
	boxsep = 0pt,
	boxrule = 0pt,
	borderline west = {3pt}{0pt}{orange},
	sharp corners = all,
	enhanced,
}

\newtheorem{example}{Example}[section]{\bfseries}{}
\tcolorboxenvironment{example}{
	% title = \textbf{Example},
	% detach title,
	% before upper = {\tcbtitle\quad},
	breakable,
	coltitle = black,
	colback = white,
	frame hidden,
	boxrule = 0pt,
	boxsep = 0pt,
	borderline west={3pt}{0pt}{green!70!black},
	sharp corners = all,
	enhanced,
}

\newtheoremstyle{remark}{0pt}{4pt}{}{}{\bfseries\itshape}{\normalfont.}{0.5em}{}
\theoremstyle{remark}
\newtheorem*{remark}{Remark}


% TColorBoxes
\newtcolorbox{week}{
	colback = black,
	coltext = white,
	fontupper = {\large\bfseries},
	width = 1.2\paperwidth,
	size = fbox,
	halign upper = center,
	center
}

\newcommand{\banner}[2]{
    \pagebreak
    \begin{week}
   		\section*{#1}
    \end{week}
    \addcontentsline{toc}{section}{#1}
    \addtocounter{section}{1}
    \setcounter{subsection}{0}
}

% Hyperref
\usepackage{hyperref}
\hypersetup{
	colorlinks=true,
	linktoc=all,
	linkcolor=links,
	bookmarksopen=true
}


\def\homeworknumber{3}
\usepackage{fancyhdr}
\pagestyle{fancy}
\fancyhead[R]{HW \#\thehwnumber}
\fancyhead[C]{\textbf{Math 130B}}
\fancyhead[L]{Eli Griffiths}


\begin{document}

\subsection*{4.2.1}
\begin{tasks}
    \task $(231)_2 = 1110\;0111$
    \task $(4532)_2 = 1\; 0001\; 1011\; 0100$
    \task $(97644)_2 = 1\; 0111\; 1101\; 0110\; 1100$
\end{tasks}

\subsection*{4.2.3}
\begin{tasks}
    \task $(1\; 1111)_2 = 2^5 - 1 = 31$
    \task $(10\; 0000\; 0001)_2 = 2^0 + 2^9 = 513$
    \task $(1\; 0101\; 0101)_2 = 2^0 + 2^2 + 2^4 + 2^6 + 2^8 = 341$
    \task $(110\; 1001\; 0001\; 0000)_2 = 2^4 + 2^8 + 2^11 + 2^13 + 2^14 = 26896$
\end{tasks}

\subsection*{4.2.5}
\begin{tasks}(2)
    \task $(572)_8 = 101111010$
    \task $(1604)_8 = 11\; 1000\; 0100$
    \task $(423)_8 = 1\; 0001\; 0011$
    \task $(2417)_8 = 101\; 0000\; 1111$
\end{tasks}

\subsection*{4.2.7}
\begin{tasks}
    \task $(80\text{E})_{16} = 1000\; 0000\; 1110$
    \task $(135\text{AB})_{16} = 1\; 0011\; 1010\; 1011$
    \task $(\text{ABBA})_{16} = 1010\; 1011\; 1011\; 1010$
    \task $(\text{DEFACED})_{16} = 1101\; 1110\; 1111\; 1010\; 1100\; 1110\; 1101$
\end{tasks}

\subsection*{4.2.9}
\[
    (\text{ABCDEF})_{16} = 1010\; 1011\; 1100\; 1101\; 1110\; 1111
.\]

\subsection*{4.2.11}
\[
    (1011\; 0111\; 1011)_2 = (B7B)_{16}
.\]

\subsection*{4.3.1}
\begin{tasks}(2)
    \task $21 = 3 \cdot 7 \implies$ not prime
    \task $29$ is prime
    \task $71$ is prime
    \task $97$ is prime
    \task $111 = 3 \cdot 37 \implies$ not prime
    \task $143 = 11 \cdot 13 \implies$ not prime
\end{tasks}

\subsection*{4.3.3}
\begin{tasks}(2)
    \task $88 = 8 \cdot 11 = 2^3 \cdot 11$
    \task $126 = 2 \cdot 63 = 2 \cdot 3^2 \cdot 7$
    \task $729 = 3 \cdot 243 = 3^2 \cdot 81 = 3^2 \cdot 3^4 = 3^6$
    \task $1001 = 11 \cdot 91 = 7 \cdot 11 \cdot 13$
    \task $1111 = 11 \cdot 101$
    \task $909,090 = 2 \cdot 3^3 \cdot 5 \cdot 7 \cdot 13 \cdot 37$
\end{tasks}

\subsection*{4.3.5}
The prime factorization of $10!$ is the product of the prime factorizations of the integers $\leq 10$. Therefore
\[
    10! = 2^{(1+2+3+1+1)} \cdot 3^{(1+2+1)} \cdot 5^{(1+1)} \cdot 7 = 2^8 \cdot 3^4 \cdot 5^2 \cdot 7
.\]

\subsection*{4.3.15}
Since $30 = 2 \cdot 3 \cdot 5$ its all integers below $30$ that do not have any of those as prime factors, hence
\[
    1, 7, 11, 13, 17, 23
.\]

\subsection*{4.3.17}
\begin{tasks}
    \task Yes they are pairwise relatively prime
    \task No since $\gcd(15,21) = 3 \neq 1$
    \task Yes they are pairwise relatively prime 
    \task Yes they are pairwise relatively prime 
\end{tasks}

\subsection*{4.3.19}
We prove via contraposition.

\begin{proof}
    Assume that $n \in \Z^+$ is not prime. Then $\exists a,b \in \Z^+$ such that $ab = n$ and both $a,b > 1$. Note then that
    \[
    2^n - 1 = 2^{ab} - 1 = \qty(2^a - 1) \qty(2^{a(b-1)} + 2^{a(b-2)} + \ldots + 2^a + 1)
    .\]
    Since $a > 1$, then $2^a > 2 \implies 2^a - 1 > 1$. Since $b > 1$ the right hand side of the factorization is also guranteed to be larger than $1$. Therefore $2^n - 1$ factors into two integers larger than $1$, meaning it is must be not prime.
\end{proof}

\subsection*{4.3.21}
\begin{tasks}
    \task $\phi(4) = \#\qty{1,3} = 2$
    \task $\phi(10) = \#\qty{1,3,7,9} = 4$
    \task $\phi(13) = \#\qty{1,2,3,4,5,6,7,8,9,10,11,12} = 12$
\end{tasks}

\subsection*{4.3.23}
There are $p^k$ candidates to consider. Since $p$ is prime, any number is coprime to it except for the powers of $p$ up to $p^k$. There are $p^{k-1}$ powers of $p$ less than or equal to $p^k$ meaning
\[
    \phi\qty(p^k) = p^k - p^{k-1}
.\]

\subsection*{4.3.25}
\begin{tasks}(3)
    \task $3^5 \cdot 5^3 = 30,375$
    \task $1$
    \task $23^{17}$
    \task $41 \cdot 43 \cdot 53$
    \task $1$
    \task $1111$
\end{tasks}

\subsection*{4.3.27}
\begin{tasks}(3)
    \task $2^{11} \cdot 3^7 \cdot 5^9 \cdot 7^3$
    \task $2^9 \cdot 3^7 \cdot 5^5 \cdot 7^3 \cdot 11 \cdot 13 \cdot 17$
    \task $23^{31}$
    \task $41 \cdot 43 \cdot 53$
    \task $2^{12} \cdot 3^{13} \cdot 5^{17} \cdot 7^{21}$
    \task Undefined
\end{tasks}

\subsection*{4.3.33}
\begin{tasks}
    \task Since $18 = 1(12) + 6$ and $12 = 2(6) + 0$, the last non zero remainder is $6$ and hence $\gcd(12, 18) = 6$.
    \task Using the Euclidean Algorithm:
    \begin{align*}
        201 &= 1(111) + 90 \\
        111 &= 1(90) + 21 \\
        90 &= 4(21) + 6 \\
        21 &= 3(6) + 3 \\
        6 &= 2(3) + 0
    \end{align*}
    Since the last non-zero remainder was $3$, $\gcd(201, 111) = 3$
    \task Using the Euclidean Algorithm:
    \begin{align*}
        1331 &= 1(1001) + 330 \\
        1001 &= 3(330) + 11 \\
        330 &= 11(33) + 0
    \end{align*}
    Since the last non-zero remainder was $11$, $\gcd(1331, 1001) = 11$
    \task Using the Euclidean Algorithm:
    \begin{align*}
        54321 &= 4(12345) + 4941 \\
        12345 &= 2(4941) + 2463 \\
        4941 &= 2(2463) + 15 \\
        2463 &= 164(15) + 3 \\
        15 &= 5(3) + 0
    \end{align*}
    Since the last non-zero remainder was $3$, $\gcd(12345, 54321) = 3$
    \task Using the Euclidean Algorithm:
    \begin{align*}
        5040 &= 5(1000) + 40
        1000 &= 40(25) + 0
    \end{align*}
    Since the last non-zero remainder was $40$, $\gcd(1000, 5040) = 40$
    \task Using the Euclidean Algorithm:
    \begin{align*}
        9888 &= 1(6060) + 3828 \\
        6060 &= 1(3828) + 2232 \\
        3828 &= 1(2232) + 1596 \\
        2232 &= 1(1596) + 636 \\
        1596 &= 2(636) + 324 \\
        636 &= 1(324) + 312 \\
        324 &= 1(312) + 12 \\
        312 &= 26(12) + 0
    \end{align*}
    Since the last non-zero remainder was $12$, $\gcd(9888, 6060) = 12$
\end{tasks}

\subsection*{4.3.39}
\begin{tasks}(3)
    \task $1 = 11 - 10$
    \task $1 = 21(21) - 10(44)$
    \task $12 = 48 - 36$
    \task $1 = 13(55) - 21(34)$
    \task $3 = 11(213) - 20(117)$
    \task $223 = 0 + 223$
    \task $1 = 37(2347) - 706(123)$
    \task $2 = 1128(3454) - 835(4666)$
\end{tasks}

\subsection*{4.4.1}
\[
    15(7) \equiv 105 \equiv 26(4) + 1 \equiv 1 \pmod{24}
.\]

\subsection*{4.4.3}
\[
    4(7) \equiv 28 \equiv 3(9) + 1 \equiv 1 \pmod{9}
.\]

\subsection*{4.4.5}
\begin{tasks}
    \task Since $1 = 9 - 2(4)$, we have $-2 \equiv 7 \pmod{9}$ as the inverse of $4$
    \task Since $1 = 52(19) - 7(141)$, we have $52$ as the inverse of $19$
    \task Since $1 = 34(55) - 21(89)$, we have $34$ as the inverse of $55$
    \task Since $1 = 73(89) - 28(232)$, we have $73$ as the inverse of $89$
\end{tasks}

\subsection*{4.4.9}
The solution will be $5 \cdot 4^{-1} \equiv 5 \cdot 7 \equiv 35 \equiv 8 \pmod{9}$

\subsection*{4.4.11}
\begin{tasks}
    \task $x \equiv 67 \pmod{141}$
    \task $x \equiv 88 \pmod{89}$
    \task $x \equiv 146 \pmod{232}$
\end{tasks}

\subsection*{4.4.15}
\begin{proof}
    Let $m' = \frac{m}{\gcd(c,m)}$. Since all shared factors between $c$ and $m$ are removed from $m$, $m'$ is coprime to $c$. Note that $m$ divides $ac - bc = (a-b)c$ and that $m'$ divides it is as well. Therefore since $c$ and $m'$ are coprime, it follows $m'$ must divides $a - b$. Therefore $a \equiv b \pmod{m'}$.
\end{proof}

\subsection*{4.4.21}
We can construct the solution
\[
    x = 1(3)(5)(11)(1) + 2(2)(5)(11)(2) + 3(2)(3)(11)(1) + 4(2)(3)(5)(7)
.\]

Therefore the solutions are all of the form
\[
    x \equiv 323 \pmod{330}
.\]

\subsection*{4.4.37}
\begin{tasks}
    \task $2^{340} \equiv \qty(2^{10})^{34} \equiv \qty(2^{11 - 1})^{34} \equiv 1^{34} \equiv 1 \pmod{11}$
    \task $2^{340} \equiv 32^{68} \equiv 1^68 \equiv 1 \pmod{31}$
    \task $2^{340} \equiv 1 \cdot 1 \equiv 1 \pmod{31 \cdot 11}$ and $31 \cdot 11 = 341$
\end{tasks}

\subsection*{4.5.15}
The check digit is $4$

\subsection*{4.5.17}
The ISBN is $125967651X$. Note that
\[
    1(1) + 2(2) + 5(3) + 9(4) + 6(5) + 7(6) + 6(7) + 5(8) + 1(9) + 10(10) = 319 \equiv 0 \pmod{11}
.\]

Therefore the ISBN is valid.

\subsection*{4.5.21}
\begin{tasks}(4)
    \task The code cannot be recovered
    \task $Q = 5$
    \task $Q = 7$
    \task $Q = 8$
\end{tasks}

\subsection*{5.1.3}
\begin{tasks}
    \task $P(1)$ is the statement that $1^2 = \frac{1(1 + 1)(2(1) + 1)}{6}$
    \task $P(1)$ is true since $1^2 = 1 = \frac{6}{6} = \frac{1(2)(3)}{6} = \frac{1(1+1)(2(1)+1)}{6}$
    \task $P(k)$, that is $1^2 + 2^2 + \ldots + (k-1)^2 + k^2 = \frac{k(k+1)(2k+1)}{6}$
    \task $P(k) \to P(k+1)$
    \task \begin{proof}
        Consider the sum
        \[
            1^2 + 2^2 + \ldots + (k-1)^2 + k^2 + (k+1)^2
        .\]
        By the inductive hypothesis
        \[
            1^2 + 2^2 + \ldots + (k-1)^2 + k^2 + (k+1)^2 = \frac{k(k+1)(2k+1)}{6} + (k+1)^2
        .\]
        Since
        \begin{align*}
            \frac{k(k+1)(2k+1)}{6} + (k+1)^2 &= \frac{2k^3 + 3k^2 + k}{6} + \frac{k^2 + 2k + 1}{6} \\
            &= \frac{2k^3 + 4k^2 + 3k + 1}{6} \\
            &= \frac{(k+1)(k+2)(2k+3)}{6}
        \end{align*}

        we have what we sought to show.
    \end{proof}
    \task $P(1)$ is true and $P(k) \to P(k+1)$ meaning by the principal of mathematical induction $P(n)$ is true for all $n \in \Z^+$
\end{tasks}

\subsection*{5.1.7}
\begin{proof}
    Proceed with induction. Consider the base case $n = 1$. Note that
    \[
        3 = \frac{3(4)}{4} = \frac{3(5^1-1)}{4}
    .\]
    Therefore the base case is established. Let $n \in \Z^+$ and assume that
    \[
        3 + 3\cdot 5 + 3\cdot 5^2 + \ldots + 3\cdot 5^n = \frac{3\qty(5^{n+1} - 1)}{4}
    .\]
    Note that
    \[
        3 + 3\cdot 5 + 3\cdot 5^2 + \ldots + 3\cdot 5^n + 3\cdot 5^{n+1} = \frac{3\qty(5^{n+1} - 1)}{4} + 3 \cdot 5^{n+1}

    \]
    by the inductive hypothesis. Therefore
    \begin{align*}
        \frac{3\qty(5^{n+1} - 1)}{4} + 3 \cdot 5^{n+1} &= \frac{3\qty(5^{n+1} - 1) + 3 \cdot 4 \cdot 5^{n+1}}{4} \\
                                                       &= \frac{3\qty(5^{n+1} - 1) + 3 \cdot (5-1) \cdot 5^{n+1}}{4} \\
                                                       &= \frac{3\qty(5^{n+1} - 1) + 3 \cdot (5^{n+2}-5^{n+1})}{4} \\
                                                       &= \frac{3\qty(5^{n+2} - 1)}{4} \\
    \end{align*}
    hence we have
    \[
        3 + 3\cdot 5 + 3\cdot 5^2 + \ldots + 3\cdot 5^n + 3\cdot 5^{n+1} = \frac{3\qty(5^{n+2} - 1)}{4}
    \]
    which was to be shown. Hence the statement holds for all $n \in \Z^+$.
\end{proof}

\subsection*{5.1.15}
\begin{proof}
    Proceed with induction. Consider the base case $n = 1$. Note that
    \[
        1 \cdot 2 = 2 = \frac{6}{3} = \frac{1(2)(3)}{6} = \frac{1(1+1)(1+2)}{6}
    .\]
    Therefore the base case holds. Let $n \in \Z^+$ and assume that
    \[
        1(2) + 2(3) + \ldots + n(n+1) = \frac{n(n+1)(n+2)}{3}
    .\]
    By the inductive hypothesis
    \[
        1(2) + 2(3) + \ldots + n(n+1) + (n+1)(n+2) = \frac{n(n+1)(n+2)}{3} + (n+1)(n+2)
    .\]
    We have
    \begin{align*}
        \frac{n(n+1)(n+2)}{3} + (n+1)(n+2) = \frac{n(n+1)(n+2) + 3(n+1)(n+2)}{3} \\
        &= \frac{(n+1)(n+2)(n+3)}{3}
    \end{align*}
    Therefore
    \[
        1(2) + 2(3) + \ldots + n(n+1) + (n+1)(n+2) = \frac{(n+1)(n+2)(n+3)}{3}
    \]
    which was to be shown. Therefore the original statement is true for all $n \in \Z^+$.
\end{proof}

\subsection*{5.1.23}
The integers for which the statement holds are $n \geq 4$.
\begin{proof}
    Proceed with induction. Consider the base case $n = 4$. Note that
    \[
        2(4) + 3 = 8 + 3 = 11 \leq 16 = 2^4
    .\]
    Therefore the base case holds. Let $n \in \Z^+$ with $n \geq 4$ and assume that
    \[
        2n + 3 \leq 2^n
    .\]
    Note that
    \[
        2(n+1) + 3 = 2n + 5 \leq 4n + 6
    .\]
    Therefore by the inductive hypotheis
    \[
        2^{n+1} = 2 \cdot 2^n \geq 2(2n + 3) = 4n + 6 \geq 2(n+1) + 3
    \]
    which was to be shown. Therefore the original statement holds for all integers $n \geq 4$.
\end{proof}

\subsection*{5.2.37}
\begin{proof}
    Assume that $a = dq + r = dq' + r'$ where $0 \leq r, r' < d$. Note that
    \[
        dq + r = dq' + r' \implies d(q-q') = r - r'
    .\]
    Therefore $d$ divides $r - r'$. However $-d < r - r' < d$ meaning $r - r' = 0 \implies r = r'$. This means as well that $q = q'$. Hence the original solution was unique.
\end{proof}

\subsection*{5.2.41}
\begin{proof}
    Assume towards contradiction that the well ordering principle is false. Then we can find a set $\varnothing \neq S \subset \Z^+$ such that there is no least element. We can then define the statement
    \[
        P(n) \coloneq i \notin S, 0 \leq i \leq n
    .\]
$P(0)$ is true since if $0 \in S$, $0$ must be a least element which is assumed not possible. It also follows that $P(n) \to P(n+1)$ because otherwise we would have $n+1$ as a least elements of $S$. Therefore by induction we have that $P(n)$ is true for all $n \in \Z^+$. However, this means no $n \in \Z^+$ is in $S$ and hence $S = \varnothing$, a contradiction.
\end{proof}

\end{document}
