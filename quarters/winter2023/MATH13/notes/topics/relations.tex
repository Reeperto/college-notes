\documentclass[../notes.tex]{subfiles}

\begin{document}

\banner{Relations}

Relations will serve useful in concretizing the idea of functions and in general how elements of sets are related to each other.

\hfil\linebreak

\begin{definition}[Relation]
	A relation $\mathcal{R}$ on a set $A$ is defined as $\mathcal{R} \subseteq A \times A$ with 3 possible properties
	\begin{align*}
		&\textbf{Reflexive} &\forall &a \in A, (a,a) \in \mathcal{R} \\
		&\textbf{Symmetric} &\forall &a,b \in A, (a,b) \in \mathcal{R} \implies (b,a) \in \mathcal{R} \\
		&\textbf{Transitive} &\forall &a,b,c \in A, (a,b), (b,c) \in \mathcal{R} \implies (a,c) \in \mathcal{R}
	.\end{align*}
\end{definition}

Consider the relation $\mathcal{R}$ defined as $(\leq, \mathbb{R})$. Which properties of a relation does it satisfy?

\begin{proof}
	Let 
\end{proof}

If a relation $\mathcal{R}$ obeys all 3 possible properties of a relation, it is called an \textbf{Equivalence Relation} often denoted by a $\sim$. Let $\mathcal{R}$ be the relation $\sim$ on $\mathbb{Z}$ such that
\[
	x \sim y \Longleftrightarrow x - y \text{ is even}
.\]
Is $\mathcal{R}$ an equivalence relation?

\begin{proof}
	Proceed to show that $\mathcal{R}$ is an equivalence relation.
	\hfill
	\hfill\linebreak

	\quad\begin{minipage}{\dimexpr\textwidth-2cm}
		(Reflexivity) Let $a \in \mathbb{Z}$. It follows that
		\begin{align*}
			a \sim a &\implies 2 \vert a - a \\
			 &\implies 2 \vert 0
		\end{align*}
		which is true. Therefore $\mathcal{R}$ is reflexive.
	\end{minipage}

	\hfill\linebreak

	\quad\begin{minipage}{\dimexpr\textwidth-2cm}
		(Symmetry) Let $a,b \in \mathbb{Z}$. It follows that
		\begin{align*}
			a \sim b &\implies a - b = 2k \\
			 &\implies b - a = 2(-k) \\
			 &\implies b \sim a
		\end{align*}
		hence $\mathcal{R}$ is symmetric.
	\end{minipage}

	\hfill\linebreak

	\quad\begin{minipage}{\dimexpr\textwidth-2cm}
		(Transitivity) Let $a,b,c \in \mathbb{Z}$. 
	\end{minipage}
\end{proof}

\end{document}
