\documentclass[../notes.tex]{subfiles}

\begin{document}
\newcommand{\powerset}[1]{\mathcal{P}(#1)}

\banner{Set Theory II}

\subsection{Cartesian Product}
\begin{definition}[Cartesian Product]
	Let $A$ and $B$ be sets. Their Cartesian product is defined as
	\[
		A \times B = \qty{(a,b) : a \in A, b \in B}
	.\]
\end{definition}

For example, consider the 2-dimensional plane. Each point can be defined in Cartesian coordinates, and therefore as an ordered pair $(x,y)$ where $x,y \in \mathbb{R}$. This means that all elements of the Cartesian plane can be expressed as elements of the set
\[
	\mathbb{R} \times \mathbb{R} = \mathbb{R}^2 = \qty{(x,y) : x,y \in \mathbb{R}}
.\]

\begin{theorem}
	\label{thm:cartesiancardinality}
	If $A$ and $B$ are finite sets, then $\qty|A \times B| = |A|\cdot |B|$
\end{theorem}

\begin{proof}
	\newcolumntype{L}{>{$}c<{$}}
	Let $|A| = m$ and $|B| = n$. Listing out $A \times B$ in a grid pattern results in
	\[
		\begin{tabular}{L L L L L L L L}
			A\times B =& \{& (a_1, b_1) & (a_1, b_2) & (a_1, b_3) & \ldots & (a_1, b_n) & \\
			   && (a_2, b_1) & (a_2, b_2) & (a_2, b_3) & \ldots & (a_2, b_n) & \\
			   && \vdots & \vdots & \vdots & \ddots & \vdots & \\
				 && (a_m, b_1) & (a_m, b_2) & (a_m, b_3) & \ldots & (a_m, b_n) & \}.\\
		\end{tabular}
	\]
	Every ordered pair is written only once. Since there are $m$ rows and $n$ columns, the number of elements is $mn$, therefore $\qty|A \times B| = mn$.
\end{proof}

\pagebreak

Here is a basic set relationship involving the Cartesian product.
\begin{multicols}{2}
	\begin{theorem}
		Let $A,B,C,D$ be any sets. Then $(A\times B)\cup (C\times D) \subseteq (A\cup C)\times (B\cup D)$. 
	\end{theorem}
	Visually, this can be seen as two regions in the Cartesian plane being contained within a larger region where the boundaries are equivalent to the union of the individual region's sides.

	\columnbreak

	\centering
	\tikz[scale=0.8]{
		\newcommand\XM{6}
		\newcommand\YM{6}
		\newcommand\POS{0.6}
		\draw[color=gray, step=0.5] (0,0) grid (\XM,\YM);
		\draw[color=gray, very thick, ->] (0,-0.2) -- (0,\YM + 0.2);
		\draw[color=gray, very thick, ->] (-0.2,0) -- (\XM + 0.2,0);
		\draw[color=red, very thick, Bracket-Bracket] (2.5,0) -- (5.5,0)  node[pos=\POS, below] {$C$};
		\draw[color=black, very thick, Bracket-Bracket] (0,2.5) -- (0,5.5)  node[pos=\POS, left] {$D$};
		\draw[color=blue, very thick, Bracket-Bracket] (1,0) -- (3.5,0) node[pos=1 - \POS, below] {$A$};
		\draw[color=green, very thick, Bracket-Bracket] (0,1) -- (0,3.5)  node[pos=1 - \POS, left] {$B$};
		\fill[fill = yellow] (2.5, 2.5) rectangle (5.5,5.5);
		\fill[fill = yellow] (1, 1) rectangle (3.5,3.5);
		\draw[color = black, very thick] (2.5, 2.5) rectangle (5.5,5.5);
		\draw[color = black, very thick] (1, 1) rectangle (3.5,3.5);
		\draw[color = black, dashed, very thick] (1, 1) rectangle (5.5,5.5);
	}
\end{multicols}

\begin{proof}
	Let $(x,y) \in (A \times B) \cup (C \times D)$. If $(x,y)$ is in $(A \times B)$, then $x \in A$ and $y \in B$. Therefore $x \in A \cup C$ and $y \in B \cup D$, meaning $(x,y) \in (A \cup C) \times (B \cup D)$. If $(x,y)$ is in $(C \times D)$, then $x \in C$ and $y \in D$. Therefore $x \in A \cup C$ and $y \in B \cup D$, meaning $(x,y) \in (A \cup C) \times (B \cup D)$, as required.
\end{proof}

Here is a proof of a more generalized version of Theorem \ref{thm:cartesiancardinality} using induction.

\begin{theorem}
	For all $n\in\mathbb{N}$, if $A_1, \ldots , A_n$ are finite sets, then 
	\[
		\qty|A_1 \times \ldots \times A_n| = \qty|A_1| \ldots \qty|A_n|
	.\]
\end{theorem}


\begin{proof}
	Proceed with induction to show that for all $n\in\mathbb{N}$, if $A_1, \ldots , A_n$ are finite sets, then $\qty|A_1 \times \ldots \times A_n| = \qty|A_1| \ldots \qty|A_n|$. Consider the base case when $n = 1$. Then $\qty|A_1| = \qty|A_1|$, hence the base case is true. Assume for a fixed $n \in \mathbb{N}$ that $\qty|A_1 \times \ldots \times A_n| = \qty|A_1| \ldots \qty|A_n|$. Consider then the Cartesian product $A_1 \times \ldots \times A_{n+1}$. This will result in every ordered pair in $A_1 \times \ldots \times A_n$ being repeated with a new element from $A_{n+1}$ added in each time. Hence the number of ordered pairs in the set $A_1 \times \ldots \times A_{n+1}$ will be the same as the number of elements of $A_1 \times \ldots A_n$ multiplied by the number of elements in $A_{n+1}$. By the induction hypothesis, the number of elements in $A_1 \times \ldots \times A_n = \qty|A_1| \ldots \qty|A_n|$ and the number of elements in $A_{n+1}$ is $\qty|A_n+1|$. Hence
	\[
		\qty|A_1 \times \ldots \times A_{n+1}| = \qty|A_1| \ldots \qty|A_{n+1}|
	.\]
	Therefore for all $n\in\mathbb{N}$, if $A_1, \ldots , A_n$ are finite sets, then 
	\[
		\qty|A_1 \times \ldots \times A_n| = \qty|A_1| \ldots \qty|A_n|
	.\]
\end{proof}

\subsection{Power Sets}

\begin{definition}[Power Set]
	The \textit{power set} of a set $A$ is the set $\powerset{A}$ of all subsets of $A$. That is
	\[
		\powerset{A} = \qty{B : B \subseteq A}
	.\]
	Equivalently $B \in \powerset{A} \Longleftrightarrow B \subseteq A$.
\end{definition}

\subsection{Indexed Collections}

\begin{definition}[Indexed Collection]
	Given a family of indexed sets $\qty{A_n : n \in I}$, the union or intersection of the family can be formed as
	\begin{align*}
		\bigcup_{n \in I} A_n &= \qty{x : x \in A_n \text{ for some } n \in I} \\
		\bigcap_{n \in I} A_n &= \qty{x : x \in A_n \text{ for all } n \in I}
	.\end{align*}
	Equivalently,
	\begin{align*}
		x \in \bigcup_{n \in I} A_n \Longleftrightarrow \exists n \in I, x \in A_n \\
		x \in \bigcap_{n \in I} A_n \Longleftrightarrow \forall n \in I, x \in A_n
	\end{align*}
\end{definition}

\end{document}
