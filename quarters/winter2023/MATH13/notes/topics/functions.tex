\documentclass[../notes.tex]{subfiles}
\graphicspath{
    {"../figures"}
}

\begin{document}

\banner{Functions}

\subsection{Introduction to Functions}

\begin{definition}[Function]
	A function $f$ is a \textit{'rule'} that assigns elements from a domain set $A$ to a codomain set $B$ with a \textbf{one-to-one} correspondence.
\end{definition}

If a function $f$ maps elements from a set $A$ to a set $B$, then it is notated as
\[
	f: A \rightarrow B
.\]

When considering an element $a \in A$ and its corresponding functional mapping $b \in B$, we say that $b = f(a)$ and also that
\[
	f: a \mapsto b
.\]
which reads as $f$ maps $a$ to $b$.

\subsection{Classification}

There are three important classifications of functions.

\begin{definition}[Injectivity]
	A function $f: A \rightarrow B$ is considered injective, an injection, or one-to-one if it never has the same output twice. Equivalaltenly
	\[
		\forall a_1, a_2 \in A, f(a_1) = f(a_2) \implies a_1 = a_2
	.\]
\end{definition}

\begin{wrapfigure}{r}{0.63\linewidth}
	\centering
	\begin{tikzpicture}[scale=1]
		\begin{axis}[
		    axis lines = left,
		    grid,
				grid style=dashed,
		]
			\addplot [
			    domain=0:2, 
			    samples=15, 
			    color=red,
			]
			{x^2};
		\end{axis}
	\end{tikzpicture}
\end{wrapfigure}
Consider the function 
\[
	f : [0,2] \rightarrow \mathbb{R} : x \mapsto x^2
.\] 
Graphically it is obvious that for every y-value, there is only a singularlly associated x-value. However, the function $f$ maps to all the real numbers. Consider an output of $16$. This would require an input of $4$, however that is outside the range. We say that in this case that $f$ is not \textit{surjective}.

\begin{theorem}[Function Cardinality]
	All of the following statements are equivalent.

	\begin{enumerate}
		\item $\qty|A| \leq \qty|B|$
		\item $\exists f: A \rightarrow B, f$ is injective
		\item $\exists f: B \rightarrow A, f$ is surjective
	\end{enumerate}
\end{theorem}

\end{document}
