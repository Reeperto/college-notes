\documentclass[12pt]{extarticle}

% Document Layout and Font
\usepackage{subfiles}
\usepackage[margin=2cm, headheight=15pt]{geometry}
\usepackage{fancyhdr}
\usepackage{enumitem}	
\usepackage{wrapfig}
\usepackage{multicol}
\usepackage{caption, subcaption}

\usepackage[p,osf]{scholax}

\renewcommand*\contentsname{Table of Contents}
\renewcommand{\headrulewidth}{0pt}
\pagestyle{fancy}
\fancyhf{}
\fancyfoot[R]{$\thepage$}
\setlength{\parindent}{0cm}
\setlength{\headheight}{17pt}
\hfuzz=9pt

% Utility Management
\usepackage{color}
\usepackage{colortbl}
\usepackage{xcolor}
\usepackage{xpatch}
\usepackage{xparse}

\definecolor{links}{HTML}{1c73a5}
\definecolor{bar}{HTML}{584AA8}

% Math Packages
\usepackage{mathtools, amsmath, amsthm, thmtools, amssymb, physics}
\usepackage[scaled=1.075,ncf,vvarbb]{newtxmath}

\newcommand\B{\mathbb{B}}
\newcommand\C{\mathbb{C}}
\newcommand\R{\mathbb{R}}
\newcommand\Q{\mathbb{Q}}
\newcommand\N{\mathbb{N}}
\newcommand\Z{\mathbb{Z}}

\newcommand\Prob[1]{\mathbb{P}\qty(#1)}
\newcommand\Var[1]{\text{Var}\qty(#1)}
\newcommand\Exp[1]{\mathbb{E}\qty[#1]}
\newcommand\ball[1]{\B\qty(#1)}
\newcommand\res[1]{\underset{#1}{\operatorname{Res}}\;}
\renewcommand\pv{\mathrm{p.v.}}

\newcommand\conj[1]{\overline{#1}}
\DeclareMathOperator{\Arg}{Arg}
\DeclareMathOperator{\Log}{Log}
\DeclareMathOperator{\cis}{cis}

\DeclareMathOperator{\dom}{dom}
\DeclareMathOperator{\spann}{span}
\DeclareMathOperator{\nullity}{nullity}

\newcommand\st{\text{ s.t. }}

% TIKZ
\usepackage{tikz}
\usepackage{pgfplots}
\usetikzlibrary{arrows.meta}
\usetikzlibrary{math}
\usetikzlibrary{cd}
\usetikzlibrary{patterns}
\usetikzlibrary{decorations.markings}
\usetikzlibrary{calc}

% Boxes and Theorems
\usepackage[most]{tcolorbox}
\tcbuselibrary{skins}
\tcbuselibrary{breakable}
\tcbuselibrary{theorems}

\newtheoremstyle{default}{0pt}{0pt}{}{}{\bfseries}{\normalfont.}{0.5em}{}
\theoremstyle{default}

\renewcommand*{\proofname}{\textit{\textbf{Proof.}}}
\renewcommand*{\qedsymbol}{$\blacksquare$}
\tcolorboxenvironment{proof}{
	breakable,
	coltitle = black,
	colback = white,
	frame hidden,
	boxrule = 0pt,
	boxsep = 0pt,
	borderline west={3pt}{0pt}{bar},
	sharp corners = all,
	enhanced,
}

\newtheorem{theorem}{Theorem}[section]{\bfseries}{}
\tcolorboxenvironment{theorem}{
	breakable,
	enhanced,
	boxrule = 0pt,
	frame hidden,
	coltitle = black,
	colback = blue!7,
	left = 0.5em,
	sharp corners = all,
}

\newtheorem{corollary}{Corollary}[section]{\bfseries}{}
\tcolorboxenvironment{corollary}{
	breakable,
	enhanced,
	boxrule = 0pt,
	frame hidden,
	coltitle = black,
	colback = white!0,
	left = 0.5em,
	sharp corners = all,
}

\newtheorem{lemma}{Lemma}[section]{\bfseries}{}
\tcolorboxenvironment{lemma}{
	breakable,
	enhanced,
	boxrule = 0pt,
	frame hidden,
	coltitle = black,
	colback = green!7,
	left = 0.5em,
	sharp corners = all,
}

\newtheorem{definition}{Definition}[section]{\bfseries}{}
\tcolorboxenvironment{definition}{
	breakable,
	coltitle = black,
	colback = white,
	frame hidden,
	boxsep = 0pt,
	boxrule = 0pt,
	borderline west = {3pt}{0pt}{orange},
	sharp corners = all,
	enhanced,
}

\newtheorem{example}{Example}[section]{\bfseries}{}
\tcolorboxenvironment{example}{
	% title = \textbf{Example},
	% detach title,
	% before upper = {\tcbtitle\quad},
	breakable,
	coltitle = black,
	colback = white,
	frame hidden,
	boxrule = 0pt,
	boxsep = 0pt,
	borderline west={3pt}{0pt}{green!70!black},
	sharp corners = all,
	enhanced,
}

\newtheoremstyle{remark}{0pt}{4pt}{}{}{\bfseries\itshape}{\normalfont.}{0.5em}{}
\theoremstyle{remark}
\newtheorem*{remark}{Remark}


% TColorBoxes
\newtcolorbox{week}{
	colback = black,
	coltext = white,
	fontupper = {\large\bfseries},
	width = 1.2\paperwidth,
	size = fbox,
	halign upper = center,
	center
}

\newcommand{\banner}[2]{
    \pagebreak
    \begin{week}
   		\section*{#1}
    \end{week}
    \addcontentsline{toc}{section}{#1}
    \addtocounter{section}{1}
    \setcounter{subsection}{0}
}

% Hyperref
\usepackage{hyperref}
\hypersetup{
	colorlinks=true,
	linktoc=all,
	linkcolor=links,
	bookmarksopen=true
}


\fancyhead[R]{\textbf{Homework \#5}}
\setlength\parindent{0pt}

\begin{document}

\section*{Problem 5.2.2}

Prove by induction that for each natural number $n$, we have $\displaystyle\sum_{j=0}^{n} 2^j = 2^{n+1} - 1$.

\subsection*{Solution}

\begin{proof}
	Proceed with induction. Let $P(n): \displaystyle\sum_{j=0}^{n} 2^j = 2^{n+1} - 1$. Consider the base case when $n=1$. Then
	\begin{align*}
		\sum_{j=0}^{1} 2^j &= 2^2 - 1 \\
		2^0 + 2^1 &= 2^2 - 1 \\
		3 &= 3.
	\end{align*} $P(1)$ is true. Assume for some fixed $n \in \mathbb{N}$ that $P(n)$ is true. Then,
	\begin{align*}
		\sum_{j=0}^{n+1} 2^j &= 2^{n+1} + \sum_{j=0}^n 2^j \\
												 &= 2^{n+1} + 2^{n+1} - 1 \\
												 &= 2^{(n+1) + 1} - 1.
	\end{align*}
	Therefore $P(n+1)$ is true, meaning that for each natural number $n$, we have 
	\[
		\sum_{j=0}^{n} 2^j = 2^{n+1} - 1
	.\]
\end{proof}

\section*{Problem 5.2.5}

Show by induction that for every $n \in \mathbb{N}$ we have: $n \equiv 5 \pmod{3}$ or $n \equiv 6 \pmod{3}$ or $n \equiv 7 \pmod{3}$.

\subsection*{Solution}

\begin{proof}
	Proceed with induction. Let $P(n): (n \equiv 5 \mod{3}) \lor (n \equiv 6 \mod{3}) \lor (n \equiv 7 \mod{3})$. By the properties of modular arithmetic, $P(n)$ can be restated as
	\[
		P(n): (n \equiv 0\text{ mod }3 ) \lor (n \equiv 1 \text{ mod }3) \lor (n \equiv 2 \text{ mod }3)
	.\]

	Consider the base case when $n=1$. Then $n \equiv 1 \pmod{3}$, therefore $P(1)$ is true. Assume for a fixed $n \in \mathbb{N}$ that $P(n)$ is true. Consider then three cases.

	\begin{enumerate}[leftmargin=1cm]
		\item If $n \equiv 0 \pmod{3}$, then $n + 1 \equiv 1 \pmod{3}$, meaning that $P(n+1)$ is true.
		\item If $n \equiv 1 \pmod{3}$, then $n + 1 \equiv 2 \pmod{3}$, meaning that $P(n+1)$ is true.
		\item If $n \equiv 2 \pmod{3}$, then $n + 1 \equiv 0 \pmod{3}$, meaning that $P(n+1)$ is true.
	\end{enumerate}

	Therefore $P(n)$ implies $P(n+1)$, meaning for every $n \in \mathbb{N}$ we have: $n \equiv 5 \pmod{3}$ or $n \equiv 6 \pmod{3}$ or $n \equiv 7 \pmod{3}$.
\end{proof}

\section*{Problem 5.2.6}

Prove by induction that, for all $n \in \mathbb{N}$, $1\cdot 2+2\cdot 3+3\cdot 4+\ldots +n(n+1) = \frac{1}{3} n(n+1)(n+2)$.

\subsection*{Solution}

\begin{proof}
	Proceed with induction. Let $P(n): \displaystyle\sum_{j=0}^{n} j(j+1) = \frac{1}{3} n (n+1) (n+2)$. Consider the base case $n=1$. Then
	\begin{align*}
		\sum_{j=0}^{1} j(j+1) &= \frac{1}{3} (1) (1+1) (1+2) \\
		2 &= \frac{1}{3} (6) \\
		2 &= 2
	.\end{align*} $P(1)$ is true. Assume for some fixed $n \in \mathbb{N}$ that $P(n)$ is true. Then it follows that
	\begin{align*}
		\sum_{j=0}^{n+1} j(j+1) &= (n+1)(n+2) + \sum_{j=0}^{n} j(j+1) \\
									 &= (n+1)(n+2) + \frac{1}{3} (n) (n+1) (n+2) \\
									 &= (\frac{1}{3} n + 1) (n + 1) (n + 2) \\
									 &= \frac{1}{3} (n + 1) (n + 2) (n + 3)
	.\end{align*}
	Therefore $P(n+1)$ is true, meaning for all $n \in \mathbb{N}$, $1\cdot 2+2\cdot 3+3\cdot 4+\ldots +n(n+1) = \frac{1}{3} n(n+1)(n+2)$
\end{proof}

\section*{Problem 5.3.2}

Suppose that $n \geq 3$. Prove that $\qty(\frac{n+1}{n})^2 < 2$.

\subsection*{Solution}

\begin{proof}
	Proceed with induction. Let $P(n): \qty(\frac{n+1}{n})^2 < 2$. Consider the base case when $n=3$. Then $\qty(\frac{3+1}{3})^2 = \qty(\frac{4}{3})^2 = \frac{16}{9} < 2$. Therefore $P(3)$ is true. Assume for a fixed $n \in \mathbb{N} \geq 3$ that $P(n)$ is true. Then
	\begin{align*}
		\qty(\frac{n+2}{n+1})^2 &= \qty(\frac{n+2}{n+1})^2 \qty(\frac{n+1}{n})^2 \qty(\frac{n}{n+1})^2 \\
														&< 2 \cdot \qty(\frac{n+2}{n+1})^2 \qty(\frac{n}{n+1})^2 \\
														&= 2 \cdot \qty(\frac{n^2 (n+2)^2}{(n+1)^4}) \\
														&= 2 \cdot \qty(\frac{n^4 + 4n^3 + 4n^2}{n^4 + 4n^3 + 6n^2 + 4n + 1}) \tag{$*$}
\end{align*}
	Note that $n^4 + 4n^3 + 4n^2 \leq n^4 + 4n^3 + 4n^2 + a$ when $a \geq 0$. Let $a = 2n^2 + 4n + 1$. Since $n$ is positive, $2n^2 + 4n + 1$ will always be greater than or equal to zero. Therefore $a \geq 0$. This means that 
	\begin{align*}
		n^4 + 4n^3 + 4n^2 &\leq n^4 + 4n^3 + 4n^2 + a \\
		n^4 + 4n^3 + 4n^2 &\leq n^4 + 4n^3 + 4n^2 + 2n^2 + 4n + 1 \\
		\frac{n^4 + 4n^3 + 4n^2}{n^4 + 4n^3 + 4n^2 + 2n^2 + 4n + 1} &\leq 1
	.\end{align*}
	Therefore returning back to $(*)$,
	\begin{align*}
		\qty(\frac{n+2}{n+1})^2 &< 2 \cdot \qty(\frac{n^4 + 4n^3 + 4n^2}{n^4 + 4n^3 + 6n^2 + 4n + 1}) \\
														&< 2 \cdot 1 \\
														&< 2
	.\end{align*}
	Therefore $P(n+1)$ is true, meaning that for all $n \geq 3$, $\qty(\frac{n+1}{n})^2 < 2$. 
\end{proof}

\section*{Problem 5.3.3}

Consider the following result. For every natural number $n \geq 2,$
\[
	\qty(1 - \frac{1}{4}) \qty(1 - \frac{1}{9}) \qty(1 - \frac{1}{16}) \ldots \qty(1 - \frac{1}{n^2}) = \frac{n+1}{2n}
.\]

\begin{enumerate}
	\item[(a)] If the statement is written in the form $\forall n \in\mathbb{N} \geq 2, P(n)$, what is the proposition $P(n)$?
	\item[(b)] Rewrite the statement using $\Pi$-notation.
	\item[(c)] Prove the result by induction.
\end{enumerate}

\subsection*{Solution}
\subsection*{Part A}

\[
	P(n): \qty(1 - \frac{1}{4}) \qty(1 - \frac{1}{9}) \qty(1 - \frac{1}{16}) \ldots \qty(1 - \frac{1}{n^2}) = \frac{n+1}{2n}
.\]

\subsection*{Part B}

\[
	P(n) : \prod_{i = 2}^n \qty(1-\frac{1}{i^2}) = \frac{n+1}{2n}
.\]

\subsection*{Part C}

\begin{proof}
	Proceed with induction. Let $P(n) : \displaystyle{\prod_{i = 2}^n} \qty(1-\frac{1}{i^2}) = \frac{n+1}{2n}$. Consider the base case when $n = 2$. Then
	\begin{align*}
		\prod_{i = 2}^2 \qty(1-\frac{1}{i^2}) &= \frac{2+1}{2(2)} \\
		\qty(1 - \frac{1}{4})  &= \frac{3}{4} \\
		\frac{3}{4} &= \frac{3}{4} \\
	\end{align*}
	$P(2)$ is true. Assume for some fixed $n \in \mathbb{N} \geq 2$ that $P(n)$ is true. Then
	\begin{align*}
		\prod_{i = 2}^{n+1} \qty(1-\frac{1}{i^2}) &= \qty(1 - \frac{1}{(n+1)^2}) \cdot \prod_{i = 2}^n \qty(1-\frac{1}{i^2}) \\
		&= \frac{(n+1)^2 - 1}{(n+1)^2} \cdot \frac{n + 1}{2n} \\
		&= \frac{n^2 + 2n + 1 - 1}{2n (n+1)} \\
		&= \frac{n(n + 2)}{2n (n+1)} \\
		&= \frac{n + 2}{2 (n+1)}
	.\end{align*}
	Therefore $P(n+1)$ is true, meaning $\forall n \in\mathbb{N} \geq 2, \displaystyle\prod_{i = 2}^n \qty(1-\frac{1}{i^2}) = \frac{n+1}{2n}$.
\end{proof}

\section*{Problem 5.3.4}

Recall the geometric series formula from calculus: if $r \neq 1$ is constant, and $n \in \mathbb{N}_0$, then

\begin{equation*}
	\sum_{k=0}^k r^n = \frac{1-r^{n-1}}{1 - r} \tag{$*$}
\end{equation*}

\begin{enumerate}
	\item[(a)] Explain why the given proof by induction is incorrect.	
	\item[(b)] Provide a correct proof of $(*)$.
\end{enumerate}

\subsection*{Part A}

The given proof is incorrect as it starts with $P(n+1)$. $P(n+1)$ is the goal of the proof, therefore attempting to prove $P(n) \implies P(n+1)$ by starting with $P(n+1)$ is incorrect.

\subsection*{Part B}

\begin{proof}
	Proceed with induction. Let $P(n): \displaystyle\sum_{k=0}^n r^k = \frac{1-r^{n-1}}{1 - r}$. Consider the base case when $n = 0$. Then $\displaystyle\sum_{k=0}^0 r^k = r^0 = 1 = \frac{1 - r^{0+1}}{1 - r}$, meaning $P(0)$ is true. Assume for some fixed $n \in \mathbb{N}_0$ that $P(n)$ is true. Then
	\begin{align*}
		\sum_{k=0}^{n+1} r^k &= r^{n+1} + \sum_{k=0}^{n} r^k \\
												 &= r^{n+1} + \frac{1-r^{n-1}}{1-r} \\
												 &= \frac{r^{n+1} - r^{n-2}}{1-r} + \frac{1-r^{n-1}}{1-r} \\
												 &= \frac{1 - r^{n-2}}{1-r}
	.\end{align*}
	Therefore $P(n+1)$ is true, meaning if $r \neq 1$ is constant, and $n \in \mathbb{N}_0$, then $\displaystyle \sum_{k=0}^k r^n = \frac{1-r^{n-1}}{1 - r}$ is true.
\end{proof}

\section*{Problem 5.3.8}

Prove that if $A \subseteq \mathbb{R}$ is a \textit{finite} set, then $A$ is well-ordered.

\subsection*{Solution}

Proof that any finite subset of the real numbers contains a minimum element, hence any finite subset of $A$ will contain a minimum element, meaning $A$ is well-ordered.

\begin{proof}
	Proof via induction that any finite subset of the real numbers has a minimum element. Let $X_n \subseteq \mathbb{R}$ such that it is finite and contains $n \in \mathbb{N}$ elements. Consider the base case of $X_1$. Then $\exists a \in \mathbb{R}$ such that $X_1 = \qty{a}$. It is obvious then that $X_1$ contains a minimum element since $a \leq a$. Assume for a fixed $n \in \mathbb{N}$ that $X_n$ has a miminum element $p$. Consider the set $X_{n+1}$. There exists $q \in \mathbb{R} \neq p$ such that $X_{n+1} = \qty{q} \cup X_n$. There are now two cases.
	\begin{enumerate}[leftmargin=2cm]
		\item[$(q < p):$] If $q$ is smaller than $p$, then the minimum element of $X_{n+1}$ will be $q$ since it is smaller than the minimum element of $X_{n}$.
		\item[$(q > p):$] If $q$ is greater than or equal to $p$, then the minimum element of $X_{n+1}$ will be $p$ since $p$ is smaller than $q$.
	\end{enumerate}
	In both cases, $X_{n+1}$ will have a minimum element. Therefore all finite subsets of the real numbers contain a minimum element.
\end{proof}

\section*{Problem 5.4.1}

Define a sequence $\qty(b_n)_{n=1}^{\infty}$ as follows:
\[
\begin{cases}
	b_n = b_{n-1} + b_{n-2} \\
	b_1 = 3, b_2 = 6
\end{cases}
.\]

Prove: $\forall n \in \mathbb{N}, b_n$ is divisible by $3$.

\subsection*{Solution}

\begin{proof}
	Proceed with strong induction. Consider the base cases where $n = 1$ and $n = 2$. Then $b_1 = 3 = 3(1)$ which is divisible by $3$ and $b_2 = 6 = 3(2)$ which is divisible by $3$. Fix $n \in \mathbb{N}_{\geq 2}$ and assume that $b_k$ is divisible by $3$ for all $ k \in \mathbb{N}, 1 \leq k \leq n$. Then
	\[
		b_{n+1} = b_{n} + b_{n - 1}
	.\]
	By the induction hypothesis, $b_n$ and $b_{n-1}$ are both divisible by $3$. Therefore there exists integers $a,b$ such that $b_n = 3a$ and $b_{n-1} = 3b$. Therefore
	\begin{align*}
		b_{n+1} &= b_n + b_{n-1} \\
						&= 3a + 3b \\
						&= 3(a + b)
	.\end{align*}
	Since $a + b \in \mathbb{Z}$, then $b_{n+1}$ is divisible by $3$. By strong induction we see that $b_n$ is divisible by $3$ for all $n \in \mathbb{N}$.
\end{proof}


\end{document}
