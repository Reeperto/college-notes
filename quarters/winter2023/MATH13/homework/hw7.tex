\documentclass[12pt]{extarticle}

\usepackage{multicol}

% Document Layout and Font
\usepackage{subfiles}
\usepackage[margin=2cm, headheight=15pt]{geometry}
\usepackage{fancyhdr}
\usepackage{enumitem}	
\usepackage{wrapfig}
\usepackage{multicol}
\usepackage{caption, subcaption}

\usepackage[p,osf]{scholax}

\renewcommand*\contentsname{Table of Contents}
\renewcommand{\headrulewidth}{0pt}
\pagestyle{fancy}
\fancyhf{}
\fancyfoot[R]{$\thepage$}
\setlength{\parindent}{0cm}
\setlength{\headheight}{17pt}
\hfuzz=9pt

% Utility Management
\usepackage{color}
\usepackage{colortbl}
\usepackage{xcolor}
\usepackage{xpatch}
\usepackage{xparse}

\definecolor{links}{HTML}{1c73a5}
\definecolor{bar}{HTML}{584AA8}

% Math Packages
\usepackage{mathtools, amsmath, amsthm, thmtools, amssymb, physics}
\usepackage[scaled=1.075,ncf,vvarbb]{newtxmath}

\newcommand\B{\mathbb{B}}
\newcommand\C{\mathbb{C}}
\newcommand\R{\mathbb{R}}
\newcommand\Q{\mathbb{Q}}
\newcommand\N{\mathbb{N}}
\newcommand\Z{\mathbb{Z}}

\newcommand\Prob[1]{\mathbb{P}\qty(#1)}
\newcommand\Var[1]{\text{Var}\qty(#1)}
\newcommand\Exp[1]{\mathbb{E}\qty[#1]}
\newcommand\ball[1]{\B\qty(#1)}
\newcommand\res[1]{\underset{#1}{\operatorname{Res}}\;}
\renewcommand\pv{\mathrm{p.v.}}

\newcommand\conj[1]{\overline{#1}}
\DeclareMathOperator{\Arg}{Arg}
\DeclareMathOperator{\Log}{Log}
\DeclareMathOperator{\cis}{cis}

\DeclareMathOperator{\dom}{dom}
\DeclareMathOperator{\spann}{span}
\DeclareMathOperator{\nullity}{nullity}

\newcommand\st{\text{ s.t. }}

% TIKZ
\usepackage{tikz}
\usepackage{pgfplots}
\usetikzlibrary{arrows.meta}
\usetikzlibrary{math}
\usetikzlibrary{cd}
\usetikzlibrary{patterns}
\usetikzlibrary{decorations.markings}
\usetikzlibrary{calc}

% Boxes and Theorems
\usepackage[most]{tcolorbox}
\tcbuselibrary{skins}
\tcbuselibrary{breakable}
\tcbuselibrary{theorems}

\newtheoremstyle{default}{0pt}{0pt}{}{}{\bfseries}{\normalfont.}{0.5em}{}
\theoremstyle{default}

\renewcommand*{\proofname}{\textit{\textbf{Proof.}}}
\renewcommand*{\qedsymbol}{$\blacksquare$}
\tcolorboxenvironment{proof}{
	breakable,
	coltitle = black,
	colback = white,
	frame hidden,
	boxrule = 0pt,
	boxsep = 0pt,
	borderline west={3pt}{0pt}{bar},
	sharp corners = all,
	enhanced,
}

\newtheorem{theorem}{Theorem}[section]{\bfseries}{}
\tcolorboxenvironment{theorem}{
	breakable,
	enhanced,
	boxrule = 0pt,
	frame hidden,
	coltitle = black,
	colback = blue!7,
	left = 0.5em,
	sharp corners = all,
}

\newtheorem{corollary}{Corollary}[section]{\bfseries}{}
\tcolorboxenvironment{corollary}{
	breakable,
	enhanced,
	boxrule = 0pt,
	frame hidden,
	coltitle = black,
	colback = white!0,
	left = 0.5em,
	sharp corners = all,
}

\newtheorem{lemma}{Lemma}[section]{\bfseries}{}
\tcolorboxenvironment{lemma}{
	breakable,
	enhanced,
	boxrule = 0pt,
	frame hidden,
	coltitle = black,
	colback = green!7,
	left = 0.5em,
	sharp corners = all,
}

\newtheorem{definition}{Definition}[section]{\bfseries}{}
\tcolorboxenvironment{definition}{
	breakable,
	coltitle = black,
	colback = white,
	frame hidden,
	boxsep = 0pt,
	boxrule = 0pt,
	borderline west = {3pt}{0pt}{orange},
	sharp corners = all,
	enhanced,
}

\newtheorem{example}{Example}[section]{\bfseries}{}
\tcolorboxenvironment{example}{
	% title = \textbf{Example},
	% detach title,
	% before upper = {\tcbtitle\quad},
	breakable,
	coltitle = black,
	colback = white,
	frame hidden,
	boxrule = 0pt,
	boxsep = 0pt,
	borderline west={3pt}{0pt}{green!70!black},
	sharp corners = all,
	enhanced,
}

\newtheoremstyle{remark}{0pt}{4pt}{}{}{\bfseries\itshape}{\normalfont.}{0.5em}{}
\theoremstyle{remark}
\newtheorem*{remark}{Remark}


% TColorBoxes
\newtcolorbox{week}{
	colback = black,
	coltext = white,
	fontupper = {\large\bfseries},
	width = 1.2\paperwidth,
	size = fbox,
	halign upper = center,
	center
}

\newcommand{\banner}[2]{
    \pagebreak
    \begin{week}
   		\section*{#1}
    \end{week}
    \addcontentsline{toc}{section}{#1}
    \addtocounter{section}{1}
    \setcounter{subsection}{0}
}

% Hyperref
\usepackage{hyperref}
\hypersetup{
	colorlinks=true,
	linktoc=all,
	linkcolor=links,
	bookmarksopen=true
}


\newcommand{\powerset}[1]{\mathcal{P}(#1)}
\fancyhead[R]{\textbf{Homework \#7}}
\renewcommand{\headrulewidth}{1pt}
\setlength\parindent{0pt}

\usetikzlibrary{arrows.meta}

\begin{document}

% Section 6.3: 1-, 2-, 3-, 4, 7-

\section*{Problem 6.3.1}

For each integer $n$, consider the set $B_n = \qty{n} \times \mathbb{R}$

\begin{enumerate}
	\item[(a)] Draw a picture of $\bigcup\limits_{n=2}^4 B_n$ (in the Cartesian plane).
	\item[(b)] Draw a picture of the set $C = [1,5] \times \qty{-2 ,2}$
	\item[(c)] Compute $\qty(\;\bigcup\limits_{n=2}^4 B_n) \cap C$ 
	\item[(d)] Compute $\bigcup\limits_{n=2}^4 (B_n \cap C)$ 
	\item[(e)] Compare $\qty(\;\bigcup\limits_{n=2}^4 B_n) \cap C$ and $\bigcup\limits_{n=2}^4 (B_n \cap C)$ 
\end{enumerate}

\subsection*{Solution}
\newcommand\XGm{-1}
\newcommand\YGm{-4}
\newcommand\XGM{6}
\newcommand\YGM{4}
\begin{multicols}{2}

	\centering
	\subsection*{Part A}

	\begin{tikzpicture}[domain=0:4]
	  \draw[very thin,color=gray] (\XGm,\YGm) grid (\XGM,\YGM);
		\draw[<->, very thick, color=gray] (\XGm - 0.1,0) -- (\XGM + 0.1,0);
		\draw[<->, very thick, color=gray] (0,\YGm - 0.1) -- (0,\YGM + 0.1);
	
		\foreach \x in {2,3,4}
			\draw[<->, very thick, color=blue] (\x,\YGm - 0.1) -- (\x,\YGM + 0.1);
		
	\end{tikzpicture}

	\columnbreak

	\subsection*{Part B}
	\begin{tikzpicture}[domain=0:4]
	  \draw[very thin,color=gray] (\XGm,\YGm) grid (\XGM,\YGM);
		\draw[<->, very thick, color=gray] (\XGm,0) -- (\XGM + 0.1,0);
		\draw[<->, very thick, color=gray] (0,\YGm - 0.1) -- (0,\YGM + 0.1);
	
		\foreach \y in {-2,2}
		\draw[Bracket-Bracket, very thick, color=blue] (1, \y) -- (5, \y);
		
	\end{tikzpicture}
\end{multicols}

\subsection*{Part C}
\[
	\qty(\;\bigcup\limits_{n=2}^4 B_n) \cap C = \qty{
		(2,2),
		(3,2),
		(4,2),
		(2,-2),
		(3,-2),
		(4,-2)
	}
.\]


\subsection*{Part D}

\[
	\bigcup\limits_{n=2}^4 (B_n \cap C) = \qty{
		(2,2),
		(3,2),
		(4,2),
		(2,-2),
		(3,-2),
		(4,-2)
	}
.\]

\subsection*{Part E}

Both sets are the same regardless whether the intersection with $C$ happens within the indexed collection or outside the indexed collection.

\section*{Problem 6.3.2}

For each real number $r$, define the interval $S_r = [r-1, r+3]$. Let $I = \qty{1,3,4}$. Determine $\bigcup\limits_{r\in I} S_r$ and $\bigcap\limits_{r\in I} S_r$.

\subsection*{Solution}

\begin{align*}
	\bigcup\limits_{r\in I} S_r = [0,4] \cup [2,6] \cup [3,7] &= [0,7] \\
	\bigcap\limits_{r\in I} S_r = [0,4] \cap [2,6] \cap [3,7] &= [3,4]
.\end{align*}


\section*{Problem 6.3.3}

Give an example of four different subsets $A, B, C$ and $D$ of $\qty{1, 2, 3, 4}$ such that all intersections of two subsets are different.

\subsection*{Solution}

\begin{align*}
	A &= \qty{1,2,3,4} \\
	B &= \qty{2,3} \\
	C &= \qty{3,4} \\
	D &= \qty{4,1}
.\end{align*}
\begin{align*}
	A \cap B &= \qty{2,3} & A \cap C &= \qty{3,4} & A \cap D &= \qty{4,1} \\
	B \cap C &= \qty{3} & B \cap D &= \varnothing & C \cap D &= \qty{4}
\end{align*}

\section*{Problem 6.3.4}

For each of the following collections of intervals, define an interval An for each $n \in \mathbb{N}$ such that indexed collection $\qty{A_n}_{n \in \mathbb{N}}$ is the given collection of sets. Then find both the union and intersection of the indexed collections of sets.

\begin{enumerate}
	\item[(a)] $\qty{\left[1, 2+1\right), \left[1, 2 + \frac{1}{2}\right), \left[1, 2+ \frac{1}{3}\right), \ldots}$
	\item[(b)] $\qty{
			\qty(-1,2), \qty(-\frac{3}{2},4),\qty(-\frac{5}{3}, 6), \qty(-\frac{7}{4}, 8), \ldots
		}$
	\item[(c)] $\qty{
			\qty(\frac{1}{4},1), \qty(\frac{1}{8},\frac{1}{2}), \qty(\frac{1}{16},\frac{1}{4}), \qty(\frac{1}{32},\frac{1}{8}), \qty(\frac{1}{64},\frac{1}{16}), \ldots
		}$
\end{enumerate}

\subsection*{Solution}
\subsection*{Part A}

\begin{align*}
	A_n = \left[ 1, 2 + \frac{1}{n} \right) \text{ with } \bigcup\limits_{n \in \mathbb{N}} A_n &= [1,3) \\
	\bigcap\limits_{n \in \mathbb{N}} A_n &= [1,2]
.\end{align*}

\begin{proof}
	Firstly consider $\bigcup_{n \in \mathbb{N}} A_n$. Let $x \in \bigcup_{n \in \mathbb{N}} A_n$. Then $x \in \left[1, 2 + \frac{1}{n} \right)$ for some $n \geq 1$. Therefore $1 \leq x < 2 + \frac{1}{n}$. If $n = 1$, then $1 \leq x < 3$, therefore $x \in [1,3)$, meaning $\bigcup_{n \in \mathbb{N}} A_n \subseteq [1,3)$. Let $x \in [1,3)$. It follows that $x \in A_1$ and therefore $x \in \bigcup_{n \in \mathbb{N}} A_n$, meaning $[1,3) \subseteq \bigcup_{n \in \mathbb{N}}$. Therefore $\bigcup_{n \in \mathbb{N}} A_n = [1,3)$.

	\hfill\linebreak

	Consider now $\bigcap_{n \in \mathbb{N}} A_n$. Let $x \in \bigcap_{n \in \mathbb{N}} A_n$. Then $\forall n \in \mathbb{N}, x \in [1, 2 + \frac{1}{n})$. If $x < 1$, then $x \notin A_1$. For all $n \in \mathbb{N}$, if $x = 2$ then $x \in A_n$ since $2 < 2 + \frac{1}{n}$. If $x > 2$, then $\frac{1}{N} \leq x$ with $N = \lceil \frac{1}{x} \rceil$, hence $x \notin A_N$. Therefore $\bigcap_{n \in \mathbb{N}} A_n \subseteq [1,2]$. Let $x \in [1,2]$. For all $n \in \mathbb{N}$, $x \in [1,2 + \frac{1}{n})$, hence $x \in \bigcap_{n \in \mathbb{N}} A_n$. Therefore $\bigcap_{n \in \mathbb{N}} A_n = [1,2]$.
\end{proof}

\subsection*{Part B}

\begin{align*}
	A_n = \qty(\frac{1-2n}{n}, 2n) \text{ with } \bigcup\limits_{n \in \mathbb{N}} A_n &= (-2,\infty) \\
	\bigcap\limits_{n \in \mathbb{N}} A_n &= (-1, 2)
.\end{align*}

\begin{proof}
	First consider $\bigcup_{n \in \mathbb{N}} A_n$. Let $x \in \bigcup_{n \in \mathbb{N}} A_n$. Assume that $x \leq -2$. Note that $-2 < -2 + \frac{1}{n} = \frac{1-2n}{n}$ for all $n \in \mathbb{N}$. Therefore $x$ cannot be in any $A_n$, meaning $x \notin \bigcup_{n \in \mathbb{N}} A_n$, meaning $x \nleq -2$. Therefore $x \in (-2, \infty)$, hence $\bigcup_{n \in \mathbb{N}} A_n \subseteq (-2, \infty)$. Let $x \in (-2, \infty)$. If $x = 0$, then $x \in A_n$ for all $n \in \mathbb{N}$ since the lower bound is always negative and the upper bound is always positive. If $x > 0$, then choose $N = \lceil x \rceil$. It follows that $x \in A_N$, meaning $x \in \bigcup_{n \in \mathbb{N}} A_n$. If $-2 < x < 0$, then since $\lim_{n \to \infty} \frac{1-2n}{n} = -2$, then a $N \in \mathbb{N}$ can be chosen such that $\frac{1-2N}{N} < x$. Therefore $x \in A_N$.

	\hfill\linebreak

	Now consider $\bigcap_{n \in \mathbb{N}} A_n$. Let $x \in \bigcap_{n \in \mathbb{N}} A_n$. Let $x \in (-1,2)$. Therefore $-1 < x < 2$. Note that for all $n \in \mathbb{N}$ that
	\begin{align*}
		-2 + \frac{1}{n} < -1 &< x < 2 \leq 2n \\
		\frac{1 - 2n}{n} &< x < 2n
	.\end{align*}
	Therefore $x \in A_n$ for all $n \in \mathbb{N}$, meaning $x \in \bigcap_{n \in \mathbb{N}} A_n$. Let $x \in \bigcap_{n \in \mathbb{N}} A_n$. Note that for all $n \in \mathbb{N}$ that $A_1 \subseteq A_n$. Since $x \in \bigcap_{n \in \mathbb{N}} A_n$, then $x \in A_n$ for all $n \in \mathbb{N}$. Therefore since $A_1 \subseteq A_n$ and $x$ is in all sets $A_n$, $x \in A_1$. Hence $x \in (-1,2)$. Therefore $\bigcap\limits_{n \in \mathbb{N}} A_n = (-1, 2)$.
\end{proof}

\subsection*{Part C}

\begin{align*}
	A_n = \qty(\frac{1}{2^{n+1}}, \frac{1}{2^{n-1}}) \text{ with } \bigcup\limits_{n \in \mathbb{N}} A_n &= (0,1) \\
	\bigcap\limits_{n \in \mathbb{N}} A_n &= \varnothing
.\end{align*}

\subsection*{Solution}

\begin{proof}
		First consider $\bigcup_{n \in \mathbb{N}} A_n$. Let $x \in \bigcup_{n \in \mathbb{N}} A_n$. Assume that $x \leq 0$. Note that for all $n \in \mathbb{N}$ that the lower bound $\frac{1}{2^{n+1}}$ is strictly positive or otherwords strictly greater than zero. Therefore $x$ cannot be in $A_n$ for any $n$ and hence $x \nleq 0$. Assume that $x = 1$. Note that the upper bound $\frac{1}{2^{n-1}}$ only equals $1$ when $n = 1$. However the range is non-inclusive and hence $1 \notin \bigcup_{n \in \mathbb{N}} A_n$, meaning $x \neq 1$. Assume $x > 1$. Note that for all $n \in \mathbb{N}$ that $1 \geq \frac{1}{2^{n-1}}$. Therefore $x > \frac{1}{2^{n-11}}$, meaning $x \notin \bigcup_{n \in \mathbb{N}} A_n$ and therefore $x \ngtr 1$. Therefore $x \in (0,1)$. Let $x \in (0,1)$.

		\hfill\linebreak

		Now consider $\bigcap_{n \in \mathbb{N}} A_n$. Assume towards contradiction that there is an element $x \in \bigcap_{n \in \mathbb{N}} A_n$. This means that for all $n \in \mathbb{N}$ that $x \in A_n$. Let $N \in \mathbb{N}$ be the index of the set $A_N$ that $x$ is in. Consider now the set $A_L$ where $L = N + 3$. Then $A_N = \qty(\frac{1}{2^{N+1}}, \frac{1}{2^{N-1}})$ and $A_L = \qty(\frac{1}{2^{N+4}}, \frac{1}{2^{N+2}})$. It follows that $A_N \cap A_L = \varnothing$. Therefore $x$ is not in every set $A_n$, hence a contradiction. Therefore there are no elements in the intersection meaning $\bigcap_{n \in \mathbb{N}} A_n = \varnothing$.
\end{proof}

\section*{Problem 6.4.7}

Use Definition 6.7 to prove the following results about nested sets.

\begin{enumerate}
	\item[(a)] $A_1 \supseteq A_2 \supseteq A_3 \supseteq\ldots \implies \bigcup\limits_{n \in \mathbb{N}} A_n = A_1 $
	\item[(b)] $A_1 \subseteq A_2 \subseteq A_3 \subseteq\ldots \implies \bigcap\limits_{n \in \mathbb{N}} A_n = A_1 $
\end{enumerate}

\subsection*{Solution}
\subsection*{Part A}

\begin{proof}
	Let $A_1, A_2, A_3, \ldots$ be sets and assume that $A_1 \supseteq A_2 \supseteq A_3 \supseteq\ldots\;$. Let $x \in \bigcup_{n \in \mathbb{N}} A_n$. Then $\exists l \in \mathbb{N}$ such that $x \in A_l$. By the transitivity of subsets, $\forall n \in \mathbb{N}, A_n \subseteq A_1$. Therefore since $x$ is in $A_l$, $x$ is also in $A_1$. Therefore $\bigcup_{x \in \mathbb{N}} A_n \subseteq A_1$. Let $x \in A_1$. By the definition of the indexed union, since $1$ is in the index set $\mathbb{N}$, $x \in \bigcup_{n \in \mathbb{N}} A_n$. Therefore $A_1 \subseteq \bigcup_{n \in \mathbb{N}} A_n$, meaning $\bigcup_{n\in \mathbb{N}} A_n = A_1$.
\end{proof}

\subsection*{Part B}

\begin{proof}
	Let $A_1, A_2, A_3, \ldots$ be sets and assume that $A_1 \supseteq A_2 \supseteq A_3 \supseteq\ldots\;$. Let $x \in \bigcap_{n \in \mathbb{N}} A_n$. Then $\forall n \in \mathbb{N}$, $x$ is in $A_n$ meaning $x \in A_1$. Therefore $\bigcap_{n \in \mathbb{N}} A_n \subseteq A_1$. Let $x \in A_1$. By the transitivity of subsets, $A_1 \subseteq A_n$ for all $n \in \mathbb{N}$. Therefore since $x\in A_1$, it follows that $x \in A_n$ for all $n \in \mathbb{N}$, which by definition of the indexed intersection means that $x \in \bigcap_{n \in \mathbb{N}}A_n$. Therefore $A_1 \subseteq \bigcap_{n \in \mathbb{N}}A_n$. Hence $\bigcap_{n \in \mathbb{N}}A_n = A_1$.
\end{proof}


\section*{Additional Problem \#1}

Let $A$ and $B$ be disjoint sets and define a function
\[
	f: \powerset{A} \times \powerset{B} \rightarrow \powerset{A \cup B} : (X, Y) \mapsto X \cup Y
.\]

Prove that $f$ is bijective.

\subsection*{Solution}

\begin{proof}
	Let $A$ and $B$ be disjoint sets associated with the function $f$ as defined.
	\hfill\linebreak

	\quad\begin{minipage}{\dimexpr\textwidth-1cm}
	(Injectivity) \quad Suppose that $(X_1, Y_1), (X_2, Y_2) \in \powerset{A} \times \powerset{B}$. Therefore $X_1, X_2 \subseteq A$ and $Y_1,Y_2 \subseteq B$. Since $A$ and $B$ are disjoint, both $X_1$ and $X_2$ are disjoint to both $Y_1$ and $Y_2$ and vice versa. Assume that $f(X_1, Y_1) = f(X_2, Y_2)$.
	\begin{align*}
		f(X_1, Y_1) &= f(X_2, Y_2) \\
		X_1 \cup Y_1 &= X_2 \cup Y_2
	.\end{align*}
	Since $X_1$ and $X_2$ are disjoint to $Y_1$ and $Y_2$, this implies that $X_1$ must equal $X_2$. The same argument implies that $Y_1 = Y_2$. Hence $f$ is injective
	\end{minipage}

	\hfill\linebreak

	\quad\begin{minipage}{\dimexpr\textwidth-1cm}
		(Surjectivity) \quad Let $S \in \powerset{A \cup B}$. Therefore $S \subseteq A \cup B$. Let $K_1 = S \cap A$ and $K_2 = S \cap B$. Note that $K_1 \subseteq A$ and $K_2 \subseteq B$, therefore $K_1 \in \powerset{A}$ and $K_2 \in \powerset{B}$. It also follows that
		\begin{align*}
			f(K_1, K_2) &= K_1 \cup K_2 \\
			&= (S \cap A) \cup (S \cap B) \\
			&= S \cap (A \cup B) \\
			&= S
		.\end{align*}
	Therefore $f$ is surjective.
	\end{minipage}

	\hfill\linebreak

	Since $f$ is both injective and surjective, it follows that $f$ is bijective.
\end{proof}

\section*{Additional Problem \#2}

Let $A_n = \qty{x \in \mathbb{R} : \qty|x^2| < \frac{1}{n}}$. Determine $\bigcup_{n \in \mathbb{N}} A_n$ and $\bigcap_{n \in \mathbb{N}} A_n$ and prove your claims.

\subsection*{Solution}

\begin{align*}
	\bigcup_{n \in \mathbb{N}} A_n &= (-1, 1) & \bigcap_{n \in \mathbb{N}} A_n = \qty{0}
.\end{align*}

\begin{proof}
 	First note that $A_n$ can be rewritten as $A_n = \left(-\sqrt{\frac{1}{n}}, \sqrt{\frac{1}{n}}\;\right)$. Let $x \in (-1,1)$. It follows that $x \in A_1$ since $A_1 = (-1, 1)$. Therefore $x \in \bigcup_{n \in \mathbb{N}} A_n$, meaning $(-1, 1) \subseteq \bigcup_{n \in \mathbb{N}} A_n$. Now let $x \in \bigcup_{n \in \mathbb{N}} A_n$. Note that for all $n \in \mathbb{N}$, $A_n \subseteq A_1$ since $\frac{1}{n} \leq 1$. Therefore $x \in (-1, 1)$, meaning $\bigcup_{n \in \mathbb{N}} A_n \subseteq (-1, 1)$, therefore $\bigcup_{n \in \mathbb{N}} A_n = (-1, 1)$.

	\hfill\linebreak

	Now consider $\bigcap_{n \in \mathbb{N}} A_n$. Let $x \in \bigcap_{n \in \mathbb{N}} A_n$. Therefore $x \in A_n$ for all $n \in \mathbb{N}$. Note that $x = 0$ works because $0 \in A_n$ for all $n$. Assume towards contradiction that $x > 0$ or $x < 0$. If $x>0$, since $\lim_{n \to \infty} \sqrt{\frac{1}{n}} = 0$, there exists an $N \in \mathbb{N}$ such that $\frac{1}{N} < x$, meaning $x \notin A_N$. The same argument applies if $x < 0$. Therefore $x$ can only be zero, meaning $\bigcap_{n \in \mathbb{N}} A_n = \qty{0}$.
\end{proof}

\section*{Additional Problem \#3}

Suppose you are given access to an infinite number of 3-pound weights and 10-pound weights. Prove that you can stack these weights to get a total of $N$-pounds of weight for any $N$ greater than or equal to 18. (e.g., you can form a 19-pound weight by combining 10+3+3+3 but you cannot form a 7-pound weight).

\subsection*{Solution}

\begin{proof}
	We proceed with strong induction. Consider the following base cases.
	\begin{align*}
		n=18 &\implies n = 6(3) + 0(10), \text{ hence 18 pounds is possible} \\
		n=19 &\implies n = 3(3) + 1(10), \text{ hence 19 pounds is possible} \\
		n=20 &\implies n = 0(6) + 2(10), \text{ hence 20 pounds is possible}
	\end{align*}
	Now fix an $n \in \mathbb{N}$ where $n \geq 21$ and assume that for all $k \in \mathbb{N}$ with $18 \leq k \leq n$ that a $k$-pound weight can be made of 3 and 10 pound weights. Otherwisely stated, $\exists a,b \in \mathbb{N}_0$ such that $k = 3a + 10b$. Consider the $n+1$ case. Then
	\begin{align*}
		n + 1 - 3 &= n - 2 \\
		\intertext{By the induction hypothesis, $\exists a,b \in \mathbb{N}_0$ such that $n - 2 = 3a + 10b$. Therefore}
		n + 1 -3 &= 3a + 10b \\
		n + 1 &= 3(a + 1) + 10b
	.\end{align*}
	Therefore for all $n \in \mathbb{N}$ greater than or equal to $18$, an $n$-pound can be made from 3 and 10 pound weights.
\end{proof}


\end{document}
