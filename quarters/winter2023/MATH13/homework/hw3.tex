\documentclass[12pt]{extarticle}

% Document Layout and Font
\usepackage{subfiles}
\usepackage[margin=2cm, headheight=15pt]{geometry}
\usepackage{fancyhdr}
\usepackage{enumitem}	
\usepackage{wrapfig}
\usepackage{multicol}
\usepackage{caption, subcaption}

\usepackage[p,osf]{scholax}

\renewcommand*\contentsname{Table of Contents}
\renewcommand{\headrulewidth}{0pt}
\pagestyle{fancy}
\fancyhf{}
\fancyfoot[R]{$\thepage$}
\setlength{\parindent}{0cm}
\setlength{\headheight}{17pt}
\hfuzz=9pt

% Utility Management
\usepackage{color}
\usepackage{colortbl}
\usepackage{xcolor}
\usepackage{xpatch}
\usepackage{xparse}

\definecolor{links}{HTML}{1c73a5}
\definecolor{bar}{HTML}{584AA8}

% Math Packages
\usepackage{mathtools, amsmath, amsthm, thmtools, amssymb, physics}
\usepackage[scaled=1.075,ncf,vvarbb]{newtxmath}

\newcommand\B{\mathbb{B}}
\newcommand\C{\mathbb{C}}
\newcommand\R{\mathbb{R}}
\newcommand\Q{\mathbb{Q}}
\newcommand\N{\mathbb{N}}
\newcommand\Z{\mathbb{Z}}

\newcommand\Prob[1]{\mathbb{P}\qty(#1)}
\newcommand\Var[1]{\text{Var}\qty(#1)}
\newcommand\Exp[1]{\mathbb{E}\qty[#1]}
\newcommand\ball[1]{\B\qty(#1)}
\newcommand\res[1]{\underset{#1}{\operatorname{Res}}\;}
\renewcommand\pv{\mathrm{p.v.}}

\newcommand\conj[1]{\overline{#1}}
\DeclareMathOperator{\Arg}{Arg}
\DeclareMathOperator{\Log}{Log}
\DeclareMathOperator{\cis}{cis}

\DeclareMathOperator{\dom}{dom}
\DeclareMathOperator{\spann}{span}
\DeclareMathOperator{\nullity}{nullity}

\newcommand\st{\text{ s.t. }}

% TIKZ
\usepackage{tikz}
\usepackage{pgfplots}
\usetikzlibrary{arrows.meta}
\usetikzlibrary{math}
\usetikzlibrary{cd}
\usetikzlibrary{patterns}
\usetikzlibrary{decorations.markings}
\usetikzlibrary{calc}

% Boxes and Theorems
\usepackage[most]{tcolorbox}
\tcbuselibrary{skins}
\tcbuselibrary{breakable}
\tcbuselibrary{theorems}

\newtheoremstyle{default}{0pt}{0pt}{}{}{\bfseries}{\normalfont.}{0.5em}{}
\theoremstyle{default}

\renewcommand*{\proofname}{\textit{\textbf{Proof.}}}
\renewcommand*{\qedsymbol}{$\blacksquare$}
\tcolorboxenvironment{proof}{
	breakable,
	coltitle = black,
	colback = white,
	frame hidden,
	boxrule = 0pt,
	boxsep = 0pt,
	borderline west={3pt}{0pt}{bar},
	sharp corners = all,
	enhanced,
}

\newtheorem{theorem}{Theorem}[section]{\bfseries}{}
\tcolorboxenvironment{theorem}{
	breakable,
	enhanced,
	boxrule = 0pt,
	frame hidden,
	coltitle = black,
	colback = blue!7,
	left = 0.5em,
	sharp corners = all,
}

\newtheorem{corollary}{Corollary}[section]{\bfseries}{}
\tcolorboxenvironment{corollary}{
	breakable,
	enhanced,
	boxrule = 0pt,
	frame hidden,
	coltitle = black,
	colback = white!0,
	left = 0.5em,
	sharp corners = all,
}

\newtheorem{lemma}{Lemma}[section]{\bfseries}{}
\tcolorboxenvironment{lemma}{
	breakable,
	enhanced,
	boxrule = 0pt,
	frame hidden,
	coltitle = black,
	colback = green!7,
	left = 0.5em,
	sharp corners = all,
}

\newtheorem{definition}{Definition}[section]{\bfseries}{}
\tcolorboxenvironment{definition}{
	breakable,
	coltitle = black,
	colback = white,
	frame hidden,
	boxsep = 0pt,
	boxrule = 0pt,
	borderline west = {3pt}{0pt}{orange},
	sharp corners = all,
	enhanced,
}

\newtheorem{example}{Example}[section]{\bfseries}{}
\tcolorboxenvironment{example}{
	% title = \textbf{Example},
	% detach title,
	% before upper = {\tcbtitle\quad},
	breakable,
	coltitle = black,
	colback = white,
	frame hidden,
	boxrule = 0pt,
	boxsep = 0pt,
	borderline west={3pt}{0pt}{green!70!black},
	sharp corners = all,
	enhanced,
}

\newtheoremstyle{remark}{0pt}{4pt}{}{}{\bfseries\itshape}{\normalfont.}{0.5em}{}
\theoremstyle{remark}
\newtheorem*{remark}{Remark}


% TColorBoxes
\newtcolorbox{week}{
	colback = black,
	coltext = white,
	fontupper = {\large\bfseries},
	width = 1.2\paperwidth,
	size = fbox,
	halign upper = center,
	center
}

\newcommand{\banner}[2]{
    \pagebreak
    \begin{week}
   		\section*{#1}
    \end{week}
    \addcontentsline{toc}{section}{#1}
    \addtocounter{section}{1}
    \setcounter{subsection}{0}
}

% Hyperref
\usepackage{hyperref}
\hypersetup{
	colorlinks=true,
	linktoc=all,
	linkcolor=links,
	bookmarksopen=true
}


\fancyhead[R]{\textbf{Homework \#3}}
\setlength\parindent{0pt}

\begin{document}

\section*{Problem 2.3.3}

Suppose that $P(x), Q(y)$ and $R(x, y, z)$ are propositional functions. Compute the negation of the following quantified propositions: 
\begin{enumerate}[label=(\alph*)]
	\item $\forall x, \exists y, P(x) \land Q(y) $
	\item $\forall x, \exists y, \forall z, R(x, y, z)$
\end{enumerate}

\subsection*{Solution}

\begin{enumerate}[label=(\alph*)]
	\item $\exists x, \forall y, \lnot P(x) \lor \lnot Q(y)$
	\item $\exists x, \forall y, \exists z, \lnot R(x,y,z)$
\end{enumerate}

\section*{Problem 2.3.10}

Consider the propositional function $P(x, y, z) : (x - 3)^2 + (y - 2)^2 + (z - 7)^2 > 0$ where the domain of each of the variables $x, y$ and $z$ is $\mathbb{R}$. 
\begin{enumerate}[label=(\alph*)]
	\item Express the quantified statement $\forall x \in \mathbb{R}, \forall y \in \mathbb{R}, \forall z \in \mathbb{R}, P(x, y, z)$ in words. 
	\item Is the quantified statement in (a) true or false? Explain. 
	\item Express the negation of the quantified statement in (a) in symbols. 
	\item Express the negation of the quantified statement in (a) in words. 
	\item Is the negation of the quantified statement in (a) true or false? Explain.
\end{enumerate}

\subsection*{Solution}
\begin{enumerate}[label=(\alph*)]
	\item For all real numbers $x,y$ and $z$, $(x - 3)^2 + (y - 2)^2 + (z - 7)^2$ is strictly greater than 0.
	\item The quantified statement is false. Consider the case where $x=3, y=2, z=7$. Therefore $(x - 3)^2 + (y - 2)^2 + (z - 7)^2 \implies 0 > 0$ which is false.
	\item $\exists x \in \mathbb{R}, \exists y \in \mathbb{R}, \exists z \in \mathbb{R}, \lnot P(x, y, z)$
	\item There exists 3 real numbers $x,y,z$ such that $(x - 3)^2 + (y - 2)^2 + (z - 7)^2$ is less than or equal to 0.
	\item The negation of the quantified statement in a is true. Consider the same case as in (b). That is, $x=3, y=2, z=7$. Therefore $(x - 3)^2 + (y - 2)^2 + (z - 7)^2 \implies 0 \leq 0$ which is true.
\end{enumerate}

\section*{Problem 2.3.11}

The following statements are about positive real numbers. Which one is true? Explain your answer. 
\begin{enumerate}[label=(\alph*)]
	\item $\forall x, \exists y$ such that $xy < y^2$. 
	\item $\exists x$ such that $\forall y, xy < y^2$.
\end{enumerate}

\subsection*{Solution}

A is true since it can be simplified to for any positive real number $x$ there exists a positive real number $y$ such that $x < y$. Since the positive real numbers are unbounded, for any real number there is another larger real number. Therefore for every positive real number x, there exists a larger real number y $\implies x < y$.

% B is false. 
% There exists a positive real number x such that for all positive real numbers y, x < y

\section*{Problem 2.3.16}

You are given the following definition (you do not have to know what is meant by a field). 

\boxemph{Let $x$ be an element of a field $\mathbb{F}$. An inverse of $x$ is an element $y$ in $\mathbb{F}$ such that $xy = 1$.}

Consider the following proposition: 

\boxemph{All non-zero elements in a field have an inverse.}

\begin{enumerate}[label=(\alph*)]
	\item Restate the proposition using both of the quantifiers $\forall$ and $\exists$. 
	\item Find the negation of the proposition, again using quantifiers.
\end{enumerate}

\subsection*{Solution}

\begin{enumerate}[label=(\alph*)]
	\item $\forall x \neq 0, \exists y$ such that $xy = 1$.
	\item $\exists x \neq 0, \forall y$ we have $xy \neq 1$ 
\end{enumerate}

\section*{Problem 2.3.19}

Recall from calculus the definitions of the limit of a sequence $(x_{n}) = (x_{1}, x_{2}, x_{3},\ldots)$. 

\begin{table}[h!]
	\centering
	\begin{tabular}{l l}
		'$x_n$ diverges to $\infty$' means: & $\forall M > 0, \exists N \in \mathbb{N}$ such that $n > N \implies x_n > M$. \\
		'$x_n$ converges to L' means: & $\forall \epsilon > 0, \exists N \in \mathbb{N}$ such that $n > N \implies \mid x_n - L\mid  < \epsilon.$ 
	\end{tabular}
\end{table}

Here we assume that all elements of ($x_{n}$) are real numbers. 

\begin{itemize}
	\item State what it means for a sequence $x_n$ not to converge at all. 
\end{itemize}

\subsection*{Solution}

In symbols: $\forall L, \exists \epsilon > 0, \forall N \in \mathbb{N}, \qty(n > N) \land \qty(\mid x_n - L\mid  \geq \epsilon)$. Or in words, there exists an $\epsilon > 0$ such that for all natural numbers $N$, there exists a natural number $n$ larger than $N$ with $\mid x_n - L\mid  \geq \epsilon$.

% \section*{Problem 3.1.5 (For Fun)}
%
% Find the remainder when $17^{251} \cdot 23^{12} - 19^{41}$ is divided by 5
%
% \subsection*{Solution}
% For all lines, congruency will be (mod 5).
% \begin{align*}
% 	17^{251} \cdot 23^{12} - 19^{41} &\equiv 2^{251} \cdot 3^{12} - (-1)^{41} \\
% 																	 &\equiv 2\cdot 2^{250} \cdot \qty(3^2)^{6} + 1 \\
% 																	 &\equiv 2\cdot \qty(2^2)^{125} \cdot \qty(3^2)^{6} + 1 \\
% 																	 &\equiv 2\cdot (-1)^{125} \cdot (-1)^{6} + 1 \\
% 																	 &\equiv -2 + 1 \\
% 																	 &\equiv -1 \\
% 	17^{251} \cdot 23^{12} - 19^{41} &\equiv 4 \;(\mathrm{mod}\;5).
% \end{align*}

\section*{Problem 3.1.13}

If $a \mid  b$ and $b \mid  c$, prove that $a \mid  c$.

\subsection*{Solution}

\begin{proof}
	Let $a,b,c \in \mathbb{Z}$ . Assume that $a\mid b$ and $b\mid c$. By definition $a$ and $b$ are non zero and there exists $m,n \in \mathbb{Z}$ such that $b = ma$ and $c = nb$. It follows that $b = \frac{c}{n}$. Therefore
	\begin{align*}
		b &= ma \\
		\frac{c}{n} &= ma \\
		c &= nma \\
	\end{align*}
	Since $nm \in \mathbb{Z}$, by definition $a \mid  c$.
\end{proof}

\section*{Problem 3.1.15}

Here are two conjectures. 
\begin{table}[h!]
	\centering
	\begin{tabular}{r l}
		\textit{Conjecture 1}: & $a \mid  b$ and $a \mid  c \implies a \mid  bc$ \\
		\textit{Conjecture 2}: & $a \mid  c$ and $b \mid  c \implies ab \mid  c $
	\end{tabular}
\end{table}
Decide whether each conjecture is true or false and prove/disprove your assertions.

\subsection*{Solution}


Proof that conjecture 1 is true.
\begin{proof}
	Let $a,b,c \in \mathbb{Z}$. Assume that $a \mid b$ and $a \mid c$. By definition there exists $m,n \in \mathbb{Z}$ such that $b = ma$ and $c = na$. It follows then 
	\begin{align*}
		bc &= (ma)(na) \\
			 &= (mna)a
	\end{align*}
	Since $mna \in \mathbb{Z}$, then by definition $a \mid bc$.
\end{proof}

Proof that conjecture 2 is false by counterexample.

\begin{proof}
	Let $a = 3, b = 6$ and $c=12$. It is true that $a\mid c \Longleftrightarrow 3 \mid  12$ and that $b\mid c \Longleftrightarrow 6\mid 12$. However, it is not true that $ab\mid c$ since $ab = 18 \implies 18 \nmid 12$.
\end{proof}

% a = 3
% c = 12
% b = 6

\end{document}
