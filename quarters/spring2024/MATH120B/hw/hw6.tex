\documentclass[12pt,titlepage]{extarticle}
% Document Layout and Font
\usepackage{subfiles}
\usepackage[margin=2cm, headheight=15pt]{geometry}
\usepackage{fancyhdr}
\usepackage{enumitem}	
\usepackage{wrapfig}
\usepackage{multicol}
\usepackage{caption, subcaption}

\usepackage[p,osf]{scholax}

\renewcommand*\contentsname{Table of Contents}
\renewcommand{\headrulewidth}{0pt}
\pagestyle{fancy}
\fancyhf{}
\fancyfoot[R]{$\thepage$}
\setlength{\parindent}{0cm}
\setlength{\headheight}{17pt}
\hfuzz=9pt

% Utility Management
\usepackage{color}
\usepackage{colortbl}
\usepackage{xcolor}
\usepackage{xpatch}
\usepackage{xparse}

\definecolor{links}{HTML}{1c73a5}
\definecolor{bar}{HTML}{584AA8}

% Math Packages
\usepackage{mathtools, amsmath, amsthm, thmtools, amssymb, physics}
\usepackage[scaled=1.075,ncf,vvarbb]{newtxmath}

\newcommand\B{\mathbb{B}}
\newcommand\C{\mathbb{C}}
\newcommand\R{\mathbb{R}}
\newcommand\Q{\mathbb{Q}}
\newcommand\N{\mathbb{N}}
\newcommand\Z{\mathbb{Z}}

\newcommand\Prob[1]{\mathbb{P}\qty(#1)}
\newcommand\Var[1]{\text{Var}\qty(#1)}
\newcommand\Exp[1]{\mathbb{E}\qty[#1]}
\newcommand\ball[1]{\B\qty(#1)}
\newcommand\res[1]{\underset{#1}{\operatorname{Res}}\;}
\renewcommand\pv{\mathrm{p.v.}}

\newcommand\conj[1]{\overline{#1}}
\DeclareMathOperator{\Arg}{Arg}
\DeclareMathOperator{\Log}{Log}
\DeclareMathOperator{\cis}{cis}

\DeclareMathOperator{\dom}{dom}
\DeclareMathOperator{\spann}{span}
\DeclareMathOperator{\nullity}{nullity}

\newcommand\st{\text{ s.t. }}

% TIKZ
\usepackage{tikz}
\usepackage{pgfplots}
\usetikzlibrary{arrows.meta}
\usetikzlibrary{math}
\usetikzlibrary{cd}
\usetikzlibrary{patterns}
\usetikzlibrary{decorations.markings}
\usetikzlibrary{calc}

% Boxes and Theorems
\usepackage[most]{tcolorbox}
\tcbuselibrary{skins}
\tcbuselibrary{breakable}
\tcbuselibrary{theorems}

\newtheoremstyle{default}{0pt}{0pt}{}{}{\bfseries}{\normalfont.}{0.5em}{}
\theoremstyle{default}

\renewcommand*{\proofname}{\textit{\textbf{Proof.}}}
\renewcommand*{\qedsymbol}{$\blacksquare$}
\tcolorboxenvironment{proof}{
	breakable,
	coltitle = black,
	colback = white,
	frame hidden,
	boxrule = 0pt,
	boxsep = 0pt,
	borderline west={3pt}{0pt}{bar},
	sharp corners = all,
	enhanced,
}

\newtheorem{theorem}{Theorem}[section]{\bfseries}{}
\tcolorboxenvironment{theorem}{
	breakable,
	enhanced,
	boxrule = 0pt,
	frame hidden,
	coltitle = black,
	colback = blue!7,
	left = 0.5em,
	sharp corners = all,
}

\newtheorem{corollary}{Corollary}[section]{\bfseries}{}
\tcolorboxenvironment{corollary}{
	breakable,
	enhanced,
	boxrule = 0pt,
	frame hidden,
	coltitle = black,
	colback = white!0,
	left = 0.5em,
	sharp corners = all,
}

\newtheorem{lemma}{Lemma}[section]{\bfseries}{}
\tcolorboxenvironment{lemma}{
	breakable,
	enhanced,
	boxrule = 0pt,
	frame hidden,
	coltitle = black,
	colback = green!7,
	left = 0.5em,
	sharp corners = all,
}

\newtheorem{definition}{Definition}[section]{\bfseries}{}
\tcolorboxenvironment{definition}{
	breakable,
	coltitle = black,
	colback = white,
	frame hidden,
	boxsep = 0pt,
	boxrule = 0pt,
	borderline west = {3pt}{0pt}{orange},
	sharp corners = all,
	enhanced,
}

\newtheorem{example}{Example}[section]{\bfseries}{}
\tcolorboxenvironment{example}{
	% title = \textbf{Example},
	% detach title,
	% before upper = {\tcbtitle\quad},
	breakable,
	coltitle = black,
	colback = white,
	frame hidden,
	boxrule = 0pt,
	boxsep = 0pt,
	borderline west={3pt}{0pt}{green!70!black},
	sharp corners = all,
	enhanced,
}

\newtheoremstyle{remark}{0pt}{4pt}{}{}{\bfseries\itshape}{\normalfont.}{0.5em}{}
\theoremstyle{remark}
\newtheorem*{remark}{Remark}


% TColorBoxes
\newtcolorbox{week}{
	colback = black,
	coltext = white,
	fontupper = {\large\bfseries},
	width = 1.2\paperwidth,
	size = fbox,
	halign upper = center,
	center
}

\newcommand{\banner}[2]{
    \pagebreak
    \begin{week}
   		\section*{#1}
    \end{week}
    \addcontentsline{toc}{section}{#1}
    \addtocounter{section}{1}
    \setcounter{subsection}{0}
}

% Hyperref
\usepackage{hyperref}
\hypersetup{
	colorlinks=true,
	linktoc=all,
	linkcolor=links,
	bookmarksopen=true
}


\def\homeworknumber{6}
\usepackage{fancyhdr}
\pagestyle{fancy}
\fancyhead[R]{HW \#\thehwnumber}
\fancyhead[C]{\textbf{Math 130B}}
\fancyhead[L]{Eli Griffiths}


\begin{document}

\subsection*{27.4}
The prime and maximal ideals of $\Z_2 \times \Z_4$ will be the same. Furthermore the possible ideals for $\Z_2 \times \Z_4$ will be of the form $I \times J$ where $I$ is an ideal of $\Z_2$ and $J$ an ideal of $\Z_4$. The possible prime ideal choices for $I$ are $\langle 1 \rangle$ and $\langle 0 \rangle$. For $J$ the possible choices are $\langle 1 \rangle$ and $\langle 2 \rangle$. By eliminating non-prime candidates it leaves
\[
    \langle (0,1) \rangle, \langle (1,2) \rangle
.\]

\subsection*{27.8}
Note that $\Z_5[x] / \langle x^2 + x + c \rangle$ will be a field when $\langle x^2 + x + c \rangle$ is maximal since $\Z_5[x]$ is a commutative ring with unity. Furthermore $\langle x^2 + x + c \rangle$ will be maximal when it is irreducible over $\Z_5$. Therefore the values of $c$ are those that make $x^2 + x + c$ irreducible in $\Z_5$. Let $f(x) = x^2 + x$. Note that
\begin{align*}
    f(0) &= 0 \\
    f(1) &= 2 \\
    f(2) &= 1 \\
    f(-2) &= 2 \\
    f(-1) &= 0
\end{align*}

Therefore $c$ has to be chosen such that $f(x) + c \neq 0$ for all $x \in \Z_5$. This is only possible for $c = 1$ and $c = 2$. Therefore $c = 1$ or $2$ makes $\Z_5[x] / \langle x^2 + x + c \rangle$ a field.

\subsection*{27.15}
Since $\Z \times \Z$ is a commutative ring with unity, any maximal ideal lends itself to a factor ring that is a field and vice versa. Therefore $\Z \times 3 \Z$ is a maximal ideal of $\Z \times \Z$ since $\Z \times \Z / \Z \times 3 \Z \simeq \Z_3$ which is a field.

\subsection*{27.16}
Note that $\Z \times \qty{0}$ is a prime ideal. However $\Z \times \Z / \Z \times \qty{0} \simeq \Z$ which is not a field and therefore $\Z \times \qty{0}$ cannot be maximal.

\subsection*{27.17}
Note that $\Z \times 4 \Z$ is a nontrivial proper ideal of $\Z \times \Z$. However, since $\Z \times \Z / \Z \times 4 \Z \simeq \Z_4$ which is not an integral domain, $\Z \times 4 \Z$ is not a prime ideal.

\subsection*{27.18}
$\Q[x] / \langle x^2 - 5x + 6 \rangle$ will be a field if $x^2 - 5x + 6$ is irreducible in $\Q$. The roots of the polynomial are
\[
    \frac{5 \pm \sqrt{25 - 4(1)(6)}}{2} = \frac{5 \pm 1}{2}
.\]
Since both roots are rational numbers, $x^2 - 5x + 6$ is not irreducible in $\Q$ and therefore the original factor ring is not a field.

\subsection*{27.30}
\begin{proof}
    Note that every ideal of $F[x]$ must be principal since $F$ is a field. Every proper non-trivial ideal prime ideal will be in the form $\langle f(x) \rangle \neq 0$. Assume towards contradiction that $\langle f(x) \rangle$ is not maximal. Then that means that $f(x)$ is reducible over $F$ meaning there exists some $g(x)$ and $h(x)$ with degrees less than $f(x)$ such that $f(x) = g(x) h(x)$. Since every polynomial in $\langle f(x) \rangle$ has degree greater than or equal to $f(x)$, then $g(x)$ and $h(x)$ cannot be in $\langle f(x) \rangle$. However, since $\langle f(x) \rangle$ is a prime ideal, then since $f(x) = h(x) g(x)$ either $h(x)$ or $g(x)$ must be in $\langle f(x) \rangle$. This is a contradiction and therefore $\langle f(x) \rangle$ must be maximal.
\end{proof}

\subsection*{27.34}
\subsubsection*{Part A}
\begin{proof}
    Let $x \in A + B$. Then $x = a + b$ for some $a \in A$ and $b \in B$. Since $A$ and $B$ are ideals, then for any $r \in R$ it follows $ra, ar \in A$ and $rb, br \in B$. Therefore both $ra + rb$ and $ar + br$ are in $A + B$. Since
    \[
        rx = r(a+b) = ra + rb
    \]
    and
    \[
        xr = (a+b)r = ar + br
    \]
    it follows that $rx, xr \in A+B$. Note for $(a_1 + b_1),(a_2 + b_2) \in A+B$ that
    \[
        (a_1 + b_1) + (a_2 + b_2) = (a_1 + a_2) + (b_1 + b_2) \in A + B
    \]
    and $0 + 0 \in A + B$. Furthermore $(-a_1 - b_1) \in A + B$ and $(a_1 + b_1) + (-a_1 - b_1) = 0$. Therefore $A + B$ is an ideal of $R$.
\end{proof}

\subsubsection*{Part B}
\begin{proof}
    Since $0 \in R$ and for any ideal $N$ of $R$ it follows $rx \in N$ for any $x \in R$, $0 \in N$ by choosing $r = 0$. Therefore $0 \in A$ and $0 \in B$ meaning
    \[
        A = \qty{a + 0 : a \in A} \subseteq \qty{a + b : a \in A, b \in B} = A + B
    \]
    and
    \[
        B = \qty{0 + b : b \in B} \subseteq \qty{a + b : a \in A, b \in B} = A + B
    .\qedhere\]
\end{proof}

\subsection*{27.35}
\subsubsection*{Part A}
\begin{proof}
    $AB$ is closed under addition since 
    \[
        \sum_{i = 1}^n a_i b_i + \sum_{j = 1}^m a_j b_j = \sum_{k = 1}^{m +n} a_k b_k \in AB
    .\]
    Since $A$ and $B$ are both ideals, then for any $r \in R$
    \[
        r \sum_{i = 1}^n a_i b_i = \sum_{i = 1}^n (r a_i) b_i \in AB
    \]
    since $r a_i \in A$ and
    \[
        \qty(\sum_{i = 1}^n a_i b_i)r = \sum_{i = 1}^n a_i (b_i r) \in AB
    \]
    since $b_i r \in B$. Furthermore since $0 \in A, B$ then $(0) (0) = 0 \in AB$ and $-(a_i b_i) = (-a_i) b_i$ then $AB$ has both an additive identity and inverses. Therefore $AB$ is an ideal of $R$.
\end{proof}

\subsubsection*{Part B}
\begin{proof}
    Since $a_i b_i$ must be in $A$ since $A$ is an ideal and must also simultaneously be in $B$ since it is an ideal, then $a_i b_i$ is in $A \cap B$. Since $A$ and $B$ are closed under addition, then any sum of $a_i b_i$ must therefore be in $A$ and $B$ simultaneously meaning any element of $AB$ must be in $A \cap B$. Hence $AB \subseteq A\cap B$.
\end{proof}

\end{document}
