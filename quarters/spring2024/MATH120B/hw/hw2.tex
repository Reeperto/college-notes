\documentclass[12pt,titlepage]{extarticle}
% Document Layout and Font
\usepackage{subfiles}
\usepackage[margin=2cm, headheight=15pt]{geometry}
\usepackage{fancyhdr}
\usepackage{enumitem}	
\usepackage{wrapfig}
\usepackage{multicol}
\usepackage{caption, subcaption}

\usepackage[p,osf]{scholax}

\renewcommand*\contentsname{Table of Contents}
\renewcommand{\headrulewidth}{0pt}
\pagestyle{fancy}
\fancyhf{}
\fancyfoot[R]{$\thepage$}
\setlength{\parindent}{0cm}
\setlength{\headheight}{17pt}
\hfuzz=9pt

% Utility Management
\usepackage{color}
\usepackage{colortbl}
\usepackage{xcolor}
\usepackage{xpatch}
\usepackage{xparse}

\definecolor{links}{HTML}{1c73a5}
\definecolor{bar}{HTML}{584AA8}

% Math Packages
\usepackage{mathtools, amsmath, amsthm, thmtools, amssymb, physics}
\usepackage[scaled=1.075,ncf,vvarbb]{newtxmath}

\newcommand\B{\mathbb{B}}
\newcommand\C{\mathbb{C}}
\newcommand\R{\mathbb{R}}
\newcommand\Q{\mathbb{Q}}
\newcommand\N{\mathbb{N}}
\newcommand\Z{\mathbb{Z}}

\newcommand\Prob[1]{\mathbb{P}\qty(#1)}
\newcommand\Var[1]{\text{Var}\qty(#1)}
\newcommand\Exp[1]{\mathbb{E}\qty[#1]}
\newcommand\ball[1]{\B\qty(#1)}
\newcommand\res[1]{\underset{#1}{\operatorname{Res}}\;}
\renewcommand\pv{\mathrm{p.v.}}

\newcommand\conj[1]{\overline{#1}}
\DeclareMathOperator{\Arg}{Arg}
\DeclareMathOperator{\Log}{Log}
\DeclareMathOperator{\cis}{cis}

\DeclareMathOperator{\dom}{dom}
\DeclareMathOperator{\spann}{span}
\DeclareMathOperator{\nullity}{nullity}

\newcommand\st{\text{ s.t. }}

% TIKZ
\usepackage{tikz}
\usepackage{pgfplots}
\usetikzlibrary{arrows.meta}
\usetikzlibrary{math}
\usetikzlibrary{cd}
\usetikzlibrary{patterns}
\usetikzlibrary{decorations.markings}
\usetikzlibrary{calc}

% Boxes and Theorems
\usepackage[most]{tcolorbox}
\tcbuselibrary{skins}
\tcbuselibrary{breakable}
\tcbuselibrary{theorems}

\newtheoremstyle{default}{0pt}{0pt}{}{}{\bfseries}{\normalfont.}{0.5em}{}
\theoremstyle{default}

\renewcommand*{\proofname}{\textit{\textbf{Proof.}}}
\renewcommand*{\qedsymbol}{$\blacksquare$}
\tcolorboxenvironment{proof}{
	breakable,
	coltitle = black,
	colback = white,
	frame hidden,
	boxrule = 0pt,
	boxsep = 0pt,
	borderline west={3pt}{0pt}{bar},
	sharp corners = all,
	enhanced,
}

\newtheorem{theorem}{Theorem}[section]{\bfseries}{}
\tcolorboxenvironment{theorem}{
	breakable,
	enhanced,
	boxrule = 0pt,
	frame hidden,
	coltitle = black,
	colback = blue!7,
	left = 0.5em,
	sharp corners = all,
}

\newtheorem{corollary}{Corollary}[section]{\bfseries}{}
\tcolorboxenvironment{corollary}{
	breakable,
	enhanced,
	boxrule = 0pt,
	frame hidden,
	coltitle = black,
	colback = white!0,
	left = 0.5em,
	sharp corners = all,
}

\newtheorem{lemma}{Lemma}[section]{\bfseries}{}
\tcolorboxenvironment{lemma}{
	breakable,
	enhanced,
	boxrule = 0pt,
	frame hidden,
	coltitle = black,
	colback = green!7,
	left = 0.5em,
	sharp corners = all,
}

\newtheorem{definition}{Definition}[section]{\bfseries}{}
\tcolorboxenvironment{definition}{
	breakable,
	coltitle = black,
	colback = white,
	frame hidden,
	boxsep = 0pt,
	boxrule = 0pt,
	borderline west = {3pt}{0pt}{orange},
	sharp corners = all,
	enhanced,
}

\newtheorem{example}{Example}[section]{\bfseries}{}
\tcolorboxenvironment{example}{
	% title = \textbf{Example},
	% detach title,
	% before upper = {\tcbtitle\quad},
	breakable,
	coltitle = black,
	colback = white,
	frame hidden,
	boxrule = 0pt,
	boxsep = 0pt,
	borderline west={3pt}{0pt}{green!70!black},
	sharp corners = all,
	enhanced,
}

\newtheoremstyle{remark}{0pt}{4pt}{}{}{\bfseries\itshape}{\normalfont.}{0.5em}{}
\theoremstyle{remark}
\newtheorem*{remark}{Remark}


% TColorBoxes
\newtcolorbox{week}{
	colback = black,
	coltext = white,
	fontupper = {\large\bfseries},
	width = 1.2\paperwidth,
	size = fbox,
	halign upper = center,
	center
}

\newcommand{\banner}[2]{
    \pagebreak
    \begin{week}
   		\section*{#1}
    \end{week}
    \addcontentsline{toc}{section}{#1}
    \addtocounter{section}{1}
    \setcounter{subsection}{0}
}

% Hyperref
\usepackage{hyperref}
\hypersetup{
	colorlinks=true,
	linktoc=all,
	linkcolor=links,
	bookmarksopen=true
}


\def\homeworknumber{2}
\usepackage{fancyhdr}
\pagestyle{fancy}
\fancyhead[R]{HW \#\thehwnumber}
\fancyhead[C]{\textbf{Math 130B}}
\fancyhead[L]{Eli Griffiths}


\begin{document}

% §19: 2, 10, 12, 23, 27, 28, 29, 30;

\subsection*{2}
For $\Z_7$ the solution is $3$ and in $\Z_{23}$ it is $16$.

\subsection*{10}
The characteristic of $\Z_6 \times \Z_{15}$ is $\lcm(6,15) = 30$

\subsection*{12}
Since $\mathcal{R}$ has characteristic $3$, $3\cdot x = 0$ for all $x \in \mathcal{R}$. Therefore
\begin{align*}
    (a+b)^9 &= \qty((a+b)^3)^3 \\
    &= \qty(a^3 + 3a^2 b + 3ab^2 + b^3)^3 \\
    &= \qty(a^3 + b^3)^3 \\
    &= a^9 + 3a^6 b^3 + 3a^3 b^6 + b^9 \\
    &= a^9 + b^9
\end{align*}

\subsection*{23}
\begin{proof}
    Let $\mathcal{R}$ be a division ring. Note that $0^2 = 0$ and $1^2 = 1$, hence $0$ and $1$ are idempotent. Assume towards contradiction there is some $a \in \mathcal{R}$ that is idempotent and $a \neq 0$ and $a \neq 1$. Then $a^2 = a \implies a(a-1) = 0$. Since $\mathcal{R}$ is a division ring and $a \neq 0$, there exists $a^{-1}$ meaning $a - 1 = 0 \implies a = 1$, a contradiction. Hence $\mathcal{R}$ only has 2 idempotents ($0$ and $1$).
\end{proof}

\subsection*{27}
By the previous exercise, the unity of an integral domain is the unique non-zero idempotent element of $\mathcal{D}$. Therefore any subdomain of $\mathcal{D}$ has the same unity as $\mathcal{D}$. Therefore since characteristic is defined as the smallest $n \in \Z_+$ such that $n \cdot 1 = 0$ or $0$ if $n$ doesn't exist, then any subdomain will have the same characteristic since it has the same unity.

\subsection*{28}
\begin{proof}
    Let $X$ be a subdomain of an integral domain $\mathcal{D}$. Note $X$ contains the same unity as $\mathcal{D}$. Therefore since $X$ is closed under addition, $n \cdot 1 \in X$ for all $n \in \Z$. Hence the set $R = \qty{n \cdot 1 : n \in \Z}$ is a subset of $X$. $R$ is closed under addition since $(n \cdot 1) + (m \cdot 1) = (m + n) \cdot 1$. Since $(-n \cdot 1) + (n \cdot 1) = 0$ and $0 \cdot 1 = 0$, $R$ also contains $0$ and has all additive inverses meaning $\langle R, + \rangle$ is an abelian group. $R$ is closed under multiplication since $(n \cdot 1) (m \cdot 1) = (mn) \cdot 1$. It also follows $1 \cdot 1 = 1$ meaning $R$ must be a commutative ring with unity. Since any product $xy = 0$ in $R$ is also a product in $\mathcal{X}$, $R$ must have no zero divisors. Therefore $R$ is a subdomain of all subdomains $\mathcal{X}$.
\end{proof}

\subsection*{29}
\begin{proof}
    Assume towards contradiction that an integral domain $\mathcal{D}$ has a characteristic of $mn$ where $m,n > 1$. Then by the distributive laws $(m \cdot 1)(n \cdot 1) = (mn) \cdot 1 = 0$. Since $\mathcal{D}$ is an integral domain, this means that either $m \cdot 1 = 0$ or $n \cdot 1 = 0$. However, $m, n < mn$ meaning if either case was true, the characteristic would be smaller than $mn$. This is a contradiction since the characteristic is the smallest possible integer $k$ such that $k \cdot 1 = 0$. Therefore $\mathcal{D}$ must have a zero or prime characteristic.
\end{proof}

\subsection*{30}
\subsubsection*{Part A}
\begin{proof}
    Examine the axioms for $S$ to be a ring.
    \begin{enumerate}
        \item[$\mathcal{R}_1 )$]
            Since both $\langle R, + \rangle$ and $\langle Z, + \rangle$ (or $\langle \Z_n, + \rangle$) are abelian groups, their direct product is also an abelian group. Since addition on $S$ is defined in the same manner as the direct product, $\langle S, + \rangle$ is an abelian group.
        \item[$\mathcal{R}_2 )$]
            Let $(r_1,n_1), (r_2,n_2), (r_3,n_3) \in S$. Then
            \begin{align*}
                (r_1, n_1) \qty[(r_2, n_2)(r_3, n_3)] &= (r_1, n_1) \qty[(r_2 r_3 + n_2 \cdot r_3 + n_3 \cdot r_2, n_2 n_3)] \\
                &= (r_1 r_2 r_3 + n_2 \cdot r_1 r_3 + n_3 \cdot r_1 r_2 \;+ \\
                &\hspace{0.7cm}n_1 \cdot r_2 r_3 + (n_1 n_2) \cdot r_3 + (n_1 n_3) \cdot r_2 \;+\\
                &\hspace{0.63cm}(n_2 n_3) \cdot r_1, n_1 n_2 n_3)
            \end{align*}
            which equals
            \[
                (r_1r_2r_3 + (n_2 n_3) \cdot r_1 + (n_1 n_3) \cdot r_2 + (n_1 n_2) \cdot r_3 + n_3 \cdot r_1 r_2 + n_1 \cdot r_2 r_3 + n_2 \cdot r_1 r_3, n_1 n_2 n_3)
            .\]
            Grouping the first two terms gives
            \begin{align*}
                \qty[(r_1, n_1) (r_2, n_2)](r_3, n_3) &= \qty[(r_1 r_2 + n_1 \cdot r_2 + n_2 \cdot r_1, n_1 n_2)] (r_3, n_3) \\
                &= (r_1 r_2 r_3 + n_1 \cdot r_2 r_3 + n_2 \cdot r_1 r_3 \;+ \\
                &\hspace{0.7cm} n_3 \cdot r_1 r_2 + (n_1 n_3) \cdot r_2 + (n_2 n_3) \cdot r_1 \;+\\
                &\hspace{0.63cm}(n_1 n_2) \cdot r_3, n_1 n_2 n_3).
            \end{align*}
            Since addition is commutative and the distributivity laws hold, it is equal to
            \[
                (r_1r_2r_3 + (n_2 n_3) \cdot r_1 + (n_1 n_3) \cdot r_2 + (n_1 n_2) \cdot r_3 + n_3 \cdot r_1 r_2 + n_1 \cdot r_2 r_3 + n_2 \cdot r_1 r_3, n_1 n_2 n_3)
            .\]
            Therefore multiplication is associative.
        \tcbbreak
        \item[$\mathcal{R}_3 )$]
            Checking the left distributive law
            \begin{align*}
                (r_1, n_1) \qty[(r_2, n_2) + (r_3, n_3)] &= (r_1, n_1) (r_2 + r_3, n_2 + n_3) \\
                &= (r_1 (r_2 + r_3) + (n_2 + n_3)\cdot r_1 + n_1 \cdot (r_2 + r_3), n_1(n_2 + n_3)) \\
                &= (r_1 r_2 + n_2 \cdot r_1 + n_1 \cdot r_2, n_1 n_2) + (r_1 r_3 + n_3 \cdot r_2 + n_2\cdot r_3, n_1 n_3) \\
                &= (r_1, n_1) (r_2, n_2) + (r_1, n_1) (r_3, n_3)
            \end{align*}
            Therefore the left distributivity law holds. The right law follows from a similar argument. \qedhere
    \end{enumerate}
\end{proof}

\subsubsection*{Part B}
\begin{proof}
    Consider $(0, 1) \in S$. Note that
    \[
        (0,1) (r, n) = (0r + 1\cdot r + n \cdot 0, 1\cdot n) = (r, n)
    \]
    and
    \[
        (r, n)(0, 1) = (r0 + n \cdot 0 + 1 \cdot r, n \cdot 1) = (r, n)
    .\]
    Therefore $(0,1) \in S$ is unity.
\end{proof}

\subsubsection*{Part C}
\begin{proof}
    By the previous part, $(0, 1)$ is the unity of $S$. Assume that $R$ has characteristic $n \neq 0$. Note $\Z_n$ is a ring of characteristic $n$, meaning $n$ is the smallest integer such that $n \cdot 1_{\Z_{n}} = 0_{\Z_{n}}$. Since $n \cdot 0_{R} = 0_{R}$ for any $n$, it follows $n \cdot (0,1) = (0,0)$. Therefore $n$ is the characteristic of $S$. Assume that $R$ has characteristic $0$. Then $S = R \times \Z$. $\Z$ has characteristic zero meaning there is no $n \in \Z_+$ such that $n \cdot 1 = 0$. Note then that for any $n \in \Z_+$ that $n \cdot (0,1) = (n \cdot 0, n \cdot 1) \neq (0,0)$. Hence $S$ has characteristic $0$.
\end{proof}

\subsubsection*{Part D}
\begin{proof}
    Let $\conj{S} = \qty{(r,0) : r \in \R} \subseteq S$ and $r_1, r_2 \in R$. Note that $(0,0) \in \conj{S}$, $(r_1, 0) - (r_2, 0) = (r_1 - r_2, 0) \in \conj{S}$, and $(r_1, 0)(r_2, 0) = (r_1r_2, 0) \in \conj{S}$. Therefore $\conj{S}$ is a subring of $S$. Consider the requirements for $\phi$ to be an isomorphism between $R$ and $\conj{S}$.
    \begin{itemize}
        \item
            Note that $\phi(r_1 + r_2) = (r_1 + r_2, 0) = (r_1, 0) + (r_2, 0) = \phi(r_1) + \phi(r_2)$ and $\phi(r_1 r_2) = (r_1 r_2, 0) = (r_1, 0) (r_2, 0) = \phi(r_1) \phi(r_2)$. Therefore $\phi$ is a homomorphism.
        \item
            Assume that $\phi(r_1) = \phi(r_2)$. Then $(r_1, 0) = (r_2, 0)$ meaning $(r_1 - r_2, 0) = (0, 0)$. Therefore $r_1 = r_2$ hence $\phi$ is injective. Let $(r, 0) \in \conj{S}$. Note that $\phi(r) = (r,0)$, hence $\phi$ is onto. Therefore $\phi$ is a bijection between $R$ and $\conj{S}$
    \end{itemize}
    Since $\phi$ is a one-to-one and onto homomorphism between $R$ and $\conj{S}$, the statement holds.
\end{proof}

\end{document}
