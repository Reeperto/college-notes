\documentclass[../notes.tex]{subfiles}
\graphicspath{
    {"../figures"}
}

\begin{document}

\subsection{Existence}

\begin{theorem}{Picard's Theorem}{picard}
				If $y' = f( x_0,y_0 )$ is continuous in $x$ and $y$, and  $\partial_y f$ exists and is continuous around  $( x_0,y_0 ) $ then:
				\[
				\begin{cases}
								y' &= f( x,y ) \\
								y( x_0 ) &= y_0 
				\end{cases}
				\] 
\end{theorem}

\namethmref{th:picard} doesn't necessarily guarantee a global solution; only a local solution for some $\epsilon$ distance away from the input  $x_0$.

\subsection{Separability}

\begin{center}
When $f( x,y ) = \underbrace{h( x ) g( y )}_{\clap{\scriptsize Can be easily integrated}} $ 
\end{center}

For the general case:

\begin{align*}
				y' = f( x_0,y_0 ) &= h( x ) g( y ) \\
				y' &= h( x ) g( y ) \\
				\frac{y'}{g( y )} &= h( x ) \\
				\int \frac{y'}{g( y ) } \dd x &= \int h( x ) \dd x \\
				\int \frac{1}{g( y ) } dy &= \int h( x ) \dd x
.\end{align*}

\begin{example}{Find the general solution of $y' = xy$}
\begin{align*}
				y' = xy \implies \frac{y'}{t} &= x \\
				\int \frac{1}{y} dy &= \int x \dd x \\
				\ln \abs{y}  &= \frac{1}{2}x^2 + c \\
				\abs{y} &= e^{\frac{1}{2}x^2 + c} \\
				y &= Ae^{\frac{x^2}{2}}; A \in \mathbb{R}
.\end{align*}
\end{example}

\begin{example}{Find the general solution of $y' = 1-x^2+y^2-y^2 x^2$; $y( 1 ) = 0$}
\begin{align*}
				y' = ( 1+y^2 )( 1-x^2 ) \implies \int \frac{1}{1+y} \dd t &= \int 1-x^2 \dd x \\
				\arctan(y)  &= x - \frac{x^3}{3} + c \\
				y  &= \tan( x - \frac{x^3}{3} + c )
\end{align*}
Solve with initial condition:
\begin{align*}
				y( 1 ) = 0 &= \tan(\frac{2}{3}) \\
				\arctan(0) &= \frac{2}{3} + c \\
				\left\{ n\pi : n \in \mathbb{Z} \right\} &= \frac{2}{3} + c
\intertext{Therefore:}
				y  &= \tan( x - \frac{x^3}{3} + c ) \\
				c &= \qty{ n\pi - \frac{2}{3} : n \in \mathbb{Z} }
.\end{align*}
\end{example}

\subsection{Linear and Non-Linear ODE}

\begin{stickynote}{Linear vs Non-Linear ODE}
				\begin{align*}
								\text{Linear } & \implies \text{Dependent variables and their derivatives appear linearly} \\
								\text{Non-Linear } & \implies \text{Dependent variables and their derivatives have a power $\ge 2$}
				.\end{align*}
\end{stickynote}

\subsubsection{Solving $1^{st}$ Order Linear ODE}

Linear first order ODEs follow the form:

\[
y' + p( x ) \cdot y = f( x ) 
.\] 

To solve such equations, utilize the \textbf{Integration Factor}:

\[
				r( x ) = e^{\int\! p( x ) \dd x}
.\] 

Here is how the integration factor is utilized.

\begin{align*}
				y' + p( x ) \cdot y &= f( x ) \\
				y'r( x )  + \underbrace{r( x ) p( x )}_{\clap{Equivalent to $r'( x ) $}}  \cdot y &= r( x ) f( x ) \\
				\underbrace{y'r( x ) + r'( x ) \cdot y}_{\clap{Use inverse product rule}}  &= r( x ) f( x ) \\
				\dv{x} \left[ y\cdot r( x ) \right] &= r( x ) f( x ) \\
				\int\dv{x} \left[ y\cdot r( x ) \right] \dd x &= \int r( x ) f( x ) \dd x \\
				y\cdot r( x ) &= \int r( x ) f( x ) \dd x \\
				y &= \frac{1}{r( x )}\int r( x ) f( x ) \dd x
.\end{align*}

\end{document}
