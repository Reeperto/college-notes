\documentclass[../notes.tex]{subfiles}
\graphicspath{
    {"../figures"}
}

\begin{document}

\subsection{Inverse Laplace Transform}

The inverse Laplace Transform is defined as
\newcommand{\equaldef}{
    \overset{\text{def}}{=}
}
\[
    \invlaplace{F(s)} \equaldef f(t)
.\]

Since it is the inverse of a linear transformation, the inverse is also linear in nature.

\begin{example}{Find $\invlaplace{\frac{s^2 + s + 1}{s^3 + s}}$}
Note that this doesn't fit any of the common forms for know Laplace Transforms. However, the polynomial division hints towards partial fraction decomposition.
\begin{align*}
    \frac{s^2+s+1}{s^3+s} = \frac{s^2 + s + 1}{s (s^2 + 1)} &= \frac{A}{s} + \frac{Bs + C}{s^2 + 1}\\
    &\Downarrow \\
    s^2 + s + 1 &= A(s^2 + 1) + s(Bs + C) \\
    s^2 + s + 1 &= As^2 + A + Bs^2 + Cs \\
    &\Downarrow \\
    (A,B,C) &= (1, 0, 1) \\
    &\Downarrow \\
    \frac{s^2+s+1}{s^3+s} &= \frac{1}{s} + \frac{1}{s^2 + 1}
.\end{align*}
Taking the Laplace transform of this is much easier, resulting in
\[
    \invlaplace{\frac{1}{s} + \frac{1}{s^2 + 1}} = 1 + \sin(t)
.\]
\end{example}

\subsubsection{Shifting Principle}

In some cases, it is easier to write $F(s)$ as  $s$ shifted by some number  $a$, such that the function in frequency space is  $F(s+a)$. Note that
 \begin{align*}
     \laplace{e^{-at}f(t)} &= \int_0^\infty e^{-st} e^{-at} f(t) dt \\
     &= \int_0^\infty e^{-(s+a)t} f(t) \\
     &= F(s+a)
.\end{align*}
Or equivalently,
\[
    \invlaplace{F(s+a)} = e^{-at} f(t)
.\]
This is helpful in cases where a polynomial in a denominator can only be partially factored via complete the square

\begin{example}{Find the inverse Laplace of $F(s) = \frac{1}{s^2 + 4s + 13}$}
In this case, $F(s)$ can not be factored in the denominator. However, it can be partially factored by completing the square.
\begin{align*}
    F(s) &= \frac{1}{s^2 + 4s + 13} \\
    &= \frac{1}{(s^2 + 4s + 4) + 9} \\
    &= \frac{1}{(s+2)^2 + 9}
.\end{align*}
Note that this result evokes the Laplace transformation of $ \sin(\omega t)$
\[
\laplace{\sin(\omega t)} = \frac{\omega}{s^2 + \omega^2}
.\]
The numerator doesn't fit the structure, but a simple manipulation resolves that issue
\[
\frac{3}{3}\cdot \frac{1}{(s+2)^2 + 9} = \frac{1}{3} \cdot \frac{3}{(s+2)^2 + 9}
.\]
Therefore the shifting principle can be utilized
\[
    \invlaplace{\frac{1}{3} \cdot \frac{3}{(s+2)^2 + 9}} = \frac{1}{3} e^{-2t} \sin(3t)
.\]
\end{example}

Note that the shifting property appears when the Heaviside Function is involved. Consider
\[
\laplace{f(t-a)u(t-a)} = e^{-as}\laplace{f(t)} = e^{-as} F(s)
.\]
Another useful form is
\[
\laplace{f(t)u(t-a)} = e^{-as} \laplace{f(t+a)}
.\]

\subsection{Laplace Transform of Derivatives}

Since the Laplace transform is an integral transform, it would be advantageous to plug in time derivatives of functions to see their resultant function in frequency space. Assume a function $g(t)$ such that  $\laplace{g(t)} = G(t)$.
 \begin{align*}
     \laplace{g'(t)} &= \int_0^\infty e^{-st} g'(t) dt \\
    \intertext{Utilize integration by parts with $u=e^{-st}$ and $dv = g'(t)dt$}
    &= \eval{g(t)e^{-st}}_0^{\infty} + s \int_0^{\infty} e^{st} g(t) dt \\
    &= \lim_{t \to \infty} g(t)e^{-st} - g(0) + s G(s)
 \end{align*}
In order for this to exist, $g(t)$ must grow slower than the exponential. Stated formally, $\qty| g(t) | < Me^{ct}$ for appropriate positive constants $M$ and $c$. Assuming this holds,
 \[
     \laplace{g'(t)} = sG(s) - g(0)
.\]

\subsubsection{Laplace Transform of Integrals}

Laplace transforms act quite nicely when the input function is an integral. Consider an integral of the form
 \[
     \int_0^{t} f(\tau) \dd\tau
.\]
Then the Laplace transform of $f(t)$ is
 \[
\laplace{\int_0^{t} f(\tau) \dd\tau} = \frac{1}{s} \laplace{f(t)} = \frac{F(s)}{s}
.\]
This can help solve harder Laplace transforms that would involve processes like partial fraction decomposition.

\begin{example}{Find inverse Laplace of $\frac{1}{s(s^2+1)}$}
    Notice that
    \[
    \frac{1}{s(s^2+1)} = \frac{1}{s} \laplace{\sin(t)}
    .\]
    In this case, $f(t) = \sin(t)$, therefore,
    \begin{align*}
        \invlaplace{\frac{1}{s(s^2+1)}} &= \int_0^{t} \sin(\tau) \dd\tau \\
                                        &= \eval{-\cos(t)}_0^{t} \\
                                        &= 1-\cos(t)
    .\end{align*}
\end{example}

\end{document}
