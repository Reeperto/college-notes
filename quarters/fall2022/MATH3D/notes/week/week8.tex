\documentclass[../notes.tex]{subfiles}
\graphicspath{
    {"../figures"}
}

\newcommand{\equaldef}{
    \overset{\text{def}}{=}
}

\begin{document}

\subsection{Different properties of Laplace Transforms}

In certain cases, a function may be related to a simpler function through a change of basis via a linear transformation of the input. Examples of this are
\begin{align*}
    f(kt) &
    & F(ks) &
    & F(as + b) &
\end{align*}

Consider $\laplace{f(kt)}$ with the restriction that $s > a \geq 0$ and $k > 0$
\begin{align*}
    \laplace{f(kt)} &= \int_0^{\infty} e^{-st} f(kt) \dd{t}\\
    \intertext{Let $u = kt \implies \dd u = k \dd{t}$}
    &= \frac{1}{k}\int_0^{\infty} e^{-\frac{su}{k}} f(u) \dd{u}\\
    \intertext{Let $\omega = \frac{s}{k}$}
    &= \frac{1}{k}\int_0^{\infty} e^{-\omega u } f(u) \dd{u}\\
    &= \frac{1}{k} F(\omega) \\
    &= \frac{1}{k} F\qty(\frac{s}{k}); \; s > ka
.\end{align*}

\subsection{Power Series}

For many, many functions, they can be represented as a long form polynomial known as a power series. They have the general form of
\[
     f(x) = \sum_{n=1}^{\infty} a_k (x - x_0)^{k}
.\]

Some famous power series are as follows
\begin{align*}
    e^{x} &= \sum_{n=0}^{\infty} \frac{1}{n!} x^{n} = 1 + x + \frac{x^2}{2} + \frac{x^3}{6} + \ldots \\
    \cos(x) &= \sum_{n=0}^{\infty} \frac{(-1)^{n}}{2n!} x^{2n} = 1 - \frac{x^2}{2} + \frac{x^{4}}{24} - \ldots \\
    \sin(x) &= \sum_{n=0}^{\infty} \frac{(-1)^{n}}{(2n+1)!} x^{2n+1} = x - \frac{x^3}{6} + \frac{x^5}{120} - \ldots
.\end{align*}

\end{document}
