\documentclass[12pt]{extarticle}

% Document Layout and Font
\usepackage{subfiles}
\usepackage[margin=2cm, headheight=15pt]{geometry}
\usepackage{fancyhdr}
\usepackage{enumitem}	
\usepackage{wrapfig}
\usepackage{multicol}
\usepackage{caption, subcaption}

\usepackage[p,osf]{scholax}

\renewcommand*\contentsname{Table of Contents}
\renewcommand{\headrulewidth}{0pt}
\pagestyle{fancy}
\fancyhf{}
\fancyfoot[R]{$\thepage$}
\setlength{\parindent}{0cm}
\setlength{\headheight}{17pt}
\hfuzz=9pt

% Utility Management
\usepackage{color}
\usepackage{colortbl}
\usepackage{xcolor}
\usepackage{xpatch}
\usepackage{xparse}

\definecolor{links}{HTML}{1c73a5}
\definecolor{bar}{HTML}{584AA8}

% Math Packages
\usepackage{mathtools, amsmath, amsthm, thmtools, amssymb, physics}
\usepackage[scaled=1.075,ncf,vvarbb]{newtxmath}

\newcommand\B{\mathbb{B}}
\newcommand\C{\mathbb{C}}
\newcommand\R{\mathbb{R}}
\newcommand\Q{\mathbb{Q}}
\newcommand\N{\mathbb{N}}
\newcommand\Z{\mathbb{Z}}

\newcommand\Prob[1]{\mathbb{P}\qty(#1)}
\newcommand\Var[1]{\text{Var}\qty(#1)}
\newcommand\Exp[1]{\mathbb{E}\qty[#1]}
\newcommand\ball[1]{\B\qty(#1)}
\newcommand\res[1]{\underset{#1}{\operatorname{Res}}\;}
\renewcommand\pv{\mathrm{p.v.}}

\newcommand\conj[1]{\overline{#1}}
\DeclareMathOperator{\Arg}{Arg}
\DeclareMathOperator{\Log}{Log}
\DeclareMathOperator{\cis}{cis}

\DeclareMathOperator{\dom}{dom}
\DeclareMathOperator{\spann}{span}
\DeclareMathOperator{\nullity}{nullity}

\newcommand\st{\text{ s.t. }}

% TIKZ
\usepackage{tikz}
\usepackage{pgfplots}
\usetikzlibrary{arrows.meta}
\usetikzlibrary{math}
\usetikzlibrary{cd}
\usetikzlibrary{patterns}
\usetikzlibrary{decorations.markings}
\usetikzlibrary{calc}

% Boxes and Theorems
\usepackage[most]{tcolorbox}
\tcbuselibrary{skins}
\tcbuselibrary{breakable}
\tcbuselibrary{theorems}

\newtheoremstyle{default}{0pt}{0pt}{}{}{\bfseries}{\normalfont.}{0.5em}{}
\theoremstyle{default}

\renewcommand*{\proofname}{\textit{\textbf{Proof.}}}
\renewcommand*{\qedsymbol}{$\blacksquare$}
\tcolorboxenvironment{proof}{
	breakable,
	coltitle = black,
	colback = white,
	frame hidden,
	boxrule = 0pt,
	boxsep = 0pt,
	borderline west={3pt}{0pt}{bar},
	sharp corners = all,
	enhanced,
}

\newtheorem{theorem}{Theorem}[section]{\bfseries}{}
\tcolorboxenvironment{theorem}{
	breakable,
	enhanced,
	boxrule = 0pt,
	frame hidden,
	coltitle = black,
	colback = blue!7,
	left = 0.5em,
	sharp corners = all,
}

\newtheorem{corollary}{Corollary}[section]{\bfseries}{}
\tcolorboxenvironment{corollary}{
	breakable,
	enhanced,
	boxrule = 0pt,
	frame hidden,
	coltitle = black,
	colback = white!0,
	left = 0.5em,
	sharp corners = all,
}

\newtheorem{lemma}{Lemma}[section]{\bfseries}{}
\tcolorboxenvironment{lemma}{
	breakable,
	enhanced,
	boxrule = 0pt,
	frame hidden,
	coltitle = black,
	colback = green!7,
	left = 0.5em,
	sharp corners = all,
}

\newtheorem{definition}{Definition}[section]{\bfseries}{}
\tcolorboxenvironment{definition}{
	breakable,
	coltitle = black,
	colback = white,
	frame hidden,
	boxsep = 0pt,
	boxrule = 0pt,
	borderline west = {3pt}{0pt}{orange},
	sharp corners = all,
	enhanced,
}

\newtheorem{example}{Example}[section]{\bfseries}{}
\tcolorboxenvironment{example}{
	% title = \textbf{Example},
	% detach title,
	% before upper = {\tcbtitle\quad},
	breakable,
	coltitle = black,
	colback = white,
	frame hidden,
	boxrule = 0pt,
	boxsep = 0pt,
	borderline west={3pt}{0pt}{green!70!black},
	sharp corners = all,
	enhanced,
}

\newtheoremstyle{remark}{0pt}{4pt}{}{}{\bfseries\itshape}{\normalfont.}{0.5em}{}
\theoremstyle{remark}
\newtheorem*{remark}{Remark}


% TColorBoxes
\newtcolorbox{week}{
	colback = black,
	coltext = white,
	fontupper = {\large\bfseries},
	width = 1.2\paperwidth,
	size = fbox,
	halign upper = center,
	center
}

\newcommand{\banner}[2]{
    \pagebreak
    \begin{week}
   		\section*{#1}
    \end{week}
    \addcontentsline{toc}{section}{#1}
    \addtocounter{section}{1}
    \setcounter{subsection}{0}
}

% Hyperref
\usepackage{hyperref}
\hypersetup{
	colorlinks=true,
	linktoc=all,
	linkcolor=links,
	bookmarksopen=true
}


\fancyhead[R]{Homework \#$4$}
\fancyhead[L]{Eli Griffiths}
\renewcommand{\headrulewidth}{1pt}
\setlength\parindent{0pt}

% Section 5:
% 8, 15, 20, 21, 33, 35, 36, 42, 46, 47, 49, 50, 51, 54

\begin{document}
\DeclarePairedDelimiter\bangle\langle\rangle

\section*{5.8}
The set of $n \times n$ matrices with determinant $2$ under matrix multiplication does not form a subgroup of $GL(n, \mathbb{R})$. Let $A,B$ be $n \times n$ matrices with $\det(A) = \det(B) = 2$. Note then that $\det(AB) = \det(A) \det(B) = 4 \neq 2$. Therefore $AB$ is not contained within the set, hence closure is not satisfied and not a subgroup.

\section*{5.15}
Let $F_0$ denote the subset of all $f \in F$ such that $f(1) = 0$
\subsection*{Part A}
$F_0$ does form a subgroup of $F$ under addition.
\begin{proof}
	Let $F_0$ denote the subset of all $f \in F$ such that $f(1) = 0$. Let $f,g \in F_0$. Then $f(1) + g(1) = 0 + 0 = 0$, therefore $F_0$ is closed under functional addition. The identity element of $F$ is the zero constant function, that is $e(x) = 0$. Note that $e(1) = 0$, therefore $e \in F_0$, hence the identity element of $F$ is in $F_0$. Let $f \in F_0$. Let $f^{-1} = -f$, that is the negative of $f$. $-f \in F_0$ since $-f(1) = 0$. Additionally, $f + (-f) = 0 = e$, thefore every element of $F_0$ has an inverse. Therefore $F_0 \leq F$ under addition.
\end{proof}

\subsection*{Part B}
$F_0$ does not form a subgroup of $\tilde{F}$ under multiplication. Note that every element in $F_0$ by definition has a zero value at $1$, hence $F_0 \nsubseteq \tilde{F}$, meaning $F_0$ cannot be a subgroup of $\tilde{F}$ under multiplication.

\section*{5.20}
\begin{align*}
	G_i &\leq G_i \text{ for } i = \qty{1, 2, \ldots , 9} \\
	G_2 &< G_8 < G_7 < G_1 < G_4 \\
	G_9 &< G_3 < G_5 \\
	G_6 &< G_5
.\end{align*}

\section*{5.21}
\subsection*{Part A}
\[
	\qty{\ldots, -50, -25, 0, 25, 50, \ldots}
.\]
\subsection*{Part B}
\[
	\qty{\ldots, 4, 2 , 0, \frac{1}{2}, \frac{1}{4} \ldots}
.\]
\subsection*{Part C}
\[
	\qty{\ldots, \frac{1}{\pi^2}, \frac{1}{\pi}, 0, \pi, \pi^2 \ldots}
.\]

\section*{5.33}
Note that
\[
	\mqty(
		0 & 0 & 1 & 0 \\
 		0 & 0 & 0 & 1 \\
 		1 & 0 & 0 & 0 \\
 		0 & 1 & 0 & 0 \\
	)
	\mqty(
		0 & 0 & 1 & 0 \\
 		0 & 0 & 0 & 1 \\
 		1 & 0 & 0 & 0 \\
 		0 & 1 & 0 & 0 \\
	)
	=
	\mqty(
		0 & 0 & 1 & 0 \\
 		0 & 0 & 0 & 1 \\
 		1 & 0 & 0 & 0 \\
 		0 & 1 & 0 & 0 \\
	)
.\]
And also that
\[
	\mqty(
		0 & 0 & 1 & 0 \\
 		0 & 0 & 0 & 1 \\
 		1 & 0 & 0 & 0 \\
 		0 & 1 & 0 & 0 \\
	)^{-1} = 
	\mqty(
		0 & 0 & 1 & 0 \\
 		0 & 0 & 0 & 1 \\
 		1 & 0 & 0 & 0 \\
 		0 & 1 & 0 & 0 \\
	)
.\]
Therefore the only elements that are generated by the matrix are the identity element and itself, hence the order of the subgroup is $2$.

\section*{5.35}
Note that
\[
	\mqty(
		0 & 1 & 0 & 0 \\
		0 & 0 & 0 & 1 \\
		0 & 0 & 1 & 0 \\
		1 & 0 & 0 & 0 \\
	)
	\mqty(
		0 & 1 & 0 & 0 \\
		0 & 0 & 0 & 1 \\
		0 & 0 & 1 & 0 \\
		1 & 0 & 0 & 0 \\
	) =
	\mqty(
		0 & 1 & 0 & 0 \\
		0 & 0 & 0 & 1 \\
		0 & 0 & 1 & 0 \\
		1 & 0 & 0 & 0 \\
	)
.\]
The inverse of the matrix is
\[
	\mqty(
		0 & 0 & 0 & 1 \\
		1 & 0 & 0 & 0 \\
		0 & 0 & 1 & 0 \\
		0 & 1 & 0 & 0 \\
	)
\]
which under repeated multiplication
\[
	\mqty(
		0 & 0 & 0 & 1 \\
		1 & 0 & 0 & 0 \\
		0 & 0 & 1 & 0 \\
		0 & 1 & 0 & 0 \\
	)
	\mqty(
		0 & 0 & 0 & 1 \\
		1 & 0 & 0 & 0 \\
		0 & 0 & 1 & 0 \\
		0 & 1 & 0 & 0 \\
	) =
	\mqty(
		0 & 0 & 0 & 1 \\
		1 & 0 & 0 & 0 \\
		0 & 0 & 1 & 0 \\
		0 & 1 & 0 & 0 \\
	)
\]
Therefore the generated elements are the inverse of the matrix, the matrix itself, and the identity element. Hence the order of the subgroup is $3$.

\section*{5.36}
\subsection*{Part A}
\[
	\begin{array}{c|c|c|c|c|c|c}
		+ & 0 & 1 & 2 & 3 & 4 & 5 \\\hline
		0 & 0 & 1 & 2 & 3 & 4 & 5 \\\hline
		1 & 1 & 2 & 3 & 4 & 5 & 0 \\\hline
		2 & 2 & 3 & 4 & 5 & 0 & 1 \\\hline
		3 & 3 & 4 & 5 & 0 & 1 & 2 \\\hline
		4 & 4 & 5 & 0 & 1 & 2 & 3 \\\hline
		5 & 5 & 0 & 1 & 2 & 3 & 4
	\end{array}
.\]

\subsection*{Part B}
\begin{align*}
	\langle 0 \rangle &= \qty{0} \\
	\langle 1 \rangle &= \qty{0,1,2,3,4,5} \\
	\langle 2 \rangle &= \qty{0, 2, 4} \\
	\langle 3 \rangle &= \qty{0, 3} \\
	\langle 4 \rangle &= \qty{0, 2, 4} \\
	\langle 5 \rangle &= \qty{0,1,2,3,4,5}
.\end{align*}

\subsection*{Part C}
Both $1$ and $5$ are generators for $\mathbb{Z}_6$.

\subsection*{Part D}
\begin{center}
	\begin{tikzpicture}[scale=2]
		\node (1) at (0,2) {$\bangle{1} = \bangle{5}$};
		\node (2) at (1.2,1) {$\bangle{2} = \bangle{4}$};
		\node (3) at (-1.2,1) {$\bangle{3}$};
		\node (0) at (0,0) {$\bangle{0}$};
	
		\draw (1) -- (2) -- (0) -- (3) -- (1);
	\end{tikzpicture}
\end{center}

\section*{5.42}
\begin{proof}
	Let $\bangle{G, *}$ and $\bangle{G', *'}$ be groups and let $\phi: G \to G'$ be an isomorphism between $G$ and $G'$. Assume that $G$ is a cylic group. Therefore there exists $g \in G$ such that $\bangle{g} = G$. Examine $\phi(g)$ as a candidate for a generator of $G'$. Let $a' \in G'$. Since $\phi$ is an isomorphism, there is an $a \in G$ such that $\phi(a) = a'$. Since $g$ is a generator, there is an $n \in \mathbb{Z}$ such that $a = g^n$. Therefore $a' = \phi(g^n)$. By repeated application of the homorphism property of $\phi$, $a' = \phi(g)^n$. There all elements of $G'$ can be generated by $\phi(g)$, hence $G'$ is cyclic.

\end{proof}

\section*{5.46}
\begin{proof}
	Let $G$ be a cyclic group and assume it has only one generator. Since $G$ is cyclic there is an $a \in G$ such that 
	\[
		G = \qty{e, a, a^2, \ldots, a^{n-1}}
	.\]
	Note that $a^{-1} = a^{n-1}$ is also a generator of $G$ since for all $k$ from $1$ to $n-1$ since
	\[
		(a^{-1})^k = (a^k)^{-1} = a^{n-k}
	.\]
	Therefore if $G$ has only one generator, $a = a^{n-1}$, or equivalently by examining the powers
	\begin{align*}
		n-1 &= 1 \\
		n &= 2
	.\end{align*}
	Therefore the group must be of size $2$. Note that if $G = \qty{e}$, it would also work. Hence if a cyclic group has a single generator it has an order of at most $2$.
\end{proof}

\section*{5.47}
\begin{proof}
	Let $G$ be an abelian group. Define the set $H = \qty{x \in G : x^2 = e}$. Let $a,b \in H$. Then
	\begin{align*}
		a^2 b^2 &= e \\
		aa bb &= e \\
		abab &= e \tag{Since $G$ is abelian} \\
		(ab)(ab) &= e \tag{By associativity}\\
		(ab)^2 &= e
	.\end{align*}
	Therefore $ab \in H$, meaning $H$ is closed under the group operation of $G$. Consider the identity $e$ of $G$. Since $ee = e$, it is in $H$. Let $a \in H$. Since $aa = aa = e$, $a$ is its own inverse and therefore every element of $H$ has an inverse. Therefore since $H$ is closed under the group operation of $G$, has the identity of $G$, and has an inverse for every element, $H \leq G$.
\end{proof}

\section*{5.49}
\begin{proof}
	Let $G$ be a finite group and let $a \in G$. Consider the set $S = \qty{a, a^2, a^3, \ldots, a^m, a^{m+1}}$ where $m = |G|$. Since there are $m+1$ elements in $S$, there has to be a repeat otherwise $S$ would contain $m+1$ unique elements which is larger than $|G|$. Therefore there exists $\alpha, \beta \in \mathbb{Z}^+$ such that $\alpha \neq \beta$ and $a^\alpha = a^\beta$. Without loss of generality let $\alpha < \beta$. Then
	\begin{align*}
		a^\beta &= a^\alpha \\
		a^{\beta - \alpha} &= e
	.\end{align*}
	Since $\alpha < \beta$, $\beta - \alpha > 0$ meaning $\beta - \alpha \in \mathbb{Z}^+$. Therefore for any $a \in G$ there exists a $n \in \mathbb{Z}^+$ such that $a^n = e$.
\end{proof}

\section*{5.50}
\begin{proof}
	Let $G$ be a finite group. Let $H \subseteq G$ where $|H| = m$ and $m$ is finite and assume $H$ is closed under the binary operation of $G$. Let $a \in H$. Consider the set $S = \qty{a, a^2, a^3, \ldots, a^m, a^{m+1}}$. Every element of $S$ is in $H$ since $H$ is closed. Since there are $m+1$ elements in $S$, there has to be a repeat otherwise $S$ would contain $m+1$ unique elements which is larger than $|H|$. Therefore there exists $\alpha, \beta \in \mathbb{Z}^+$ such that $\alpha \neq \beta$ and $a^\alpha = a^\beta$. Without loss of generality let $\alpha < \beta$. Then
	\begin{align*}
		a^\beta &= a^\alpha \\
		a^{\beta - \alpha} &= e
	.\end{align*}
	Since $\alpha < \beta$, $\beta - \alpha > 0$ meaning $\beta - \alpha \in \mathbb{Z}^+$. Therefore $e \in H$. Additionally every element of $H$ has an inverse since
	\begin{align*}
		a^{\beta - \alpha - 1} a &= a^{\beta - \alpha} \\
														 &= e
	.\end{align*}
	Therefore since $H$ is closed under the group operation of $G$, has the identity of $G$, and has an inverse for every element, $H \leq G$.
\end{proof}

\section*{5.51}
\begin{proof}
	Let $G$ be a group and let $a \in G$. Define $H_a = \qty{x \in G : xa = ax}$. Let $x,y \in H_a$. Then note that $xya = x(ya) = xay = (xa)y = axy$, therefore $xy \in H_a$. Note the identity of $G$ is in $H_a$ since $ea = a = ae$. Let $x \in H_a$. Then
	\begin{align*}
		xa &= ax \\
		a &= x^{-1}ax \\
		ax^{-1} &= x^{-1}a
	.\end{align*}
	Therefore $x^{-1} \in H_a$. Therefore since $H$ is closed under the group operation of $G$, has the identity of $G$, and has an inverse for every element, $H \leq G$.
\end{proof}

\section*{5.54}
\begin{proof}
	Let $G$ be a group. Let $H$ and $K$ be sets such that $H \leq G$ and $K \leq G$. Consider $H \cap K$. Let $a,b \in H \cap K$. Then $a,b \in H$ and $a,b \in K$. Additionally since both are subgroups of $G$, $ab \in H$ and $ab \in K$ by closure. Therefore $ab \in H \cap K$. Since both $H$ and $K$ are subgroups of $G$, they both contain the identity element of $G$, and therefore $H \cap K$ contains the identity element. Let $a \in H \cap K$. Then $a^{-1} \in H$ and $a^{-1} \in K$ since both are subgroups and hence have inverses for every element. Therefore since $a^{-1}$ is in $H$ and $K$, $a^{-1} \in H \cap K$. Hence $H \cap K \leq G$.
\end{proof}

\end{document}
