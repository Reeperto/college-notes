\documentclass[12pt]{extarticle}

% Document Layout and Font
\usepackage{subfiles}
\usepackage[margin=2cm, headheight=15pt]{geometry}
\usepackage{fancyhdr}
\usepackage{enumitem}	
\usepackage{wrapfig}
\usepackage{multicol}
\usepackage{caption, subcaption}

\usepackage[p,osf]{scholax}

\renewcommand*\contentsname{Table of Contents}
\renewcommand{\headrulewidth}{0pt}
\pagestyle{fancy}
\fancyhf{}
\fancyfoot[R]{$\thepage$}
\setlength{\parindent}{0cm}
\setlength{\headheight}{17pt}
\hfuzz=9pt

% Utility Management
\usepackage{color}
\usepackage{colortbl}
\usepackage{xcolor}
\usepackage{xpatch}
\usepackage{xparse}

\definecolor{links}{HTML}{1c73a5}
\definecolor{bar}{HTML}{584AA8}

% Math Packages
\usepackage{mathtools, amsmath, amsthm, thmtools, amssymb, physics}
\usepackage[scaled=1.075,ncf,vvarbb]{newtxmath}

\newcommand\B{\mathbb{B}}
\newcommand\C{\mathbb{C}}
\newcommand\R{\mathbb{R}}
\newcommand\Q{\mathbb{Q}}
\newcommand\N{\mathbb{N}}
\newcommand\Z{\mathbb{Z}}

\newcommand\Prob[1]{\mathbb{P}\qty(#1)}
\newcommand\Var[1]{\text{Var}\qty(#1)}
\newcommand\Exp[1]{\mathbb{E}\qty[#1]}
\newcommand\ball[1]{\B\qty(#1)}
\newcommand\res[1]{\underset{#1}{\operatorname{Res}}\;}
\renewcommand\pv{\mathrm{p.v.}}

\newcommand\conj[1]{\overline{#1}}
\DeclareMathOperator{\Arg}{Arg}
\DeclareMathOperator{\Log}{Log}
\DeclareMathOperator{\cis}{cis}

\DeclareMathOperator{\dom}{dom}
\DeclareMathOperator{\spann}{span}
\DeclareMathOperator{\nullity}{nullity}

\newcommand\st{\text{ s.t. }}

% TIKZ
\usepackage{tikz}
\usepackage{pgfplots}
\usetikzlibrary{arrows.meta}
\usetikzlibrary{math}
\usetikzlibrary{cd}
\usetikzlibrary{patterns}
\usetikzlibrary{decorations.markings}
\usetikzlibrary{calc}

% Boxes and Theorems
\usepackage[most]{tcolorbox}
\tcbuselibrary{skins}
\tcbuselibrary{breakable}
\tcbuselibrary{theorems}

\newtheoremstyle{default}{0pt}{0pt}{}{}{\bfseries}{\normalfont.}{0.5em}{}
\theoremstyle{default}

\renewcommand*{\proofname}{\textit{\textbf{Proof.}}}
\renewcommand*{\qedsymbol}{$\blacksquare$}
\tcolorboxenvironment{proof}{
	breakable,
	coltitle = black,
	colback = white,
	frame hidden,
	boxrule = 0pt,
	boxsep = 0pt,
	borderline west={3pt}{0pt}{bar},
	sharp corners = all,
	enhanced,
}

\newtheorem{theorem}{Theorem}[section]{\bfseries}{}
\tcolorboxenvironment{theorem}{
	breakable,
	enhanced,
	boxrule = 0pt,
	frame hidden,
	coltitle = black,
	colback = blue!7,
	left = 0.5em,
	sharp corners = all,
}

\newtheorem{corollary}{Corollary}[section]{\bfseries}{}
\tcolorboxenvironment{corollary}{
	breakable,
	enhanced,
	boxrule = 0pt,
	frame hidden,
	coltitle = black,
	colback = white!0,
	left = 0.5em,
	sharp corners = all,
}

\newtheorem{lemma}{Lemma}[section]{\bfseries}{}
\tcolorboxenvironment{lemma}{
	breakable,
	enhanced,
	boxrule = 0pt,
	frame hidden,
	coltitle = black,
	colback = green!7,
	left = 0.5em,
	sharp corners = all,
}

\newtheorem{definition}{Definition}[section]{\bfseries}{}
\tcolorboxenvironment{definition}{
	breakable,
	coltitle = black,
	colback = white,
	frame hidden,
	boxsep = 0pt,
	boxrule = 0pt,
	borderline west = {3pt}{0pt}{orange},
	sharp corners = all,
	enhanced,
}

\newtheorem{example}{Example}[section]{\bfseries}{}
\tcolorboxenvironment{example}{
	% title = \textbf{Example},
	% detach title,
	% before upper = {\tcbtitle\quad},
	breakable,
	coltitle = black,
	colback = white,
	frame hidden,
	boxrule = 0pt,
	boxsep = 0pt,
	borderline west={3pt}{0pt}{green!70!black},
	sharp corners = all,
	enhanced,
}

\newtheoremstyle{remark}{0pt}{4pt}{}{}{\bfseries\itshape}{\normalfont.}{0.5em}{}
\theoremstyle{remark}
\newtheorem*{remark}{Remark}


% TColorBoxes
\newtcolorbox{week}{
	colback = black,
	coltext = white,
	fontupper = {\large\bfseries},
	width = 1.2\paperwidth,
	size = fbox,
	halign upper = center,
	center
}

\newcommand{\banner}[2]{
    \pagebreak
    \begin{week}
   		\section*{#1}
    \end{week}
    \addcontentsline{toc}{section}{#1}
    \addtocounter{section}{1}
    \setcounter{subsection}{0}
}

% Hyperref
\usepackage{hyperref}
\hypersetup{
	colorlinks=true,
	linktoc=all,
	linkcolor=links,
	bookmarksopen=true
}


\fancyhead[R]{Homework \#6}
\fancyhead[L]{Eli Griffiths}
\renewcommand{\headrulewidth}{1pt}
\setlength\parindent{0pt}

\begin{document}

\section*{Problem 1}
\subsection*{Part A}
The confidence interval will be of the form $\overline{X} \pm 1.96\cdot\frac{s}{\sqrt{n}}$ with $\overline{X} = 38, s = 5, n = 36$, giving a confidence interval of
\[
	(36.37, 39.63)
.\]

\subsection*{Part B}
The confidence interval will be of the form $\overline{X} - \overline{Y} \sqrt{\frac{s_1^2}{n} + \frac{s_2^2}{n}}$ with $\overline{X} = 38, \overline{Y} = 36, s_1 = 5, s_2 = 7, n_1 = 36, n_2 = 25$, giving a confidence interval of
\[
	(-1.19, 5.19)
.\]

\subsection*{Part C}
We can say with 95\% confidence that $\mu_1 - \mu_2$ can be equal to $0$, thus we cannot claim to have evidence that compact and economy cars differ in fuel efficiency.

\section*{Problem 2}
The confidence interval will be of the form $\overline{X} \pm 1.96 \cdot\sqrt{\frac{\overline{X} (1 - \overline{X})}{n}}$ with $\overline{X} = \frac{710}{1000}, n = 1000$, giving a confidence interval of
\[
	(0.68, 0.74)
.\]

\section*{Problem 3}
\subsection*{Part A}
The confidence interval will be of the form $\overline{X} \pm 1.96 \cdot \frac{s}{\sqrt{n}}$ with $\overline{X} = 55, s = 7, n = 36$, giving a confidence interval of
\[
	(52.7, 57.2)
.\]

\subsection*{Part B}
The confidence interval will be of the form $\overline{X} \pm 2.57 \cdot \frac{s}{\sqrt{n}}$ with $\overline{X} = 55, s = 7, n = 36$, giving a confidence interval of
\[
	(52, 58)
.\]

\subsection*{Part C}
The interval with a higher confidence interval is wider, which is a consequence of needing a larger margin of error to be more certain that the true population paramter is within the interval.

\subsection*{Part D}
We have 95\% confidence that $\mu$ lies in the interval $(52.7, 57.2)$ which lies entirely above 50.

\subsection*{Part E}
\begin{align*}
	H_0 &: \mu \leq 50 \\
	H_a &: \mu > 50
.\end{align*}

\section*{Problem 4}
\subsection*{Part A}
The confidence interval will be of the form $\overline{X} \pm 1.96 \cdot \frac{s}{\sqrt{n}}$ with $\overline{X} = 4.5, s = 3.6, n = 100$, giving a confidence interval of
\[
	(3.79, 5.21)
.\]

\subsection*{Part B}
We can say with 95\% confidence that the average difference between scores of nutritious and light breakfasts is between 3.79 and 5.21, hence we are 95\% confident that a nutritious breakfast will have 3.79 to 5.21 more points on average.

\section*{Problem 5}
$\overline{X}$ will follow an approximate normal distribution with $\mu = 2.6$ and $s = \frac{1.4}{\sqrt{100}} = 0.14$. Therefore
\[
	\Prob{\overline{X} > 3} = 1 - \verb|pnorm(3, 2.6, 0.14)| \approx 0.002
.\]

\end{document}
