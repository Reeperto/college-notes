\documentclass[12pt]{extarticle}

\usepackage{multicol}

% Document Layout and Font
\usepackage{subfiles}
\usepackage[margin=2cm, headheight=15pt]{geometry}
\usepackage{fancyhdr}
\usepackage{enumitem}	
\usepackage{wrapfig}
\usepackage{multicol}
\usepackage{caption, subcaption}

\usepackage[p,osf]{scholax}

\renewcommand*\contentsname{Table of Contents}
\renewcommand{\headrulewidth}{0pt}
\pagestyle{fancy}
\fancyhf{}
\fancyfoot[R]{$\thepage$}
\setlength{\parindent}{0cm}
\setlength{\headheight}{17pt}
\hfuzz=9pt

% Utility Management
\usepackage{color}
\usepackage{colortbl}
\usepackage{xcolor}
\usepackage{xpatch}
\usepackage{xparse}

\definecolor{links}{HTML}{1c73a5}
\definecolor{bar}{HTML}{584AA8}

% Math Packages
\usepackage{mathtools, amsmath, amsthm, thmtools, amssymb, physics}
\usepackage[scaled=1.075,ncf,vvarbb]{newtxmath}

\newcommand\B{\mathbb{B}}
\newcommand\C{\mathbb{C}}
\newcommand\R{\mathbb{R}}
\newcommand\Q{\mathbb{Q}}
\newcommand\N{\mathbb{N}}
\newcommand\Z{\mathbb{Z}}

\newcommand\Prob[1]{\mathbb{P}\qty(#1)}
\newcommand\Var[1]{\text{Var}\qty(#1)}
\newcommand\Exp[1]{\mathbb{E}\qty[#1]}
\newcommand\ball[1]{\B\qty(#1)}
\newcommand\res[1]{\underset{#1}{\operatorname{Res}}\;}
\renewcommand\pv{\mathrm{p.v.}}

\newcommand\conj[1]{\overline{#1}}
\DeclareMathOperator{\Arg}{Arg}
\DeclareMathOperator{\Log}{Log}
\DeclareMathOperator{\cis}{cis}

\DeclareMathOperator{\dom}{dom}
\DeclareMathOperator{\spann}{span}
\DeclareMathOperator{\nullity}{nullity}

\newcommand\st{\text{ s.t. }}

% TIKZ
\usepackage{tikz}
\usepackage{pgfplots}
\usetikzlibrary{arrows.meta}
\usetikzlibrary{math}
\usetikzlibrary{cd}
\usetikzlibrary{patterns}
\usetikzlibrary{decorations.markings}
\usetikzlibrary{calc}

% Boxes and Theorems
\usepackage[most]{tcolorbox}
\tcbuselibrary{skins}
\tcbuselibrary{breakable}
\tcbuselibrary{theorems}

\newtheoremstyle{default}{0pt}{0pt}{}{}{\bfseries}{\normalfont.}{0.5em}{}
\theoremstyle{default}

\renewcommand*{\proofname}{\textit{\textbf{Proof.}}}
\renewcommand*{\qedsymbol}{$\blacksquare$}
\tcolorboxenvironment{proof}{
	breakable,
	coltitle = black,
	colback = white,
	frame hidden,
	boxrule = 0pt,
	boxsep = 0pt,
	borderline west={3pt}{0pt}{bar},
	sharp corners = all,
	enhanced,
}

\newtheorem{theorem}{Theorem}[section]{\bfseries}{}
\tcolorboxenvironment{theorem}{
	breakable,
	enhanced,
	boxrule = 0pt,
	frame hidden,
	coltitle = black,
	colback = blue!7,
	left = 0.5em,
	sharp corners = all,
}

\newtheorem{corollary}{Corollary}[section]{\bfseries}{}
\tcolorboxenvironment{corollary}{
	breakable,
	enhanced,
	boxrule = 0pt,
	frame hidden,
	coltitle = black,
	colback = white!0,
	left = 0.5em,
	sharp corners = all,
}

\newtheorem{lemma}{Lemma}[section]{\bfseries}{}
\tcolorboxenvironment{lemma}{
	breakable,
	enhanced,
	boxrule = 0pt,
	frame hidden,
	coltitle = black,
	colback = green!7,
	left = 0.5em,
	sharp corners = all,
}

\newtheorem{definition}{Definition}[section]{\bfseries}{}
\tcolorboxenvironment{definition}{
	breakable,
	coltitle = black,
	colback = white,
	frame hidden,
	boxsep = 0pt,
	boxrule = 0pt,
	borderline west = {3pt}{0pt}{orange},
	sharp corners = all,
	enhanced,
}

\newtheorem{example}{Example}[section]{\bfseries}{}
\tcolorboxenvironment{example}{
	% title = \textbf{Example},
	% detach title,
	% before upper = {\tcbtitle\quad},
	breakable,
	coltitle = black,
	colback = white,
	frame hidden,
	boxrule = 0pt,
	boxsep = 0pt,
	borderline west={3pt}{0pt}{green!70!black},
	sharp corners = all,
	enhanced,
}

\newtheoremstyle{remark}{0pt}{4pt}{}{}{\bfseries\itshape}{\normalfont.}{0.5em}{}
\theoremstyle{remark}
\newtheorem*{remark}{Remark}


% TColorBoxes
\newtcolorbox{week}{
	colback = black,
	coltext = white,
	fontupper = {\large\bfseries},
	width = 1.2\paperwidth,
	size = fbox,
	halign upper = center,
	center
}

\newcommand{\banner}[2]{
    \pagebreak
    \begin{week}
   		\section*{#1}
    \end{week}
    \addcontentsline{toc}{section}{#1}
    \addtocounter{section}{1}
    \setcounter{subsection}{0}
}

% Hyperref
\usepackage{hyperref}
\hypersetup{
	colorlinks=true,
	linktoc=all,
	linkcolor=links,
	bookmarksopen=true
}


\newcommand{\powerset}[1]{\mathcal{P}(#1)}
\fancyhead[R]{Homework \#1}
\fancyhead[L]{Eli Griffiths}
\renewcommand{\headrulewidth}{1pt}
\setlength\parindent{0pt}

\usetikzlibrary{arrows.meta}

\begin{document}

\section*{Problem 1}
\subsection*{Part A}
\begin{align*}
	P(A) &= 0.4 \\
	P(B) &= 0.3 \\
	P(AB) &= 0.2
.\end{align*}

\subsection*{Part B}
The events of having a wireless mouse and wireless keyboard are not mutually exclusive since $P(AB) \neq 0$.

\subsection*{Part C}
\begin{align*}
	P(A \cup B) &= P(A) + P(B) - P(AB) \\
	&= 0.4 + 0.2 - 0.2 \\
	&= 40\%
.\end{align*}

\subsection*{Part D}
\begin{align*}
	P(A^\complement \cup B^\complement) &= P((AB)^\complement) \\
	&= 1 - P(AB) \\
	&= 1 - 0.2 \\
	&= 80\%
.\end{align*}

\subsection*{Part E}
The probability for each person is independent of the others choice, meaning for each person there is a $40\%$ chance they have a wireless mouse. Therefore the probability that both have a wireless mouth is $(0.4)^2 = 16\%$.

\section*{Problem 2}
Let the event of getting at least one $3$ in $100$ rolls be denoted by $A$. Then $A^\complement$ is the event of not getting any $3$'s in $100$ rolls. The chance of not rolling a $3$ each roll is $\frac{5}{6}$. Therefore $P(A^\complement) = \qty(\frac{5}{6})^{100}$, meaning that $P(A) = 1 - \qty(\frac{5}{6})^{100} \approx 99.9999987925\%$.

\section*{Problem 3}
\subsection*{Part A}
% 
% P(A u B) = P(A) + P(B) - P(AB)
% P(A u B) + P(AB) = P(A) + P(B) - P(AB)
%
% 
\begin{proof}
	Let $A$ and $B$ be events from some sample space $S$. Therefore
	\begin{align*}
		P((AB)^\complement) &= P(A^\complement \cup B^\complement) \leq P(A^\complement) + P(B^\complement) \\
		P((AB)^\complement) & \leq P(A^\complement) + P(B^\complement)
	.\end{align*}
	Note that $P((AB)^\complement) = 1 - P(AB)$, therefore
	\begin{align*}
		P((AB)^\complement) & \leq P(A^\complement) + P(B^\complement) \\
		- P(AB) & \leq - P(A) + 1 - P(B) \\
		P(AB) & \geq P(A) + P(B) - 1
	.\end{align*}
\end{proof}

\subsection*{Part B}
Since $AB \subseteq B$, then $P(AB) \leq P(B)$ meaning, therefore using that and the Bonferroni inequlity:
\begin{align*}
	&P(AB) \geq P(A) + P(B) - 1 \\
	&P(AB) \geq \frac{1}{12} \\
	P(B) \geq &P(AB) \geq \frac{1}{12} \\
	\frac{1}{12} \leq &P(AB) \leq \frac{1}{3}
.\end{align*}

\section*{Problem 4}
\subsection*{Part A}
\[
	P(X) = 0.2 + 0.3 + 0.1 + 0.3 + 0.1 = 1.0 \;\checkmark
.\]

\subsection*{Part B}
\[
	P(X < 3) = 0.2 + 0.3 + 0.1 = 0.6 = 60%
.\]

\subsection*{Part C}
\[
	P((X=0) \cup (X=4)) = P(X=0) + P(X=4) = 0.2 + 0.1 = 0.3 = 30\%
.\]

\subsection*{Part D}
\begin{align*}
	P(X=1 | X>0) &= \frac{P((X=1)\cap (X>0))}{P(X > 0)} \\
	&= \frac{P(X=1)}{P(X>0)} \\
	&= \frac{0.3}{0.3 + 0.1 + 0.3 + 0.1} \\
	&= 0.375 = 37.5\%
.\end{align*}

\subsection*{Part E}
Let $A \Rightarrow$ patient 1 has 0 limbs injure and $B \Rightarrow$ patient 2 has 4 limbs injured. Since $A$ and $B$ are independent, $P(AB) = P(A) \cdot P(B)$. $P(A) = 0.2$ and $P(B) = 0.1$, therefore $P(AB) = 0.02$, meaning
\begin{align*}
	P(A | B) &= \frac{P(AB)}{P(B)} \\
	&= \frac{0.02}{0.1} \\
	&= 0.2 = 20\%
.\end{align*}


\end{document}
