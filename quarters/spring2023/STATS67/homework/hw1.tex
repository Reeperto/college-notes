\documentclass[12pt]{extarticle}

\usepackage{multicol}

% Document Layout and Font
\usepackage{subfiles}
\usepackage[margin=2cm, headheight=15pt]{geometry}
\usepackage{fancyhdr}
\usepackage{enumitem}	
\usepackage{wrapfig}
\usepackage{multicol}
\usepackage{caption, subcaption}

\usepackage[p,osf]{scholax}

\renewcommand*\contentsname{Table of Contents}
\renewcommand{\headrulewidth}{0pt}
\pagestyle{fancy}
\fancyhf{}
\fancyfoot[R]{$\thepage$}
\setlength{\parindent}{0cm}
\setlength{\headheight}{17pt}
\hfuzz=9pt

% Utility Management
\usepackage{color}
\usepackage{colortbl}
\usepackage{xcolor}
\usepackage{xpatch}
\usepackage{xparse}

\definecolor{links}{HTML}{1c73a5}
\definecolor{bar}{HTML}{584AA8}

% Math Packages
\usepackage{mathtools, amsmath, amsthm, thmtools, amssymb, physics}
\usepackage[scaled=1.075,ncf,vvarbb]{newtxmath}

\newcommand\B{\mathbb{B}}
\newcommand\C{\mathbb{C}}
\newcommand\R{\mathbb{R}}
\newcommand\Q{\mathbb{Q}}
\newcommand\N{\mathbb{N}}
\newcommand\Z{\mathbb{Z}}

\newcommand\Prob[1]{\mathbb{P}\qty(#1)}
\newcommand\Var[1]{\text{Var}\qty(#1)}
\newcommand\Exp[1]{\mathbb{E}\qty[#1]}
\newcommand\ball[1]{\B\qty(#1)}
\newcommand\res[1]{\underset{#1}{\operatorname{Res}}\;}
\renewcommand\pv{\mathrm{p.v.}}

\newcommand\conj[1]{\overline{#1}}
\DeclareMathOperator{\Arg}{Arg}
\DeclareMathOperator{\Log}{Log}
\DeclareMathOperator{\cis}{cis}

\DeclareMathOperator{\dom}{dom}
\DeclareMathOperator{\spann}{span}
\DeclareMathOperator{\nullity}{nullity}

\newcommand\st{\text{ s.t. }}

% TIKZ
\usepackage{tikz}
\usepackage{pgfplots}
\usetikzlibrary{arrows.meta}
\usetikzlibrary{math}
\usetikzlibrary{cd}
\usetikzlibrary{patterns}
\usetikzlibrary{decorations.markings}
\usetikzlibrary{calc}

% Boxes and Theorems
\usepackage[most]{tcolorbox}
\tcbuselibrary{skins}
\tcbuselibrary{breakable}
\tcbuselibrary{theorems}

\newtheoremstyle{default}{0pt}{0pt}{}{}{\bfseries}{\normalfont.}{0.5em}{}
\theoremstyle{default}

\renewcommand*{\proofname}{\textit{\textbf{Proof.}}}
\renewcommand*{\qedsymbol}{$\blacksquare$}
\tcolorboxenvironment{proof}{
	breakable,
	coltitle = black,
	colback = white,
	frame hidden,
	boxrule = 0pt,
	boxsep = 0pt,
	borderline west={3pt}{0pt}{bar},
	sharp corners = all,
	enhanced,
}

\newtheorem{theorem}{Theorem}[section]{\bfseries}{}
\tcolorboxenvironment{theorem}{
	breakable,
	enhanced,
	boxrule = 0pt,
	frame hidden,
	coltitle = black,
	colback = blue!7,
	left = 0.5em,
	sharp corners = all,
}

\newtheorem{corollary}{Corollary}[section]{\bfseries}{}
\tcolorboxenvironment{corollary}{
	breakable,
	enhanced,
	boxrule = 0pt,
	frame hidden,
	coltitle = black,
	colback = white!0,
	left = 0.5em,
	sharp corners = all,
}

\newtheorem{lemma}{Lemma}[section]{\bfseries}{}
\tcolorboxenvironment{lemma}{
	breakable,
	enhanced,
	boxrule = 0pt,
	frame hidden,
	coltitle = black,
	colback = green!7,
	left = 0.5em,
	sharp corners = all,
}

\newtheorem{definition}{Definition}[section]{\bfseries}{}
\tcolorboxenvironment{definition}{
	breakable,
	coltitle = black,
	colback = white,
	frame hidden,
	boxsep = 0pt,
	boxrule = 0pt,
	borderline west = {3pt}{0pt}{orange},
	sharp corners = all,
	enhanced,
}

\newtheorem{example}{Example}[section]{\bfseries}{}
\tcolorboxenvironment{example}{
	% title = \textbf{Example},
	% detach title,
	% before upper = {\tcbtitle\quad},
	breakable,
	coltitle = black,
	colback = white,
	frame hidden,
	boxrule = 0pt,
	boxsep = 0pt,
	borderline west={3pt}{0pt}{green!70!black},
	sharp corners = all,
	enhanced,
}

\newtheoremstyle{remark}{0pt}{4pt}{}{}{\bfseries\itshape}{\normalfont.}{0.5em}{}
\theoremstyle{remark}
\newtheorem*{remark}{Remark}


% TColorBoxes
\newtcolorbox{week}{
	colback = black,
	coltext = white,
	fontupper = {\large\bfseries},
	width = 1.2\paperwidth,
	size = fbox,
	halign upper = center,
	center
}

\newcommand{\banner}[2]{
    \pagebreak
    \begin{week}
   		\section*{#1}
    \end{week}
    \addcontentsline{toc}{section}{#1}
    \addtocounter{section}{1}
    \setcounter{subsection}{0}
}

% Hyperref
\usepackage{hyperref}
\hypersetup{
	colorlinks=true,
	linktoc=all,
	linkcolor=links,
	bookmarksopen=true
}


\newcommand{\powerset}[1]{\mathcal{P}(#1)}
\fancyhead[R]{Homework \#1}
\fancyhead[L]{Eli Griffiths}
\renewcommand{\headrulewidth}{1pt}
\setlength\parindent{0pt}

\usetikzlibrary{arrows.meta}

\begin{document}

\section*{Problem 1}
\begin{multicols}{2}
\subsection*{Part A}
\[
	S = \qty{2,3,4,5,6,7,8,9,10,11,12}
.\]
\subsection*{Part B}
\[
	S = \qty{1,2,3,4,5,6,7}
.\]

\subsection*{Part C}
	\def\Heads{\mathbf{H}}
	\def\Tails{\mathbf{T}}
	\begin{align*}
		S = \{
			&(\Tails,\Tails,\Tails), (\Heads,\Heads,\Heads),\\
			&(\Tails,\Tails,\Heads), (\Heads,\Tails,\Tails),\\
			&(\Tails,\Heads,\Heads), (\Heads,\Heads,\Tails), \\
			&(\Tails,\Heads,\Tails), (\Heads,\Tails,\Heads)
		\}
	\end{align*}

\columnbreak

\subsection*{Part D}
\[
	S = \qty{0,1,2,3,4,5}
.\]

\subsection*{Part E}
\[
	S = \qty{1,2,3,4,5,6}
.\]

\subsection*{Part F}
\[
	S = \qty{0,1,2,3,4,5}
.\]
\end{multicols}

\section*{Problem 2}
\subsection*{Part A}
\[
	A = \qty{
		(J,M),
		(J,A),
		(S,M),
		(S,A)
	}
.\]

\subsection*{Part B}
\[
	B = \qty{
		(J,M),
		(J,A),
		(S,M),
		(S,A),
		(M,A)
	}
.\]

\subsection*{Part C}
\[
	A^\complement = \qty{
		(J,S),
		(M,A)
	}
.\]

\subsection*{Part D}
\[
	A\cap B = \qty{
		(J, M),
		(J, A),
		(S, M),
		(S, A)
	}
.\]

\section*{Problem 3}
\subsection*{Part A}
A probability distribution was not used as the probability of getting a 2 is negative which is not a valid probability since it does not lie between 0 or 1.

\subsection*{Part B}
A probability distribution was used as each probability is in the range of 0 to 1 and the sum of the probability of each element in the sample space adds up to 1.

\subsection*{Part C}
A probability distribution was not used as the sum of the probabilities is $0.95 \neq 1$.

\section*{Problem 4}
$A$ and $B$ cannot be mutually exclusive as the probability $P(A \cup B) = 1.10 > 1$, meaning that $A$ and $B$ could potentially occur at the same time. Hence they cannot be mutually exclusive.

\section*{Problem 5}
\subsection*{Part A}
\begin{align*}
	P(A^\complement B^\complement) &= P((A \cup B)^\complement) \\
																			&= 1 - P(A \cup B) \\
																			&= 1 - 0.8 \\
																			&= 20\%
\end{align*}
\subsection*{Part B}
\begin{align*}
	P(A\cup B) &= P(A) + P(B) - P(AB) \\
	P(AB) &= P(A) + P(B) - P(A \cup B) \\
	&= 0.5 + 0.45 - 0.8 \\
	&= 15\%
\end{align*}
\subsection*{Part C}
\begin{align*}
	P(AB^\complement) &= P(A) - P(AB) \\
										&= 0.5 - 0.15 \\
										&= 35\%
\end{align*}
\subsection*{Part D}
The events of having a Visa card and a having a Mastercard are not mutually exlcusive since $P(AB) > 0$.

\section*{Problem 6}
\subsection*{Part A}

\begin{align*}
	\hat{p}_1 &= 0.6000 \\
	\hat{p}_2 &= 0.3000 \\
	\hat{p}_3 &= 0.7000 \\
	\hat{p}_4 &= 0.5000 \\
	\hat{p}_5 &= 0.4000
\end{align*}

The true proportion of heads should be 0.5 since the coin is fair.

\subsection*{Part B}

\begin{align*}
	\hat{p}_1 &= 0.5010  \\
	\hat{p}_2 &= 0.4945 \\
	\hat{p}_3 &= 0.5062 \\
	\hat{p}_4 &= 0.4981 \\
	\hat{p}_5 &= 0.4993
\end{align*}

\section*{Problem 7}
\subsection*{Part A}

Probability student doesn't miss any days $= 1 - 0.25 - 0.15 - 0.28 = 32\%$.

\subsection*{Part B}
Probability student misses one day or less $= 1 - 0.15 - 0.28 = 57\%$.

\section*{Problem 8}
The meaning of a $75\%$ chance is that if I play many games against an opponent and look at the proportion of games I won to the games I played, in the long run that proportion will converge towards $0.75$ or $75\%$.

\end{document}
