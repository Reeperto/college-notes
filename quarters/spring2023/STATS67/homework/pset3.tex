\documentclass[12pt]{extarticle}

% Document Layout and Font
\usepackage{subfiles}
\usepackage[margin=2cm, headheight=15pt]{geometry}
\usepackage{fancyhdr}
\usepackage{enumitem}	
\usepackage{wrapfig}
\usepackage{multicol}
\usepackage{caption, subcaption}

\usepackage[p,osf]{scholax}

\renewcommand*\contentsname{Table of Contents}
\renewcommand{\headrulewidth}{0pt}
\pagestyle{fancy}
\fancyhf{}
\fancyfoot[R]{$\thepage$}
\setlength{\parindent}{0cm}
\setlength{\headheight}{17pt}
\hfuzz=9pt

% Utility Management
\usepackage{color}
\usepackage{colortbl}
\usepackage{xcolor}
\usepackage{xpatch}
\usepackage{xparse}

\definecolor{links}{HTML}{1c73a5}
\definecolor{bar}{HTML}{584AA8}

% Math Packages
\usepackage{mathtools, amsmath, amsthm, thmtools, amssymb, physics}
\usepackage[scaled=1.075,ncf,vvarbb]{newtxmath}

\newcommand\B{\mathbb{B}}
\newcommand\C{\mathbb{C}}
\newcommand\R{\mathbb{R}}
\newcommand\Q{\mathbb{Q}}
\newcommand\N{\mathbb{N}}
\newcommand\Z{\mathbb{Z}}

\newcommand\Prob[1]{\mathbb{P}\qty(#1)}
\newcommand\Var[1]{\text{Var}\qty(#1)}
\newcommand\Exp[1]{\mathbb{E}\qty[#1]}
\newcommand\ball[1]{\B\qty(#1)}
\newcommand\res[1]{\underset{#1}{\operatorname{Res}}\;}
\renewcommand\pv{\mathrm{p.v.}}

\newcommand\conj[1]{\overline{#1}}
\DeclareMathOperator{\Arg}{Arg}
\DeclareMathOperator{\Log}{Log}
\DeclareMathOperator{\cis}{cis}

\DeclareMathOperator{\dom}{dom}
\DeclareMathOperator{\spann}{span}
\DeclareMathOperator{\nullity}{nullity}

\newcommand\st{\text{ s.t. }}

% TIKZ
\usepackage{tikz}
\usepackage{pgfplots}
\usetikzlibrary{arrows.meta}
\usetikzlibrary{math}
\usetikzlibrary{cd}
\usetikzlibrary{patterns}
\usetikzlibrary{decorations.markings}
\usetikzlibrary{calc}

% Boxes and Theorems
\usepackage[most]{tcolorbox}
\tcbuselibrary{skins}
\tcbuselibrary{breakable}
\tcbuselibrary{theorems}

\newtheoremstyle{default}{0pt}{0pt}{}{}{\bfseries}{\normalfont.}{0.5em}{}
\theoremstyle{default}

\renewcommand*{\proofname}{\textit{\textbf{Proof.}}}
\renewcommand*{\qedsymbol}{$\blacksquare$}
\tcolorboxenvironment{proof}{
	breakable,
	coltitle = black,
	colback = white,
	frame hidden,
	boxrule = 0pt,
	boxsep = 0pt,
	borderline west={3pt}{0pt}{bar},
	sharp corners = all,
	enhanced,
}

\newtheorem{theorem}{Theorem}[section]{\bfseries}{}
\tcolorboxenvironment{theorem}{
	breakable,
	enhanced,
	boxrule = 0pt,
	frame hidden,
	coltitle = black,
	colback = blue!7,
	left = 0.5em,
	sharp corners = all,
}

\newtheorem{corollary}{Corollary}[section]{\bfseries}{}
\tcolorboxenvironment{corollary}{
	breakable,
	enhanced,
	boxrule = 0pt,
	frame hidden,
	coltitle = black,
	colback = white!0,
	left = 0.5em,
	sharp corners = all,
}

\newtheorem{lemma}{Lemma}[section]{\bfseries}{}
\tcolorboxenvironment{lemma}{
	breakable,
	enhanced,
	boxrule = 0pt,
	frame hidden,
	coltitle = black,
	colback = green!7,
	left = 0.5em,
	sharp corners = all,
}

\newtheorem{definition}{Definition}[section]{\bfseries}{}
\tcolorboxenvironment{definition}{
	breakable,
	coltitle = black,
	colback = white,
	frame hidden,
	boxsep = 0pt,
	boxrule = 0pt,
	borderline west = {3pt}{0pt}{orange},
	sharp corners = all,
	enhanced,
}

\newtheorem{example}{Example}[section]{\bfseries}{}
\tcolorboxenvironment{example}{
	% title = \textbf{Example},
	% detach title,
	% before upper = {\tcbtitle\quad},
	breakable,
	coltitle = black,
	colback = white,
	frame hidden,
	boxrule = 0pt,
	boxsep = 0pt,
	borderline west={3pt}{0pt}{green!70!black},
	sharp corners = all,
	enhanced,
}

\newtheoremstyle{remark}{0pt}{4pt}{}{}{\bfseries\itshape}{\normalfont.}{0.5em}{}
\theoremstyle{remark}
\newtheorem*{remark}{Remark}


% TColorBoxes
\newtcolorbox{week}{
	colback = black,
	coltext = white,
	fontupper = {\large\bfseries},
	width = 1.2\paperwidth,
	size = fbox,
	halign upper = center,
	center
}

\newcommand{\banner}[2]{
    \pagebreak
    \begin{week}
   		\section*{#1}
    \end{week}
    \addcontentsline{toc}{section}{#1}
    \addtocounter{section}{1}
    \setcounter{subsection}{0}
}

% Hyperref
\usepackage{hyperref}
\hypersetup{
	colorlinks=true,
	linktoc=all,
	linkcolor=links,
	bookmarksopen=true
}


\fancyhead[R]{PSET \#$3$}
\fancyhead[L]{Eli Griffiths}
\renewcommand{\headrulewidth}{1pt}
\setlength\parindent{0pt}

\begin{document}

\section*{Problem 1}
\subsection*{Part A}
\[
	\Prob{X = 2} = 0
.\]

\subsection*{Part B}
\[
	\Prob{X \geq 2} = 1 - \Prob{X < 2} = e^{-20} \approx 2.061 \times 10^{-9}
.\]

\subsection*{Part C}
\[
	\Prob{X > 3} = e^{-30} \approx 9.358 \times 10^{-14}
.\]

\subsection*{Part D}
\[
	\Exp{5X} = 5 \cdot \Exp{X} = 5\cdot \frac{1}{10} = \frac{1}{2}
.\]
\[
	\Var{5X} = 25 \cdot \Var{X} = 25 \cdot \frac{1}{100} = \frac{1}{4}
.\]

\subsection*{Part E}
\[
	250\cdot\Exp{X} = 250\cdot \frac{1}{10} = 25\$
.\]
\[
	250\cdot\Exp{5X} = 1250\cdot \frac{1}{10} = 125\$
.\]

\section*{Problem 2}
\subsection*{Part A}
$f(x)$ will be a valid density if
\[
	\int_0^2 f(x) \dd x = 1
.\]
Therefore
\begin{align*}
	\int_0^2 f(x) dx &= \int_0^2 cx^3 \dd x \\
	1 &= \int_0^2 cx^3 \dd x \\
	1 &= \int_0^2 cx^3 \dd x \\
	1 &= \frac{c}{4}\cdot x^4 \eval_0^2 \\
	1 &= \frac{c}{4}\cdot 2^4 \\
	1 &= 4c \implies \boxed{c = \frac{1}{4}}
.\end{align*}

\subsection*{Part B}
\[
	\Prob{X = 1} = \int_1^1 f(x) \dd x = 0
.\]
\[
	\Prob{X = 1 \text{ or } X = 2} = \Prob{X = 1} + \Prob{X = 2} = \int_1^1 f(x) \dd x + \int_2^2 f(x) \dd x = 0
.\]

\subsection*{Part C}
\[
	\Exp{X} = \int_0^2 x f(x) \dd x = \int_0^2 \frac{x^4}{4} \dd x = \frac{8}{5}
.\]

\subsection*{Part D}
\[
	\Prob{0.5 < X < 1.5} = \int_{0.5}^{1.5} \frac{x^3}{4} \dd x = \frac{x^4}{16} \eval_{0.5}^{1.5} = \frac{5}{16}
.\]

\subsection*{Part E}
\[
	\Prob{X > 1.5 | X > 0.5} = \frac{\displaystyle\int_{1.5}^2 x^3 \dd x}{\displaystyle\int_{0.5}^2 x^3 \dd x} = 0.6863
.\]

\section*{Problem 3}
\subsection*{Part A}
\[
	X \sim \text{Binom}(100, 0.85)
.\]

\subsection*{Part B}
\[
	\Exp{X} = 100(0.85) = 85
.\]
\[
	\Var{X} = 100(0.85)(0.15) = 12.75
.\]

\subsection*{Part C}
\[
	\Prob{X \leq 80} = \sum_{n=0}^{80} \binom{100}{n} (0.85)^n (0.15)^{100-n} \approx 0.1065
.\]

\subsection*{Part D}
\[
	\Prob{X \leq 80} = 1 - \Prob{X > 80} = 1 - \sum_{n=81}^{100} \binom{100}{n} (0.85)^n (0.15)^{100-n}
.\]

\subsection*{Part E}
\[
	\Prob{X \leq 80}^2 = (0.1065)^2 = 0.01134
.\]

\subsection*{Part F}
No, since the probability of success will change between succesive trials since it improves.

\section*{Problem 4}
\subsection*{Part A}
\[
	\mathcal{R}_X = \mathbb{N}
.\]

\subsection*{Part B}
\[
	\Exp{X} = \frac{1}{0.85} \approx 1.18
.\]

\subsection*{Part C}
\[
	\Prob{X > 2} = 0.003375
.\]

\subsection*{Part D}
\[
	2\cdot\Exp{2X} = 4\cdot \frac{1}{0.85} \approx 9.44
.\]

\section*{Problem 5}
\subsection*{Part A}
The poisson distribution can be used. Therefore
\[
	\Exp{X} = 100
.\]
\[
	\Var{X} = 100
.\]
\[
	\sigma = \sqrt{100} = 10
.\]

\subsection*{Part B}
\[
	\Prob{X = 100} = \frac{100^{100}}{100!} e^{-100} \approx 0.0399
.\]

\subsection*{Part C}
\[
	\Prob{X \leq 100} = 0.5266
.\]

\subsection*{Part D}
\[
	\Exp{3X} = 3\cdot 100 = 300
.\]
\[
	\Var{3X} = 9\cdot 100 = 900
.\]
\[
	\sigma = \sqrt{900} = 30
.\]

\subsection*{Part E}
\[
	\Prob{X = 100}\cdot (0.2)^{100} = 0.0399 (0.2)^{100} 
.\]

\section*{Problem 6}
\subsection*{Part A}
\begin{align*}
	\int_0^{10} cx^2 \dd x &= 1 \\
	\frac{cx^3}{3} \eval_{0}^{10} &= 1 \\
	\frac{1000c}{3} &= 1 \implies \boxed{c = \frac{3}{1000}}
.\end{align*}

\subsection*{Part B}
\[
	\Prob{X = 5} = \Prob{X = 5\text{ or }X=9} = 0
.\]

\subsection*{Part C}
\[
	\Prob{X > 5} = \int_5^{10} cx^2 \dd x = \frac{3}{1000} \cdot \frac{875}{3} = \frac{7}{8}
.\]

\subsection*{Part D}
\begin{align*}
	F(a) &= \int_0^a cx^2 \dd x \\
	&= \frac{3}{1000} \frac{x^3}{3} \eval_0^a \\
	&= \frac{a^3}{1000}
.\end{align*}

\end{document}
