\documentclass[12pt]{extarticle}

% Document Layout and Font
\usepackage{subfiles}
\usepackage[margin=2cm, headheight=15pt]{geometry}
\usepackage{fancyhdr}
\usepackage{enumitem}	
\usepackage{wrapfig}
\usepackage{multicol}
\usepackage{caption, subcaption}

\usepackage[p,osf]{scholax}

\renewcommand*\contentsname{Table of Contents}
\renewcommand{\headrulewidth}{0pt}
\pagestyle{fancy}
\fancyhf{}
\fancyfoot[R]{$\thepage$}
\setlength{\parindent}{0cm}
\setlength{\headheight}{17pt}
\hfuzz=9pt

% Utility Management
\usepackage{color}
\usepackage{colortbl}
\usepackage{xcolor}
\usepackage{xpatch}
\usepackage{xparse}

\definecolor{links}{HTML}{1c73a5}
\definecolor{bar}{HTML}{584AA8}

% Math Packages
\usepackage{mathtools, amsmath, amsthm, thmtools, amssymb, physics}
\usepackage[scaled=1.075,ncf,vvarbb]{newtxmath}

\newcommand\B{\mathbb{B}}
\newcommand\C{\mathbb{C}}
\newcommand\R{\mathbb{R}}
\newcommand\Q{\mathbb{Q}}
\newcommand\N{\mathbb{N}}
\newcommand\Z{\mathbb{Z}}

\newcommand\Prob[1]{\mathbb{P}\qty(#1)}
\newcommand\Var[1]{\text{Var}\qty(#1)}
\newcommand\Exp[1]{\mathbb{E}\qty[#1]}
\newcommand\ball[1]{\B\qty(#1)}
\newcommand\res[1]{\underset{#1}{\operatorname{Res}}\;}
\renewcommand\pv{\mathrm{p.v.}}

\newcommand\conj[1]{\overline{#1}}
\DeclareMathOperator{\Arg}{Arg}
\DeclareMathOperator{\Log}{Log}
\DeclareMathOperator{\cis}{cis}

\DeclareMathOperator{\dom}{dom}
\DeclareMathOperator{\spann}{span}
\DeclareMathOperator{\nullity}{nullity}

\newcommand\st{\text{ s.t. }}

% TIKZ
\usepackage{tikz}
\usepackage{pgfplots}
\usetikzlibrary{arrows.meta}
\usetikzlibrary{math}
\usetikzlibrary{cd}
\usetikzlibrary{patterns}
\usetikzlibrary{decorations.markings}
\usetikzlibrary{calc}

% Boxes and Theorems
\usepackage[most]{tcolorbox}
\tcbuselibrary{skins}
\tcbuselibrary{breakable}
\tcbuselibrary{theorems}

\newtheoremstyle{default}{0pt}{0pt}{}{}{\bfseries}{\normalfont.}{0.5em}{}
\theoremstyle{default}

\renewcommand*{\proofname}{\textit{\textbf{Proof.}}}
\renewcommand*{\qedsymbol}{$\blacksquare$}
\tcolorboxenvironment{proof}{
	breakable,
	coltitle = black,
	colback = white,
	frame hidden,
	boxrule = 0pt,
	boxsep = 0pt,
	borderline west={3pt}{0pt}{bar},
	sharp corners = all,
	enhanced,
}

\newtheorem{theorem}{Theorem}[section]{\bfseries}{}
\tcolorboxenvironment{theorem}{
	breakable,
	enhanced,
	boxrule = 0pt,
	frame hidden,
	coltitle = black,
	colback = blue!7,
	left = 0.5em,
	sharp corners = all,
}

\newtheorem{corollary}{Corollary}[section]{\bfseries}{}
\tcolorboxenvironment{corollary}{
	breakable,
	enhanced,
	boxrule = 0pt,
	frame hidden,
	coltitle = black,
	colback = white!0,
	left = 0.5em,
	sharp corners = all,
}

\newtheorem{lemma}{Lemma}[section]{\bfseries}{}
\tcolorboxenvironment{lemma}{
	breakable,
	enhanced,
	boxrule = 0pt,
	frame hidden,
	coltitle = black,
	colback = green!7,
	left = 0.5em,
	sharp corners = all,
}

\newtheorem{definition}{Definition}[section]{\bfseries}{}
\tcolorboxenvironment{definition}{
	breakable,
	coltitle = black,
	colback = white,
	frame hidden,
	boxsep = 0pt,
	boxrule = 0pt,
	borderline west = {3pt}{0pt}{orange},
	sharp corners = all,
	enhanced,
}

\newtheorem{example}{Example}[section]{\bfseries}{}
\tcolorboxenvironment{example}{
	% title = \textbf{Example},
	% detach title,
	% before upper = {\tcbtitle\quad},
	breakable,
	coltitle = black,
	colback = white,
	frame hidden,
	boxrule = 0pt,
	boxsep = 0pt,
	borderline west={3pt}{0pt}{green!70!black},
	sharp corners = all,
	enhanced,
}

\newtheoremstyle{remark}{0pt}{4pt}{}{}{\bfseries\itshape}{\normalfont.}{0.5em}{}
\theoremstyle{remark}
\newtheorem*{remark}{Remark}


% TColorBoxes
\newtcolorbox{week}{
	colback = black,
	coltext = white,
	fontupper = {\large\bfseries},
	width = 1.2\paperwidth,
	size = fbox,
	halign upper = center,
	center
}

\newcommand{\banner}[2]{
    \pagebreak
    \begin{week}
   		\section*{#1}
    \end{week}
    \addcontentsline{toc}{section}{#1}
    \addtocounter{section}{1}
    \setcounter{subsection}{0}
}

% Hyperref
\usepackage{hyperref}
\hypersetup{
	colorlinks=true,
	linktoc=all,
	linkcolor=links,
	bookmarksopen=true
}

\usepackage{svg}
\svgsetup{inkscapeexe=inkscape, inkscapearea=drawing, inkscapeversion=1}
\svgpath{{figures/}}

\fancyhead[R]{Homework \#6}
\fancyhead[L]{Eli Griffiths}
\renewcommand{\headrulewidth}{1pt}
\setlength\parindent{0pt}

\begin{document}

\section*{Problem 1}
\subsection*{Part 1}
\begin{align*}
	\Prob{X > 5} &= \Prob{\frac{X - 10}{6} > \frac{5 - 10}{6}} \\
	&= \Prob{Z > -\frac{5}{6}} \\
	&= 1 - \Phi\qty(-\frac{5}{6}) \approx 0.79767
.\end{align*}

\subsection*{Part 2}
\begin{align*}
	\Prob{4 < X < 16} &= \Prob{\frac{4-10}{6} < \frac{X - 10}{6} < \frac{16 - 10}{6}} \\
					  &= \Prob{-1 < Z < 1} \\
					  &= \Phi\qty(1) - \Phi(-1) = 2\cdot\Phi(1) - 1 \approx 0.68268
.\end{align*}

\subsection*{Part 3}
\begin{align*}
	\Prob{X < 8} &= \Prob{\frac{X - 10}{6} < \frac{8 - 10}{6}} \\
	&= \Prob{Z < -\frac{1}{3}} \\
	&= \Phi\qty(-\frac{1}{3}) \approx 0.36944
.\end{align*}

\subsection*{Part 4}
\begin{align*}
	\Prob{X < 20} &= \Prob{\frac{X - 10}{6} < \frac{20 - 10}{6}} \\
	&= \Prob{Z < \frac{5}{3}} \\
	&= \Phi\qty(\frac{5}{3}) \approx 0.95221
.\end{align*}

\subsection*{Part 5}
\begin{align*}
	\Prob{X > 16} &= \Prob{\frac{X - 10}{6} > \frac{16 - 10}{6}} \\
	&= \Prob{Z > 1} \\
	&= 1 - \Phi\qty(1) \approx 0.15866
.\end{align*}

\section*{Problem 2}
Assuming that the annual rainfall does not change from year to year and that the rainfall from each year is independent, the probability is going to be
\[
	\Prob{X \leq 50}^{10} = 
	\Prob{\frac{X-40}{4} \leq \frac{50-40}{4}}^{10} =
	\Prob{Z \leq 2.5}^{10} = \Phi(2.5)^{10} \approx 0.93961
.\]

\section*{Problem 3}
Let $X$ denote the salaries of the physicians in thousands of dollars. By the given information,
\begin{align*}
	\Prob{X < 180} = \Prob{X > 320} = 0.25
.\end{align*}
Therefore
\begin{align*}
	\Prob{X < 180} &= 0.25 \\
	\Prob{\frac{X - \mu}{\sigma} < \frac{180 - \mu}{\sigma}} &= 0.25 \\
	\Prob{Z < \frac{180 - \mu}{\sigma}} &= 0.25 \\
	\Phi\qty(\frac{180 - \mu}{\sigma}) &= 0.25 \implies \frac{180 - \mu}{\sigma} = -0.67449
\end{align*}
and
\begin{align*}
	\Prob{X > 320} &= 0.25 \\
	\Prob{\frac{X - \mu}{\sigma} > \frac{320 - \mu}{\sigma}} &= 0.25 \\
	\Prob{Z > \frac{320 - \mu}{\sigma}} &= 0.25 \\
	1 - \Phi\qty(\frac{320 - \mu}{\sigma}) &= 0.25 \implies \frac{320 - \mu}{\sigma} = 0.67449
\end{align*}
Therefore this gives a system of 2 equations
\begin{align*}
	180 - \mu &= -0.67449\cdot\sigma \\
	320 - \mu &= 0.67449\cdot\sigma \\
.\end{align*}
Solving gives $\mu = 250$ and $\sigma = 103.704$.

\subsection*{Part 1}
\begin{align*}
	\Prob{X < 200} &= \Prob{\frac{X - 250}{103.704} < \frac{200 - 250}{103.704}} \\
				   &= \Prob{Z < -0.4821414796} \\
				   &= \Phi{-0.4821414796} \approx 0.31485
.\end{align*}

\subsection*{Part 2}
\begin{align*}
	\Prob{280 < X < 320} &= \Prob{\frac{280 - 250}{103.704} < \frac{X - 250}{103.704} < \frac{320 - 250}{103.704} } \\
						 &= \Prob{0.28928 < Z < 0.67499} \\
						 &= \Phi(0.67499) - \Phi(0.28928) \approx 0.13634
\end{align*}

\section*{Problem 4}
\begin{align*}
	\Prob{X > c} &= 0.1 \\
	\Prob{\frac{X - 12}{2} > \frac{c - 12}{2}} &= 0.1 \\
	\Prob{Z > \frac{c - 12}{2}} &= 0.1 \\
	1 - \Phi(Z > \frac{c - 12}{2}) &= 0.1 \\
	\Phi(Z > \frac{c - 12}{2}) &= 0.9 \\
	\intertext{Using a lookup table for when $\Phi$ is $0.9$ gives $1.2816$, therefore}
	\frac{c - 12}{2} &= 1.2816 \\
	c &= 2(1.2816) + 12 \implies \boxed{c = 14.5632}
\end{align*}

\section*{Problem 5}
\[
	\Prob{X > 2} = 1 - \Prob{X \leq 2} = 1 - \int_0^2 e^{-x} \dd x = 1 - \qty[-e^{-x}]_0^2 = 1 - 1 + e^{-2} = e^{-2}
.\]

\section*{Problem 6}
Since the exponential distribution is memoryless, the probability it will last an additional 8 years will be the same as the probability that a new radio would last 8 years. Therefore the probability is
\[
	\Prob{X > 8} = 1 - \Prob{X < 8} = 1 - \int_0^8 18e^{-18x} \dd x = 1 - \qty[-e^{-18x}]_0^8 = 1 - 1 + e^{-144} = e^{-144}
.\]

\section*{Problem 7}
A quadratic's roots are both real when its discriminant is positive. The discriminant in this case is $16Y^2 - 4(4)(Y+2)$. It follows that
\begin{align*}
	16Y^2 - 16Y - 32 \geq 0 \\
	Y^2 - Y - 2 \geq 0 \\
	(Y + 1) (Y - 2) \geq 0
.\end{align*}
Since $0 < Y < 5$, this only holds when $Y - 2 \geq 0$. Therefore
\begin{align*}
	\Prob{Y - 2 \geq 0} &= \Prob{Y \geq 2} \\
	&= \int_2^5 \frac{1}{5} \dd x = \frac{3}{5}
.\end{align*}

\section*{Problem 8}
\begin{align*}
	&F_Y(t) = \Prob{Y \leq t} = \Prob{\log(X) \leq t}
					= \Prob{X \leq e^t}
					= F_X (e^t) \\
	&\Downarrow \dv{t} \\
	&f_Y (t) = f_X (e^t) \cdot e^t \implies f_Y (t) = e^{t-e^t}, t \in \mathbb{R}
.\end{align*}

\section*{Problem 9}
\subsection*{Part 1}
\begin{align*}
	\Exp{|X - a|} &= \int_0^A \frac{1}{A} \cdot |x - a| \dd x \\
				  &= \frac{1}{A} \cdot \qty[
				  \int_0^a (a - x) \dd x + \int_a^A (x - a) \dd x
				  ] \\
				  &= \frac{1}{A} \cdot \qty[
				  \eval{ax - \frac{x^2}{2}}_0^a + \frac{x^2}{2} - ax \eval_a^A
				  ] \\
				  &= \frac{1}{A} \cdot \qty[
				  \frac{a^2}{2} + \frac{A^2}{2} - Aa - \frac{a^2}{2} + a^2
				  ] \\
				  &= \frac{1}{A} \cdot \qty[
				  \frac{A^2}{2} - Aa + a^2
				  ] \\
				  &= \frac{a^2}{A} - a + \frac{A}{2}
.\end{align*}
Therefore by minimizing with respect to $a$,
\begin{align*}
	\dv{a} \Exp{|X-a|} &= 0 \\
	\dv{a} \qty(\frac{a^2}{A} - a + \frac{A}{2}) &= 0 \\
	\frac{2a}{A} - 1 &= 0 \implies a = \frac{A}{2}
.\end{align*}
The concavity at $\frac{a}{2}$ is positive, therefore $\frac{a}{2}$ minimizes the expected distance from the fire.

\subsection*{Part 2}
\begin{align*}
	\Exp{|X - a|} &= \lambda \qty[\int_0^\infty |x - a| \cdot e^{-\lambda x} \dd x ]\\
				  &= \lambda \qty[\int_0^a (a-x) e^{-\lambda x} \dd x + \int_a^\infty (x - a) e^{-\lambda x} \dd x ]\\
				  &= \lambda \qty[\int_0^a ae^{-\lambda x} - xe^{-\lambda x} \dd x + \int_a^\infty xe^{-\lambda x} - ae^{-\lambda x}  ]\\
				  &= \lambda \qty[a\int_0^a e^{-\lambda x} \dd x - \int_0^a xe^{-\lambda x} \dd x + \int_0^\infty xe^{-\lambda x} \dd x - a \int_a^\infty e^{-\lambda x} \dd x ]\\
				  &= \lambda \qty[a \qty[-\frac{e^{-\lambda x}}{\lambda}]_0^a - a \qty[-\frac{e^{-\lambda x}}{\lambda}]_a^\infty - \int_0^a xe^{-\lambda x} \dd x + \int_0^\infty xe^{-\lambda x} \dd x ]\\
				  &= \lambda \qty[\frac{a}{\lambda} \qty[-e^{-\lambda a} + 1 + 0 - e^{-\lambda a}] - \int_0^a xe^{-\lambda x} \dd x + \int_a^\infty xe^{-\lambda x} \dd x ]\\
				  &= \lambda \qty[\frac{a}{\lambda} \qty[1-2e^{-\lambda a}] - \qty(-\frac{xe^{-\lambda x}}{\lambda} \eval_0^a + \int_0^a \frac{e^{-\lambda x}}{\lambda} \dd x) + \qty(-\frac{xe^{-\lambda x}}{\lambda} \eval_a^\infty + \int_a^\infty \frac{e^{-\lambda x}}{\lambda} \dd x) ]\\
				  &= \lambda \qty[\frac{a}{\lambda} \qty[1-2e^{-\lambda a}] - \qty(-\frac{xe^{-\lambda x}}{\lambda} \eval_0^a + \qty(-\frac{e^{-\lambda x}}{\lambda^2} \eval_0^a)) + \qty(-\frac{xe^{-\lambda x}}{\lambda} \eval_a^\infty + \qty(-\frac{e^{-\lambda x}}{\lambda^2} \eval_a^\infty)) ]\\
				  &= \lambda \qty[\frac{a}{\lambda} \qty[1-2e^{-\lambda a}] - \qty(-\frac{ae^{-\lambda a}}{\lambda} - \frac{e^{-\lambda a}}{\lambda^2} + \frac{1}{\lambda^2}) + \qty(0 + \frac{ae^{-\lambda a}}{\lambda} + \frac{e^{-\lambda a}}{\lambda^2} ) ] \\
				  &= \lambda \qty[\frac{a}{\lambda} \qty[1-2e^{-\lambda a}] + \frac{ae^{-\lambda a}}{\lambda} + \frac{e^{-\lambda a}}{\lambda^2} - \frac{1}{\lambda^2} + \frac{ae^{-\lambda a}}{\lambda} + \frac{e^{-\lambda a}}{\lambda^2} ] \\
				  &= \lambda \qty[\frac{a}{\lambda} \qty[1-2e^{-\lambda a}] + \frac{2ae^{-\lambda a}}{\lambda} + \frac{2e^{-\lambda a}}{\lambda^2} - \frac{1}{\lambda^2} ] \\
				  &= a\qty(1-2e^{-\lambda a}) + 2ae^{-\lambda a} + \frac{2e^{-\lambda a}}{\lambda} - \frac{1}{\lambda} \\
				  &= a-2ae^{-\lambda a} + 2ae^{-\lambda a} + \frac{2e^{-\lambda a}}{\lambda} - \frac{1}{\lambda} \\
				  &= a + \frac{2e^{-\lambda a}}{\lambda} - \frac{1}{\lambda}
.\end{align*}
Therefore by minimizing with respect to $a$,
\begin{align*}
	\dv{a} \qty(a + \frac{2e^{-\lambda a}}{\lambda} - \frac{1}{\lambda}) &= 0 \\
	1 - 2e^{-\lambda a} &= 0 \\
	e^{-\lambda a} &= \frac{1}{2} \\
	\lambda a &= \log(2) \implies \boxed{a = \frac{\log(2)}{\lambda}}
.\end{align*}
The concavity is always positive so $\frac{\log(2)}{\lambda}$ minimizes the expected distance from the fire.

\section*{Problem 10}
Let $Y \sim \text{Binom}\qty(1000, \frac{1}{6})$ be the number of $6$'s that appear in 1000 rolls. Since $np$ and $nq$ are large, $Y$ can be approximated by a normal random variable $X \sim N(np, \sqrt{npq}) = N\qty(\frac{500}{3}, \frac{25\sqrt{2}}{3})$. Let $\mu = \frac{500}{3}$ and $\sigma = \frac{25\sqrt{2}}{3}$. Then
\begin{align*}
	\Prob{150 \leq Y \leq 200} &\approx \Prob{150 \leq X \leq 200} \\
							   &= \Prob{\frac{150 - \mu}{\sigma} \leq \frac{X - \mu}{\sigma} \leq \frac{200 - \mu}{\sigma}} \\
							   &= \Prob{-\sqrt{2} \leq Z \leq 2\sqrt{2}} \\
							   &= \Phi\qty(-\sqrt{2}) - \Phi\qty(2\sqrt{2}) \approx 0.919
.\end{align*}

Consider now if 6 has appeared 200 times. Then the number of times 5 appears in the remaining 800 rolls can be represented by the random variable $Y \sim \text{Binom}\qty(800, \frac{1}{5})$ since there are 800 trials, and since 6 will not appear again, 5 has a $\frac{1}{5}$ chance of appearing. Since $np$ and $nq$ are large, $Y$ can be approximated by $X \sim N\qty(np, \sqrt{npq}) = N\qty(160, 8\sqrt{2})$. Therefore
\begin{align*}
	\Prob{Y < 150} &\approx \Prob{X < 150} \\
				   &= \Prob{\frac{X - 160}{8\sqrt{2}} < \frac{150 - 160}{8\sqrt{2}}} \\
				   &= \Prob{Z < -\frac{5}{4 \sqrt{2}}} \\
				   &= \Phi\qty(-\frac{5}{4 \sqrt{2}}) \approx 0.18838
.\end{align*}

\end{document}
