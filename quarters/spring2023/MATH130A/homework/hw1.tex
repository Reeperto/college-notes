\documentclass[12pt]{extarticle}

\usepackage{multicol}

% Document Layout and Font
\usepackage{subfiles}
\usepackage[margin=2cm, headheight=15pt]{geometry}
\usepackage{fancyhdr}
\usepackage{enumitem}	
\usepackage{wrapfig}
\usepackage{multicol}
\usepackage{caption, subcaption}

\usepackage[p,osf]{scholax}

\renewcommand*\contentsname{Table of Contents}
\renewcommand{\headrulewidth}{0pt}
\pagestyle{fancy}
\fancyhf{}
\fancyfoot[R]{$\thepage$}
\setlength{\parindent}{0cm}
\setlength{\headheight}{17pt}
\hfuzz=9pt

% Utility Management
\usepackage{color}
\usepackage{colortbl}
\usepackage{xcolor}
\usepackage{xpatch}
\usepackage{xparse}

\definecolor{links}{HTML}{1c73a5}
\definecolor{bar}{HTML}{584AA8}

% Math Packages
\usepackage{mathtools, amsmath, amsthm, thmtools, amssymb, physics}
\usepackage[scaled=1.075,ncf,vvarbb]{newtxmath}

\newcommand\B{\mathbb{B}}
\newcommand\C{\mathbb{C}}
\newcommand\R{\mathbb{R}}
\newcommand\Q{\mathbb{Q}}
\newcommand\N{\mathbb{N}}
\newcommand\Z{\mathbb{Z}}

\newcommand\Prob[1]{\mathbb{P}\qty(#1)}
\newcommand\Var[1]{\text{Var}\qty(#1)}
\newcommand\Exp[1]{\mathbb{E}\qty[#1]}
\newcommand\ball[1]{\B\qty(#1)}
\newcommand\res[1]{\underset{#1}{\operatorname{Res}}\;}
\renewcommand\pv{\mathrm{p.v.}}

\newcommand\conj[1]{\overline{#1}}
\DeclareMathOperator{\Arg}{Arg}
\DeclareMathOperator{\Log}{Log}
\DeclareMathOperator{\cis}{cis}

\DeclareMathOperator{\dom}{dom}
\DeclareMathOperator{\spann}{span}
\DeclareMathOperator{\nullity}{nullity}

\newcommand\st{\text{ s.t. }}

% TIKZ
\usepackage{tikz}
\usepackage{pgfplots}
\usetikzlibrary{arrows.meta}
\usetikzlibrary{math}
\usetikzlibrary{cd}
\usetikzlibrary{patterns}
\usetikzlibrary{decorations.markings}
\usetikzlibrary{calc}

% Boxes and Theorems
\usepackage[most]{tcolorbox}
\tcbuselibrary{skins}
\tcbuselibrary{breakable}
\tcbuselibrary{theorems}

\newtheoremstyle{default}{0pt}{0pt}{}{}{\bfseries}{\normalfont.}{0.5em}{}
\theoremstyle{default}

\renewcommand*{\proofname}{\textit{\textbf{Proof.}}}
\renewcommand*{\qedsymbol}{$\blacksquare$}
\tcolorboxenvironment{proof}{
	breakable,
	coltitle = black,
	colback = white,
	frame hidden,
	boxrule = 0pt,
	boxsep = 0pt,
	borderline west={3pt}{0pt}{bar},
	sharp corners = all,
	enhanced,
}

\newtheorem{theorem}{Theorem}[section]{\bfseries}{}
\tcolorboxenvironment{theorem}{
	breakable,
	enhanced,
	boxrule = 0pt,
	frame hidden,
	coltitle = black,
	colback = blue!7,
	left = 0.5em,
	sharp corners = all,
}

\newtheorem{corollary}{Corollary}[section]{\bfseries}{}
\tcolorboxenvironment{corollary}{
	breakable,
	enhanced,
	boxrule = 0pt,
	frame hidden,
	coltitle = black,
	colback = white!0,
	left = 0.5em,
	sharp corners = all,
}

\newtheorem{lemma}{Lemma}[section]{\bfseries}{}
\tcolorboxenvironment{lemma}{
	breakable,
	enhanced,
	boxrule = 0pt,
	frame hidden,
	coltitle = black,
	colback = green!7,
	left = 0.5em,
	sharp corners = all,
}

\newtheorem{definition}{Definition}[section]{\bfseries}{}
\tcolorboxenvironment{definition}{
	breakable,
	coltitle = black,
	colback = white,
	frame hidden,
	boxsep = 0pt,
	boxrule = 0pt,
	borderline west = {3pt}{0pt}{orange},
	sharp corners = all,
	enhanced,
}

\newtheorem{example}{Example}[section]{\bfseries}{}
\tcolorboxenvironment{example}{
	% title = \textbf{Example},
	% detach title,
	% before upper = {\tcbtitle\quad},
	breakable,
	coltitle = black,
	colback = white,
	frame hidden,
	boxrule = 0pt,
	boxsep = 0pt,
	borderline west={3pt}{0pt}{green!70!black},
	sharp corners = all,
	enhanced,
}

\newtheoremstyle{remark}{0pt}{4pt}{}{}{\bfseries\itshape}{\normalfont.}{0.5em}{}
\theoremstyle{remark}
\newtheorem*{remark}{Remark}


% TColorBoxes
\newtcolorbox{week}{
	colback = black,
	coltext = white,
	fontupper = {\large\bfseries},
	width = 1.2\paperwidth,
	size = fbox,
	halign upper = center,
	center
}

\newcommand{\banner}[2]{
    \pagebreak
    \begin{week}
   		\section*{#1}
    \end{week}
    \addcontentsline{toc}{section}{#1}
    \addtocounter{section}{1}
    \setcounter{subsection}{0}
}

% Hyperref
\usepackage{hyperref}
\hypersetup{
	colorlinks=true,
	linktoc=all,
	linkcolor=links,
	bookmarksopen=true
}


\fancyhead[R]{\textbf{Math 130A: Homework \#1}}
\fancyhead[L]{Eli Griffiths}
\renewcommand{\headrulewidth}{1pt}
\setlength\parindent{0pt}

\usetikzlibrary{arrows.meta}

\begin{document}

\section*{Problem 1}
\subsection*{Part A}

There are $26^2$ ways to choose 2 letters and $10^5$ ways to choose 5 numbers. Therefore there are $26^2 \cdot 10^5 = 67,600,000$ different 7-place license plate numbers.

\subsection*{Part B}
There are $26$ ways to choose the first letter and $25$ ways to choose the next letter. With a similar argument there are $10\cdot 9\cdot 8\cdot 7\cdot 6 = 30,240$ ways to choose 5 unique numbers. Therefore there are $26\cdot 25 \cdot 30,240 = 19,656,000$ unique 7 place license plates.

\section*{Problem 2}
There are $\binom{10}{5}$ ways to choose $5$ men and $\binom{12}{5}$ ways to choose $5$ women. There are $5!$ ways to pair up $5$ men with $5$ women, meaning that there are $5! \cdot \binom{10}{5} \cdot \binom{12}{5} = 23,950,080$ distinct possible results.

\section*{Problem 3}
The first gift can be given to $10$ children, the second gift can be given to $9$ children and so forth. Therefore the number of distinct results possible is $10\cdot 9 \cdot 8\cdot 7 \cdot 6\cdot 5\cdot 4 = 604,800$. 

\section*{Problem 4}
\[
	S = \qty{
		(R, R),
		(R, G),
		(R, B),
		(G, R),
		(G, G),
		(G, B),
		(B, R),
		(B, G),
		(B, B)
	}
.\]

The probability associated with each point in the sample space is $\frac{1}{9} \approx 11.11\%$

\section*{Problem 5}
\[
	S = \qty{
		(R, G),
		(R, B),
		(G, R),
		(G, B),
		(B, R),
		(B, G)
	}
.\]

\section*{Problem 6}
The sample space is infinite.

\def\Heads{\mathbf{H}}
\def\Tails{\mathbf{T}}
\begin{align*}
	S = \{
		&(\Heads, \Heads), \\
		&(\Tails, \Heads, \Heads), \\
		&(\Tails, \Tails, \Heads, \Heads),
		(\Heads, \Tails, \Heads, \Heads), \\
		&(\Tails, \Tails, \Tails, \Heads, \Heads),
		(\Heads, \Tails, \Tails, \Heads, \Heads),
		(\Tails, \Heads, \Tails, \Heads, \Heads) \ldots \}
\end{align*}

There are $2^4 = 16$ possible outcomes from tossing a coin 4 times and there are $2$ outcomes in the sample space that are $4$ tosses, meaning the probability of the coin being tossed exactly four times if $\frac{2}{16} = \frac{1}{8} = 12.5\%$.

\section*{Problem 7}
The number of ways to pick $5$ people from a group of $15$ is $\binom{15}{5} = 3,003$. The number of ways to pick $3$ and $2$ women is $\binom{6}{3} \cdot \binom{9}{2} = 720$. Therefore the probability of the committee consisteng of $3$ men and $2$ women is $\frac{720}{3,003} \approx 23.98\%$.

\section*{Problem 8}
Consider the set of $n$ balls. The group of marked balls has $1$ ball in it (the marked ball) and the group of unmarked balls has $n - 1$ balls in it. The total number of ways to draw $k$ balls from $n$ balls is $\binom{n}{k}$. The number of ways to draw $k$ balls with the marked ball in it is $\binom{1}{1} \cdot \binom{n-1}{k-1}$. The ways to choose $1$ marked ball from a set of $1$ marked ball is $\binom{1}{1} = 1$ and the ways to choose the remaining $k-1$ balls is $\binom{n-1}{k-1}$. Therefore the probability of drawing the marked ball is
\[
	\frac{\binom{n-1}{k-1}}{\binom{n}{k}} = \frac{k}{n}
.\]

\section*{Problem 9}
\subsection*{Part 1}
Since $A$ and $B$ are mutually exclusive, $P(AB) = 0$. Therefore 
\begin{align*}
	P(A \cup B) &= P(A) + P(B) + P(AB) \\ 
							&= P(A) + P(B) \\ 
							&= 80\%
.\end{align*}

\subsection*{Part 2}
\begin{align*}
	P(AB^\complement) &= P(A) - P(AB) \\
										&= P(A) \\
										&= 30\%
.\end{align*}

\subsection*{Part 3}
Since $A$ and $B$ are mutually exclusive, $P(AB) = 0$.

\section*{Problem 10}
Firstly, the number of ways $5$ cards can be dealt from a deck is $\binom{52}{5} = 2,598,960$. A straight hand will be comprised of picking $5$ cards with $4$ possible suits per card. Therefore given a straight range the number of possibilities is $4^5 - 4$. The substraction of $4$ accounts for the $4$ hands where all the cards are the same suit which would make the hand a straight flush and not a straight. There are $10$ different possible ranges for a straight meaning that there are $(4^5 - 4)\cdot 10 = 10,200$ possible straight hands. Therefore the probability of being dealt a straight hand from a deck is $\frac{10,200}{2598960} \approx 0.392\%$.


\end{document}
