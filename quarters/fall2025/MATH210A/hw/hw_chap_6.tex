\documentclass[hw_all.tex]{subfiles}

\begin{document}

\section*{Exercise 6.12}

\begin{proof}
    \begin{enumerate}
        \item
            By the definition of the outer measure, for any $\eps > 0$ there exists a cover $A_n \in \mathcal{B}^\N$ of $E$ such that
            \[
                \sum_{n \in \N} \mu_0(A_n) \leq \mu^*(E) + \eps
            .\]
            Clearly $A_n$ is a cover for $A$, meaning that
            \[
                \mu^*(A) \leq \sum_{n \in \N} \mu_0(A_n) \leq \mu^*(E) + \eps
            .\]
            Since each $A_n \in \mathcal{B}$ and $A$ is a countable union, it follows that $A \in \mathcal{B}_{\sigma}$.

        \item
            Suppose $E$ is $\mu^*$-measurable and $\mu^*(E) < \infty$. Using part $(a)$, let $B_n \in \mathcal{B}_{\sigma}^\N$ such that $E \subset B_n$ with $\mu^*(B_n) \leq \mu^*(E) + \frac{1}{n}$. Take then $B = \bigcap_{n \in \N} B_n$. Note that both $E \subset B \subset B_n$ and $B \in \mathcal{B}_{\sigma \delta}$. Thus
            \[
                \mu^*(E) \leq \mu^*(B) \leq \mu^*(E) + \frac{1}{n}
            \]
            which in the limit $n \to \infty$ gives $\mu^*(B) = \mu^*(E)$. Since $E$ is $u^*$-measurable, it follows that
            \[
                \mu^*(E) = \mu^*(E \cap B) + \mu^*(E \setminus B) \implies \mu^*(E \setminus B) = \mu^*(E) - \mu^*(B)
            .\]
            But since $\mu^*(B) = \mu^*(E)$, it follows that $\mu^*(E \setminus B) = 0$.

            Suppose then some $E \subset B \in \mathcal{B}_{\sigma \delta}$ where $\mu^*(E \setminus B) = 0$. Since $\mu^*(E \setminus B) = 0$, clearly the Caratheodory criterion holds for $E \setminus B$ and thus it is $\mu^*$-measurable. Since the set of $\mu^*$-measurable sets is a $\sigma$-algebra containing $\mathcal{B}$, it follows that $B$ is $\mu^*$-measurable and thus $(E \setminus B) \cup B = E$ is $\mu^*$-measurable.

        \item 
            Note that the reverse direction did not require $\mu^*(E) < \infty$, so we only consider the forward direction. Suppose $X$ is $\sigma$-finite. Then there exists $X_k \in \mathcal{B}^\N$ such that $\mu^*(X_k) < \infty$ and $X = \bigcup_{k \in \N} X_k$. Let $E_k = E \cap X_k$ and note that from part $(a)$ that there exist $E_k \subset O_{k,n} \in \mathcal{B}_{\sigma}^\N$ with $\mu^*(O_{k,n}) \leq \mu^*(E_k) + \frac{1}{n \cdot 2^k}$. 

            Take then $B_n = \bigcup_{k \in \N} O_{k,n}$ and note that $B_n \in \mathcal{B}_{\sigma}$ and
            \[
                B_n \setminus E \subset \bigcup_{k \in \N} (O_{k,n} \setminus E_k)
            .\]
            Therefore by subadditivity it follows
            \[
                \mu^*(B_n \setminus E) \leq \sum_{k \in \N} \mu^*(O_{k,n} \setminus E) \leq \sum_{k \in \N} \frac{1}{n\cdot 2^k} = \frac{1}{n}
            .\]
            The exact same argument in $(b)$ then works from here.
    \end{enumerate}
\end{proof}

\section*{Exercise 6.14}

\begin{proof}
    \begin{enumerate}
        \item
            Let $A_n \in \mathcal{A}^*$ be a cover of $E$. Since $\mu^*$ is an outer measure, it is subadditive and monotonic meaning
            \[
                \mu^*(E) \leq \mu^*\qty(\bigcup_{n \in \N} A_n) \leq \sum_{n \in \N} \mu^*(A_n)
            .\]
            Since $\mu^+$ is the infimum over all such sums, it follows that $\mu^*(E) \leq \mu^+(E)$.

            Suppose then there exists an $A \supset E$ with $\mu^*(A) = \mu^*(E)$. Since $A$ covers $E$, by definition $\mu^+(E) \leq \mu^*(A) = \mu^*(E)$. Combined with first inequality gives $\mu^+(E) = \mu^*(E)$.

            Suppose then that $\mu^*(E) = \mu^+(E)$.
            \begin{itemize}
                \item
                    Suppose $\mu^*(E) < \infty$. By the definition of $\mu^+$ there then exists $A_n \in \mathcal{A}^*$ such that $E \subset A_n$ and
                    \[
                        \mu^*(A_n) \leq \mu^+(E) + \frac{1}{n}
                    .\]
                    Take then $A = \bigcap_{n \in \N} A_n$ and note that both $A \in \mathcal{A}^*$ and $E \subset A$. In the limit as $n \to \infty$ it then follows
                    \[
                        \mu^*(A) \leq \mu^*(A_n) \leq \mu^+(E) = \mu^*(E)
                    .\]
                    Therefore $\mu^*(E) = \mu^*(A)$.

                \item
                    Suppose $\mu^*(E) = \infty$. Since $\mu^*(E) \leq \mu^+(E)$, it follows $\mu^+(E) = \infty$. Take $A = X$. Clearly then $A \in \mathcal{A}^*$ and $E \subset A$. Thus by monotonicity $\infty = \mu^*(E) \leq \mu^*(A)$ meaning $\mu^*(A) = \mu^*(E) = \infty$
            \end{itemize}

        \item
            If $\mu^*$ is induced from a pre-measure over some algebra $\mathcal{B}$, then from problem 1 part $(a)$, for any $\eps > 0$ there is some $A \in \mathcal{A}_{\sigma}$ such that $E \subset A$ and $\mu^*(A) \leq \mu^*(E) + \eps$. By Caratheodory's, every set in $\mathcal{B}$ is measurable and thus in $\mathcal{A}^*$. Since $\mathcal{A}^*$ is a $\sigma$-algebra, countable unions are contained in it and thus $\mathcal{B}_{\sigma} \subset \mathcal{A}^*$. Since then $A \in \mathcal{A}^*$ and covers $E$, it follows $\mu^+(E) \leq \mu^*(A) \leq \mu^*(E) + \eps$. Letting $\eps \to 0$ thus gives $\mu^+(E) \leq \mu^*(E)$, which combined with the initial result gives equality.

        \item
            Define $\mu^*$ on $2^X$ where
            \begin{align*}
                \mu^*(\varnothing) &= 0 \\ 
                \mu^*(\qty{0}) &= 1 \\ 
                \mu^*(\qty{1}) &= 1 \\ 
                \mu^*(\qty{0,1}) &= 1.5
            \end{align*}
            Clearly it is monotonic an subadditive, and $\mu^*(\varnothing) = 0$. Therefore $\mu^*$ is an outer measure on $X$. Note that
            \[
                \mu^*({0,1}) = 1.5 \neq 2 = 1 + 1 = \mu^*(\qty{0,1} \cap \qty{0}) + \mu^*(\qty{0,1} \cap \qty{1})
            .\]
            Thus $\qty{0}$ and $\qty{1}$ are not measurable, leaving just $\varnothing$ and $\qty{0,1}$ which are trivially measurable. Note that the only measurable set containing $\qty{0}$ is $\qty{0,1}$ and so $\mu^+(\qty{0}) = \mu^*(\qty{0,1}) = 1.5$, but
            \[
                \mu^*(\qty{0}) = 1 \neq 1.5 = \mu^+(\qty{0})
            .\]
    \end{enumerate}
\end{proof}

\section*{Exercise 6.18}

\begin{proof}
    \begin{enumerate}
        \item
            First note that $(a,b] \cap \Q$ forms an elementary family.
            \begin{itemize}
                \item[iii)]
                    Note that $(a,b]^c = (b, \infty]$ and thus complements are maintained in $\Q$.

                \item[ii)]
                    Note that $(a_1, b_1] \cap (a_2, b_2] = (\max(a_1, a_2), \min(b_1, b_2)]$ and thus intersections are maintained in $\Q$

                \item[i)]
                    Note that $(a, b] \cap (a, b]^c = \varnothing$ and thus the empty set is in family
            \end{itemize}
            Since they form an elementary family, then $\mathcal{B}$ is an algebra since is the finite union of sets in the family.

        \item 
            Note that any singleton $\qty{q} \subset \Q$ can be achieved by
            \[
                \qty{q} = \bigcap_{n \in \N} \left(q-\frac{1}{n}, q\right]
            .\]
            Since all the inner sets are contained in $\mathcal{B}$, it follows that $\qty{q} \in \sigma(\mathcal{B})$ for all $q \in \Q$. But then that means that any $E \subset \Q$ can be made in $\sigma(\mathcal{B})$ since $\Q$ is countable. Thus $2^\Q \subset \sigma(\mathcal{B})$. By definition $\sigma(\mathcal{B}) \subset 2^\Q$, hence equality.

        \item 
            Consider the following measures on $2^\Q$:
            \[
                \nu_1(E) = |E| \hspace{2cm} \nu_2(E) = \begin{cases}
                    0 & E = \varnothing \\
                    \infty & E = \varnothing \\
                \end{cases}
            .\]
            When restricted to $\mathcal{B}$, $\nu_1(E) = \infty$ since any set in $\mathcal{B}$ has countably many elements. Similarly, $\varnothing \notin \mathcal{B}$ so $\nu_2(E) = \infty$ for all $E$ as well. Therefore both are equal to the premeasure on $\mathcal{B}$. However the measures are not the same on say $\qty{q} \in 2^\Q$ since $\nu_1(\qty{q}) = 1$ and $\nu_2(\qty{q}) = \infty$.
    \end{enumerate}
\end{proof}

\section*{Exercise 6.19}

\begin{proof}
    \begin{enumerate}
        \item
            Let $D = A \Delta B$. Note that both $D \in \mathcal{A}$ and $D \cap E = \varnothing$ since
            \begin{itemize}
                \item If $x \in A \cap E$, then $x \in B$ meaning $x \notin A \setminus B$
                \item If $x \in B \cap E$, then $x \in A$ meaning $x \notin B \setminus A$
                \item If $x \in E$ but $x \notin A$ or $x \notin B$, then $x \notin D$
            \end{itemize}
            Thus $E \subset D^c$, meaning $\mu(E) \leq \mu(D^c)$. Since $\mu$ is a finite measure, it follows that $\mu(D^c) = \mu(X) - \mu(D)$. Since $\mu(X) = \mu^*(X) = \mu^*(E) = \mu(E)$, then
            \[
                \mu(X) \leq \mu(X) - \mu(D) \implies \mu(D) \leq 0
            .\]
            Therefore $\mu(D) = 0$. Since $A \setminus B \subset D$ and $B \setminus A \subset D$, as well as $A = (A \cap B) \cup (A \setminus B)$ and $B = (B \cap A) \cup (B \setminus A)$, it follows by monotonicity and additivity
            \begin{align*}
                \mu(A) &= \mu(A \cap B) + \mu(A \setminus B) \\
                       &= \mu(A \cap B) + 0                  \\
                       &= \mu(B \cap A) + \mu(B \setminus A) \\
                       &= \mu(B)
            \end{align*}

        \item
            It has been shown in prior exercises that a $\sigma$-algebra restricted to some set is still a $\sigma$-algebra, so all that needs to be shown is that $\nu$ is a measure on $\mathcal{A}_E$. Note that $\nu$ is well defined by part $(a)$. Let $B_n \in \mathcal{A}_E^\N$ be pairwise disjoint. Then there exists $A_n \in \mathcal{A}$ such that $B_n = A_n \cap E$. Let then
            \[
                C_n = A_n \setminus \bigcup_{k=1}^{n - 1} A_k
            \]
            and note that all the $C_n$ are pairwise disjoint. Also note
            \[
                C_n \cap E = (A_n \cap E) \setminus \bigcup_{k=1}^{n-1} (A_k \cap E) = B_n \setminus \bigcup_{k=1}^{n-1} B_k
            .\]
            Since the $B_k$ are disjoint, it follows then that $C_n \cap E = B_n$. Therefore
            \[
                \bigcup_{n \in \N} B_n = \bigcup_{n\in \N} (C_n \cap E) = E \cap \bigcup_{n \in \N} C_n
            \]
            meaning
            \[
                \nu\qty(\bigcup_{n \in \N} B_n) = \mu\qty(\bigcup_{n\in \N} C_n) = \sum_{n \in \N} \mu(C_n) = \sum_{n \in \N} \nu(B_n)
            .\]
            Thus $\nu$ is countably additive and hence a measure.
    \end{enumerate}
\end{proof}

\section*{Exercise 6.29}

\begin{proof}
    Since each $\mathcal{A}_i$ generates themselves ($\sigma(\mathcal{A}_i) = \mathcal{A}_i$), then by proposition 6.22 it follows that
    \[
        \mathcal{A}_1 \otimes \mathcal{A}_2 \otimes \mathcal{A}_3 = \sigma\qty(\qty{A_1 \times A_2 \times A_3 : A_i \in \mathcal{A}_i})
    .\]
    But note that $\qty{A_1 \times A_2 : A_i \in \mathcal{A}_i}$ generates $\mathcal{A}_1 \otimes \mathcal{A}_2$, so again by proposition 6.22
    \[
        (\mathcal{A}_1 \otimes \mathcal{A}_2) \otimes \mathcal{A}_3 = \sigma\qty(\qty{(A_1 \times A_2) \times A_3 : A_i \in \mathcal{A}_i})
    .\]
    Since $(A_1 \times A_2) \times A_3 = A_1 \times A_2 \times A_3$, equality of the two product algebras follows.

    Suppose each of the $\mu_i$ are $\sigma$-finite. Then $X_i = \bigcup_{j \in \N} X_{i}^j$ where $\mu_i(X_i^j) < \infty$. Therefore $X_1 \times X_2 \times X_3 = \bigcup_{j,k,l \in \N} X_1^j \times X_2^k \times X_3^l$ and
    \[
        (\mu_{1} \times \mu_2 \times \mu_3)(X_1^j \times X_2^k \times X_3^l) = \mu_1(X_1^j) \mu_2(X_2^k) \mu_3(X_3^l) < \infty
    .\]
    Therefore $\mu_1 \times \mu_2 \times \mu_3$ is $\sigma$-finite. Therefore by Theorem 6.10, it follows that $\mu_1 \times \mu_2 \times \mu_3$ is the unique measure on $\mathcal{A}_1 \times \mathcal{A}_2 \times \mathcal{A}_3$ in which
    \[
        (\mu_1 \times \mu_2 \times \mu_3)(A_1 \times A_2 \times A_3) = \mu_1(A_1) \mu_2(A_2) \mu_3(A_3), \quad A_i \in \mathcal{A_i}
    .\]
    However since for any $A_i \in \mathcal{A}_i$ we have
    \[
        \qty((\mu_1 \times \mu_2) \times \mu_3)(A_1 \times A_2 \times A_3) = \qty(\mu_1 \times \mu_2) (A_1 \times A_2) \mu_3(A_3) = \mu_1(A_1) \mu_2(A_2) \mu_3(A_3)
    \]
    it follows $\mu_1 \times \mu_2 \times \mu_3 = (\mu_1 \times \mu_2) \times \mu_3$ by uniqueness.
\end{proof}

\section*{Exercise 6.34}

\begin{proof}
    %TODO: Other stuff

    Suppose towards contradiction that $(\mu_1 \times \mu_2)(D) < \infty$. Let $A_n \times B_n$ be a cover of $D$ in $X_1 \times X_2$. Without loss of generality, it can be assumed that $m(A_0) = 0$ and $m(A_n) > 0$ for all $n \geq 1$, as well as $\mu_2(B_n) \geq 1$ for all $n$. Note that since $(\mu_1 \times \mu_2)(D) < \infty$, it follows
    \[
        \sum_{n \geq 0} \mu_1(A_n) \mu_2(B_n) < \infty
    \]
    which clearly means that $\mu_2(B_n) < \infty$ and thus the $B_n$ are finite for all $n \geq 1$. Therefore letting
    \[
        C = \bigcup_{n \geq 1} A_n \cap B_n
    \]
    is a countable set and thus $m(C) = 0$. Since $D$ is this special diagonal set, it follows that $x \in A_n \times B_n$ implies $x \in A_n$ and $x \in B_n$, thus
    \[
        [0,1] \subset \bigcup_{n \geq 0} A_n \cap B_n = (A_0 \cap B_0) \cup C
    .\]
    But then this means that
    \[
        [0,1] \setminus C \subset A_0 \cap B_0 \implies A_0 \supset [0,1] \setminus C
    .\]
    Since $m(C) = 0$, then by monotonicity it follows $m(A_0) \geq m([0,1]) = 1$, a contradiction. Therefore 
    \[
        \int \ind{D} d \mu_1 \times \mu_2 = (\mu_1 \times \mu_2)(D) = \infty
    .\]
\end{proof}

\end{document}
