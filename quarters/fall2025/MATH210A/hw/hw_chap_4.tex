\documentclass[hw_all.tex]{subfiles}

\begin{document}

\section*{Exercise 4.9}

\begin{proof}
    \begin{enumerate}
        \item
            Clearly if such a bijection exists the result holds. Suppose then that
            \[
                \mathcal{B}((B_{\alpha})_{\alpha \in I}) = \mathcal{B}((B'_{\alpha})_{\alpha \in I'})
            .\]
            Since they are equal, it follows that for any $B_{\alpha}$ there is some $J \subset I'$ such that
            \[
                B_{\alpha} = \bigcup_{\alpha' \in J} B'_{\alpha'}
            .\]
            By the same argument it follows that there exists for each $B'_{\alpha'}$ some $I_{\alpha'} \subset I$ such that
            \[
                B'_{\alpha'} = \bigcup_{\alpha \in I_{\alpha'}} B_{\alpha}
            .\]
            Therefore
            \[
                B_{\alpha} = \bigcup_{\alpha' \in J} \bigcup_{\beta \in I_{\alpha'}} B_{\beta}
            .\]
            If either $|J| > 1$ or $|I_{\alpha'}| > 1$, then $B_{\alpha}$ could not be an atom as it would be the union of at least two other atoms. Thus $|J| = 1$ and $|I_{\alpha'}| = 1$, meaning $B_{\alpha} = B'_{\alpha'}$ for some $\alpha' \in I'$. Define then $\phi(\alpha) = \alpha'$. This same argument applies to all $B_{\alpha}$ as well as all $B'_{\alpha'}$, hence $\phi : I \to I'$ is a bijection.

        \item 
            Since $\mathcal{B}$ is atomic and finite, there a finite number of atoms $A_1, \ldots, A_n$ that generate $\mathcal{B}$. Since $B \in \mathcal{B}$ can be expressed as the union of some subset of these atoms, and it is a binary choice to either include an atom or not in the union, there are $2^n$ possible $B$. Hence $|\mathcal{B}| = 2^n$.

            Note then that every finite Boolean algebra over $X$ can be described as the atomic algebra generated by some finite partition of $X$. Any two partitions that are the same under relabelling will still preserve equality of the corresponding Boolean algebras under relabelling by part $(i)$.

        \item 
            In the case of either the elementary or Jordan algebra, the non-empty atoms would have to be non-trivial intervals. However any non-trivial interval $I$ can be split in half into two non-trivial intervals $I_1$ and $I_2$. Clearly $I_1$ and $I_2$ themselves are elementary and Jordan measurable, but cannot be in generated algebra since $I$ is an atom. Therefore neither the elementary or Jordan algebra's are atomic.

            Suppose that the Lebesgue algebra (which is just the collection of all Lebesgue measurable sets) was atomic. Note that there must be an atom $A$ such that $|A| \geq 2$. If not, then each atom would be a singleton and thus generate the discrete algebra, which does not equal the Lebesgue algebra. Note that $A$ must be Lebesgue measurable, and since $\qty{x}$ is also Lebesgue measurable, $A \setminus \qty{x}$ is also Lebesgue measurable. But since $A \setminus \qty{x} \subset A$, then $A$ cannot be an atom, a contradiction. Therefore the Lebesgue algebra is not atomic

            The same argument above holds for the null algebra since the Lebesgue measure of a singleton is $0$. Hence the null algebra is not atomic.

        \item 
            Since $\varnothing$ and $X$ must be contained all $\mathcal{B}_{\alpha}$ by definition, then $\varnothing, X \in \mathcal{B}$. Take $A, B \in \mathcal{B}$.
            \begin{itemize}
                \item
                    Since $A \in \mathcal{B}$, then $A \in \mathcal{B}_{\alpha}$ for all $\alpha$, meaning $X \setminus A \in \mathcal{B}_{\alpha}$ for all $\alpha$. Thus $X \setminus B \in \mathcal{B}$.

                \item 
                    Since $A, B \in \mathcal{B}$, then $A,B \in \mathcal{B}_{\alpha}$ for all $\alpha$, meaning $A \cup B \in \mathcal{B}_{\alpha}$ for all $\alpha$. Thus $A \cup B \in \mathcal{B}$
            \end{itemize}
            Therefore $\mathcal{B}$ is a boolean algebra. Since $\mathcal{B} \subset B_{\alpha}$ for all $\alpha$, it follows that $\mathcal{B}$ is coarser than all $B_{\alpha}$. Note that if $\mathcal{C}$ is a boolean algebra contained in all $B_{\alpha}$, then any $A \in \mathcal{C}$ is contained in all $B_{\alpha}$. Hence $A \in \mathcal{B}$, meaning $\mathcal{C} \subset \mathcal{B}$. Thus $\mathcal{B}$ is the finest Boolean algebra coarser than all $\mathcal{B}_{\alpha}$.
    \end{enumerate}
\end{proof}

\section*{Exercise 4.12}

\begin{proof}
    \begin{enumerate}
        \item 
            Let $\mathcal{F}$ be the collection of all boxes in $\R^d$. Since the elementary algebra by definition contains all boxes since boxes are elementary sets, it follows that $b(\mathcal{F}) \subset \tilde{\mathcal{E}}$. Note that for any Boolean algebra $\mathcal{B}$ containing $\mathcal{F}$, that the collection of all finite unions of boxes and the complement of those sets must be contained in it. But that is exactly the definition of $\tilde{\mathcal{E}}$, hence $\tilde{\mathcal{E}} \subset \mathcal{B}$. Therefore $\tilde{\mathcal{E}} \subset b(\mathcal{F})$, meaning $\tilde{\mathcal{E}} = b(\mathcal{F})$.

        \item 
            Let $\mathcal{F} = \qty{F_1, \ldots, F_n}$. For all $n$ long binary string $\eps = (\eps_1, \eps_2, \ldots, \eps_n)$, let
            \[
                A_{\eps} = \bigcap_{k=1}^n B_k, \qquad B_k = \begin{cases}
                    F_k & \eps_k = 1 \\
                    F_k^c & \eps_k = 0 \\
                \end{cases}
            .\]
            Note that the collection of all $A_{\eps}$ for every binary string partitions $X$.
            \begin{itemize}
                \item
                    If $A_{\eps} \neq A_{\eps'}$, then $\eps \neq \eps'$ meaning some $\eps_k \neq \eps'_k$. Therefore $B_k \cap B'_k = \varnothing$ meaning $A_{\eps} \cap A_{\eps'} = \varnothing$.
                \item 
                    Let $x \in X$ and define $\eps_k = 1$ if $x \in F_k$ and $0$ otherwise. Then clearly $x \in A_{\eps}$, meaning
                    \[
                        X \subset \bigcup_{\eps \in \qty{0,1}^n} A_{\eps}
                    .\]
                    But clearly the reverse inclusion holds, hence equality holds.
            \end{itemize}
            Therefore we can construct the atomic algebra $\mathcal{A}$ from the atoms $A_{\eps}$. Since we have $2^n$ atoms (since there are $2^n$ binary strings of length $n$), from the previous problem we have $|\mathcal{A}| \leq 2^{2^n}$. But clearly $b(\mathcal{F}) \subset \mathcal{A}$, hence $|b(\mathcal{F})| \leq 2^{2^n}$.

            Consider $X = \qty{0,1}^N$ and let $F_k = \qty{x \in X : x_k = 1}$. Note that by the same construction above, $A_x = \qty{x}$ for every $x \in \qty{0,1}^N$. Since every Boolean algebra containing $\mathcal{F} = \qty{F_1, \ldots, F_N}$ must be closed under finite union, intersection, and complement it follows that every Boolean algebra contains $A_x$ for all $x \in X$. But that means every singleton of $X$ is contained in every Boolean algebra. Therefore every Boolean algebra must equal $\mathcal{P}(X)$, hence $|b(\mathcal{F})| = |\mathcal{F}| = 2^{|X|} = 2^{2^N}$.


        \item 
            Let $G = \bigcup_{n=0}^\infty \mathcal{F}_n$. Let $E, F \in G$.
            \begin{itemize}
                \item If $\mathcal{F} \neq \varnothing$, then take $A \in \mathcal{F}$. Note then that $A \cup A^c \in \mathcal{F_1}$ and thus $(A \cup A^c)^c \in \mathcal{F_2} \subset G$. But $(A\cup A^c)^c = \varnothing$. Thus $\varnothing \in G$.

                \item Since $E, F$ are in $G$, then WLOG $E \in \mathcal{F}_{n}$ and $F \in \mathcal{F}_m$ for some $n > m \geq 0$. Note that $\mathcal{F}_m \subset \mathcal{F}_n$. Therefore $F \in \mathcal{F}_n$, meaning $E \cup F \in \mathcal{F}_{n+1} \subset G$. Hence $G$ is closed under union.

                \item Since $E$ is in $G$, then $E \in \mathcal{F}_n$ for some $n \geq 0$. Thus by construction, $E^c \in \mathcal{F}_{n+1} \subset G$. Hence $G$ is closed under complement.
            \end{itemize}

            Therefore $G$ is a boolean algebra containing $\mathcal{F}$, hence $b(\mathcal{F}) \subset G$. Note that $\mathcal{F}_0 \subset b(\mathcal{F})$ and $\mathcal{F}_n \subset b(\mathcal{F}) \implies \mathcal{F}_{n+1} \subset b(\mathcal{F})$ since $b(\mathcal{F})$ is closed under finite intersection, union, and complement. Therefore by induction $\mathcal{F}_n \subset b(\mathcal{F})$ for all $n$ and thus $G \subset b(\mathcal{F})$. Hence $G = b(\mathcal{F})$.
    \end{enumerate}
\end{proof}

\section*{Exercise 4.14}

\begin{proof}
    \begin{enumerate}
        \item
            Since atomic algebras are Boolean algebras, the first two properties follow by definition. Let $\mathcal{A}((A_{\alpha})_{\alpha \in I})$ be an atomic algebra and $(E_n)_{n \in \N} \in \mathcal{A}^{\N}$. Note that for each $E_n$ there is some $I_n \subset I$ such that $E_n = \bigcup_{\alpha \in I_n} A_{\alpha}$. Therefore if $I_E = \bigcup_{n \in \N} I_n$ then
            \[
                \bigcup_{n \in \N} E_n = \bigcup_{\alpha \in I_E} A_{\alpha}
            .\]
            Since $I_E \subset I$, then $\bigcup_{n \in \N} E_n \in \mathcal{A}$. Hence $\mathcal{A}$ is a $\sigma$-algebra.

            Since the trival, discrete, and dyadic algebras are atomic, then they are $\sigma$-algebras. Furthermore every finite algebra is also a $\sigma$-algebra since there are only finitely many sets in the algebra, hence any countable union of sets in the algebra can be reduced to a finite union which is closed.

        \item 
            Clearly $\mathcal{L}$ is a $\sigma$-algebra by its closure properties. Consider then $\mathcal{N}$. Since $\mathcal{N}$ is a Boolean algebra, all that is left to show is that it is closed under countable union. Let $(E_n)_{n \in \N} \subset \mathcal{N}^\mathcal{N}$. Consider then two cases
            \begin{itemize}
                \item If there is at least one $E_k$ such that $m(E_k^c) = 0$, then
                    \[
                        \qty(\bigcup_{n \in \N} E_n)^c = \bigcap_{n \in \N} E_n^c \subset E_k^c
                    .\]
                    Therefore
                    \[
                        m\qty(\qty(\bigcup_{n \in \N} E_n)^c) = 0 \implies \bigcup_{n \in \N} E_n \in \mathcal{N}
                    .\]

                \item If $m(E_n) = 0$ for all $n$, then
                    \[
                        m\qty(\bigcup_{n \in \N} E_n) \leq \sum_{n \in \N} m(E_n) = 0 \implies \bigcup_{n \in \N} E_n \in \mathcal{N}
                    .\]
            \end{itemize}
            In either case, we have closure under countable union, hence $\mathcal{N}$ is a $\sigma$-algebra.

            The elementary and Jordan algebra's are not $\sigma$-algebras, since for any $r \in \Q$, $\qty{r}$ is in both, but $\Q$ is not measurable in either and $\Q = \bigcup_{r \in \Q} \qty{r}$ which is a countable union.

        \item 
            Consider the requirements of a $\sigma$-algebra:
            \begin{itemize}
                \item 
                    Clearly $\varnothing \in \mathcal{A}_Y$ since $\varnothing \in \mathcal{A}$ and $Y \cap \varnothing = \varnothing$.

                \item 
                    Let $E \cap Y \in \mathcal{A}_Y$ with $E \in \mathcal{A}$. Note then the complement of $E \cap Y$ in $Y$ is
                    \[
                        Y \setminus (E \cap Y) = Y \cap (E^C \cup Y) = (Y \cap Y^c) \cup (E^c \cap Y) = E^c \cap Y
                    .\]
                    Since $E^c \in \mathcal{A}$, it follows the complement of $E^c \cap Y$ is $Y$ is in $\mathcal{A}_Y$.

                \item 
                    Let $(E_n \cap Y)_{n \in \N} \in \mathcal{A}_Y^\N$. Note that
                    \[
                        \bigcup_{n \in \N} (E_n \cap Y) = Y \cap \bigcup_{n \in \N} E_n
                    .\]
                    Since $\bigcup_{n \in \N} E_n \in \mathcal{A}$, it follows $\mathcal{A}_Y$ is closed under countable union.
            \end{itemize}

            Therefore $\mathcal{A}_Y$ is a $\sigma$-algebra on $Y$.

        \item 
            Consider the requirements of a $\sigma$-algebra:

            \begin{itemize}
                \item 
                    $\varnothing \in \mathcal{A}_{\alpha}$ for every $\alpha$, thus $\varnothing \in \mathcal{A}$.

                \item 
                    If $E \in \mathcal{A}$, then $E \in \mathcal{A}_{\alpha}$ for every $\alpha$. Thus $E^c \in \mathcal{A}_{\alpha}$ for every $\alpha$ meaning $E^c \in \mathcal{A}$.

                \item 
                    If $(E_n)_{n \in \N} \in \mathcal{A}^\N$, then for every $n$ we have $E_n \in \mathcal{A}_{\alpha}$ for every $\alpha$. Therefore
                    \[
                        \bigcup_{n \in \N} E_n \in \mathcal{A}_{\alpha}, \forall \alpha
                    .\]
                    Thus $\bigcup_{n \in \N} E_n \in \mathcal{A}$.
            \end{itemize}

            Therefore $\mathcal{A}$ is a $\sigma$-algebra.
    \end{enumerate}
\end{proof}

\section*{Exercise 4.20}

\begin{proof}
    \begin{enumerate}
        \item
            Consider first the case where $E_1$ is a box in $\R^{d_1}$. Define the property $P(B)$ where $B \subset \R^{d_2}$ by
            \[
                P(B) \Leftrightarrow E_1 \times B \in \mathcal{B}(\R^{d_1 + d_2})
            .\]
            
            Let $\mathcal{F}$ be the family of all boxes in $\R^{d_2}$. Note then that
            \begin{itemize}
                \item 
                    $P(\varnothing)$ holds since $E_1 \times \varnothing = \varnothing$ is Borel

                \item 
                    $P(B)$ holds for every $B \in \mathcal{F}$ since the product of boxes is also a box, and all boxes are Borel.

                \item 
                    Suppose $P(B)$ holds. Then $E_1 \times B$ is Borel, hence $(E_1 \times B)^c$ is Borel. Note then that
                    \[
                        E_1 \times B^c = (E_1 \times \R^{d_2}) \setminus (E_1 \times B) = (E_1 \times \R^{d_2}) \cap (E_1 \times B)^c
                    .\]

                    Since $E_1 \times \R^{d_2}$ is Borel, it follows that $E_1 \times B^c$ is Borel and hence $P(B^c)$ holds.

                \item 
                    Suppose $P(B_n)$ holds for every $n$. Then $E_1 \times B_n$ is Borel for every $n$, and
                    \[
                        E_1 \times \bigcup_{n \in \N} B_n = \bigcup_{n \in \N} (E_1 \times B_n)
                    .\]
                    Therefore $E_1 \times \bigcup_{n \in \N} B_n$ is a countable union of Borel sets, and thus Borel. Hence $P(\bigcup_{n \in \N} B_n)$ holds.
            \end{itemize}

            Therefore $P(B)$ holds for all $B \in \sigma(\mathcal{F}) = \mathcal{B}(\R^{d_2})$. If then $E_2$ is fixed to be Borel, the same argument applies over the family of boxes in $\R^{d_1}$, hence $E_1 \times E_2$ is Borel for all $E_1$ Borel and $E_2$ fixed. But this establishes the desired result.

        \item
            First we prove that the preimage of Borel sets under a continuous function are Borel. Let $f : X \to Y$ be continuous and $\mathcal{B}_X$ and $\mathcal{B}_Y$ the Borel algebras of $X$ and $Y$ respectively. Let $\mathcal{F}$ be the family of all open sets in $Y$. Define the property $P(E)$ for $E \subset Y$ by
            \[
                P(E) \Leftrightarrow f^{-1}(E) \in \mathcal{B}_X
            .\]
            Note that
            \begin{itemize}
                \item 
                    $P(\varnothing)$ holds since $f^{-1}(\varnothing) = \varnothing \in \mathcal{B}_X$.

                \item 
                    If $U \subset Y$ is open, then $f^{-1}(U)$ is open in $X$ and hence Borel. Therefore $P(U)$ holds for all $U \in \mathcal{F}$.

                \item 
                    Suppose that $P(E)$ holds. Then $f^{-1}(E)$ is Borel meaning $\qty(f^{-1}(E))^c$ is Borel. But $\qty(f^{-1}(E))^c = f^{-1}(E^c)$, thus $P(E^c)$ holds.

                \item 
                    Suppose $P(E_n)$ holds for all $n$. Then $f^{-1}(E_n)$ is Borel for all $n$, thus
                    \[
                        f^{-1}\qty(\bigcup_{n \in \N} E_n) = \bigcup_{n \in \N} f^{-1}(E_n)
                    \]
                    is a countable union of Borel sets, hence it is Borel. Therefore $P(\bigcup_{n \in \N} E_n)$ holds.
            \end{itemize}

            Therefore it follows $P(E)$ holds for all $E \in \sigma(\mathcal{F}) = \mathcal{B}_Y$. Hence the preimage of Borel sets under a continuous map are Borel.

            Now take $E \in \mathcal{B}(\R^{d_1 + d_2})$ and fix $x_1 \in \R^{d_1}$. Let
            \[
                f_{x_1} : \R^{d_2} \to \R^{d_1 + d_2} : y \mapsto (x_1, y)
            .\]
            Note that $E_{x_1} = f^{-1}_{x_1}(E)$ and that $f$ is continuous. Thus from the previous lemma, it follows $E_{x_1}$ is Borel. The same argument with the coordinates reversed establishes the desired result.

            This result does not hold if Borel is substituted for Lebesgue. Let $V$ be the Vital set and $x \in \R$. Then clearly $E = V \times \qty{x}$ is a null set in $\R^2$, but $E_x = V$ is not Lebesgue measurable in $\R$.

        \item 
            Since every borel and null set are Lebesgue measurable, it follows $\mathcal{B} \cup \mathcal{N} \subset \mathcal{L}$, and thus $\sigma(\mathcal{B} \cup \mathcal{N}) \subset \sigma(\mathcal{L})$. Consider then some Lebesgue measurable set $E \in \mathcal{L}$. Then there exists a $F_{\sigma}$ set $B$ and null set $N$ such that $E = B \cup N \in \mathcal{B} \cup \mathcal{N}$. Therefore $\sigma(\mathcal{L}) = \mathcal{L} \subset \sigma(\mathcal{B} \cup \mathcal{N})$. Hence $\sigma(\mathcal{L}) = \sigma(\mathcal{B} \cup \mathcal{N})$.
    \end{enumerate}
\end{proof}

\section*{Exercise 4.31}

\begin{proof}
    \begin{enumerate}
        \item 
            Let $(F_n)_{n \in \N} \in \mathcal{A}^\N$ where
            \[
                F_1 = E_1, \qquad F_{n+1} = E_{n+1} \setminus \qty(\bigcup_{k=1}^n E_n)
            .\]
            Note that each $F_n$ are disjoint, $E_n \subset F_n$, and that
            \[
                \bigcup_{n \in \N} F_n = \bigcup_{n \in \N} E_n
            .\]
            Therefore
            \[
                \mu\qty(\bigcup_{n \in \N} E_n) = \mu\qty(\bigcup_{n \in \N} F_n) = \sum_{n \in \N} \mu(F_n) \leq \sum_{n \in \N} \mu(E_n)
            \]
            which was to be shown.

        \item 
            If $\mu(E_n) = \infty$ for some $n \in \N$, then clearly the result holds by montonocity. Suppose then that $\mu(E_n) < \infty$ for all $n$. Let $F_{n+1} = E_{n+1} \setminus E_n$ with $F_1 = E_1$. Note that $F_n$ is a disjoint sequence, and that $\bigcup_{n \in \N} F_n = \bigcup_{n \in \N} E_n$. Therefore
            \[
                \mu\qty(\bigcup_{n \in \N} E_n) = \mu\qty(\bigcup_{n \in \N} F_n) = \sum_{n \in \N} \mu(F_n)
            .\]
            Note that the partial sums of the right hand side for some $N$ are
            \begin{align*}
                \sum_{n=1}^N \mu(F_n) &= \mu(E_1) + \mu(E_2 \setminus E_1) + \ldots + \mu(E_{N-1} \setminus E_{N-2}) + \mu(E_N \setminus E_{N-1}) \\
                &= \mu(E_1) - \mu(E_1) + \mu(E_2) - \ldots - \mu(E_{N - 2}) + \mu(E_{N-1}) - \mu(E_{N-1}) + \mu(E_{N}) \\
                &= \mu(E_{N})
            \end{align*}
            Therefore $\sum_{n \in \N} \mu(F_n) = \lim_{n \to \infty} \mu(E_n)$. Note also by monotonicty that $\mu(E_n) \leq \mu(E_{n+1})$, therefore $\lim_{n\to \infty} \mu(E_n) = \sup_{n \in \N} \mu(E_n)$. In total then
            \[
                \mu\qty(\bigcup_{n \in \N} E_n) = \lim_{n \to \infty} \mu(E_n) = \sup_{n \in \N} \mu(E_n)
            .\]

        \item 
            Let $E_k$ be the set such that $\mu(E_k) < \infty$ and $F = \bigcap_{n \in \N} E_n$. Note then that
            \[
                E_k = F \;\dot{\cup}\; \qty(\bigcup_{n=k}^\infty E_{n} \setminus E_{n+1}) \tag{$\star_1$}
            \]
            since $E_{n} \setminus E_{n+1}$ doesn't contain $E_{n+1}$ and $F \subset E_{n+1}$. Each $E_{n} \setminus E_{n+1}$ is disjoint since the sequence of $E_n$ is non-increasing, therefore
            \[
                \mu(E_k) = \mu(F) + \mu\qty(\bigcup_{n=k}^\infty E_{n} \setminus E_{n+1}) = \mu(F) + \sum_{n=k}^\infty \mu(E_{n} \setminus E_{n+1})
            \]
            By montonicity $\mu(F) \leq \mu(E_k) < \infty$, thus
            \[
                \sum_{n=k}^\infty \mu(E_n \setminus E_{n+1}) = \mu(E_k) - \mu(F) < \infty \tag{$\star_2$}
            .\]
            Note that $(\star_1)$ holds for any $n \geq k$, therefore if $n \geq k$
            \[
                \mu(E_n) - \mu(F) = \sum_{j=n}^\infty \mu(E_n \setminus E_{n+1})
            .\]
            The RHS is the tail of the sum in $(\star_2)$ which a convergent series. Therefore in the limit $n \to \infty$, it goes to $0$. Thus
            \[
                \lim_{n \to \infty} \mu(E_n) - \mu(F) = 0 \implies \lim_{n \to \infty} \mu(E_n) = \mu(F) = \mu\qty(\bigcap_{n \in \N} E_n)
            .\]
            Since $\mu(E_n) \geq \mu(E_{n+1})$ by monotonicity, $\lim_{n \to \infty} \mu(E_n) = \inf_{n \in \N} \mu(E_n)$. In total then
            \[
                \mu\qty(\bigcap_{n \in \N} E_n) = \lim_{n \to \infty} \mu(E_n) = \inf_{n \in \N} \mu(E_n)
            .\]
    \end{enumerate}

    In $(iii)$, suppose the condition for one of the sets being finite is dropped. Consider then the non-increasing sequence $E_n = [n, \infty)$. Then $m(E_n) = \infty$ for all $n$, meaning
    \[
        \lim_{n \to \infty} m(E_n) = \infty
    .\]
    But then
    \[
        m\qty(\bigcap_{n = 1}^\infty E_n) = m(\varnothing) = 0
    .\]
    Hence the condition for eventual finiteness is needed.
\end{proof}

\section*{Exercise 4.31}

\begin{proof}
    \begin{enumerate}
        \item
            Take $x \in E$. From pointwise convergence it follows $\exists N \in \N$ such that $|\mathds{1}_{E_n}(x) - \mathds{1}_{E}(x)| < \frac{1}{2}$ for every $n \geq N$. But this means that $\mathds{1}_{E_n}(x) = \mathds{1}_{E}(x)$ since any indicator function only takes on the values $\qty{0,1}$. Therefore $x \in E_n$ for every $n \geq N$. Thus
            \[
                E \subset \bigcup_{N \in \N} \bigcap_{n \geq N} E_n \liminf_{n \to \infty} E_n \eqcolon L
            .\]
            Now take $x \in L$. Then there is some $N \in \N$ such that $x \in E_n$ for all $n \geq N$. Thus $\mathds{1}_{E_n}(x) = 1$ for every $n \geq N$. It follows then $\mathds{1}_{E} = \lim_{n \to \infty} \mathds{1}_{E_n}(x) = 1$, meaning $x \in E$. Therefore we have the other subset inclusion giving $E = L$.

            Since every $E_n$ is measurable, the countable intersection of them is also measurable. This collection of countable intersections are all measurable, thus the countable union of them is also measurable. Hence $L$ is measurable, which means $E$ is measurable.

        \item 
            By the same observations as above, if $x \notin E$, then $1_{E_n}(x) = 0$ for sufficiently large $n$, and hence $x \notin E_n$. Therefore $x \in E_n$ for only finitely many $n$ and thus $x \notin \limsup_{n} E_n$. Hence $\limsup_{n} E_n \subset E$ meaning
            \[
                E = \liminf_{n \to \infty} E_n \subset \limsup_{n \to \infty} E_n \subset E \implies E = \limsup_{n \to \infty} E_n = \bigcup_{n \in N} \bigcap_{k \geq n} E_n
            .\]
            Take then $C_n = \bigcap_{k \geq n} E_n$ and $D_n = \bigcup_{k \geq n}$. Note that we have
            \[
                C_n \subset E_n \subset D_n \subset F
            .\]
            Since $\mu(F) < \infty$, continuity from above and below can be applied to get
            \[
                \mu(E) = \lim_{n \to \infty} \mu(C_n) \hspace{1.5cm} \mu(E) = \lim_{n \to \infty} \mu(D_n)
            .\]
            But note that by montonicty $\mu(C_n) \leq \mu(E_n)$ and $\mu(D_n) \geq \mu(E_n)$, thus
            \[
                \lim_{n \to \infty} \mu(C_n) \leq \liminf_{n \to \infty} \mu(E_n) \hspace{1.5cm} \lim_{n \to \infty} \mu(D_n) \geq \limsup_{n \to \infty} \mu(E_n)
            .\]
            Therefore
            \[
                \limsup_{n \to \infty} \mu(E_n) \leq \mu(E) \leq \liminf_{n \to \infty} \mu(E_n)
            \]
            which gives $\lim_{n \to \infty} \mu(E_n) = \mu(E)$
        \item 
            Consider the Lebesgue algebra and measure. Let $E_n = n + [0,1]$. Note that $m(E_n) = 1$ for all $n$ by translation invariance. However, for any $x \in \R$, there exists some $N \in \N$ such that $x < n$ for all $n > N$. Thus $x \notin E_n$ for $n > N$, and hence $\mathds{1}_{E_n}(x) \neq 1$ for $n > N$. This means that $\lim_{n \to \infty} \mathds{1}_{E_n} = \mathds{1}_{\varnothing}$, and since $m(\varnothing) = 0 \neq 1$, (b) does not hold.
    \end{enumerate}
\end{proof}

\section*{Exercise 4.35}

\begin{proof}
    \newcommand\aRefine{\conj{\mathcal{A}}}
    \newcommand\aNull{\mathcal{N}}
    \newcommand\aNullS{\mathcal{N}^*}
    Define the following 
    \begin{align*}
    \aNull &= \qty{N \in \mathcal{A} : \mu(N) = 0} \\
    \aNullS &= \qty{M \in 2^X : M \subset N \in \aNull} \\
    \aRefine &= \qty{A \cup M : A \in \mathcal{A}, M \in \aNullS}
    \end{align*}
    Clearly $\varnothing$ and $X$ are in $\aRefine$ since $M$ can be taken as $\varnothing$ in both cases. Let then $E_n = A_n \cup M_n \in \aRefine$ with $M_n \subset N_n \in \aNull$. Note that
    \[
        \bigcup_{n \in \N} E_n = \qty(\bigcup_{n \in \N} A_n) \cup \qty(\bigcup_{n \in \N} M_n)
    .\]
    Since $\mathcal{A}$ is a $\sigma$-algebra, $\bigcup A_n \in \mathcal{A}$. Furthermore $\mu(\bigcup N_n) = 0$ and since $\bigcup M_n \subset \bigcup N_n$, it follows $\bigcup E_n \in \aRefine$. Take now $E = A \cup M \in \aRefine$ and $M \subset N \in \aNull$. Note that
    \[
        E^c = A^c \cap M^c = (A^c \cap (X \setminus N)) \cup (A^c \cap (N \setminus M))
    .\]
    Since $A^c \in \mathcal{A}$ and $X \setminus N \in \mathcal{A}$, it follows $A^c \cap (X \setminus N) \in N$. Furthermore, $A^c \cap (N \setminus M) \subset N$ meaning $E^c \in \aRefine$. Hence $\aRefine$ is a $\sigma$-algebra.

    Take $E = A \cup M \in \aRefine$ with and define $\conj{\mu}(E) \coloneq \mu(A)$. Clearly if $E \in \mathcal{A}$ ($M = \varnothing$) then $\conj{\mu}(E) = \mu(E)$. Suppose then that $A \cup M = B \cup M' \in \aRefine$ and $M \subset N \in \aNull$ and $M' \subset N' \in \aNull$. Note that
    \[
        A \setminus B \subset (A \cup M) \setminus (B \cup M') \subset M \cup M'
    \]
    which also holds for $B \setminus A$. Therefore $A \Delta B \subset M \cup M' \subset N \cup N'$. Since $\mu(N \cup N') = 0$, it follows $\mu(A \Delta B) = 0$ and thus $\mu(A) = \mu(B)$. Hence $\conj{\mu}$ is well-defined. Clearly $\conj{\mu}$ enjoys the measure properties of $\mu$ directly except for countable additivity. Let $E_n = A_n \cup M_n$ with $M_n \subset N_n \in \aNull$ pairwise disjoint. Note that
    \[
        \bigcup_{n \in \N} E_n = \qty(\bigcup_{n \in \N} A_n) \cup \qty(\bigcup_{n \in \N} M_n)
    .\]
    Since $E_n$ are pairwise disjoint, then so are the $A_n$ meaning
    \[
        \mu\qty(\bigcup_{n \in \N} A_n) = \sum_{n \in \N} \mu(A_n)
    .\]
    Furthermore since $\bigcup M_n \subset \bigcup N_n$ and $\bigcup N_n$ is a null set,
    \[
        \conj{\mu}\qty(\bigcup_{n \in \N} E_n) = \mu\qty(\bigcup_{n \in \N} A_n) = \sum_{n \in \N} \mu(A_n) = \sum_{n \in \N} \conj{\mu}(E_n)
    .\]
    Therefore $\conj{\mu}$ is a measure on $\aRefine$.

    Completeness of $\aRefine$ under $\conj{\mu}$ follows directly by its construction and the fact that $\varnothing \in \mathcal{A}$, thus leaving the question of coarsity. Let $\mathcal{B} \supset \mathcal{A}$ be a complete $\sigma$-algebra with meausre $\nu$ that extends $\mu$. Note that $\aNullS \subset \mathcal{B}$ since for any $M \subset N \in \aNull \subset \mathcal{B}$, by completeness $\mu(N) = 0 \implies \mu(M) = 0$. Therefore $\mathcal{A} \cup \aNullS \subset \mathcal{B}$, meaning it must contain $\sigma(A \cup \aNullS)$ which is simply $\aRefine$. Thus $\aRefine$ is the coarset complete refinement of $\mathcal{A}$.
\end{proof}

\section*{Exercise 4.41}

\begin{proof}
    \begin{enumerate}
        \item
            Let
            \[
                \mathcal{C} = \qty{E \in \sigma(\mathcal{B}) : \forall \eps > 0, \exists F \in \mathcal{B} \text{ s.t. } \mu(E \Delta F) < \eps }
            .\]

            Note that $\mathcal{B} \subset \mathcal{B}$ since $\mu(A \Delta A) = 0$ for $A \in \mathcal{B}$. We proceed to show $\mathcal{C}$ is a montone class.

            \begin{itemize}
                \item
                    Let $E_n \in \mathcal{C}^\N$ be non-decreasing and $E = \bigcup_{n \in \N} E_n$. Take $\eps > 0$. For each $n$ take $F_n \in \mathcal{B}$ such that
                    \[
                        \mu(E_n \Delta F_n) < \eps \cdot 2^{-(n+1)}
                    .\]
                    Let $G_N = \bigcup_{k \leq N} F_k$ and note $G_N \in \mathcal{B}$ since it is a finite union. Note that we have
                    \[
                        E \setminus G_N \subset \qty(\bigcup_{k \leq N} E_k \setminus G_N) \cup \qty(\bigcup_{k > N} E_k) \subset \qty(\bigcup_{k \leq N} E_k \setminus F_k) \cup \qty(\bigcup_{k > N} E_k)
                    \]
                    and
                    \[
                        G_N \setminus E \subset \bigcup_{k \leq N} F_k \setminus E_k \subset \bigcup_{k \leq N} E_k \Delta F_k
                    .\]
                    Therefore
                    \begin{align*}
                        \mu(E \Delta G_N) &= \mu((E \setminus G_N) \cup (G_N \setminus E)) \\
                                          &\leq \mu\qty(\bigcup_{k > N} E_k) + \mu(\bigcup_{k \leq N} E_k \Delta F_k) \\
                                          &\leq \mu(E \setminus E_N) + \sum_{k \leq N} \mu(E_k \Delta F_k)
                    \end{align*}
                    By continuity from below, it follows $\mu(E \setminus E_N) \to 0$ as $N \to \infty$. Take $N$ large enough such that $\mu(E \setminus E_N) \leq \frac{\eps}{2}$. Note then by above
                    \[
                        \mu(E \Delta G_N) \leq \frac{\eps}{2} + \sum_{k \leq N} \mu(E_k \Delta F_k) \leq \frac{\eps}{2} + \sum_{k \in \N} \eps \cdot 2^{-(n+1)} = \frac{e}{2} + \frac{\eps}{2} = \eps
                    .\]
                    Therefore $E \in \mathcal{C}$.

                \item 
                    Let $E_n \in \mathcal{C}^\N$ be non-increasing and $E = \bigcap_{n \in \N} E_n$. Since $\mu(X) < \infty$, the sequence $E_n^c$ and $E^c$ match the above criteria, thus there is some $F \in \mathcal{B}$ such that $\mu(E^c \Delta F) < \eps$. Note then that $\mu(E \Delta F^c) = \mu(E^c \Delta F) < \eps$ which can be done again since $\mu(X) < \infty$. Thus $E \in \mathcal{C}$ as $F^c \in \mathcal{B}$.
            \end{itemize} 

            Therefore $\mathcal{C}$ is a monotone class that is generated by $\mathcal{B}$, thus $\mathcal{C} = \sigma(\mathcal{B})$ meaning every set in $\sigma(\mathcal{B})$ has the desired property.

        \item 
            Take $\eps > 0$ and $E \in \sigma(\mathcal{B})$ with finite measure. For any $N$,consider the restricted subalgebra $\sigma(\mathcal{B}) \cap B_N$. On this sub algebra, the previous result holds and thus there is some $F \in \mathcal{B}$ such that
            \[
                \mu((E \cap B_N) \Delta F) < \frac{\eps}{2}
            .\]
            Since $E$ has finite measure and $X = \bigcup_{n \in \N} B_n$, then
            \[
                E = \bigcup_{E \cap B_n}
            .\]
            Therefore by continuity from below it follows $\mu(E) = \lim_{n \to \infty} \mu(E \cap B_n)$. Therefore there exists $N$ large enough such that $\mu(E \setminus B_N) < \frac{\eps}{2}$. Note then that
            \[
                E \Delta F \subset \qty(\qty(E \cap B_N) \Delta F) \cup (E \setminus B_N)
            \]
            meaning
            \[
                \mu(E \Delta F) \leq \mu((E \cap B_N) \Delta F) \leq \mu(E \setminus B_N) < \frac{\eps}{2} + \frac{\eps}{2} = \eps
            \]
            which was to be shown.
    \end{enumerate}
\end{proof}

\section*{Exercise 4.46}

\begin{proof}
    Suppose the backwards direction. Since each $A_{\alpha}$ is measurable, each $\ind{A_{\alpha}}$ is measurable and thus $f$ is measurable since it is sum of measurable functions.

    Suppose then the forwards direction. $f : \mathcal{A} \to \C$ being measurable is equivalent to its components being measurable, so we reduce to the case of $f : \mathcal{A} \to [0,\infty]$ since the resulting atomic sum representations can be added together. Consider some atom $A_{\alpha}$ and suppose towards contradiction that there are $x,y \in A_{\alpha}$ such that $f(x) \neq f(y)$. WLOG, suppose $f(x) < f(y)$. Let $B = [0, t)$ where $f(x) < t < f(y)$. Since $f$ is measurable, it follows $f^{-1}(B) \in \mathcal{A}$. Consider then $A_{\alpha} \cap f^{-1}(B)$.

    \begin{itemize}
        \item Since $x \in A_{\alpha}$ and $f(x) \in B$, then $x \in A_{\alpha} \cap f^{-1}(B)$ meaning it is non-empty
        \item Since $f(y) \notin B$, it follows $y \notin A_{\alpha} \cap f^{-1}(B)$ meaning is a proper subset.
    \end{itemize}

    Therefore $A_{\alpha} \cap f^{-1}(B)$ is a proper non-empty subset of $A_{\alpha}$. But since $A_{\alpha}$ and $f^{-1}(B)$ are measurable, $A_{\alpha} \cap f^{-1}(B)$ is measurable and smaller than $A_{\alpha}$, a contradiction. Therefore $f$ must be constant on every atom $A_{\alpha}$ with some value $c_{\alpha}$. Since the partition of $X$ is a disjoint partition, it then follows
    \[
        f = \sum_{\alpha \in I} c_{\alpha} \ind{A_{\alpha}} \qedhere
    \]
\end{proof}

\section*{Exercise 4.47}

\begin{proof}
    First note that we can modify the $f_n$ on a null set $K$ to converge pointwise everywhere to $f$, thus we assume $f_n$ converges pointwise everywhere to $f$ and defer dealing with $K$. Since the $f_n$ converge pointwise, for any $x \in X$ and $M \in \N$, there exists some $N \geq 0$ such that
    \[
        |f_n(x) - f(x)| \leq \frac{1}{M}, \forall n \geq N
    .\]
    Therefore if we take
    \[
        E_{N,M} = \qty{x \in X : |f_n(x) - f(x)| \leq \frac{1}{M}, n \geq N}
    \]
    then it follows for every $M$ that $X = \bigcup_{N \in N} E_{N, M}$ and $\varnothing = \bigcap_{N \in \N} E_{N,M}^c$. 
    
    To see that each $E_{N,M}$ (and thus $E_{N,M}^c$) is measurable, let $g_n(x) = |f_n(x) - f(x)|$. Note that $g_n$ is measurable since $f_n$ is measurable and $f$ is the limit of measurable functions and thus also measurable. Furthermore
    \[
        E_{N,M} = \bigcap_{n \geq N} g_n^{-1}\left(\left[0,\frac{1}{M}\right)\right)
    .\]
    Since $E_{N,M}$ is the countable intersection of measurable sets (since preimage of measurable sets under a measurable function are measurable), $E_{N,M}$ is measurable.

    Note that $E_{N,M}^c$ is decreasing in $N$, and since $\mu(X) < \infty$ then by continuity from above
    \[
        \lim_{N \to \infty} \mu(E_{N,M}^c) = 0
    .\]
    Note then for any $M \geq 1$, there exists some $N_M$ such that
    \[
        \mu(E_{N_M, M}^c) \leq \frac{\eps}{2^M}
    \]
    with $\eps > 0$. Therefore if we take
    \[
        A = K \cup \bigcup_{M \in \N} E_{N_M, M}^c
    \]
    then by subadditivity $\mu(A) \leq \eps$. Take $\delta > 0$ and $M$ large enough such that $\frac{1}{M} < \delta$. Then for any $x \in X \setminus A$ we have $x \in E_{N_M, M}$ for all $n \geq N_M$, hence
    \[
        |f_n(x) - f(x)| \leq \frac{1}{M} < \delta
    .\]
    Since $N_M$ only depends on $M$ which itself only depends on $\delta$, it follows $f_n$ converges uniformly to $f$.
\end{proof}

\section*{Exercise 4.54}

\begin{proof}
    Let $f = \sum_{i=j}^m c_j \ind{E_j}$ and $g = \sum_{k=1}^m d_k \ind{F_k}$ be simple functions.
    \begin{enumerate}
        \item
            If $f \leq g$ pointwise, then note on the refinement $A_{jk} = E_j \cap F_k$ that $c_j \leq d_k$. Therefore
            \[
                \sint(f, \mu) = \sum_{j,k} c_j \mu(A_{jk}) \leq \sum_{j,k} d_k \mu(A_{jk}) = \sint(g, \mu)
            .\]

        \item 
            Note the only values that $\ind{E}$ take are $0$ and $1$, and that $\ind{E}^{-1}(1) = E$. Therefore since $E \in \mathcal{A}$, we have measurability and
            \[
                \sint(\ind{E}, \mu) = 1 \cdot \mu(E) + 0 \cdot \mu(\ind{E}^{-1}(0)) = \mu(E)
            .\]

        \item 
            Note that
            \[
                cf = \sum_{j} (c \cdot c_j) \ind{E_j} = c \cdot \sum_{j} c_j \ind{E_j}
            .\]
            Therefore
            \[
                \sint(cf, \mu) = \sum_{j} (c \cdot c_j) \mu(E_j) = c \cdot \sum_{j} c_j \mu(E_j) = c\cdot\sint(f, \mu)
            .\]

        \item 
            Note that on the refinement $A_{jk} = E_j \cap F_k$ that $f + g = (c_j + d_k) \ind{A_{jk}}$. Therefore
            \begin{align*}
                \sint(f+g, \mu) = \sum_{j,k} (c_j + d_k) \mu(A_{jk}) &= \qty(\sum_{j,k} c_j A_{jk}) + \qty(\sum_{j,k} d_k A_{jk}) \\
                &= \sint(f,\mu) + \sint(g, \mu)
            \end{align*}

        \item 
            We have from definition 4.52 the finite $\sigma$-algebra $\mathcal{C} = \sigma(\qty{f^{-1}(c_1), \ldots, f^{-1}(c_m)})$. Note that $\mathcal{C} \subset \mathcal{A} \subset \mathcal{B}$. Since $\nu \vert_\mathcal{A} = \mu$, it follows that $\nu \vert_{\mathcal{C}} = \mu \vert_{\mathcal{C}}$ and thus
            \[
                \sint(f, \nu) = \sint(f, \mu\vert_\mathcal{B}) = \sint(f, \mu)
            .\]

        \item 
            Suppose $f \overset{\text{a.e.}}{=} g$ with respect to $\mu$. Consider the refinement $A_{jk} = E_j \cap F_k$ and let $I = \qty{(j,k) : c_k \neq d_k}$. Then
            \begin{align*}
                \sint(f, \mu) &= \sum_{(i,j) \notin I} c_j \mu(A_{jk}) + \sum_{(i,j) \in I} c_j A_{jk} \\
                \sint(g, \mu) &= \sum_{(i,j) \notin I} d_k \mu(A_{jk}) + \sum_{(i,j) \in I} d_k A_{jk}
            \end{align*}
            Note that if $N = \bigcup_{(i,j) \in I} A_{jk}$ that $\mu(N) = 0$ since it is comprised of sets where $f \neq g$. Thus $(i,j) \in I$ means $\mu(A_{jk}) \leq \mu(N) = 0$. Therefore
            \[
                \sint(f, \mu) = \sum_{(i,j) \notin I} c_j \mu(A_{jk}) = \sum_{(i,j) \notin I} d_k \mu(A_{jk}) = \sint(g, \mu)
            .\]
            
        \item 
            Suppose that $\sint(f, \mu) < \infty$. Then 
            \[
                c_i \mu(E_i) \leq \sum_{j} c_j \mu(E_j) = \sint(f,\mu) < \infty
            \]
            for all $i$. Note this implies that $f$ cannot be $\infty$ on a set of positive measure, thus $f < \infty$ almost everywhere. Take $c = \min\qty{c_j : c_j > 0}$. If $c$ does not exist, then clearly the statement holds. Note then that
            \[
                \sint(f, \mu) = \sum_{j} c_j \mu(E_j) = \sum_{j, c_j > 0} c_j \mu(E_j) \geq \sum_{j, c_j} c \mu(E_j) = c \cdot\mu(\supp(f))
            .\]
            Therefore $\mu(\supp(f)) \leq \frac{1}{c} \cdot \sint(f,\mu) < \infty$.

            Suppose then that $f < \infty$ almost everywhere and $\mu(\supp(f)) < \infty$. Let
            \begin{align*}
                J_{\infty} &= \qty{ j : c_j = \infty } \\
                J_{+} &= \qty{ j : 0 < c_j < \infty } \\
                J_{0} &= \qty{ j : c_j = 0 }
            \end{align*}
            Note that since $f < \infty$ almost everywhere, $c_j \mu(E_j) = 0$ for any $j \in J_{\infty}$ since $\mu(E_j) = 0$. If $J_{+}$ is empty, then clearly the result holds. Assume then $J_{+}$ is non-empty and let $c = \max_{j \in J_{+}} c_j < \infty$. Therefore
            \[
                \sint(f,\mu) = \qty[\sum_{j \in J_\infty} c_j \mu(E_j) ] + \qty[\sum_{j \in J_0} c_j \mu(E_j)] + \qty[\sum_{j \in J_+} c_j \mu(E_j)] = \sum_{j \in J_+} c_j \mu(E_j)
            .\]
            Since the $E_j$ can be taken pairwise disjoint, we get
            \[
                \sum_{j \in J_+} c_j \mu(E_j) \leq M \sum_{j \in J_+} \mu(E_j) = M \cdot \mu\qty(\bigcup_{j \in J_+} E_j)
            .\]
            Note that $\bigcup_{j \in J_+} E_j \subset \supp(f)$, thus
            \[
                \sint(f, \mu) \leq M \cdot \mu\qty(\bigcup_{j \in J_+} E_j) \leq M \cdot \mu(\supp(f)) < \infty
            .\]

        \item 
            This follows directly from $(vi)$ since $\sint(0, \mu) = 0$ always.
    \end{enumerate}
\end{proof}

\section*{Exercise 4.58}

\begin{proof}
    \begin{enumerate}
        \item[(i)]
            Suppose $f \overset{\text{a.e.}}{=} g$ with respect to $\mu$. Let $N = \qty{x \in X : f(x) \neq g(x)}$ and $\mu(N) = 0$. Take $h \leq f$ to be simple. Note that $\tilde{h} = h \cdot \ind{N^c}$ is also simple and that $\tilde{h} \leq g$. Since $h$ and $\tilde{h}$ are equal $\mu$-a.e., it follows $\sint(h, \mu) = \sint(\tilde{h}, \mu)$ as well. Therefore every simple function $h \leq f$ has some $\tilde{h} \leq g$ with the same integral, meaning
            \[
                \sup_{h \leq f} \sint(h, \mu) \leq \sup_{\tilde{h} \leq g} \sint(\tilde{h}, g)
            .\]
            The roles of $f$ and $g$ can be reversed in the previous argument, turning the previous inequality into equality. Thus $\int_X f d \mu = \int_X g d \mu$.

        \item[(ii)]
            This follows from the first part of the argument above by replacing $N = \qty{x \in X : f(x) > g(x)}$.

        \item[(v)]
            Since $f$ is simple and $f \leq f$, it follows
            \[
                \sint(f, \mu) \leq \sup_{h \leq f} \sint(h, \mu) 
            .\]
            But by montonicity of simple functions $h \leq f$, $\sint(h, \mu) \leq \sint(f, \mu)$ giving the reverse inequality. Hence $\int_X f d \mu = \sint(f, \mu)$.

        \item[(vi)]
            Note that pointwise $\lambda \cdot \ind{[f \geq \lambda]} \leq f$. Integrating both sides and applying both $(ii)$ and $(v)$ we get
            \[
                \lambda\cdot \mu([f \geq \lambda]) = \lambda \cdot \sint(\ind{[f \geq \lambda]}, \mu) = \sint(\lambda \cdot \ind{[f \geq \lambda]}, \mu) = \int_X \lambda \cdot \ind{[f \geq \lambda]} d \mu \leq \int_X f d \mu
            .\]

        \item[(viii)]
            If $\int_X f d \mu = 0$, for each $n \in \N$ we have from $(vi)$ that
            \[
                \mu([f \geq \frac{1}{n}]) \leq n \cdot \int_X f d \mu = 0
            .\]
            Therefore since the countable union of null sets is also a null set
            \[
                \mu([f > 0]) = \mu\qty(\bigcup_{n \in \N} [f \geq \frac{1}{n}]) = 0
            .\]
            Thus $f$ is $0$ $\mu$-a.e.

        \item[(iii)]
            If $c = 0$ then the result clearly holds as $cf = 0$ and the only simple function bounded by $0$ is the $0$ function itself. If $c = \infty$, then $\int_X f d \mu > 0$ gives $c \int_X f d \mu = \int_X cf d \mu = \infty$ and $\int_X f d \mu = 0$ gives $cf = 0$ a.e. and the result holds. Suppose then that $0 < c < \infty$. Then the map $h \mapsto ch$ is a bijection between simple functions $h \leq f$ and simple functions $ch \leq cf$. Since $\sint(ch, \mu) = c \sint(h, \mu)$, it follows
            \[
                \int_X cf d \mu = \sup_{h \leq cf} \sint(h, \mu) = \sup_{h \leq f} \sint(ch, \mu) = c\cdot \sup_{h \leq f} \sint(h, \mu) = c \cdot \int_X f d \mu
            .\]

        \item[(iv)]
            If $h_1, h_2$ are simple with $h_1 \leq f$ and $h_2 \leq g$ then $\tilde{h} = h_1 + h_2$ is also simple and $\tilde{h} \leq f + g$ as well as $h_1 + h_2 \leq \tilde{h}$. By linearity we have
            \[
                \sint(h_1 + h_2, \mu) = \sint(h_1, \mu) + \sint(h_2, \mu)
            .\]
            Note then that
            \[
                \sint(h_1, \mu) + \sint(h_2, \mu) = \sint(h_1 + h_2, \mu) \leq \sup_{\tilde{h} \leq f + g} \sint(\tilde{h}, \mu) = \int_X (f + g) d \mu
            .\]
            Thus taking the supremum over $h_1$ and $h_2$ we get
            \[
                \int_X f d \mu + \int_X g d \mu \leq \int_X (f+g) d \mu
            .\]

        \item[(vii)]
            Suppose that $\int_X f d \mu < \infty$. Let $E = \qty{x \in X : f(x) = \infty}$. Note that for each $n \in \N$ that $n \cdot \ind{E} \leq f$. Therefore
            \[
                n \mu(E) = \sint(n \cdot \ind{E}, \mu) \leq \int_X f d \mu
            .\]
            If $\mu(E) > 0$, then $\lim_{n \to \infty} n \mu(E) = \infty$, giving $\int_X f d \mu = \infty$, a contradiction. Therefore $\mu(E) = 0$, hence $f < \infty$ $\mu$-a.e.

        \item[(ix)]
            Let $f_n = \min(f, n)$. Note that each $f_n$ is measurable and $\lim_{n \to \infty} f_n = f$ pointwise. Since $f_n \leq f_{n+1}$ and $f_n \leq f$ for all $n$, it follows
            \[
                \lim_{n \to \infty} \int_X f_n d \mu \leq \int_X f d \mu
            .\]
            Consider then a simple function $h \leq f$.
            \begin{itemize}
                \item 
                    If $h = \infty$ on a set of positive measure $E$, then we have from montonicity that $\int_X f d \mu = \infty$. Note also that for any $r > 0$ there then must be some $N$ large such that for $n \geq N$ we have $f_n(x) \geq r$ for all $x \in E$. But this means that
                    \[
                        r \cdot \mu(E) \leq \int_X f_N d \mu
                    .\]
                    Since $r$ was arbitrary, this means that $\lim_{n\to \infty} \int_X f_n d \mu = \infty$.

                \item
                    If $h = \infty$ on a null set, then we can simply set $h$ to $0$ on the null set and get the same integral.
            \end{itemize}

            We can assume then $h$ is finite everywhere. Thus there exists some $N$ large such that for $n \geq N$, $h \leq f_n$. Therefore taking the limit and then supremum gives 
            \[
                \int_X f d \mu \leq \lim_{n \to \infty} \int_X f_n d \mu
            .\]

        \item[(x)]
            Let $f_n = f \ind{E_n}$ and $E = \bigcup_{n \in \N} E_n$. Note that $f_n \leq f \ind{E}$ for all $n$, therefore by montonicity
            \[
                \lim_{n \to \infty} \int_X f_n d \mu \leq \int_X f \ind{E} d \mu
            .\]

        \item[(xi)]
            Note that if $h$ is a simple function on $E$ such that $h \leq f$ on $E$, then extending $h$ to $\tilde{h}$ that is $0$ on $E^c$, we get $\tilde{h} \leq f \ind{E}$. Furthermore, if $g$ is a simple function on $X$ with $g \leq f \ind{E}$, then $\supp(g) = E$ and thus the restriction of $\tilde{g}$ of $g$ to $E$ is simple and $\tilde{g} \leq f$ on $E$. This establishes a bijection between simple functions $h \leq f$ on $E$ and simple function $g \leq f \ind{E}$ on $X$. Since $\sint(h, \mu\vert_E) = \sint(\tilde{h}, \mu)$ and $\sint(g, \mu) = \sint(\tilde{g}, \mu\vert_E)$, we have
            \[
                \int_X f \ind{E} d \mu = \sup_{g \leq f \ind{E}} \sint(g, \mu) = \sup_{h \leq f \text{ on } E} \sint(h, \mu\vert_E) = \int_E f d \mu\vert_E
            .\]
    \end{enumerate}
\end{proof}

\section*{Exercise 4.61}

\begin{proof}
    \begin{enumerate}
        \item
            To show $\phi_* \mu(E)$ is a measure, we check the axioms. Let $E \in \mathcal{B}$.
            \begin{itemize}
                \item $\phi_* \mu(E) = \mu(\phi^{-1}(E)) \geq 0$ since $\mu$ is non-negative.
                \item $\phi_* \mu(\varnothing) = \mu(\phi^{-1}(\varnothing)) = \mu(\varnothing) = 0$
                \item Let $E_n \in \mathcal{B}^\N$ be pairwise disjoint. Note that the preimages $\phi^{-1}(E_n)$ are therefore also disjoint and
                    \[
                        \phi^{-1}\qty(\bigcup_{n \in \N} E_n) = \bigcup_{n \in \N} \phi^{-1}(E_n)
                    .\]
                    Thus
                    \[
                        \phi_* \mu\qty(\bigcup_{n \in \N} E_n) = \mu\qty(\bigcup_{n \in \N} \phi^{-1}(E_n)) = \sum_{n \in \N} \mu(\phi^{-1}(E_n)) = \sum_{n \in \N} \phi_* \mu(E_n)
                    .\]
                    Hence countable additivity holds.
            \end{itemize}

        \item 
            First consider some simple function $s = \sum_{k} c_k \ind{E_k}$ over $Y$. Note that $\phi^{-1}\circ\ind{E_k} = \ind{\phi^{-1}(E_k)}$, thus
            \[
                \int_Y s d(\phi_* \mu) = \sum_{k} c_k \phi_* \mu(E_k) = \sum_{k} c_k \mu(\phi^{-1}(E_k)) = \int_X s \circ \phi d \mu
            .\]
            Note if $s \leq f$ that both $s \circ \phi$ is simple and $s \circ \phi \leq f \circ \phi$. Therefore by the previous result
            \[
                \int_Y s d(\phi_* \mu) = \int_X s \circ \phi d \mu \leq \int_X f \circ \phi d \mu
            .\]
            Taking the supremum of the left side gives then
            \[
                \int_Y f d(\phi_* \mu) \leq \int_X f \circ \phi d \mu
            .\]
            Take then $t \leq f \circ \phi$ simple over $X$ where $t = \sum_{j} d_j \ind{F_j}$. Note that for each $j$ there is some $F_j = \phi^{-1}(B_j)$. Thus $t$ can be rewritten as
            \[
                t = \sum_{j} d_j \ind{\phi^{-1}(B_j)} = \sum_{j} d_j \ind{B_j} \circ \phi = \qty(\sum_j d_j \ind{B_j}) \circ \phi
            .\]
            Therefore if $h = \sum_{j} d_j \ind{B_j}$, $h$ is a simple function on $Y$ and $t = h \circ \phi$. Since $t \leq f \circ \phi$, then $h \circ \phi \leq f \circ \phi$ and thus $h \leq f$ on $\phi(X)$. Setting $h = 0$ outside of $\phi(X)$ gives then $h \leq f$ everywhere on $Y$. Therefore
            \[
                \int_X t d \mu = \int_X (h \circ \phi) d \mu = \int_Y h d(\phi_* \mu) \leq \int_Y f d(\phi_* \mu)
            .\]
            Taking the supremum over all $t$ thus gives
            \[
                \int_X (f \circ \phi) d \mu \leq \int_Y f d(\phi_* \mu)
            .\]

            If $T : \R^d \to \R^d$ is invertible, from previous HW we have for every Lebesgue measurable set $E$ that
            \[
                T_* m(E) = m(T^{-1}(E)) = \frac{1}{|\det T|} m(E)
            .\]
            Therefore $T_* m = \frac{1}{|\det T|} m$.
    \end{enumerate}
\end{proof}

\section*{Exercise 4.65}

\begin{proof}
    Suppose $f_n \to f$ uniformly and take $\eps > 0$. Then there exists some $N$ such that for all $n \geq N$ and $x \in X$
    \[
        -\eps < f_n(x) - f(x) < \eps \implies f_n(x) - \eps < f(x) < f_n(x) + \eps
    .\]
    Since uniform convergence implies pointwise, it follows $f$ is measurable. Thus by montonicity of the integral we then have
    \[
        \int_X (f_n - \eps) d \mu < \int_X f d \mu < \int_X (f_n + \eps) d \mu
    .\]
    Since $\mu(X) < \infty$, we get
    \[
        \int_X f_n d \mu - \frac{\eps}{\mu(X)} < \int_X f_n d \mu + \frac{\eps}{\mu(X)} \implies -\frac{\eps}{\mu(X)} < \int_X f_n d \mu - \int_X f d \mu < \frac{\eps}{\mu(X)}
    .\]
    As $\eps \to 0$, then $n \to \infty$ and the difference between the two goes to $0$. Therefore
    \[
        \lim_{n \to \infty} \int_X f_n d \mu = \int_X f d \mu
    .\]
\end{proof}

\section*{Exercise 4.73}

\begin{proof}
    \begin{enumerate}
        \item 
            Note that
            \[
                \limsup_{n \to \infty} E_n = \bigcap_{n \in \N} \bigcup_{k \geq n} = \qty{ x \in X : x \in E_n \text{ for infinitely many } n }
            .\]
            Thus an equivalent result is to show that
            \[
                \mu\qty(\limsup_{n \to \infty} E_n) = 0
            .\]
            Note that by montonicity and subadditivity
            \[
                \mu(\limsup_{n \to \infty} E_n) = \mu\qty(\bigcap_{n \in \N} \bigcup_{k \geq n} E_k) \leq \mu\qty(\bigcup_{k \geq n} E_k) \leq \sum_{k \geq n} \mu(E_k)
            \]
            for all $n \in \N$. Since $\sum_n \mu(E_n) < \infty$, then the tail of the sum goes to $0$. Therefore
            \[
                \mu(\limsup_{n \to \infty} E_n) \leq \sum_{k \geq n} \mu(E_k) \to 0 \; (k \to \infty)
            .\]
            Thus $\mu(\limsup_{n \to \infty} E_n) = 0$.

        \item 
            We will work in the interval $[0,1]$ equipped with the Lebesgue measure. Let $H_n$ be the $n^\text{th}$ harmonic number. Let $E_n$ denote the segment with $m(E_n) = \frac{1}{n+1}$ starting at $x = H_n$ that "wraps" around the endpoints of $[0,1]$. That is an interval $[a,b]$ with $a < 1$ and $b > 1$ turns into $[a,1] \cup [0, b \Mod{1}]$. Since $\lim_{n \to \infty} H_n = \infty$, it follows that the $E_n$ will cover every point infinitely many times, however $\lim_{n \to \infty} \mu(E_n) = 0$.
    \end{enumerate}
\end{proof}

\section*{Exercise 4.81}

\begin{proof}
    Let $g_n = \min(f_n, f)$. Note that both $g_n \leq f$ everywhere and $g_n \to f$ as $n \to \infty$. Since $f$ is $\mathcal{L}^1$, then DCT applies to $g_n$ meaning
    \[
        \lim_{n \to \infty} \int_X g_n d \mu = \int_X f d \mu
    .\]
    Note that $f_n - f - |f - f_n| = 2 \cdot (f_n - g_n)$, therefore
    \[
        \int_X f_n d \mu - \int_X f d \mu - \int_X |f - f_n| d \mu = 2 \cdot \qty(\int_X f_n d \mu - \int_X g_n d \mu)
    .\]
    Since the RHS goes to $0$ as $n \to \infty$, it follows that
    \[
        \int_X f_n d \mu - \int_X f d \mu - \norm{f - f_n}_{L_1(\mu)} \to 0 \quad (n \to \infty)
    .\]
\end{proof}

\section*{Exercise 4.82}

\begin{proof}
    Consider the axioms of a measure
    \begin{itemize}
        \item 
            Since $g$ is non-negative, the integral of $g$ over any set will also be non-negative. Thus $\mu_g(E) \geq 0$ for all $E \in \mathcal{A}$.

        \item 
            Let $E,F \in \mathcal{A}$ with $E \subset F$. Note that $g\cdot\ind{E} \leq g\cdot\ind{F}$, therefore
            \[
                \mu_g(E) = \int_E g d \mu = \int_X g \cdot \ind{E} d \mu \leq \int_X g \cdot \ind{F} d \mu = \int_F g d \mu = \mu_g(F)
            .\]

        \item 
            Let $(E_n) \in \mathcal{A}^\N$ be disjoint and $E = \bigcup_{n \in \N} E_n$. Note that the sequence $F_n = \bigcup_{k \leq n} E_n$ is an increasing sequence and $\lim_{n \to \infty} F_n = E$. Therefore
            \[
                \lim_{n \to \infty} \int_X g \cdot \ind{F_n} d \mu = \int_X g \cdot \ind{E} d \mu = \mu_g(E)
            .\]
            Since $\ind{A \cup B} = \ind{A} + \ind{B}$ when $A$ and $B$ are disjoint, it follows by linearity that
            \[
               \int_X g \cdot \ind{F_n} = \int_X g \cdot \qty(\sum_{k \leq n} \ind{E_n}) d \mu = \sum_{k \leq n} \int_X g \cdot \ind{E_n} d \mu = \sum_{k \leq n} \int_{E_n} g d \mu = \sum_{k \leq n} \mu_g(E_n)
            .\]
            In total then
            \[
                \mu_g(E) = \lim_{n \to \infty} \int_X g \cdot \ind{F_n} d \mu = \sum_{n \in \N} \mu_g(E_n)
            \]
            hence $\mu_g$ is countably additive.
    \end{itemize}
    Thus $\mu_g$ is a measure.
\end{proof}

\end{document}
