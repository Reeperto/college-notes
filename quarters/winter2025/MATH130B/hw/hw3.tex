\documentclass{eeleyes}

\let\originalleft\left
\let\originalright\right
\renewcommand{\left}{\mathopen{}\mathclose\bgroup\originalleft}
\renewcommand{\right}{\aftergroup\egroup\originalright}

\NewDocumentCommand{\prob}{m}{\mathbb{P}\left[#1 \right]}
\NewDocumentCommand{\expp}{m}{\mathbb{E}\left[#1 \right]}
\DeclareMathOperator{\varr}{Var}

\def\thehwnumber{3}
\usepackage{fancyhdr}
\pagestyle{fancy}
\fancyhead[R]{HW \#\thehwnumber}
\fancyhead[C]{\textbf{Math 130B}}
\fancyhead[L]{Eli Griffiths}


\begin{document}

\section*{Problem 1}
We have the following probability mass functions for $X$ and $Y$
\[
    \prob{X = i} = \binom{n}{i} p^i (1-p)^{n - i} \qquad \prob{Y = j} = \binom{m}{j} p^j (1-p)^{m - j}
.\]
Therefore considering the probability mass function of $X + Y$ we have
\begin{align*}
    \prob{X + Y = k} &= \sum_{i = 0}^k \prob{X = k - i, Y = i} \\
                     &= \sum_{i = 0}^k \prob{X = k - i} \prob{Y = i} \\
                     &= \sum_{i = 0}^k \qty(\binom{n}{k-i} p^{k-i} (1-p)^{n-k+i}) \qty(\binom{m}{i} p^i (1-p)^{m-i}) \\
                     &= p^k (1-p)^{n+m-k}  \sum_{i = 0}^k \binom{n}{k-i} \binom{m}{i} \\
                     &= \binom{n+m}{k} p^k (1-p)^{n+m-k}
\end{align*}
which is the probability mass function of $\operatorname{Binomial}(n+m, p)$.


\section*{Problem 2}
\subsection*{Part A}
By the definition of a conditional distribution,
\[
    p_{X | Y}(x | 3) = \frac{p(x,3)}{p_{Y}(3)} = \frac{p(x,3)}{0.05 + 0.1 + 0.35} = 2 p(x,3)
.\]
Therefore
\begin{align*}
    p_{X | Y}(1 \mid 3) &= \frac{1}{10} \\
    p_{X | Y}(2 \mid 3) &= \frac{2}{10} \\
    p_{X | Y}(3 \mid 3) &= \frac{7}{10} \\
\end{align*}

\subsection*{Part B}
Again by the definition of a conditional distribution,
\[
    p_{Y | X}(y | 2) = \frac{p(2,y)}{p_{X}(2)} = \frac{p(2,y)}{0.2 + 0.1 + 0.05} = \frac{20 p(2,y)}{7}
.\]
Therefore
\begin{alignat*}{3}
    p_{Y | X}(1 \mid 2) &= \frac{2}{10} \cdot \frac{20}{7} &&= \frac{4}{7} \\
    p_{Y | X}(3 \mid 2) &= \frac{1}{10} \cdot \frac{20}{7} &&= \frac{2}{7} \\
    p_{Y | X}(5 \mid 2) &= \frac{1}{20} \cdot \frac{20}{7} &&= \frac{1}{7}
\end{alignat*}

\subsection*{Part C}
No they are not the same. We have
\[
    p_{Y|X}(3 \mid 2) = \frac{1}{7} \neq \frac{2}{10} = p_{X | Y}(2 \mid 3)
.\]

\section*{Problem 3}
\subsection*{Part A}
We can obtain the marginal density by integrating the joint density over the possible values of $y$. Thus
\begin{align*}
    f_X(x) &= \int_{0 < x < y} f(x,y) \dd y \\
           &= \int_{x}^\infty e^{-y} \dd y \\
           &= -e^{-y} \eval_{x}^\infty \\
           &= e^{x}, \quad x > 0
\end{align*}

\subsection*{Part B}
\[
    f_{Y|X}(y | x) = \frac{f(x,y)}{f_X(x)} = e^{-(x+y)}, \quad y > x
.\]

\section*{Problem 4}

\subsection*{Part A}
Since $U$ is uniform on $(0,1)$, its probability density is simply $f_{U}(u) = 1$ for $0 < u < 1$. Note that
\[
    \prob{U > a} = \int_a^1 1 \dd u = 1 - a
.\]
Therefore for $a < u < 1$
\[
    f_{U|U > a}(u) = \frac{f_{U}(u)}{\prob{U > a}} = \frac{1}{1 - a}
.\]

\subsection*{Part B}
In a similar manner to (A), we have
\[
    \prob{U < a} = \int_0^a 1 \dd u = a
.\]
Therefore for $0 < u < a$
\[
    f_{U|U < a}(u) = \frac{f_{U}(u)}{\prob{U < a}} = \frac{1}{a}
.\]

\end{document}
