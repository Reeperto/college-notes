\documentclass{eeleyes}

\let\originalleft\left
\let\originalright\right
\renewcommand{\left}{\mathopen{}\mathclose\bgroup\originalleft}
\renewcommand{\right}{\aftergroup\egroup\originalright}

\NewDocumentCommand{\prob}{m}{\mathbb{P}\left[#1 \right]}
\NewDocumentCommand{\expp}{m}{\mathbb{E}\left[#1 \right]}
\DeclareMathOperator{\varr}{Var}

\def\thehwnumber{1}
\usepackage{fancyhdr}
\pagestyle{fancy}
\fancyhead[R]{HW \#\thehwnumber}
\fancyhead[C]{\textbf{Math 130B}}
\fancyhead[L]{Eli Griffiths}


\begin{document}

\section*{Problem 1}
\subsection*{Part A}
Note that you are guranteed to play at least $5$ games. If the fifth game is won, then the expected number of additional games played is $\frac{1}{1-p}$ as the number of additional games played is described by a geometric random variable where the probability of a success ending the trials is $1-p$ (a loss has probability $1-p$). If the fifth game is lost, then the expected number of additional games played is simply $0$. The probability that the fifth game is won is $p$ meaning the expected number of games played is 
\[
    5 + p \cdot \frac{1}{1-p} + p \cdot 0 = 5 + \frac{p}{1-p}
.\]

\subsection*{Part B}
The expected number of games lost is the number of games lost in the first $5$ and then $1$ more loss if the fifth game is won. That is because the additional games only end once you lose, hence playing additional games will only contribute one loss. The expected number of losses in the the first $5$ games is simply a binomial distribution with probability $1-p$. The probability of winning the fifth game and hence adding another loss is $p$ and the probability of losing the fifth game and not adding another loss is $1-p$. Therefore the expected number of games lost is
\[
    5 \cdot (1-p) + 1 \cdot p + 0 \cdot (1 - p) = 5 - 4p
.\]


\section*{Problem 2}
\subsection*{Part A}
In order for the maximum number on a ball to be a given value $x$, the other two balls must have values less than $x$ written on them. Therefore the probability of a given value being the maximum is the probability that two other smaller value balls are drawn. There are $\binom{x-1}{2}$ possible choices for two balls with value smaller than $x$ and $\binom{20}{3}$ possible choices of three balls. Thus

\[
    f_X(x) = \frac{\binom{x - 1}{2}}{\binom{20}{3}}
.\]

\subsection*{Part B}
Using the definition of expectation we have
\begin{align*}
    \expp{X} &= \sum_{x=3}^{20} x f_X(x) \\
             &= \frac{1}{\binom{20}{3}} \sum_{x=3}^{20} x \cdot \binom{x-1}{2} \\
             &= \frac{1}{\binom{20}{3}} \sum_{x=3}^{20} 3 \cdot \binom{x}{3} \\
             &= \frac{3 \cdot \binom{21}{4}}{\binom{20}{3}} \\
             &= \frac{3 \cdot \frac{21}{4} \cdot \binom{20}{3}}{\binom{20}{3}} \\
             &= \frac{63}{4} = 15.75
\end{align*}

\section*{Problem 3}
Let $M$ be the event that $A$ gets more heads after $n+1$ flips than $B$ after $n$ flips, $H_A$ be the event that $A$ gets more heads than $B$ after both do $n$ flips, $H_B$ the event $B$ has more heads after $n$ flips, and $H_0$ the event that both have the same number of heads. Since $H_A, H_B, H_0$ are all mutually exclusive and cover all cases, by the law of total probability, we have
\[
    \prob{M} = \prob{M \mid H_A} \prob{H_A} + \prob{M \mid H_B} \prob{H_B} + \prob{M \mid H_0} \prob{H_0}
.\]
Note that
\begin{itemize}
    \item $\prob{M \mid H_A} = 1$ since $A$ would already have more heads than $B$, thus the $n+1$ flip will not change the outcome that $A$ has more
    \item $\prob{M \mid H_B} = 0$ since $A$ is at least $1$ head flip behind $B$ and thus cannot overtake $B$ in terms of heads with an additional flip
    \item $\prob{M \mid H_0} = \frac{1}{2}$ since $A$ will have more heads than $B$ if a heads is flipped which occurs with probability $\frac{1}{2}$
    \item $\prob{H_A} = \prob{H_B}$ since the coin is fair and thus the outcome of either $A$ or $B$ having more heads is symmetric
    \item $\prob{H_0} = 1 - \prob{H_A \cup H_B} = 1 - \prob{H_A} - \prob{H_B} = 1 - 2 \prob{H_A}$
\end{itemize}
Therefore in total we have
\begin{align*}
    \prob{M} &= \prob{M \mid H_A} \prob{H_A} + \prob{M \mid H_B} \prob{H_B} + \prob{M \mid H_0} \prob{H_0} \\
             &= 1\cdot(\prob{H_A}) + 0 \cdot(\prob{H_B}) + \frac{1}{2} \cdot (1 - 2\prob{H_A}) \\
             &= \prob{H_A} + \frac{1}{2} - \prob{H_A} = \boxed{\frac{1}{2}}
\end{align*}

\section*{Problem 4}
First consider the scenario that the barrel is spun again before shooting. Since the barrel is being respun, the first attempt by the opponent has no effect as a new random selection is being made. Therefore the probability of living is $\frac{4}{6}$ since there are $4$ empty slots of the $6$ total chambers.

Now consider the scenario in which the barrel is not spun again before shooting. Then the probability of being shot is the probability that a given empty chamber is followed by the adjacent loaded chambers. Since there is only $1$ empty chamber out of $4$ that is followed by the loaded chambers, the probability of getting shot is $\frac{1}{4}$ meaning a $\frac{3}{4}$ probability of survival.

\section*{Problem 5}
\subsection*{Part A}

We can use the following result to re-express $A^{(n)}$.

\begin{theorem}[]
    There exists a semicircle containing the points $P_1, \ldots, P_n$ if and only if there is a semicircle with an endpoint as some $P_i$ containing all the points.
\end{theorem}

\begin{proof}
    The reverse implication is trivial. Assume that there exists a semicircle $C$ containing all points $P_i$. Without loss of generality, take $P_1, \ldots, P_n$ in clockwise order. Since all points are contained in $C$, can rotate $C$ clockwise until one of the endpoints is at $P_1$. Call this new semicircle $C'$. Note all points must be in $C'$ since the point furthest around the circle clockwise, $P_n$, is contained in $C$ and in front $P_1$, and thus moving $C$ to $C'$ will not cause $P_n$ to no longer be contained. Thus since $P_1$ and $P_n$ are in $C'$ and all other points are between $P_1$ and $P_n$, all points are contained in $C'$.
\end{proof}
Therefore we can identify the event that such a semicircle exists for a set of randomnly sampled points with the events that a semicircle starting at one the points contains all other points, giving
\[
    A^{(n)} = \bigcup_{i = 1}^n A_{i}^{(n)}
.\]

\subsection*{Part B}
Note that
\[
    \prob{A_i^{(n)}} = \qty(\frac{1}{2})^{n-1} = \frac{1}{2^{n-1}}
\]
since the other $n-1$ uniformly selected points have $\frac{1}{2}$ probability to be in the semicircle each (due to the fact that the semicircle is half the circumference and the points are sample uniformly along the circumference). Using part $A$ we get
\[
    \prob{A^{(n)}} = \prob{\bigcup_{i} A_{i}^{(n)}} = \sum_{i} \prob{A_i^{(n)}} + \ldots
\]
with the extra terms being probabilities of intersections of different $A_i^{(n)}$ due to the inclusion exclusion principle. Note that $\prob{A_i^{(n)} \cap A_j^{(n)}} = 0$ for any $i,j$. If $P_i \neq P_j$, then if $P_i$ is in the semicricle starting at $P_j$, $P_j$ cant be in the semicircle starting at $P_i$ and vice versa. Therefore the only possible case in which you have both semicircles containing all points is if $P_i = P_j$. However since the distribution of both points is continuous, the probability of this occuring is $0$. Therefore $\prob{A_i^{(n)} \cap A_j^{(n)}} = 0$. Clearly then any collection of intersections of $A_i^{(n)}$ will also be $0$, eliminating all the extra terms. Therefore we have

\[
    \prob{A^{(n)}} = \sum_{i=1}^n \frac{1}{2^{n-1}} = \frac{n}{2^{n-1}}
.\]


\end{document}
