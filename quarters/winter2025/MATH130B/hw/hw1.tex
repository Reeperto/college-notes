\documentclass{eeleyes}

\usepackage{physics}

\usepackage{fancyhdr}
\pagestyle{fancy}
\fancyhead[R]{HW \#1}
\fancyhead[C]{\textbf{Math 130B}}
\fancyhead[L]{Eli Griffiths}

\NewDocumentCommand{\prob}{m}{\mathbb{P}\qty[#1]}
\NewDocumentCommand{\expp}{m}{\mathbb{E}\qty[#1]}

\begin{document}

\section*{Problem 1}
\subsection*{Part A}
Let $Y$ be the number of games played. Note that $Y = X + 5$ where $X$ is a geometric distribution with probability $1-p$. This is because we are guaranteed to play $5$ games and then continually play until the first failure (which has probability $1 - p$). Thus the expected number of games played is
\[
    \expp{Y} = \expp{X + 5} = \expp{X} + 5 = \frac{1}{1 - p} + 5
.\]

\subsection*{Part B}
Let $Y$ be the number of games lost. If $X$ is a binomial distribution with probability $1 - p$ and trial count $5$ then $Y = X + 1$. That is, the number of games lost is the number of games lost in the first $5$ plays plus the loss that stops further play. Thus the expected number of games lost is
\[
    \expp{Y} = \expp{X + 1} = \expp{X} + 1 = n(1 - p) + 1
.\]

\section*{Problem 2}
\subsection*{Part A}
In order for the maximum number on a ball to be a given value $x$, the other two balls must have values less than $x$ written on them. Therefore the probability of a given value being the maximum is the probability that two other smaller value balls are drawn. There are $\binom{x-1}{2}$ possible choices for two balls with value smaller than $x$ and $\binom{20}{3}$ possible choices of three balls. Thus

\[
    f_X(x) = \frac{\binom{x - 1}{2}}{\binom{20}{3}}
.\]

\subsection*{Part B}


\section*{Problem 3}
Let $M$ be the event that $A$ gets more heads after $n+1$ flips than $B$ after $n$ flips, $H_A$ be the event that $A$ gets more heads than $B$ after both do $n$ flips, $H_B$ the event $B$ has more heads after $n$ flips, and $H_0$ the event that both have the same number of heads. Since $H_A, H_B, H_0$ are all mutually exclusive and cover all cases, by the law of total probability, we have
\[
    \prob{M} = \prob{M \mid H_A} \prob{H_A} + \prob{M \mid H_B} \prob{H_B} + \prob{M \mid H_0} \prob{H_0}
.\]
Note that
\begin{itemize}
    \item $\prob{M \mid H_A} = 1$ since $A$ would already have more heads than $B$, thus the $n+1$ flip will not change the outcome that $A$ has more
    \item $\prob{M \mid H_B} = 0$ since $A$ is at least $1$ head flip behind $B$ and thus cannot overtake $B$ in terms of heads with an additional flip
    \item $\prob{M \mid H_0} = \frac{1}{2}$ since $A$ will have more heads than $B$ if a heads is flipped which occurs with probability $\frac{1}{2}$
    \item $\prob{H_A} = \prob{H_B}$ since the coin is fair and thus the outcome of either $A$ or $B$ having more heads is symmetric
    \item $\prob{H_0} = 1 - \prob{H_A \cup H_B} = 1 - \prob{H_A} - \prob{H_B} = 1 - 2 \prob{H_A}$
\end{itemize}
Therefore in total we have
\begin{align*}
    \prob{M} &= \prob{M \mid H_A} \prob{H_A} + \prob{M \mid H_B} \prob{H_B} + \prob{M \mid H_0} \prob{H_0} \\
             &= 1\cdot(\prob{H_A}) + 0 \cdot(\prob{H_B}) + \frac{1}{2} \cdot (1 - 2\prob{H_A}) \\
             &= \prob{H_A} + \frac{1}{2} - \prob{H_A} = \boxed{\frac{1}{2}}
\end{align*}

\section*{Problem 4}
First consider the scenario that the barrel is spun again before shooting. Since the barrel is being respun, the first attempt by the opponent has no effect as a new random selection is being made. Therefore the probability of living is $\frac{4}{6}$ since there are $4$ empty slots of the $6$ total chambers.

Now consider the scenario in which the barrel is not spun again before shooting. Then the probability of being shot is the probability that a given empty chamber is followed by the adjacent loaded chambers. Since there is only $1$ empty chamber out of $4$ that is followed by the loaded chambers, the probability of getting shot is $\frac{1}{4}$ meaning a $\frac{3}{4}$ probability of survival.

\section*{Problem 5}
\subsection*{Part A}

We can use the following result to re-express $A^{(n)}$.

\begin{theorem}[]
    There exists a semicircle containing the points $P_1, \ldots, P_n$ if and only if there is a semicircle with an endpoint as some $P_i$ containing all the points.
\end{theorem}

\begin{proof}
    The reverse implication is trivial. Assume that there exists a semicircle $C$ containing all points $P_i$. Without loss of generality, take $P_1, \ldots, P_n$ in clockwise order. Since all points are contained in $C$, can rotate $C$ clockwise until one of the endpoints is at $P_1$. Call this new semicircle $C'$. Note all points must be in $C'$ since the point furthest around the circle clockwise, $P_n$, is contained in $C$ and in front $P_1$, and thus moving $C$ to $C'$ will not cause $P_n$ to no longer be contained. Thus since $P_1$ and $P_n$ are in $C'$ and all other points are between $P_1$ and $P_n$, all points are contained in $C'$.
\end{proof}
Therefore we can identify the event that such a semicircle exists for a set of randomnly sampled points with the events that a semicircle starting at one the points contains all other points, giving
\[
    A^{(n)} = \bigcup_{i = 1}^n A_{i}^{(n)}
.\]

\subsection*{Part B}
Using part $A$ we get
\[
    \prob{A^{(n)}} = \prob{\bigcup_{i} A_{i}^{(n)}} = \sum_{i} \prob{A_i^{(n)}} + O(1)
.\]
We will have exact equality without any other factors if we have pairwise mutually exclusivity between the $A_i^{(n)}$. 

\[
    \prob{A_i^{(n)}} = \qty(\frac{1}{2})^{n-1} = \frac{1}{2^{n-1}}
\]
since the other $n-1$ uniformly selected points have $\frac{1}{2}$ probability to be in the semicircle each (due to the fact that the semicircle is half the circumference and the points are sample uniformly along the circumference). Therefore
\[
    \prob{A^{(n)}} = \sum_{i=1}^n \frac{1}{2^{n-1}} = \frac{n}{2^{n-1}}
.\]


\end{document}
