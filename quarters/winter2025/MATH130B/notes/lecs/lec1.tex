\documentclass{subfiles}

\begin{document}

\chapter{Review+}

\begin{definition}[Probability Space]
    A probability space is a triple $(\Omega, \mathcal{F}, P)$ such that
    \begin{enumerate}
        \item $\mathcal{F} \subseteq \mathcal{P}(\Omega)$
        \item $\forall E \in \mathcal{F}, 0 \leq P(E) \leq 1$
        \item $P(\Omega) = 1$
        \item If $E_1, E_2, \ldots, E_n \in \mathcal{F}$ are pairwise disjoint then $\displaystyle P(\cup_i E_i) = \sum_i P(E_i)$
    \end{enumerate}
    $\Omega$ is denote as the sample space, $\mathcal{F}$ the events, and $P$ the probability function.
\end{definition}

\begin{definition}[Conditional Probability]
    For events $E, F \in \mathcal{F}$, the conditional probability is
    \[
        \prob{E | F} \coloneq \frac{\prob{E \cap F}}{\prob{F}}
    .\]
\end{definition}

\begin{definition}[Independence]
    Two events $E,F \in \mathcal{F}$ are \textbf{independent} if 
    \[
     \prob{E | F} = \prob{E} \Leftrightarrow \prob{E \cap F} = \prob{E} \prob{F}
    .\]
    A collection of events $E_1, \ldots, E_n$ are \textbf{mutually independent} if
    \[
        \forall I \subseteq [n], \prob{\cap_{i \in I} E_i} = \prod_{i \in I} \prob{E_i}
    .\]
\end{definition}

\begin{theorem}[Law of Total Probability]
    \label{thm:totalprobability}
    Suppose that $F_1, \ldots, F_n \in \mathcal{F}$ are disjoint and $\Omega = \cup_{i} F_i$. Then for any other event $E \in \mathcal{F}$,
    \[
        \prob{E} = \sum_{i} \prob{E | F_i} \cdot \prob{F_i}
    .\]
\end{theorem}

\begin{example}
    Suppsoe that $S \subseteq [n]$ be a randomly chosen subset. We want to find $\prob{|S| \text{ is even}}$. Take $E_i$ to be the event that $i \in S$. Note that all $E_i$ are mutually independent for $1 \leq i \leq n$. Take
    \[
        F_{\conj{x}} = \bigcap_{i=1}^{n-1} E_i^{x_i}
    .\]
    where $\overline{x}$ is a string of either complements or nothing. Note then that
    \[
        \prob{|S| \text{ is even}} = \sum_{\conj{x}} \prob{|S| \text{ is even} \mid F_{\conj{x}}} \cdot \prob{F_{\conj{x}}}
    .\]
    Note that $\prob{|S| \text{ is even} \mid F_{\conj{x}}} = \frac{1}{2}$ since there are only two possible outcomes for adding the $n^\text{th}$ element and only one makes $|S|$ even. Furthermore since all $F_{\conj{x}}$ across all strings $\conj{x}$ are disjoint and cover $\Omega$, their total probability is $1$. Hence
    \[
        \prob{|S| \text{ is even}} = \frac{1}{2} \cdot 1 = \frac{1}{2}
    .\]
\end{example}

\end{document}
