\documentclass[12pt,titlepage]{extarticle}
% Document Layout and Font
\usepackage{subfiles}
\usepackage[margin=2cm, headheight=15pt]{geometry}
\usepackage{fancyhdr}
\usepackage{enumitem}	
\usepackage{wrapfig}
\usepackage{multicol}
\usepackage{caption, subcaption}

\usepackage[p,osf]{scholax}

\renewcommand*\contentsname{Table of Contents}
\renewcommand{\headrulewidth}{0pt}
\pagestyle{fancy}
\fancyhf{}
\fancyfoot[R]{$\thepage$}
\setlength{\parindent}{0cm}
\setlength{\headheight}{17pt}
\hfuzz=9pt

% Utility Management
\usepackage{color}
\usepackage{colortbl}
\usepackage{xcolor}
\usepackage{xpatch}
\usepackage{xparse}

\definecolor{links}{HTML}{1c73a5}
\definecolor{bar}{HTML}{584AA8}

% Math Packages
\usepackage{mathtools, amsmath, amsthm, thmtools, amssymb, physics}
\usepackage[scaled=1.075,ncf,vvarbb]{newtxmath}

\newcommand\B{\mathbb{B}}
\newcommand\C{\mathbb{C}}
\newcommand\R{\mathbb{R}}
\newcommand\Q{\mathbb{Q}}
\newcommand\N{\mathbb{N}}
\newcommand\Z{\mathbb{Z}}

\newcommand\Prob[1]{\mathbb{P}\qty(#1)}
\newcommand\Var[1]{\text{Var}\qty(#1)}
\newcommand\Exp[1]{\mathbb{E}\qty[#1]}
\newcommand\ball[1]{\B\qty(#1)}
\newcommand\res[1]{\underset{#1}{\operatorname{Res}}\;}
\renewcommand\pv{\mathrm{p.v.}}

\newcommand\conj[1]{\overline{#1}}
\DeclareMathOperator{\Arg}{Arg}
\DeclareMathOperator{\Log}{Log}
\DeclareMathOperator{\cis}{cis}

\DeclareMathOperator{\dom}{dom}
\DeclareMathOperator{\spann}{span}
\DeclareMathOperator{\nullity}{nullity}

\newcommand\st{\text{ s.t. }}

% TIKZ
\usepackage{tikz}
\usepackage{pgfplots}
\usetikzlibrary{arrows.meta}
\usetikzlibrary{math}
\usetikzlibrary{cd}
\usetikzlibrary{patterns}
\usetikzlibrary{decorations.markings}
\usetikzlibrary{calc}

% Boxes and Theorems
\usepackage[most]{tcolorbox}
\tcbuselibrary{skins}
\tcbuselibrary{breakable}
\tcbuselibrary{theorems}

\newtheoremstyle{default}{0pt}{0pt}{}{}{\bfseries}{\normalfont.}{0.5em}{}
\theoremstyle{default}

\renewcommand*{\proofname}{\textit{\textbf{Proof.}}}
\renewcommand*{\qedsymbol}{$\blacksquare$}
\tcolorboxenvironment{proof}{
	breakable,
	coltitle = black,
	colback = white,
	frame hidden,
	boxrule = 0pt,
	boxsep = 0pt,
	borderline west={3pt}{0pt}{bar},
	sharp corners = all,
	enhanced,
}

\newtheorem{theorem}{Theorem}[section]{\bfseries}{}
\tcolorboxenvironment{theorem}{
	breakable,
	enhanced,
	boxrule = 0pt,
	frame hidden,
	coltitle = black,
	colback = blue!7,
	left = 0.5em,
	sharp corners = all,
}

\newtheorem{corollary}{Corollary}[section]{\bfseries}{}
\tcolorboxenvironment{corollary}{
	breakable,
	enhanced,
	boxrule = 0pt,
	frame hidden,
	coltitle = black,
	colback = white!0,
	left = 0.5em,
	sharp corners = all,
}

\newtheorem{lemma}{Lemma}[section]{\bfseries}{}
\tcolorboxenvironment{lemma}{
	breakable,
	enhanced,
	boxrule = 0pt,
	frame hidden,
	coltitle = black,
	colback = green!7,
	left = 0.5em,
	sharp corners = all,
}

\newtheorem{definition}{Definition}[section]{\bfseries}{}
\tcolorboxenvironment{definition}{
	breakable,
	coltitle = black,
	colback = white,
	frame hidden,
	boxsep = 0pt,
	boxrule = 0pt,
	borderline west = {3pt}{0pt}{orange},
	sharp corners = all,
	enhanced,
}

\newtheorem{example}{Example}[section]{\bfseries}{}
\tcolorboxenvironment{example}{
	% title = \textbf{Example},
	% detach title,
	% before upper = {\tcbtitle\quad},
	breakable,
	coltitle = black,
	colback = white,
	frame hidden,
	boxrule = 0pt,
	boxsep = 0pt,
	borderline west={3pt}{0pt}{green!70!black},
	sharp corners = all,
	enhanced,
}

\newtheoremstyle{remark}{0pt}{4pt}{}{}{\bfseries\itshape}{\normalfont.}{0.5em}{}
\theoremstyle{remark}
\newtheorem*{remark}{Remark}


% TColorBoxes
\newtcolorbox{week}{
	colback = black,
	coltext = white,
	fontupper = {\large\bfseries},
	width = 1.2\paperwidth,
	size = fbox,
	halign upper = center,
	center
}

\newcommand{\banner}[2]{
    \pagebreak
    \begin{week}
   		\section*{#1}
    \end{week}
    \addcontentsline{toc}{section}{#1}
    \addtocounter{section}{1}
    \setcounter{subsection}{0}
}

% Hyperref
\usepackage{hyperref}
\hypersetup{
	colorlinks=true,
	linktoc=all,
	linkcolor=links,
	bookmarksopen=true
}


\def\homeworknumber{7}
\usepackage{fancyhdr}
\pagestyle{fancy}
\fancyhead[R]{HW \#\thehwnumber}
\fancyhead[C]{\textbf{Math 130B}}
\fancyhead[L]{Eli Griffiths}


\makeatletter
\renewcommand*\env@matrix[1][*\c@MaxMatrixCols c]{%
  \hskip -\arraycolsep
  \let\@ifnextchar\new@ifnextchar
  \array{#1}}
\makeatother

% Section 4.2: 1, 4, 7, 25, 30.
% Section 4.3. 1, 4, 10, 17.

\begin{document}

\subsection*{4.2.1}
\begin{enumerate}[label=\alph*)]
    \item False
    \item True
    \item True
    \item True
    \item False
    \item False
    \item False
    \item True
\end{enumerate}

\subsection*{4.2.4}
\[
    k = 
    \underbrace{(-1)}_{-R_1}
    \times\underbrace{(2)}_{R_1 \to R_1 + R_2 + R_2}
    \times\underbrace{(1)}_{R_2 -> R_2 - R_1}
    \times\underbrace{(1)}_{R_3 -> R_3 - R_1}
    \times\underbrace{(-1)}_{R_3 \leftrightarrow R_2}
    = 2
.\]

\subsection*{4.2.7}
\[
    \det A = 1\cdot \mqty|
        1 & 2 \\
        3 & 0
    | + 0 + 3 \cdot \mqty|
        0 & 1 \\
        2 & 3
    | = -6 + 3 (-2) = -12
.\]

\subsection*{4.2.25}
\begin{proof}
    Let $A \in M_{n\times n}(\mathbb{F})$. Note that $kA$ is the same as multiplying every row of $A$ by $k$. Therefore since there are $n$ rows in $A$,
    \[
        \det (kA) = k^n \det A
    .\]
\end{proof}

\subsection*{4.2.30}
By swapping the $i$th row with the $n + 1 - i$ row for $i = 1,2,\ldots, \left\lfloor \frac{n}{2} \right\rfloor$, it follows that
\[
    \det B = (-1)^{\left\lfloor \frac{n}{2} \right\rfloor} \det A
.\]

\subsection*{4.3.1}
\begin{enumerate}[label=\alph*)]
    \item False
    \item True
    \item False
    \item True
    \item False
    \item True
    \item True
    \item False
    \item False
\end{enumerate}

\subsection*{4.3.4}
\[
    A = \mqty(
        2 & 1  & -3 \\
        1 & -2 & 1 \\
        3 & 4  & -2
    ), \det A = \mqty|
        2 & 1  & -3 \\
        1 & -2 & 1 \\
        0 & 5  & 0
    | = -5(2 + 3) = -25
.\]

\begin{align*}
    x_1 &= -\frac{1}{25} \mqty|
        1  & 1  & -3 \\
        0  & -2 & 1 \\
        -5 & 4  & -2
    | = -\frac{1}{25}\qty(
    \mqty|
        -2 & 1 \\
        4 & -2
    | - 5 \mqty|
        1 & -3 \\
        -2 & 1
    |
    ) = -1 \\
    x_2 &= -\frac{1}{25} \mqty|
        2 & 1  & -3 \\
        1 & 0 & 1 \\
        3 & -5  & -2
    | = -\frac{1}{25} \mqty|
        2 & 1  & -3 \\
        1 & 0 & 1 \\
        0 & -6  & 0
    | = -\frac{1}{25} \qty(
    6 \cdot \mqty|
        2 & -3 \\
        1 & 1
    |
    ) = -\frac{6}{5} \\
    x_3 &= -\frac{1}{25} \mqty|
        2 & 1  & 1 \\
        1 & -2 & 0 \\
        3 & 4  & -5
    | = -\frac{1}{25} \mqty|
        2 & 1  & 1 \\
        1 & -2 & 0 \\
        0 & 35  & 0
    | = -\frac{1}{25} \mqty|
        1 & -2 \\
        0 & 35
    | = -\frac{7}{5}
\end{align*}

\subsection*{4.3.10}
\begin{proof}
    Let $A \in M_{n \times n}(\mathbb{F})$ and assume that $A$ is nilpotent. That is $\exists k \in \mathbb{Z}$ such that $A^k = O$. Note then that
    \[
        (\det A)^k = \det \qty(A^k) = 0 \implies \det A = 0
    .\]
    Therefore any nilpotent matrix has a zero determinant.
\end{proof}

\subsection*{4.3.17}
\begin{proof}
    Let $A, B \in M_{n \times n}(\mathbb{F})$ where $AB = -AB$. Assume that $n$ is odd and that $\mathbb{F}$ has characteristic not equal to $2$. Then
    \begin{align*}
        \det(AB) &= \det(-BA) \\
        \det(A)\det(B) &= (-1)^n\det(B)\det(A) \\
        (1 - (-1)^n)\det(A)\det(B) &= 0 \\
        (1 + 1)\det(A)\det(B) &= 0
    \end{align*}
    Since $\mathbb{F}$ doesn't have characteristic 2, $1 + 1 \neq 0$ so either $A$ or $B$ must a zero determinant and hence $A$ or $B$ are not invertible.
\end{proof}


\end{document}
