\documentclass[12pt,titlepage]{extarticle}
% Document Layout and Font
\usepackage{subfiles}
\usepackage[margin=2cm, headheight=15pt]{geometry}
\usepackage{fancyhdr}
\usepackage{enumitem}	
\usepackage{wrapfig}
\usepackage{multicol}
\usepackage{caption, subcaption}

\usepackage[p,osf]{scholax}

\renewcommand*\contentsname{Table of Contents}
\renewcommand{\headrulewidth}{0pt}
\pagestyle{fancy}
\fancyhf{}
\fancyfoot[R]{$\thepage$}
\setlength{\parindent}{0cm}
\setlength{\headheight}{17pt}
\hfuzz=9pt

% Utility Management
\usepackage{color}
\usepackage{colortbl}
\usepackage{xcolor}
\usepackage{xpatch}
\usepackage{xparse}

\definecolor{links}{HTML}{1c73a5}
\definecolor{bar}{HTML}{584AA8}

% Math Packages
\usepackage{mathtools, amsmath, amsthm, thmtools, amssymb, physics}
\usepackage[scaled=1.075,ncf,vvarbb]{newtxmath}

\newcommand\B{\mathbb{B}}
\newcommand\C{\mathbb{C}}
\newcommand\R{\mathbb{R}}
\newcommand\Q{\mathbb{Q}}
\newcommand\N{\mathbb{N}}
\newcommand\Z{\mathbb{Z}}

\newcommand\Prob[1]{\mathbb{P}\qty(#1)}
\newcommand\Var[1]{\text{Var}\qty(#1)}
\newcommand\Exp[1]{\mathbb{E}\qty[#1]}
\newcommand\ball[1]{\B\qty(#1)}
\newcommand\res[1]{\underset{#1}{\operatorname{Res}}\;}
\renewcommand\pv{\mathrm{p.v.}}

\newcommand\conj[1]{\overline{#1}}
\DeclareMathOperator{\Arg}{Arg}
\DeclareMathOperator{\Log}{Log}
\DeclareMathOperator{\cis}{cis}

\DeclareMathOperator{\dom}{dom}
\DeclareMathOperator{\spann}{span}
\DeclareMathOperator{\nullity}{nullity}

\newcommand\st{\text{ s.t. }}

% TIKZ
\usepackage{tikz}
\usepackage{pgfplots}
\usetikzlibrary{arrows.meta}
\usetikzlibrary{math}
\usetikzlibrary{cd}
\usetikzlibrary{patterns}
\usetikzlibrary{decorations.markings}
\usetikzlibrary{calc}

% Boxes and Theorems
\usepackage[most]{tcolorbox}
\tcbuselibrary{skins}
\tcbuselibrary{breakable}
\tcbuselibrary{theorems}

\newtheoremstyle{default}{0pt}{0pt}{}{}{\bfseries}{\normalfont.}{0.5em}{}
\theoremstyle{default}

\renewcommand*{\proofname}{\textit{\textbf{Proof.}}}
\renewcommand*{\qedsymbol}{$\blacksquare$}
\tcolorboxenvironment{proof}{
	breakable,
	coltitle = black,
	colback = white,
	frame hidden,
	boxrule = 0pt,
	boxsep = 0pt,
	borderline west={3pt}{0pt}{bar},
	sharp corners = all,
	enhanced,
}

\newtheorem{theorem}{Theorem}[section]{\bfseries}{}
\tcolorboxenvironment{theorem}{
	breakable,
	enhanced,
	boxrule = 0pt,
	frame hidden,
	coltitle = black,
	colback = blue!7,
	left = 0.5em,
	sharp corners = all,
}

\newtheorem{corollary}{Corollary}[section]{\bfseries}{}
\tcolorboxenvironment{corollary}{
	breakable,
	enhanced,
	boxrule = 0pt,
	frame hidden,
	coltitle = black,
	colback = white!0,
	left = 0.5em,
	sharp corners = all,
}

\newtheorem{lemma}{Lemma}[section]{\bfseries}{}
\tcolorboxenvironment{lemma}{
	breakable,
	enhanced,
	boxrule = 0pt,
	frame hidden,
	coltitle = black,
	colback = green!7,
	left = 0.5em,
	sharp corners = all,
}

\newtheorem{definition}{Definition}[section]{\bfseries}{}
\tcolorboxenvironment{definition}{
	breakable,
	coltitle = black,
	colback = white,
	frame hidden,
	boxsep = 0pt,
	boxrule = 0pt,
	borderline west = {3pt}{0pt}{orange},
	sharp corners = all,
	enhanced,
}

\newtheorem{example}{Example}[section]{\bfseries}{}
\tcolorboxenvironment{example}{
	% title = \textbf{Example},
	% detach title,
	% before upper = {\tcbtitle\quad},
	breakable,
	coltitle = black,
	colback = white,
	frame hidden,
	boxrule = 0pt,
	boxsep = 0pt,
	borderline west={3pt}{0pt}{green!70!black},
	sharp corners = all,
	enhanced,
}

\newtheoremstyle{remark}{0pt}{4pt}{}{}{\bfseries\itshape}{\normalfont.}{0.5em}{}
\theoremstyle{remark}
\newtheorem*{remark}{Remark}


% TColorBoxes
\newtcolorbox{week}{
	colback = black,
	coltext = white,
	fontupper = {\large\bfseries},
	width = 1.2\paperwidth,
	size = fbox,
	halign upper = center,
	center
}

\newcommand{\banner}[2]{
    \pagebreak
    \begin{week}
   		\section*{#1}
    \end{week}
    \addcontentsline{toc}{section}{#1}
    \addtocounter{section}{1}
    \setcounter{subsection}{0}
}

% Hyperref
\usepackage{hyperref}
\hypersetup{
	colorlinks=true,
	linktoc=all,
	linkcolor=links,
	bookmarksopen=true
}


\begin{document}

\begin{theorem}
Connectedness is a topological property.
\end{theorem}

\begin{proof}
Assume $(X,\mathcal{T}_X),(Y,\mathcal{T}_Y)$ are topological spaces and are homeomorphic, and $(Y,\mathcal{T}_Y)$ is 
not connected. Then by the negation of Definition $10.1$, there exist two nonempty open sets $U,V$ such that $U\cap V=\emptyset$ and $U\cup V=Y$. Then by Definition $9.7$, there exists a homeomorphism $f:X\to Y$ which is bijective, and both $f$ and $f^{-1}$ are continuous. We want to show that $(X,\mathcal{T}_X)$ is not connected, hence examine the four conditions that $(X, \mathcal{T}_X)$ is not connected.

\begin{enumerate}
    \item %---------------------------------------------------------
    Note $f$ is continuous and $U$ and $V$ are open sets. Then by Definition $9.1$, their preimages, $f^{-1}(U)$ and $f^{-1}(U)$, are also open. 
    \item %---------------------------------------------------------
    Since $U,V\neq\emptyset$, $f^{-1}(U)\neq\emptyset$ and $f^{-1}(V)\neq\emptyset$
    \item %---------------------------------------------------------
    Assume for the sake of contradiction that $f^{-1}(U)\cap f^{-1}(V)$ is nonempty. Then there exists an $x\in X$ such that $x\in f^{-1}(U)\cap f^{-1}(V)$. Then by Definition $3.11$, $x\in f^{-1}(U)$ and $x\in f^{-1}(V)$. Because $f$ is invertible, it is the case that $f(x)\in U$ and $f(x)\in V$. Then by Definition $3.11$, $f(x)\in U\cap V$. This contradicts our previous assumption that $U\cap V=\emptyset$, so $f^{-1}(U)\cap f^{-1}(V)$ must be empty. Equivalently, $f^{-1}(U)\cap f^{-1}(V)=\emptyset$. 
    \item %---------------------------------------------------------
    Now we will show that $f^{-1}(U)\cup f^{-1}(V)$ and $X$ are subsets of each and therefore are equal.
    \begin{enumerate}[label=\alph*)]
        \item First we will show that $f^{-1}(U)\cup f^{-1}(V) \subseteq X$. Let $a\in f^{-1}(U)\cup f^{-1}(V)$. Then by definition $3.11,$ $a\in f^{-1}(U)$ or $a\in f^{-1}(V)$. Note that $f^{-1}(U)$ and $f^{-1}(V)$ are both subsets of $X$. Therefore in both cases, $a\in X$. Since $a\in X$ for all $a\in a\in f^{-1}(U)\cup f^{-1}(V)$, by Definition $3.4$, $f^{-1}(U)\cup f^{-1}(V)\subseteq X$
        \item We will now show that $X \subseteq f^{-1}(U)\cup f^{-1}(V)$. Let $b\in X$. Then $f(b)\in Y$.  Note that $Y=U\cup V$. It follows that $f(b)\in U\cup V$. By Definition $3.11$,  $f(b)\in U$ or $f(b)\in V$. Since $f$ is invertible, either $b\in f^{-1}(U)$ or $b\in f^{-1}(V)$ . Then by Definition $3.11$, $b\in f^{-1}(U)\cup f^{-1}(V)$. Since $b\in f^{-1}(U)\cup f^{-1}(V)$ for all $b\in X$, by Definition $3.4$, $X\subseteq f^{-1}(U)\cup f^{-1}(V)$.
    \end{enumerate}
    Note we have shown that $f^{-1}(U)\cup f^{-1}(V)\subseteq X$ and $X\subseteq f^{-1}(U)\cup f^{-1}(V)$. By the theorem of equality of sets ($RQ$ $3$), $f^{-1}(U)\cup f^{-1}(V)=X$
\end{enumerate}

Note $f^{-1}(U)$ and $f^{-1}(V)$ exist, are elements of $X$, and $f^{-1}(U)\cap f^{-1}(V)=\emptyset$ and $f^{-1}(U)\cup f^{-1}(V)=X$. Therefore $(X,\mathcal{T}_X)$ is not connected by Definition $10.1$.

So far we have shown that if $(Y,\mathcal{T}_Y)$ is not connected, any space $(X,\mathcal{T}_X)$ homeomorphic to $(Y,\mathcal{T}_Y)$ cannot be connected. Recall Definition $9.10$, which states $P$ is a topological property if whenever $(X,\mathcal{T}_X)$ does not have property $P$ than neither do any spaces homeomorphic to $(X,\mathcal{T}_X)$. Thus, Connectedness is a topological property.
\end{proof}


\end{document}
