\documentclass[12pt,titlepage]{extarticle}
% Document Layout and Font
\usepackage{subfiles}
\usepackage[margin=2cm, headheight=15pt]{geometry}
\usepackage{fancyhdr}
\usepackage{enumitem}	
\usepackage{wrapfig}
\usepackage{multicol}
\usepackage{caption, subcaption}

\usepackage[p,osf]{scholax}

\renewcommand*\contentsname{Table of Contents}
\renewcommand{\headrulewidth}{0pt}
\pagestyle{fancy}
\fancyhf{}
\fancyfoot[R]{$\thepage$}
\setlength{\parindent}{0cm}
\setlength{\headheight}{17pt}
\hfuzz=9pt

% Utility Management
\usepackage{color}
\usepackage{colortbl}
\usepackage{xcolor}
\usepackage{xpatch}
\usepackage{xparse}

\definecolor{links}{HTML}{1c73a5}
\definecolor{bar}{HTML}{584AA8}

% Math Packages
\usepackage{mathtools, amsmath, amsthm, thmtools, amssymb, physics}
\usepackage[scaled=1.075,ncf,vvarbb]{newtxmath}

\newcommand\B{\mathbb{B}}
\newcommand\C{\mathbb{C}}
\newcommand\R{\mathbb{R}}
\newcommand\Q{\mathbb{Q}}
\newcommand\N{\mathbb{N}}
\newcommand\Z{\mathbb{Z}}

\newcommand\Prob[1]{\mathbb{P}\qty(#1)}
\newcommand\Var[1]{\text{Var}\qty(#1)}
\newcommand\Exp[1]{\mathbb{E}\qty[#1]}
\newcommand\ball[1]{\B\qty(#1)}
\newcommand\res[1]{\underset{#1}{\operatorname{Res}}\;}
\renewcommand\pv{\mathrm{p.v.}}

\newcommand\conj[1]{\overline{#1}}
\DeclareMathOperator{\Arg}{Arg}
\DeclareMathOperator{\Log}{Log}
\DeclareMathOperator{\cis}{cis}

\DeclareMathOperator{\dom}{dom}
\DeclareMathOperator{\spann}{span}
\DeclareMathOperator{\nullity}{nullity}

\newcommand\st{\text{ s.t. }}

% TIKZ
\usepackage{tikz}
\usepackage{pgfplots}
\usetikzlibrary{arrows.meta}
\usetikzlibrary{math}
\usetikzlibrary{cd}
\usetikzlibrary{patterns}
\usetikzlibrary{decorations.markings}
\usetikzlibrary{calc}

% Boxes and Theorems
\usepackage[most]{tcolorbox}
\tcbuselibrary{skins}
\tcbuselibrary{breakable}
\tcbuselibrary{theorems}

\newtheoremstyle{default}{0pt}{0pt}{}{}{\bfseries}{\normalfont.}{0.5em}{}
\theoremstyle{default}

\renewcommand*{\proofname}{\textit{\textbf{Proof.}}}
\renewcommand*{\qedsymbol}{$\blacksquare$}
\tcolorboxenvironment{proof}{
	breakable,
	coltitle = black,
	colback = white,
	frame hidden,
	boxrule = 0pt,
	boxsep = 0pt,
	borderline west={3pt}{0pt}{bar},
	sharp corners = all,
	enhanced,
}

\newtheorem{theorem}{Theorem}[section]{\bfseries}{}
\tcolorboxenvironment{theorem}{
	breakable,
	enhanced,
	boxrule = 0pt,
	frame hidden,
	coltitle = black,
	colback = blue!7,
	left = 0.5em,
	sharp corners = all,
}

\newtheorem{corollary}{Corollary}[section]{\bfseries}{}
\tcolorboxenvironment{corollary}{
	breakable,
	enhanced,
	boxrule = 0pt,
	frame hidden,
	coltitle = black,
	colback = white!0,
	left = 0.5em,
	sharp corners = all,
}

\newtheorem{lemma}{Lemma}[section]{\bfseries}{}
\tcolorboxenvironment{lemma}{
	breakable,
	enhanced,
	boxrule = 0pt,
	frame hidden,
	coltitle = black,
	colback = green!7,
	left = 0.5em,
	sharp corners = all,
}

\newtheorem{definition}{Definition}[section]{\bfseries}{}
\tcolorboxenvironment{definition}{
	breakable,
	coltitle = black,
	colback = white,
	frame hidden,
	boxsep = 0pt,
	boxrule = 0pt,
	borderline west = {3pt}{0pt}{orange},
	sharp corners = all,
	enhanced,
}

\newtheorem{example}{Example}[section]{\bfseries}{}
\tcolorboxenvironment{example}{
	% title = \textbf{Example},
	% detach title,
	% before upper = {\tcbtitle\quad},
	breakable,
	coltitle = black,
	colback = white,
	frame hidden,
	boxrule = 0pt,
	boxsep = 0pt,
	borderline west={3pt}{0pt}{green!70!black},
	sharp corners = all,
	enhanced,
}

\newtheoremstyle{remark}{0pt}{4pt}{}{}{\bfseries\itshape}{\normalfont.}{0.5em}{}
\theoremstyle{remark}
\newtheorem*{remark}{Remark}


% TColorBoxes
\newtcolorbox{week}{
	colback = black,
	coltext = white,
	fontupper = {\large\bfseries},
	width = 1.2\paperwidth,
	size = fbox,
	halign upper = center,
	center
}

\newcommand{\banner}[2]{
    \pagebreak
    \begin{week}
   		\section*{#1}
    \end{week}
    \addcontentsline{toc}{section}{#1}
    \addtocounter{section}{1}
    \setcounter{subsection}{0}
}

% Hyperref
\usepackage{hyperref}
\hypersetup{
	colorlinks=true,
	linktoc=all,
	linkcolor=links,
	bookmarksopen=true
}


\def\homeworknumber{8}
\usepackage{fancyhdr}
\pagestyle{fancy}
\fancyhead[R]{HW \#\thehwnumber}
\fancyhead[C]{\textbf{Math 130B}}
\fancyhead[L]{Eli Griffiths}


\makeatletter
\renewcommand*\env@matrix[1][*\c@MaxMatrixCols c]{%
  \hskip -\arraycolsep
  \let\@ifnextchar\new@ifnextchar
  \array{#1}}
\makeatother

% Section 4.4: 1, 5, 6. 
% Section 5.1: 1, 3, 8, 11, 14, 18.

\begin{document}

\subsection*{4.4.1}
\begin{enumerate}[label=\alph*)]
    \item True
    \item True (its also "wise" to check if any two columns or rows are the same)
    \item True
    \item False
    \item False
    \item True
    \item True
    \item False
    \item True
    \item True
    \item True
\end{enumerate}

\subsection*{4.4.5}
\begin{proof}
    Let $A \in M_{n\times n}(\mathbb{F})$ and $I = I_{m}$. We will show that
    \[
        \det\mqty(A & B \\ O & I) = \det A
    \]
    where $B$ is any $n\times n$ matrix. Proceed with induction on $m$. Consider the base case $m = 1$. Then by doing a cofactor expansion on the bottom row,
    \[
        \det\mqty(
          &   &   & b_1 \\
          & A &   & \vdots \\
          &   &   & b_n \\
        0 & \ldots & 0 & 1
        ) = (-1)^{(n+1) + (n+1)} \det A = \det A
    .\]
    Hence the base case holds. Assume for some fixed $m \geq 1$. Then
    \[
        \det\mqty(
            A & B \\
            O & I_{m+1}
        ) = \det\mqty(
            A & B_1 & B_2 \\
            0 & I_{m} & 0 \\
            0 & 0  & 1
        )
    \]
        where $B = \mqty(B_1 & B_2)$ with $B_2$ being a single column. Therefore by expanding on the bottom row,
        \[
        \det\mqty(
            A & B_1 & B_2 \\
            0 & I_{m} & 0 \\
            0 & 0  & 1
            ) = (-1)^{(n + m - 1) + (n + m - 1)} \det\mqty(A & B_1 \\ O & I_m) = \det A
        .\]
        Therefore the statement holds for all $m \geq 1$. Note then if there is a matrix $M$ with the form
        \[
            M = \mqty(A & B \\ O & I) \implies \det M = \det A 
        .\]
\end{proof}

\subsection*{4.4.6}
\begin{proof}
    Consider two cases. Assume that $C$ is not invertible. Then there are two rows of $C$ that are not independent, and therefore there are two rows that aren't independent in $\mqty(O & C)$. This means that $M$ cannot be invertible and therefore
    \[
        \det(A) \det(C) = 0 = \det(M)
    .\]
    Assume then that $C$ is invertible. Note that
    \[
        \mqty(
        I & O \\
        O & C^{-1}
        ) \mqty(
        A & B \\ 
        O & C
        ) = 
        \mqty(
        A & B \\
        O & I
        )
    .\]
    By the previous proof,
    \[
        \det\mqty(
        I & O \\
        O & C^{-1}
        ) \det\mqty(
        A & B \\ 
        O & C
        ) = 
        \det\mqty(
        A & B \\
        O & I
        ) \implies \det\mqty(A & B \\ O & C) \det(C^{-1}) = \det(A)
    .\]
    Therefore since $\det C^{-1} = \frac{1}{\det C}$,
    \[
        \det\mqty(A & B \\ O & C) = \det(A) \det(C)
    .\]
\end{proof}

\subsection*{5.1.1}
\begin{enumerate}[label=\alph*)]
    \item False
    \item True
    \item True
    \item False
    \item False
    \item False
    \item False
    \item True
    \item True
    \item False
    \item False
\end{enumerate}

\subsection*{5.1.3}
\subsubsection*{Part A}
\[
    \det(A - \lambda I) = \det\mqty(1 - \lambda & 2 \\ 3 & 2 - \lambda) = (1-\lambda)(2-\lambda) - 6 = \lambda^2 - 3 \lambda - 4 \implies \lambda = \qty{-1, 4}
.\]
For $\lambda = -1$,
\[
    A + I = \mqty(2 & 2 \\ 3 & 3) \implies N(A + I) = \spann\qty{\mqty(-1 \\ 1)}
.\]
For $\lambda = 4$,
\[
    A - 4I = \mqty(-3 & 2 \\ 3 & -2) \implies N(A - 4I) = \spann\qty{\mqty(\frac{2}{3} \\ 1)} = \spann\qty{\mqty{2 \\ 3}}
.\]

Since $\mqty(-1 \\ 1)$ and $\mqty(2 \\ 3)$ are linearly independent, they form a basis for $\mathbb{F}^2$. Therefore
\[
    Q = \mqty(
        -1 & 2 \\
        1 & 3
    ),
    D = \mqty(
        \dmat[0]{-1, 4}
    )
.\]

\subsubsection*{Part B}
\[
    \det(A - \lambda I) = \det\mqty(
    -\lambda & -2 & -3 \\
    -1 & 1-\lambda & -1 \\
    2 & 2 & 5 - \lambda
    ) = -\lambda ^3+6 \lambda ^2-11 \lambda +6 = -((\lambda -3) (\lambda -2) (\lambda -1))
.\]
Therefore $\lambda = \qty{1, 2, 3}$. For $\lambda = 1$,
\[
    A - I = \mqty(
        -1 & -2 & -3 \\
         -1 & 0 & -1 \\
         2 & 2 & 4 \\
         ) \implies N(A - I) = \spann\qty{
             \mqty(1 \\ 1 \\ -1)
         }
.\]
For $\lambda = 2$,
\[
    A - 2 I = \mqty(
        -2 & -2 & -3 \\
         -1 & -1 & -1 \\
         2 & 2 & 3 \\
    ) \implies N(A - 2I) = \spann\qty{
        \mqty(1 \\ -1 \\ 0)
    }
.\]

For $\lambda = 3$,
\[
    A - 3 I = \mqty(
    -3 & -2 & -3 \\
     -1 & -2 & -1 \\
     2 & 2 & 2 \\
    ) \implies N(A - 3I) = \spann\qty{
        \mqty(-1 \\ 1 \\ 0),
        \mqty(-1 \\ 0 \\ 1)
    }
.\]

The set $\qty{
    \mqty(1 \\ -1 \\ 0),
    \mqty(-1 \\ 0 \\ 1),
    \mqty(1 \\ 1 \\ -1)
}$ forms a basis for $\mathbb{F}^3$. Therefore
\[
    Q = \mqty(
    1  & 1  & -1 \\
    1  & -1 & 0  \\
    -1 & 0  & 1  
    ),
    D = \mqty(
    \dmat[0]{1,2,3}
    )
.\]

\subsubsection*{Part C}
\[
    \det\mqty(i - \lambda & 1 \\ 2 & -i - \lambda) = (\lambda - i)(\lambda + i) - 2 = \lambda^2 - 1 \implies \lambda = \qty{-1, 1}
.\]

For $\lambda = -1$,
\[
    A + I = \mqty(
        i + 1 & 1 \\
        2  & -i + 1
    ) \implies N(A - I) = \spann\qty{
        \mqty(i-1 \\ 2)
    }
.\]

For $\lambda = 1$,
\[
    A - I = \mqty(
        i - 1 & 1 \\
        2  & -i - 1
    ) \implies N(A - I) = \spann\qty{
        \mqty(i+1 \\ 2)
    }
.\]

The set $\qty{
    \mqty(i+1 \\ 2),
    \mqty(i-1 \\ 2)
}$ forms a basis of $\mathbb{C}^2$. Therefore
\[
    Q = \mqty(
        i + 1 & i - 1 \\
        2 & 2 \\
    ), D = \mqty(\dmat[0]{-1,1})
.\]


\subsubsection*{Part D}
\[
    \det(A - \lambda I) = \det\mqty(
        2-\lambda  & 0 & -1 \\
        4 & 1-\lambda  & -4 \\
        2 & 0 & -\lambda -1 \\
        ) = -\lambda ^3+2 \lambda ^2-\lambda = -(\lambda -1)^2 \lambda \implies \lambda = \qty{0, 1}
.\]

For $\lambda = 0$,
\[
    N(A) = \spann\qty{\mqty(1 \\ 4 \\ 2)}
.\]

For $\lambda = 1$,
\[
    A - I = \mqty(
        1 & 0 & -1 \\
        4 & 0 & -4 \\
        2 & 0 & -2 \\
    ) \implies N(A - I) = \spann\qty{
        \mqty(1 \\ 0 \\ 1),
        \mqty(0 \\ 1 \\ 0)
    }
.\]

The set $\qty{
    \mqty(1 \\ 0 \\ 1),
    \mqty(0 \\ 1 \\ 0),
    \mqty(1 \\ 4 \\ 2)
}$ forms a basis for $\mathbb{F}^3$. Therefore
\[
    Q = \mqty(
    1 & 0 & 1 \\
    0 & 1 & 4 \\
    1 & 0 & 2 \\
    ), D = \mqty(\dmat[0]{1,1,0})
.\]

\subsection*{5.1.8}
\subsubsection*{Part A}
\begin{proof}
    Let $T : V \to V$ be a linear operator on the finite dimensional space $V$.
    \begin{enumerate}
        \item[$\Rightarrow)$] 
            Assume towards contradiction that $T$ is invertible and $0$ is an eigenvalue. Then $\exists x \neq 0$ such that $T(x) = 0$. However, this means that $\nullity T \neq 0$ and therefore $T$ is not invertible, hence a contradiction.
        \item[$\Leftarrow$]
            Assume that $0$ is not an eigenvalue of $T$. That is, there is no $x \neq 0$ such that $T(x) = 0$. This means that $\nullity T = 0$ and hence $T$ must be invertible.
    \end{enumerate}
    Since both directions are true, the original statement is true.
\end{proof}

\subsubsection*{Part B}
\begin{proof}
    Let $T : V \to V$ be an invertible linear operator with $\lambda$ as an eigenvalue. Then $\exists x \neq 0$ such that $T(x) = \lambda x$. Note then that
    \[
        T(x) = \lambda x \implies x = T^{-1}(\lambda x) = \lambda T^{-1}(x) \implies T^{-1}(x) = \frac{1}{\lambda} x
    .\]
    Therefore $x$ is an eigenvector for $T^{-1}$ with an eigenvalue of $\lambda^{-1}$.
\end{proof}

\subsubsection*{Part C}
\begin{proof}
    Let $A \in M_{n\times n}(\mathbb{F})$. Assume that $A$ is invertible. Then $A$ has a non zero determinant. This means that $0$ cannot be an eigenvalue of $A$ since $\det(A - 0I) = \det(A) \neq 0$. Assume that $0$ is not an eigenvalue of $A$. Then there is no non-zero vector such that $Ax = 0$. Therefore $A$ is full rank and hence invertible.
\end{proof}
\begin{proof}
    Let $A \in M_{n\times n}(\mathbb{F})$. Assume that $A$ is invertible and $\lambda$ is an eigenvalue for $A$. Then $\exists x \neq 0$ such that $Ax = \lambda x$. Note then that
    \[
        Ax = \lambda x \implies x = \lambda A^{-1} x \implies A^{-1} x = \frac{1}{\lambda} x
    .\]
    Therefore $x$ is a eigenvector for $A^{-1}$ with eigenvalue $\lambda^{-1}$.
\end{proof}


\subsection*{5.1.11}
\subsubsection*{Part A}
\begin{proof}
    Let $A$ be a square matrix and $\lambda \in \mathbb{F}$. ASsume that $A \sim \lambda I$. Therefore there is an invertible matrix $P$ such that $A = P^{-1} \lambda I P$. Then
    \[
        A = P^{-1} \lambda P = \lambda P^{-1} I P = \lambda P^{-1} P = \lambda I
    .\]
\end{proof}

\subsubsection*{Part B}
\begin{proof}
    Let $A$ be a diagnolizable matrix such that it has a single eigenvalue $\lambda$. Then there exists an invertible matrix $\Lambda$ such that
    \[
        A = \Lambda^{-1} \mqty(\dmat{\lambda, \ddots, \lambda}) \Lambda
    .\]
    But note that the diagonal matrix is $\lambda I$. Therefore by the previous result $A = \lambda I$.
\end{proof}

\subsubsection*{Part C}
\begin{proof}
    Assume towards contradiction that $\mqty(1 & 1 \\ 0 & 1)$ is diagnolizable. Note that the matrix has only the eigenvalue $1$. Therefore by the previous results the matrix should equal $\mqty(1 & 0 \\ 0 & 1)$ but this is a contradiction.
\end{proof}

\subsection*{5.1.14}
\begin{proof}
    Let $A$ be a square matrix. Note that
    \[
        \det(A - \lambda I) = \det((A - \lambda I)^t) = \det(A^t - \lambda I)
    .\]
    Therefore $A$ and $A^t$ have the same characteristic polynomial and hence same eigenvalues.
\end{proof}

\subsection*{5.1.18}
\begin{proof}
    Let $A,B$ be similar $n\times n$ matrices. Since they are similar, there is an invertible matrix $Q$ such that
    \[
        A = Q^{-1} B Q
    .\]
    By exercise $2.5.14$ and noting that $P = Q$, there then must exist an $n$ dimensional vector space $V$ and $n$ dimensional vector space $W$, ordered bases $\beta$ and $\beta'$ for $V$ and $\gamma$ and $\gamma'$ for $W$, and a linear transformation $T : V \to W$
\end{proof}

\end{document}
