\documentclass[12pt,titlepage]{extarticle}
% Document Layout and Font
\usepackage{subfiles}
\usepackage[margin=2cm, headheight=15pt]{geometry}
\usepackage{fancyhdr}
\usepackage{enumitem}	
\usepackage{wrapfig}
\usepackage{multicol}
\usepackage{caption, subcaption}

\usepackage[p,osf]{scholax}

\renewcommand*\contentsname{Table of Contents}
\renewcommand{\headrulewidth}{0pt}
\pagestyle{fancy}
\fancyhf{}
\fancyfoot[R]{$\thepage$}
\setlength{\parindent}{0cm}
\setlength{\headheight}{17pt}
\hfuzz=9pt

% Utility Management
\usepackage{color}
\usepackage{colortbl}
\usepackage{xcolor}
\usepackage{xpatch}
\usepackage{xparse}

\definecolor{links}{HTML}{1c73a5}
\definecolor{bar}{HTML}{584AA8}

% Math Packages
\usepackage{mathtools, amsmath, amsthm, thmtools, amssymb, physics}
\usepackage[scaled=1.075,ncf,vvarbb]{newtxmath}

\newcommand\B{\mathbb{B}}
\newcommand\C{\mathbb{C}}
\newcommand\R{\mathbb{R}}
\newcommand\Q{\mathbb{Q}}
\newcommand\N{\mathbb{N}}
\newcommand\Z{\mathbb{Z}}

\newcommand\Prob[1]{\mathbb{P}\qty(#1)}
\newcommand\Var[1]{\text{Var}\qty(#1)}
\newcommand\Exp[1]{\mathbb{E}\qty[#1]}
\newcommand\ball[1]{\B\qty(#1)}
\newcommand\res[1]{\underset{#1}{\operatorname{Res}}\;}
\renewcommand\pv{\mathrm{p.v.}}

\newcommand\conj[1]{\overline{#1}}
\DeclareMathOperator{\Arg}{Arg}
\DeclareMathOperator{\Log}{Log}
\DeclareMathOperator{\cis}{cis}

\DeclareMathOperator{\dom}{dom}
\DeclareMathOperator{\spann}{span}
\DeclareMathOperator{\nullity}{nullity}

\newcommand\st{\text{ s.t. }}

% TIKZ
\usepackage{tikz}
\usepackage{pgfplots}
\usetikzlibrary{arrows.meta}
\usetikzlibrary{math}
\usetikzlibrary{cd}
\usetikzlibrary{patterns}
\usetikzlibrary{decorations.markings}
\usetikzlibrary{calc}

% Boxes and Theorems
\usepackage[most]{tcolorbox}
\tcbuselibrary{skins}
\tcbuselibrary{breakable}
\tcbuselibrary{theorems}

\newtheoremstyle{default}{0pt}{0pt}{}{}{\bfseries}{\normalfont.}{0.5em}{}
\theoremstyle{default}

\renewcommand*{\proofname}{\textit{\textbf{Proof.}}}
\renewcommand*{\qedsymbol}{$\blacksquare$}
\tcolorboxenvironment{proof}{
	breakable,
	coltitle = black,
	colback = white,
	frame hidden,
	boxrule = 0pt,
	boxsep = 0pt,
	borderline west={3pt}{0pt}{bar},
	sharp corners = all,
	enhanced,
}

\newtheorem{theorem}{Theorem}[section]{\bfseries}{}
\tcolorboxenvironment{theorem}{
	breakable,
	enhanced,
	boxrule = 0pt,
	frame hidden,
	coltitle = black,
	colback = blue!7,
	left = 0.5em,
	sharp corners = all,
}

\newtheorem{corollary}{Corollary}[section]{\bfseries}{}
\tcolorboxenvironment{corollary}{
	breakable,
	enhanced,
	boxrule = 0pt,
	frame hidden,
	coltitle = black,
	colback = white!0,
	left = 0.5em,
	sharp corners = all,
}

\newtheorem{lemma}{Lemma}[section]{\bfseries}{}
\tcolorboxenvironment{lemma}{
	breakable,
	enhanced,
	boxrule = 0pt,
	frame hidden,
	coltitle = black,
	colback = green!7,
	left = 0.5em,
	sharp corners = all,
}

\newtheorem{definition}{Definition}[section]{\bfseries}{}
\tcolorboxenvironment{definition}{
	breakable,
	coltitle = black,
	colback = white,
	frame hidden,
	boxsep = 0pt,
	boxrule = 0pt,
	borderline west = {3pt}{0pt}{orange},
	sharp corners = all,
	enhanced,
}

\newtheorem{example}{Example}[section]{\bfseries}{}
\tcolorboxenvironment{example}{
	% title = \textbf{Example},
	% detach title,
	% before upper = {\tcbtitle\quad},
	breakable,
	coltitle = black,
	colback = white,
	frame hidden,
	boxrule = 0pt,
	boxsep = 0pt,
	borderline west={3pt}{0pt}{green!70!black},
	sharp corners = all,
	enhanced,
}

\newtheoremstyle{remark}{0pt}{4pt}{}{}{\bfseries\itshape}{\normalfont.}{0.5em}{}
\theoremstyle{remark}
\newtheorem*{remark}{Remark}


% TColorBoxes
\newtcolorbox{week}{
	colback = black,
	coltext = white,
	fontupper = {\large\bfseries},
	width = 1.2\paperwidth,
	size = fbox,
	halign upper = center,
	center
}

\newcommand{\banner}[2]{
    \pagebreak
    \begin{week}
   		\section*{#1}
    \end{week}
    \addcontentsline{toc}{section}{#1}
    \addtocounter{section}{1}
    \setcounter{subsection}{0}
}

% Hyperref
\usepackage{hyperref}
\hypersetup{
	colorlinks=true,
	linktoc=all,
	linkcolor=links,
	bookmarksopen=true
}


\def\homeworknumber{4}
\usepackage{fancyhdr}
\pagestyle{fancy}
\fancyhead[R]{HW \#\thehwnumber}
\fancyhead[C]{\textbf{Math 130B}}
\fancyhead[L]{Eli Griffiths}


% Section 2.4: 1, 3, 16, 17, 22
% Section 2.5: 1, 4, 5, 6,

\begin{document}

\subsection*{2.4.1}
\begin{enumerate}[label=\alph*)]
    \item False
    \item True
    \item False
    \item False
    \item True
    \item False
    \item True
    \item True
    \item True
\end{enumerate}

\subsection*{2.4.3}
\begin{enumerate}[label=\alph*)]
    \item Not isomoprhic since their dimensions are not equal $(3 \neq 4)$   
    \item Yes since they are the same dimension and any vector space is isomorphic to $\mathbb{F}^n$ where $n$ is the dimension of the vector space
    \item Yes since theyre finitely dimensional with the same dimension
    \item Not isomorphic since their dimensions are not equal $(2 \neq 4)$
\end{enumerate}

\subsection*{2.4.16}
Let $\Phi^{-1}(A) = B A B^{-1}$. Note then that
\begin{align*}
    \Phi(\Phi^{-1} (A)) &= B^{-1} B A B^{-1} B = A \\
    \Phi^{-1}(\Phi(A)) &= B B^{-1} A B^{-1} B = A
\end{align*}
Therefore $\Phi \Phi^{-1} = \Phi^{-1} \Phi = I$. Therefore $\Phi$ is invertible and hence an isomorphism between $M_{n\times n} (\mathbb{F})$ and itself.

\subsection*{2.4.17}
\subsubsection*{Part A}
\begin{proof}
    Since $T$ is an isomorphism, it is linear. Let $y_1, y_2 \in T(V_0)$ where $y_1 = T(x_1)$ and $y_2 = T(x_2)$. Then $y_1 + y_2 = T(x_1) + T(x_2) = T(x_1 + x_2) \in T(V_0)$ since $x_1, x_2 \in V_0$. Additionally, $c y_1 = c T(x_1) = T(cx_1) \in T(V_0)$ by linearity of $T$. $0_W \in T(V_0)$ since $V_0$ is a subspace and hence $0_V \in V_0$ and $T(0_V) = 0_W$. Therefore $T(V_0)$ is a subspace of $W$.
\end{proof}

\subsubsection*{Part B}
\begin{proof}
    Let $T': V_0 \to W$ with $T'(x) = T(x)$. Since $T$ is invertible, it is one-to-one and onto and consequently so is $T'$. Therefore $\nullity T' = 0$ and
    \[
        \rank T' = \dim(V_0) \implies \dim(T(V_0)) = \dim(V_0)
    \]
\end{proof}

\subsection*{2.4.22}
\begin{proof}
    Note that
    \begin{align*}
        T(f + cg) &= \qty((f+cg)(c_0), \ldots (f+cg)(c_n)) \\
                  &= \qty(f(c_0) + c\cdot g(c_0), \ldots, f(c_n) + c\cdot g(c_n)) \\
                  &= \qty(f(c_0), \ldots, f(c_n)) + c\qty(g(c_0), \ldots, g(c_n)) \\
                  &= T(f) + cT(g)
    \end{align*}
    The only functions that will map to $0$ are functions that have $n+1$ zeroes, which must be the zero function. Therefore $N(T) = \qty{0}$. Since $\dim P_n (\mathbb{F}) = \dim F^{n+1}$ and $T$ is injective since $N(T) = \qty{0}$, $T$ is also onto and therefore a bijection. Hence $T$ is invertible and therefore an isomorphism.
\end{proof}

\subsection*{2.5.1}
\begin{enumerate}[label=\alph*)]
    \item False
    \item True
    \item True
    \item False
    \item True
\end{enumerate}

\subsection*{2.5.4}
\begin{align*}
    [T]_{\beta'} &= [I]_{\beta}^{\beta'} [T]_{\beta} [I]_{\beta'}^{\beta} \\
                 &= \mqty(1 & 1 \\ 1 & 2)^{-1}
                    \mqty(2 & 1 \\ 1 & -3) 
                    \mqty(1 & 1 \\ 1 & 2) \\
                 &= \mqty(2 & -1 \\ -1 & 1)
                    \mqty(2 & 1 \\ 1 & -3) 
                    \mqty(1 & 1 \\ 1 & 2) \\
                 &= \mqty(2 & -1 \\ -1 & 1)
                    \mqty(3 & 4 \\ -2 & -5) \\ 
                 &= \mqty(8 & 13 \\ -5 & -9)
\end{align*}

\subsection*{2.5.5}
\begin{align*}
    [T]_{\beta'} &= [I]_{\beta}^{\beta'} [T]_{\beta} [I]_{\beta'}^{\beta} \\
    &= 
    \mqty(
        1 & 1 \\
        1 & -1
    )^{-1}
    \mqty(
        0 & 1 \\
        0 & 0 \\
    )
    \mqty(
        1 & 1 \\
        1 & -1
    ) \\
    &=
    \mqty(
        \frac{1}{2} & \frac{1}{2} \\
        \frac{1}{2} & -\frac{1}{2}
    )
    \mqty(
        0 & 1 \\
        0 & 0 \\
    )
    \mqty(
        1 & 1 \\
        1 & -1
    ) \\
    &=
    \mqty(
        \frac{1}{2} & \frac{1}{2} \\
        \frac{1}{2} & -\frac{1}{2}
    )
    \mqty(
        1 & -1 \\
        0 & 0 \\
    ) \\
    &=
    \mqty(
        \frac{1}{2} & -\frac{1}{2} \\
        \frac{1}{2} & -\frac{1}{2} \\
    )
\end{align*}

\subsection*{2.5.6}
\begin{enumerate}
    \item $[L_A]_{\beta} = \mqty(6 & 11 \\ -2 & -4 \\)$ and $Q = \mqty(1 & 1 \\ 1 & 2)$
    \item $[L_A]_{\beta} = \mqty(3 & 0 \\ 0 & -1)$ and $Q = \mqty(1 & 1 \\ 1 & -1)$
    \item $[L_A]_{\beta} = \mqty(2 & 2 & 2 \\ -2 & -3 & -4 \\ 1 & 1 & 2)$ and $Q = \mqty(1 & 1 & 1 \\ 1 & 0 & 1 \\ 1 & 1 & 2)$
    \item $[L_A]_{\beta} = \mqty(6 & 0 & 0 \\ 0 & 12 & 0 \\ 0 & 0 & 18)$ and $Q = \mqty(1 & 1 & 1 \\ 1 & -1 & 1 \\ -2 & 0 & 1)$
\end{enumerate}

\end{document}
