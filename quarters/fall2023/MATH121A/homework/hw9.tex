\documentclass[12pt,titlepage]{extarticle}
% Document Layout and Font
\usepackage{subfiles}
\usepackage[margin=2cm, headheight=15pt]{geometry}
\usepackage{fancyhdr}
\usepackage{enumitem}	
\usepackage{wrapfig}
\usepackage{multicol}
\usepackage{caption, subcaption}

\usepackage[p,osf]{scholax}

\renewcommand*\contentsname{Table of Contents}
\renewcommand{\headrulewidth}{0pt}
\pagestyle{fancy}
\fancyhf{}
\fancyfoot[R]{$\thepage$}
\setlength{\parindent}{0cm}
\setlength{\headheight}{17pt}
\hfuzz=9pt

% Utility Management
\usepackage{color}
\usepackage{colortbl}
\usepackage{xcolor}
\usepackage{xpatch}
\usepackage{xparse}

\definecolor{links}{HTML}{1c73a5}
\definecolor{bar}{HTML}{584AA8}

% Math Packages
\usepackage{mathtools, amsmath, amsthm, thmtools, amssymb, physics}
\usepackage[scaled=1.075,ncf,vvarbb]{newtxmath}

\newcommand\B{\mathbb{B}}
\newcommand\C{\mathbb{C}}
\newcommand\R{\mathbb{R}}
\newcommand\Q{\mathbb{Q}}
\newcommand\N{\mathbb{N}}
\newcommand\Z{\mathbb{Z}}

\newcommand\Prob[1]{\mathbb{P}\qty(#1)}
\newcommand\Var[1]{\text{Var}\qty(#1)}
\newcommand\Exp[1]{\mathbb{E}\qty[#1]}
\newcommand\ball[1]{\B\qty(#1)}
\newcommand\res[1]{\underset{#1}{\operatorname{Res}}\;}
\renewcommand\pv{\mathrm{p.v.}}

\newcommand\conj[1]{\overline{#1}}
\DeclareMathOperator{\Arg}{Arg}
\DeclareMathOperator{\Log}{Log}
\DeclareMathOperator{\cis}{cis}

\DeclareMathOperator{\dom}{dom}
\DeclareMathOperator{\spann}{span}
\DeclareMathOperator{\nullity}{nullity}

\newcommand\st{\text{ s.t. }}

% TIKZ
\usepackage{tikz}
\usepackage{pgfplots}
\usetikzlibrary{arrows.meta}
\usetikzlibrary{math}
\usetikzlibrary{cd}
\usetikzlibrary{patterns}
\usetikzlibrary{decorations.markings}
\usetikzlibrary{calc}

% Boxes and Theorems
\usepackage[most]{tcolorbox}
\tcbuselibrary{skins}
\tcbuselibrary{breakable}
\tcbuselibrary{theorems}

\newtheoremstyle{default}{0pt}{0pt}{}{}{\bfseries}{\normalfont.}{0.5em}{}
\theoremstyle{default}

\renewcommand*{\proofname}{\textit{\textbf{Proof.}}}
\renewcommand*{\qedsymbol}{$\blacksquare$}
\tcolorboxenvironment{proof}{
	breakable,
	coltitle = black,
	colback = white,
	frame hidden,
	boxrule = 0pt,
	boxsep = 0pt,
	borderline west={3pt}{0pt}{bar},
	sharp corners = all,
	enhanced,
}

\newtheorem{theorem}{Theorem}[section]{\bfseries}{}
\tcolorboxenvironment{theorem}{
	breakable,
	enhanced,
	boxrule = 0pt,
	frame hidden,
	coltitle = black,
	colback = blue!7,
	left = 0.5em,
	sharp corners = all,
}

\newtheorem{corollary}{Corollary}[section]{\bfseries}{}
\tcolorboxenvironment{corollary}{
	breakable,
	enhanced,
	boxrule = 0pt,
	frame hidden,
	coltitle = black,
	colback = white!0,
	left = 0.5em,
	sharp corners = all,
}

\newtheorem{lemma}{Lemma}[section]{\bfseries}{}
\tcolorboxenvironment{lemma}{
	breakable,
	enhanced,
	boxrule = 0pt,
	frame hidden,
	coltitle = black,
	colback = green!7,
	left = 0.5em,
	sharp corners = all,
}

\newtheorem{definition}{Definition}[section]{\bfseries}{}
\tcolorboxenvironment{definition}{
	breakable,
	coltitle = black,
	colback = white,
	frame hidden,
	boxsep = 0pt,
	boxrule = 0pt,
	borderline west = {3pt}{0pt}{orange},
	sharp corners = all,
	enhanced,
}

\newtheorem{example}{Example}[section]{\bfseries}{}
\tcolorboxenvironment{example}{
	% title = \textbf{Example},
	% detach title,
	% before upper = {\tcbtitle\quad},
	breakable,
	coltitle = black,
	colback = white,
	frame hidden,
	boxrule = 0pt,
	boxsep = 0pt,
	borderline west={3pt}{0pt}{green!70!black},
	sharp corners = all,
	enhanced,
}

\newtheoremstyle{remark}{0pt}{4pt}{}{}{\bfseries\itshape}{\normalfont.}{0.5em}{}
\theoremstyle{remark}
\newtheorem*{remark}{Remark}


% TColorBoxes
\newtcolorbox{week}{
	colback = black,
	coltext = white,
	fontupper = {\large\bfseries},
	width = 1.2\paperwidth,
	size = fbox,
	halign upper = center,
	center
}

\newcommand{\banner}[2]{
    \pagebreak
    \begin{week}
   		\section*{#1}
    \end{week}
    \addcontentsline{toc}{section}{#1}
    \addtocounter{section}{1}
    \setcounter{subsection}{0}
}

% Hyperref
\usepackage{hyperref}
\hypersetup{
	colorlinks=true,
	linktoc=all,
	linkcolor=links,
	bookmarksopen=true
}


\def\homeworknumber{9}
\usepackage{fancyhdr}
\pagestyle{fancy}
\fancyhead[R]{HW \#\thehwnumber}
\fancyhead[C]{\textbf{Math 130B}}
\fancyhead[L]{Eli Griffiths}


% Section 5.2: 1, 2 (d)-(f), 3 (a)-(b), 8, 18, 19.  
% Section 5.4: 1, 2, 4, 17, 18.

\begin{document}

\subsection*{5.2.1}
\begin{enumerate}[label=\alph*)]
    \item False
    \item False
    \item False
    \item True
    \item True
    \item False
    \item True
    \item True
    \item False
\end{enumerate}

\subsection*{5.2.2}
\subsubsection*{Part D}
\[
    \det(A - \lambda I) = \det\mqty(
    7 - \lambda & -4 & 0 \\
    8 & -5 - \lambda & 0 \\
    6 & -6 & 3 - \lambda
    ) = -(\lambda - 3)^2 (1 + \lambda) \implies \lambda = \qty{-1, 3}
.\]

For $\lambda = -1$,
\[
    E_{-1} = N(A + I) = \spann\qty{
        \mqty(2 \\ 4 \\ 3)
    }
.\]

For $\lambda = 3$,
\[
    E_3 = N(A - 3I) = \spann\qty{
        \mqty(1 \\ 1 \\ 0),
        \mqty(0 \\ 0 \\ 1)
    }
.\]

Therefore $A$ is diagonalizable with
\[
    D = \mqty(\dmat[0]{-1, 3, 3}), Q = \mqty(
    2 & 1 & 0 \\
    4 & 1 & 0 \\
    3 & 0 & 1
    )
.\]

\subsubsection*{Part E}
Since \[
    \det(A - \lambda I) = (\lambda^2 + 1)(1-\lambda) 
.\]
The characteristic polynomial does not split and therefore $A$ is not diagonalizable.

\subsubsection*{Part F}
Since $A$ is upper triangular, its eigenvalues are $\lambda = \qty{1, 3}$. Note that
\[
    N(A - I) = E_1 = \spann\qty{
        \mqty(0 \\ 0 \\ 1)
    }
.\]
Since $\dim E_1 < 2$, $A$ is not diagonalizable.

\subsection*{5.2.3}
\subsubsection*{Part A}
\[
    [T]_e = \mqty(
    0 & 1 & 2 & 0 \\
    0 & 0 & 2 & 6 \\
    0 & 0 & 0 & 3 \\
    0 & 0 & 0 & 0
    )
.\]

Since $[T]_e$ is upper triangular, its singular eigenvalue is $0$. Then
\[
    E_0 = N(A) = \spann\qty{
        \mqty(0 \\ 0 \\ 0 \\ 1)
    }
.\]
Therefore since $\dim E_0 < 4$, $T$ is not diagonalizable.

\subsubsection*{Part B}
\[
    [T]_e = \mqty(
    0 & 0 & 1 \\
    0 & 1 & 0 \\
    1 & 0 & 0
    )
.\]

\[
    \det([T]_e - \lambda I) = \mqty(
    -\lambda & 0 & 1 \\
    0 & 1 - \lambda & 0 \\
    1 & 0 & -\lambda
    ) = - (\lambda - 1)^2 (\lambda + 1) \implies \lambda = \qty{-1, 1}
.\]

For $\lambda = -1$,
\[
    E_{-1} = N(A + I) = \spann\qty{
        \mqty(1 \\ 0 \\ -1)
    }
.\]

For $\lambda = 1$,
\[
    E_1 = N(A - I) = \spann\qty{
        \mqty(0 \\ 1 \\ 0),
        \mqty(1 \\ 0 \\ 1)
    }
.\]

Therefore $A$ is diagonalizable with
\[
    D = \mqty(\dmat[0]{-1,1,1}), Q = \mqty(
    1 & 0 & 1 \\
    0 & 1 & 0 \\
    -1 & 0 & 1 \\
    )
.\]

\subsection*{5.2.8}
\begin{proof}
    Since $\lambda_1$ and $\lambda_2$ are distinct, there must be an eigenvector $\vec{\lambda}$ associated with $\lambda_2$ that is independent of the vectors in $E_{\lambda_1}$. Therefore the set of all eigenvectors between them will have dimension $n - 1 + 1 = n$ and therefore $A$ will be diagonalizable.
\end{proof}

\subsection*{5.2.18}
\subsubsection*{Part A}
\begin{proof}
    Let $\beta$ be the ordered basis such that $[T]_\beta$ and $[U]_{\beta}$ are diagonal matrices. Since diagonal matrices commute,
    \[
        [T]_{\beta} [U]_{\beta} = [U]_{\beta}[T]_{\beta}
    .\]
    Therefore since the matrix representations commute, $TU = UT$.
\end{proof}

\subsubsection*{Part B}
\begin{proof}
    Let $Q^{-1}$ be the matrix that makes $A$ and $B$ simultaneously diagonalizable. Then
    \[
        (Q^{-1} A Q)(Q^{-1} B Q) = (Q^{-1} B Q)(Q^{-1} A Q)
    \]
    since $Q$ is invertible, hence $A$ and $B$ commute.
\end{proof}

\subsection*{5.2.19}
\begin{proof}
    Since $T$ and $T^m$ have the same eigenvectors, if $T$ is diagonalizable then $T^m$ is diagonalizable under the same basis.
\end{proof}

\subsection*{5.4.1}
\begin{enumerate}[label=\alph*)]
    \item False
    \item True
    \item False
    \item False
    \item True
    \item True
    \item True
\end{enumerate}

\subsection*{5.4.2}
\subsubsection*{Part A}
Yes. For any polynomial in $P_2(\mathbb{F})$, it follows that
\[
    T(ax^2 + bx + c) = 2ax + b \in P_2(\mathbb{F})
.\]

\subsubsection*{Part B}
No. For any polynomial in $W = P_2(\mathbb{F})$, it follows that
\[
    T(ax^2 + bx + c) = ax^3 + bx^2 + cx \notin W
.\]

\subsubsection*{Part C}
Yes. With $(t,t,t) \in W$,
\[
    T((t,t,t)) = (t+t+t, t+t+t, t+t+t) = 3\cdot (t,t,t) \in W
.\]

\subsubsection*{Part D}
Yes. With $at + b \in W$,
\[
    T(at+b) = t\int_{0}^1 ax + b \dd x = \qty(\frac{a}{2} + b) t \in W
.\]

\subsubsection*{Part E}
No. Note that $A = \mqty(2 & 0 \\ 0 & 1) \in W$, but
\[
    T(A) = \mqty(
    0 & 1 \\
    1 & 0
    )
    \mqty(
    2 & 0 \\
    0 & 1
    ) = \mqty(
    0 & 1 \\
    2 & 0 
    ) \neq \mqty(
    0 & 2 \\
    1 & 0
    )
.\]

\subsection*{5.4.4}
\begin{proof}
    Let $T$ be a linear operator on $V$ and $W$ be a $T$-invariant subspace. Let $g(t) = a_n t^n + \ldots + a_1 t + a_0$. Note that $W$ is a $T$-invariant subspace of any scalar multiple and positive integer power of $T$. For some $w \in W$ and $a \in \mathbb{F}$ and $k \in \mathbb{Z}_{+}$,
    \[
        T(w) \in W \implies T(T(w)) \in W \implies T^k(w) \in W
    .\]
    and
    \[
        aw \in W \implies T(aw) = aT(w) \in W
    .\]
    Furthermore,
    \[
        T(w_1 + w_2) \in W \implies T(w_1) + T(w_2) \in W
    .\]
    Therefore linear combinations of vectors in $W$ under $T$, whether $T$ is scaled or raised to a power, are still in $W$. Hence the polynomial of $T$ preserves $T$-invariance.
\end{proof}

\subsection*{5.4.17}
\begin{proof}
    Let $f(t) = (-1)^n t^n + \ldots + a_0$ be the characteristic polynomial of $A$. Then
    \[
        f(A) = (-1)^n A^n + \ldots + A a_0 = O
    .\]
    Therefore $A^n$ is a linear combination of $\qty{I, A, \ldots, A^{n-1}}$. Note then that
    \[
        A f(A) = (-1)^n A^{n+1} + \ldots A^2 A_0 = 0
    .\]
    Therefore $A^{n+1}$ is a linear combination of $\qty{I, A, \ldots, A^n}$ which reduces to $\qty{I, A, \ldots, A^{n-1}}$. This is true for any succesive power, therefore
    \[
        \dim\qty{I, A, A^2, \ldots} = \dim\qty{I, A, \ldots, A^{n-1}} \leq n
    .\]
\end{proof}

\subsection*{5.4.18}
\subsubsection*{Part A}
Since $f(t) = \det(A - tI)$, if $f(0) = 0$, then $\det(A) = 0$ which would mean $A$ is not invertible. Therefore $f(0) = a_0 \neq 0$.

\subsubsection*{Part B}
Note that
\begin{align*}
    A A^{-1} &= -A \cdot \frac{1}{a_0} \qty(
    (-1)^n A^{n-1} + a_{n-1} A^{n-2} + \ldots + a_1 I_n
    ) \\
             &= -\frac{1}{a_0} \qty((-1)^n A^n + a_{n-1} A^{n-1} + \ldots a_1 A) \\
             &= -\frac{1}{a_0} \qty(-a_0 I) = I
\end{align*}
Therefore the formula for $A^{-1}$ is valid.

\end{document}
