\documentclass{subfiles}

\begin{document}

\chapter{Cayley Hamilton Theorem}

We will build up the machinery needed to prove the following:

\begin{theorem}[Cayley Hamilton Theorem][thm:cayley]
    Given a linear operator $T : V \to V$ with $\dim(v) = n$, then
    \[
        P_T(T) = \mathbb{0}_{n\times n}
    .\]
    where $P_T$ is the characteristic polynomial of $T$.
    \addproofref
\end{theorem}

\section{Invariant Subspaces}

\begin{definition}[Invariant Subspace]
    Given a linear operator $T : V \to V$ and subspace $W \subseteq V$, if $T[W] \subseteq W$ then
    \[
        T |_W : W \to W
    \]
    is a linear operator on $W$ and $W$ is $T$ invariant.
\end{definition}

\begin{example}
    Consider some eigenvalue $\lambda$ of $T$. Then there is a subspace $E_\lambda$ of $V$ associated with that eigenvalue. Taking any $v \in E_\lambda$, note that by definition $Tv = \lambda v \in E_\lambda$. Therefore $E_\lambda$ is an invariant subspace for any eigenvalue $\lambda$ of $T$.
\end{example}

\begin{theorem}
    If $W \subseteq V$ is an invariant subspace under $T$, then for $T |_W : W \to W$ we have
    \[
        P_{T|_W}(t) \;\vert\; P_T(t)
    .\]
\end{theorem}

\begin{proof}
    Let $\beta_w = \qty{w_1, \ldots, w_k}$ be a basis of $W$ and $\beta = \beta_w \cup \qty{v_{k+1}, \ldots, v_n}$ be a basis of $V$ where $w_i$ form a basis of $W$. Then the matrix form of $T$ is
    \[
        [T]_\beta = \mqty(
        B_{1} & B_2 \\
        0 & B_3
        )
    \]
    where $B_1$ is $[T|_W]_{\beta_w}$. By subtracting $t I$ from both sides and taking the determinant, we get
    \[
        \det(T - tI) = \det(B_1 - tI) \det(B_3 - tI)
    .\]
    But this is just
    \[
        P_T(t) = P_{T|_W}(t) \cdot q(t)
    \]
    with $q(t) = \det(B_3 - t I)$.
\end{proof}

\subsection{Generating Invariant Subspaces}

Consider some linear operator $T$ on a finite dimensional space $V$ with $\dim V = n$. Then note for any $v \in V$ that
\[
    \qty{0, T v, T^2 v, \ldots}
\]
must be a linearly dependent set of vectors. If this wasn't the case, then repeated applications of $T$ would produce infinitely many linearly independent vectors within $V$. Therefore there is some $k \leq n$ such that
\[
    \qty{0, Tv, T^2 v, \ldots, T^{k-1} v}
\]
is linearly independent. The span of this set gives a subspace $W$ that is $T$ invariant, something analogous to cyclic groups in group theory. This motivates the following definition.

\begin{definition}[Cyclic Subspace]
    Let $T$ be a linear operator on $V$ and $v \in V$. Then the subspace
    \[
        W = \spann\qty{0, Tv, T^2 v, T^3 v, \ldots}
    \]
    is the \textbf{$T$-cyclic subspace of $V$ generated by $v$}.
\end{definition}

\begin{example}
    Consider the operator $T : P(\R) \to P(\R)$ with $T(p(x)) = p'(x)$. Starting with $x^3$, we see that
    \[
        \qty{0, Tx^3, T^2 x^3, \ldots} = \qty{0, 3x^2, 6x, 6}
    .\]
    The span of this set then is then $P_3(\R)$ which is invariant under $T$.
\end{example}

\begin{theorem}
    If $a_0 + a_1 Tv + a_2 T^2 v + \ldots + a_{k-1} T^{k-1} v + T^k v = 0$, then
    \[
        P_{T|_W}(t) = (-1)^k \qty(a_0 + a_1 t + \ldots a_{k-1} t^{k-1} + t^k)
    .\]
\end{theorem}

\begin{proof}
    We consider the cyclic invariant subspace $W$ spanned by the basis $\beta = \qty{v, Tv, T^2 v, \ldots, T^{k-1} v}$. If $w \in W$, then we know that
    \[
        w = a_0 v + a_1 T v + a_2 T^2 v + \ldots + a_k T^{k-1} v
    \]
    which gives
    \[
        T w = a_0 T v + a_1 T^2 v + \ldots + a_k T^k v
    .\]
\end{proof}

\subsection{The Proof}

\begin{proof}[thm:cayley]
    Let $T$ be a linear operator on $V$, $v \in V \neq 0$, and $W$ be the cyclic subspace generated by $v$.
\end{proof}

\end{document}
