\documentclass{subfiles}

\begin{document}

\begin{theorem}
    A linear operator $T : V \to V$ is an orthogonal projection if and only if $T^*$ exists and $T^2 = T^* = T$.
\end{theorem}

\begin{proof}
    We consider both directions.
    \begin{enumerate}
        \item[$\Rightarrow)$]
            Assume that $T$ is an orthogonal projection. Then note that $V = R(T) \oplus N(T)$. Therefore for any $x,y \in V$, there are decompositions
            \begin{align*}
                x &= x_R + x_N \\
                y &= y_R + y_N
            \end{align*}
            where $x_R, y_R$ are from the range and $x_N, y_N$ are from the null space. Note then
            \[
                \iprd{x, Ty} = \iprd{x_R + x_N, y_R} = \iprd{x_R, y_R} + \cancelto{0}{\iprd{x_N, y_R}} = \iprd{x_R, y_R}
            .\]
            By similar logic, $\iprd{Tx, y} = \iprd{x_R, y_R}$. Therefore $\iprd{x, Ty} = \iprd{Tx, y} = \iprd{x, T^* y}$ meaning $T = T^*$.
        \item[$\Leftarrow)$]
            Suppose $T^2 = T = T^*$. Take $x \in R(T)$ and $y \in R(T)$. By definition of the range, we have $x = Tx'$. But then $T^*x = Tx = T^2 x' = T x' = x$. Therefore
            \[
                \iprd{x,y} = \iprd{T^* x, y} = \iprd{x, Ty} = \iprd{x, 0} = 0
            \]
            meaning $x \in R(T)^\perp$.
    \end{enumerate}
\end{proof}

\end{document}
