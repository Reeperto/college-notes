\documentclass{article}

\include{hw1_preamble}

\usepackage{fancyhdr}
\pagestyle{fancy}
\fancyhead[lcr]{}
\fancyhead[l]{Eli Griffiths}
\fancyhead[c]{MATH $121$B}
\fancyhead[r]{HW \#$1$}

\newcommand{\iprd}[1]{\left\langle #1 \right\rangle}

\begin{document}

\section*{Problem 1}
\begin{proof}
    Let $n$ denote the size of $C$. We proceed with induction on $n$. Consider the base case $n = 1$. Then
    \[
        \det(C - t I) = \mqty| -a_0 - t | = (-1)^n (t + a_0)
    \]
    hence the base case holds. If we therefore we consider the case of $n+1$ expanding along the first row gives
    \[
        \det (C - t I) = -t \mqty|
            -t & 0 & \cdots & -a_1 \\
            1  & -t & \cdots & -a_2 \\
            \vdots & \ddots & \ddots & \vdots \\
            0 & 0 & 1 & -a_{n-1} - t \\
        | - a_0 I
    .\]
\end{proof}

\section*{Problem 2}
It is true for all matrices $A$. 

\begin{proof}
    Assume that $a_0 \neq 0$. Note that
    \[
        p_{A}(0) = (-1)^n (0^n + a_{n-1} \cdot 0^{n-1} + \ldots + a_0) = (-1)^n a_0 \neq 0
    .\]
    By the definition of the characteristic polynomial, we know $p_A(t) = \det(A - t I)$. Therefore
    \[
        p_A(0) = \det(A - 0(I)) = \det A = (-1)^n a_0 \neq 0
    .\]
    Since the determinant of $A$ is non-zero, it must be invertible.
\end{proof}

\section*{Problem 3}
\subsection*{Part A}
\begin{proof}
    We will show $\iprd{A,B}_{F}$ satisfies the 4 requirements of being an inner product. Let $A,B \in M_{n\times n}(\R)$ and $s \in \R$.
    \begin{enumerate}
        \item 
        We want to show linearity of the inner product. Take $C \in M_{n\times n}(\R)$. Note that both the transpose and trace are linear maps, therefore
        \begin{align*}
            \iprd{A + C, B}_F &= \tr((A+C)^T B) \\
            &= \tr((A^T + C^T) B) \\
            &= \tr(A^T B + C^T B) \\
            &= \tr(A^T B) + \tr(C^T B) = \iprd{A,B}_F + \iprd{C,B}_F
        \end{align*}
        Therefore $\iprd{\cdot, \cdot}_F$ satisfies linearity.
        \item
            We want to show $\iprd{sA,B} = s \iprd{A,B}$.
    \end{enumerate}
\end{proof}

\subsection*{Part B}
\begin{proof}
    Let $A \in M_{n \times n}(\R)$ and assume that $A$ is diagonalizable. Then $A = P D P^{-1}$ where $P$ is unitary and $D$ is a diagonal matrix with its entries being the eigenvalues of $A$. Note that
    \begin{align*}
        A^T A &= \qty(P D P^{-1})^T \qty(P D P^{-1}) \\
              &= \qty(P^{-1})^T D^{T} P^T P D P^{-1}
              \intertext{Since $P$ is unitary, its transpose is its inverse giving}
              &= P D^T I D P^{-1} \\
              &= P D^2 P^{-1}.
    \end{align*}
    Therefore $A^T A$ is also diagonalizable. Note that $D^2$ will be a diagonal matrix as well with entries $\lambda_i^2$ where $\lambda_i$ are the original entries from $D$. Since $D^2$ is the diagonal matrix in the decomposition of $A^T A$, its entries $\lambda_i^2$ are the eigenvalues of $A^T A$. Since the trace of a matrix is equal to the sum of its eigenvalues, it follows
    \[
        \tr (A^T A) = \sum_{i=1}^n \lambda_i^2 \implies \norm{A}_{F} = \sqrt{\sum_{i=1}^n \lambda_i^2}
    \]
    which was to be shown.
\end{proof}

\end{document}
