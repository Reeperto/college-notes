\documentclass{article}

\input{hw1_preamble.tex}

\usepackage{fancyhdr}
\pagestyle{fancy}
\fancyhead[lcr]{}
\fancyhead[l]{Eli Griffiths}
\fancyhead[c]{MATH $121$B}
\fancyhead[r]{HW \#$5$}

\DeclareMathOperator*{\argmin}{argmin}

\NewDocumentCommand\iprd{ s m }{% s = star, m = mandatory arg
   \IfBooleanTF{#1}{%
    \langle #2 \rangle
   }{%
    \left\langle #2 \right\rangle
   }%
}

\begin{document}

\section*{Problem 1}
\begin{proof}
    We can expand the right hand side to get
    \begin{align*}
        \text{RHS} &= \qty(\frac{x+y}{2})^* A \qty(\frac{x+y}{2}) - \qty(\frac{x-y}{2})^* A \qty(\frac{x-y}{2}) + i \qty(\frac{x+iy}{2})^* A \qty(\frac{x+iy}{2}) - i\qty(\frac{x-iy}{2})^* A \qty(\frac{x-iy}{2}) \\
            &= \frac{1}{4} \qty\Big[
                (x^*+y^*)(Ax+Ay) - (x^* - y^*)(Ax - Ay) + i(x^* - iy^*)(Ax + iAy) - i(x^*+iy^*)(Ax - iAy)
            ]
    \end{align*}
    We expand each term to get
    \begin{align*}
        (x^*+y^*)(Ax+Ay) &= x^* Ax + x^* Ay + y^* Ax + y^* A y \\[10pt]
        (x^* - y^*)(Ax - Ay) &= x^* Ax - x^* Ay - y^* Ax + y^* Ay \\[10pt]
        (x^* - iy^*)(Ax + iAy) &= x^* Ax + ix^* Ay - iy^* Ax + y^* Ay \\[10pt]
        (x^*+iy^*)(Ax - iAy) &= x^* Ax - ix^* Ay + iy^* Ax + iy^* Ay
    \end{align*}
    Substituing we have
    \begin{align*}
        \text{RHS} &= \frac{1}{4}\qty(\qty\Big[2x^* Ay + 2y^* Ax] + i\qty\Big[2i x^* Ay - 2i y^* Ax + y^* Ay - iy^* Ay] )\\
                &= \frac{1}{4} \qty(2x^* Ay + 2y^* Ax - 2x^* Ay + 2y^* Ax) \\
                &= y^* Ax
    \end{align*}
    Since $y^* Ax = \iprd{Ax, y}$ we have the desired identity.
    \\

    Suppose $W(A) = \qty{0}$. Then for any $x \in \C^n \neq 0$ we have
    \[
        \frac{Q_A(x)}{x^* x} = 0 \implies Q_A(x) = 0
    .\]
    By definition we also have that $Q_A(0) = 0$. Now take $x,y \in \C^n$ with $x \neq 0$. Using the polarization identity we have
    \[
        \iprd{Ax, y} = Q_A(\ldots) - Q_A(\ldots) + i Q_A(\ldots) - i Q_A(\ldots) = 0 - 0 + 0 - 0 = 0
    .\]
    Since $x$ was taken to be non zero and $y$ as any vector, we must have that $A = 0$.
    % Note that we can rewrite $Q_A(x) = \iprd{x, Ax}$ where $\iprd{\cdot, \cdot}$ is the standard inner product on $\C^n$. Taking the right hand side of the polarization identity we have
    % \begin{align*}
    %     \text{RHS} &= \frac{1}{4}\iprd{x+y, A(x+y)} - \frac{1}{4} \iprd{x-y, A(x-y)} + \frac{i}{4} \iprd{x+iy, A(x+iy)} - \frac{i}{4} \iprd{x-iy, A(x-iy)}\\
    %                &= \frac{1}{4} \qty\Big[\iprd{x+y, A(x+y)} -  \iprd{x-y, A(x-y)} + i \iprd{x+iy, A(x+iy)} - i \iprd{x-iy, A(x-iy)}] \\
    %                    &= \frac{1}{4} (I_x + I_y)
    % \end{align*}
    % where
    % \begin{align*}
    %     I_x &= \iprd{x, A(x+y)} - \iprd{x, A(x-y)} + i\iprd{x, A(x+iy)} - i\iprd{x, A(x-iy)} \\
    %     I_y &= \iprd{y, A(x+y)} - \iprd{-y, A(x-y)} + i \iprd{iy, A(x+iy)} - i\iprd{-iy, A(x-iy)}
    % \end{align*}
    % We can simplify both to get
    % \begin{align*}
    %     I_x &= \iprd{x, Ax + Ay} + \iprd{x, -Ax + Ay} + \iprd{x, -iAx + Ay} + \iprd{x, iAx + Ay} \\
    %     &= \iprd{x, Ax + Ay - Ax + Ay - iAx + Ay + iAx + Ay} \\
    %     &= \iprd{x, 4Ay} = 4\iprd{x, Ay} \\[10pt]
    %     I_y &= \iprd{y, Ax + Ay} + \iprd{y, Ax - Ay} + \iprd{y, -Ax-iAy)} + \iprd{y, -Ax + iAy} \\
    %         &= \iprd{y, Ax + Ay + Ax - Ay - Ax - iAy - Ax + iAy} \\
    %         &= \iprd{y, 0} = 0
    % \end{align*}
    % Therefore we have
    % \[
    %     \text{RHS} = \frac{1}{4} (4 \iprd{x, Ay} + 0) = \iprd{x, Ay} = x^* Ay
    % .\]
    % \\

    % Suppose $W(A) = \qty{0}$. Then for any $x \in \C^n \neq 0$ we have
    % \[
    %     \frac{Q_A(x)}{x^* x} = 0 \implies Q_A(x) = 0
    % .\]
    % Take $x \neq 0$ and $y \in \C^n$. We then have from the polarization identity
    % \[
    %     \iprd{Ax, y} = Q_A\qty(\ldots) - Q_A(\ldots) + iQ_A(\ldots) - iQ_A(\ldots) = 0
    % .\]
\end{proof}

\section*{Problem 2}
\begin{proof}
    Note that we can rewrite $Q_A(x) = \iprd{x, Ax}$ and $x^* x = \iprd{x, x}$ using the standard inner product on $\C^n$. The lower bound follows quickly from Cauchy Schwarz,
    \[
        r(A) = \sup_{x \in \C^n \setminus \qty{0}} \qty|\frac{\iprd{x, Ax}}{\iprd{x,x}}| \leq \sup_{x\in \C^n \setminus \qty{0}} \frac{\norm{Ax}_2 \norm{x}_2}{\norm{x}_2^2} = \sup_{x \in \C^n \setminus \qty{0}} \frac{\norm{Ax}_2}{\norm{x}_2} = \norm{A}
    .\]
    Note first that $\qty|Q_A(x)| \leq r(A) \norm{x}_2^2$. Therefore if $x \in \C^n$ from (1) and the triangle inequality we have
    % \begin{align*}
    %     \qty|\iprd{Ax,Ax}| &\leq \qty|Q_A\qty(Ax)| + \qty|Q_A\qty(0)| + \qty|Q_A\qty(\frac{Ax(1+i)}{2})| + \qty|Q_A\qty(\frac{Ax(1-i)}{2})| \\
    %                        &\leq r(A) \norm{Ax}_2^2 + r(A) \norm{\frac{1+i}{2} \cdot Ax} + r(A)
    % \end{align*}
    \begin{align*}
        \qty|\iprd{Ax,y}| &\leq \qty|Q_A\qty(\frac{x+y}{2})| + \qty|Q_A\qty(\frac{x-y}{2})| + \qty|Q_A\qty(\frac{x+iy}{2})| + \qty|Q_A\qty(\frac{x-iy}{2})| \\
        &\leq r(A) \qty(\norm{\frac{x+y}{2}}_2^2 + \norm{\frac{x-y}{2}}_2^2) + r(A) \qty(\norm{\frac{x+iy}{2}}_2^2 + \norm{\frac{x-iy}{2}}_2^2) \\
        &= r(A) \qty(2\norm{\frac{x}{2}}_2^2 + 2\norm{\frac{y}{2}}_2^2) + r(A) \qty(2\norm{\frac{x}{2}}_2^2 + 2\norm{\frac{iy}{2}}_2^2) \\
        &= r(A) \qty(2\norm{\frac{x}{2}}_2^2 + 2\norm{\frac{y}{2}}_2^2) + r(A) \qty(2\norm{\frac{x}{2}}_2^2 + 2\norm{\frac{y}{2}}_2^2) \\
        &= 2r(A) \qty(2\norm{\frac{x}{2}}_2^2 + 2\norm{\frac{y}{2}}_2^2) \\
        &= 2r(A) \qty(\frac{1}{2}\norm{x}_2^2 + \frac{1}{2}\norm{y}_2^2) \\
    &= r(A) \qty(\norm{x}_2^2 + \norm{y}_2^2) \tag{$*$}
    \end{align*}
    Using Cauchy Schwarz we can reformulate the matrix norm as
    \[
        \norm{A} = \sup_{\norm{x}_2 = \norm{y}_2 = 1} \qty|\iprd{Ax, y}|
    .\]
    By taking this supremum of the extreme sides in the previous derivation, we have
    \[
        \norm{A} = \sup_{\norm{x}_2 = \norm{y}_2 = 1} \qty|\iprd{Ax, y}| \leq \sup_{\norm{x}_2 = \norm{y}_2 = 1} r(A) \qty(\norm{x}_2^2 + \norm{y}_2^2) = r(A) (1^2 + 1^2) = 2r(A)
.\]
    Hence $\norm{A} \leq 2r(A)$.
\end{proof}

\section*{Problem 3}
\begin{example}[$\norm{A} = r(A)$]
    Note that for the identity matrix $I$ that
    \[
        \frac{Q_I(x)}{x^* x} = \frac{x^* I x}{x^* x} = \frac{x^* x}{x^* x} = 1
    \]
    when $x \neq 0$. Therefore $W(I) = \qty{1}$ and hence $r(I) = 1$. We also have
    \[
        \frac{\norm{Ix}_2}{\norm{x}_2} = \frac{\norm{x}_2}{\norm{x}_2} = 1
    \]
    when $x \neq 0$. Therefore $\norm{I} = 1$ as well meaning $\norm{I} = r(I)$. 
\end{example}

\begin{example}[$\norm{A} = 2r(A)$]
    Consider the matrix
        \[
            A = \mqty(0 & 2 \\ 0 & 0)
        .\]
        Take $x \in \C^2 \neq 0$ with components $x_1$ and $x_2$. We have
        \[
            \qty|\frac{Q_A(x)}{x^* x}| = \qty|\frac{x^* A x}{x^* x}| = \qty|\frac{\mqty(\conj{x_1} & \conj{x_2}) \mqty(2x_1 \\ 0)}{|x_1|^2 + |x_2|^2}| = \qty|\frac{2 x_1 \conj{x_2}}{|x_1|^2 + |x_2|^2}| = \frac{2|x_1||x_2|}{|x_1|^2 + |x_2|^2}
        .\]
        Note that
        \[
            0 \leq (|x_1| - |x_2|)^2 = |x_1|^2 - 2|x_1| |x_2| + |x_2|^2 \implies 0 \leq \frac{2|x_1||x_2|}{|x_1|^2 + |x_2|^2} \leq 1
        .\]
        The upper bound is achieved with $x = 1$, hence $r(A) = 1$. Consider $\norm{A}$. Note that
    \[
        \norm{A} = \qty(\sup_{x \in \C^2 \setminus \qty{0}} \frac{\norm{Ax}_2^2}{\norm{x}_2^2})^{\frac{1}{2}}
    \]
    by the monotonicity of the square root. We then have
    \[
        \frac{\norm{Ax}_2^2}{\norm{x}_2^2} = \frac{\norm{\mqty(2x_2 \\ 0)}_2}{|x_1|^2 + |x_2|^2} = \frac{4|x_2|^2}{|x_1|^2 + |x_2|^2}
    .\]
    Note that $|x_2|^2 \leq |x_2|^2 + |x_1|^2$ meaning
    \[
    \frac{|x_2|^2}{|x_1|^2+|x_2|^2} \leq 1 \implies \frac{4|x_2|^2}{|x_1|^2 + |x_2|^2} \leq 4
    .\]

    This upper bound is achieved with $x_1 = 0$ and $x_2 = 1$ hence $\norm{A} = \sqrt{4} = 2$. Therefore $\norm{A} = 2 r(A)$.
\end{example}

\section*{Problem 4}
\begin{proof}
    Consider the characteristic polynomial $p_N(t)$. Since $N$ is purely upper triangular, we have $p_N(t) = \det(N - tI) = (-1)^n t^n$. By the Cayley-Hamilton theorem, we have that $\vb{0} = p_N(N) = (-1)^n N^n$ which means $N^n = \vb{0}$. Therefore $N$ is nilpotent.
\end{proof}


\end{document}
