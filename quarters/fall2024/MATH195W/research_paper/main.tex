\documentclass[12pt]{article}
\usepackage[margin=1in]{geometry}
\usepackage{float}
\usepackage{xcolor}

\usepackage{wrapfig}
\usepackage{subcaption}
\usepackage{amsmath, amsthm, amssymb, mathtools}
\usepackage{newpxtext, newpxmath}

\usepackage{physics}

\NewDocumentCommand{\R}{}{\mathbb{R}}

\usepackage{thmtools}

\RenewDocumentCommand{\qedsymbol}{}{$\blacksquare$}

\declaretheorem{theorem}
\declaretheorem{lemma}
\declaretheorem[style=definition]{definition}

\declaretheorem[style=remark]{example}
\declaretheorem[style=remark]{remark}

\usepackage{tikz}
\usetikzlibrary{arrows}
\usetikzlibrary{arrows.meta}

\usepackage[style=alphabetic]{biblatex}
\addbibresource{sources/library.bib}

\tikzset{
    vertex/.style={
        font = \scriptsize,
        fill=blue!20,
        draw,
        circle,
        inner sep=0pt,
        minimum size=20pt,
    },
    edge/.style={}
}

\usepackage{hyperref}

\hypersetup{
    colorlinks,
    citecolor=blue,
    linkcolor=blue,
    filecolor=blue,      
    urlcolor=blue,
}

\title{The Graph Laplacian and Determining the Connectivity of Meshes}
\author{Eli Griffiths}
\date{November 2024}

\directlua{graph = require("graph")}
\directlua{examples = require("data")}

\begin{document}

\maketitle

\begin{abstract}
    Graphs lend themselves naturally to matrices that encode properties such as which vertices are connected to others and how many vertices are connected to a given vertex. These matrix representations allow us to analyze graphs outside of traditional combinatoric approachs by considering their eigenvalues and eigenvectors. Key to our study is the Laplacian matrix representation of a graph. We first show that the eigenvalues of the Laplacian matrix give insight into if every vertex of a graph is connected to each other via some path through edges, and if not how many separate components there are in which this is the case. We then frame this result within the context of of computational geometry and discuss how more advanced problems in computational geometry can be approached with this spectral framework.
\end{abstract}

\section{Introduction}

Graphs at their core encode the concept of connectivity. A graph is simply vertices and edges where vertices are objects or nodes and edges are connections between these vertices. While simple in concept, this means graphs appear in different fields and applications such modeling friend networks on a social media platform, determining flight schedules for optimal transport, model the structure of chemical compounds, and much more.

Graph theory has always been deeply tied to the field of combinatorics. One of the earliest problems in what we now consider graph theory, Euler's famous K\"onigsberg bridge problem, was solved via methods of counting, a fundamental technique in combinatorics \cite{wilson2013combinatorics}.



\section{Background}

We first outline the basic structure of a graph and related definitions that will be used throughout the paper, following the formalism and notation from \cite{diestelGraphTheory2017}. We adopt the notation that $[S]^n$ is the set of all $n$-element sized subsets of $S$.

\begin{definition}[Graph Structure]
    A \textbf{graph} is a pair $G = (V, E)$ of sets where $E \subseteq [V]^2$. The elements of $V$ are \textbf{vertices} and the elements of $E$ are \textbf{edges}. A vertex $v$ is said to be \textbf{incident} to an edge $e$ if $v \in e$. Two vertices $v_1$ and $v_2$ are \textbf{neighbors} if $\qty{v_1, v_2} \in E$. We denote $\qty{v_1, v_2} \in E$ by $v_1 \sim v_2$. The \textbf{set of neighbors} of a vertex $v$ is denoted by $N(v) \coloneq \qty{w \in V : v \sim w}$. The \textbf{degree} of a vertex $v$ is $\deg(v) \coloneq |N(v)|$. A \textbf{subgraph} $H$ of $G$, denoted by $H \subseteq G$, is a graph whose vertex and edge sets are subsets of $G$'s.
\end{definition}

\begin{remark}
    Edges importantly are defined here as two element sets and not as ordered pairs. This makes the graph \textit{undirected}. If the graph was to be directed (that is the edges were to be ordered pairs) the further results of this paper would not hold in general and would require more restrictions.
\end{remark}

\begin{figure}
    \centering
    \caption{Two example graphs in the plane}
    \label{fig:graph_visual}
    \begin{subfigure}[t]{0.45\textwidth}
        \centering
        \begin{tikzpicture}[scale=0.8]
            \directlua{graph.graph_tikz(examples.example1)}
        \end{tikzpicture}   
        \caption{A graph with $7$ vertices and $15$ edges}
        \label{fig:basic_graph}
    \end{subfigure}\hfill
    \begin{subfigure}[t]{0.45\textwidth}
        \centering
        \begin{tikzpicture}[scale=0.8]
            \directlua{graph.graph_tikz(examples.example2)}
        \end{tikzpicture}   
        \caption{A graph with $2$ connected components}
        \label{fig:connected_components}
    \end{subfigure}
\end{figure}

For the purposes of this paper, we will assume that every graph has finitely many vertices (and hence finitely many edges). A way to visualize a finite graph is by representing vertices as points in the plane and edges as line segments between these points as seen in Figure \ref{fig:graph_visual}. Notice that the graphs in Figure \ref{fig:graph_visual} differ in terms of how connected the vertices are to one other. If each vertex was a city and the edges roads, a car on the graph in \ref{fig:basic_graph} could get to any city, but cars on the outer ring in \ref{fig:connected_components} are stuck there. Intuitively, \ref{fig:connected_components} appears to be in two separate pieces. This idea of connectedness and connected sections of a graph is the key property of graphs we are interested in.

\begin{definition}[Connectedness]
    A \textbf{path} is a non-empty graph $P = (V,E)$ where
    \[
        V = \qty{v_0, v_1, \ldots, v_k}, E = \qty{\qty{v_0,v_1}, \ldots, \qty{v_{k-1}, v_k}}
    .\]
    A graph $G$ is then \textbf{connected} if between any two vertices $v_0$ and $v_f$ there exists a path $P \subseteq G$ starting at $v_0$ and ending at $v_f$. A \textbf{connected component} of a graph is a subgraph $H \subseteq G$ that is connected and is not contained in any larger connected subgraph.
\end{definition}

\begin{definition}
    The \textbf{adjacency matrix} $A_G$ and \textbf{degree matrix} $D_G$ of a finite graph $G$ with $n$ vertices are the $n \times n$ matrices such that
    \begin{align*}
        (A_G)_{ij} = \begin{cases}
            1 & v_i \sim v_j \\
            0 & \text{otherwise}
        \end{cases} & &
        (D_G)_{ij} = \begin{cases}
            \deg(v_i) & i = j \\
            0 & i \neq j
        \end{cases}
    \end{align*}
\end{definition}

\begin{example}
    The adjacency and degree matrices for the graph in Figure \ref{fig:basic_graph} are

    \begin{align*}
        A_G =    
        \begin{bmatrix}
            \directlua{graph.adj_matrix(examples.example1)}
        \end{bmatrix} & &
        D_G = \begin{bmatrix}
            \directlua{graph.deg_matrix(examples.example1)}
        \end{bmatrix}
    \end{align*}
\end{example}


\section{A Graph's Spectra}


\begin{definition}
    The \textbf{Laplacian Matrix} of a graph $G$ is $L_G \coloneq D_G - A_G$. If $G$ is understood via context, we simply refer to it as $L$.
\end{definition}

\begin{theorem}
    Let $L$ be the laplacian matrix of a finite graph $G$ with $n$ vertices. Then for a given vector $x \in \R^n$,
    \[
        x^T L x = \sum_{v_i \sim v_j} (x_i - x_j)^2.
    \]
    We call $x^T L x$ the quadratic form of $L$.
\end{theorem}

\begin{proof}
    We can rewrite $x^T L x = x^T (D - A) x$ using the definition of the Laplacian matrix. Expanding out we get
    \[
        x^T L x = x^T D x - x^T A x
    .\]
    First consider the term $x^T D x$. By the definition of $D$, we know that $(D)_{ij} = \deg(v_i)$ when $i = j$ and $0$ otherwise. Therefore
    \[
        x^T D x = \sum_{i=1}^n \sum_{j=1}^n x_i (D)_{ij} x_j = \sum_{i=1}^n \deg(v_i) \cdot x_i^2
    .\]
    Consider some vertex $v \in V$. Then $\deg(v)$ is also the number of edges containing $v$. Therefore we can rewrite the previous sum as a sum over the edges,
    \[
        \sum_{i=1}^n \deg(v_i) \cdot x_i^2 = \sum_{v_i \sim v_j} x_i^2 + x_j^2
    .\]
    Now consider $x^T A x$. By the definition of $A$, we know that $(A)_{ij} = 1$ if $v_i \sim v_j$ and $0$ otherwise. Therefore
    \[
        x^T A x = \sum_{i=1}^n \sum_{j=1}^n x_i (A)_{ij} x_j = 2 \cdot\sum_{v_i \sim v_j} x_i x_j
    .\]
    Note the factor of $2$ is needed because summing over entries of $A$ double counts edges. We have then
    \[
        x^T L x = x^T D x - x^T A x = \sum_{v_i \sim v_j} \qty(x_i^2 - 2 x_i x_j + x_j^2) = \sum_{v_i \sim v_j} (x_i - x_j)^2
        .\qedhere
    \]
\end{proof}

In order to make the connection between the eigenvalues of $L$, its quadratic form, and the connectivity of $G$, we will need some results from linear algebra. While we wont prove these here, a proof for Lemma \ref{lem:eig_multiplicity} can be found in \cite{} and a proof for Lemma \ref{lem:block_det} in \cite{silvester2000determinants}.

\begin{lemma}
    \label{lem:eig_multiplicity}
    If $\lambda$ is an eigenvalue of a symmetric matrix with algebraic multiplicity $k$, it has $k$ linearly independent associated eigenvectors.
\end{lemma}

\begin{lemma}
    \label{lem:block_det}
    For a block diagonal matrix $A$ with blocks $A_i$, $\det(A) = \det(A_1) \det(A_2) \cdots \det(A_n)$
\end{lemma}

\begin{lemma}
    A connected graph $G$ is connected if and only if the $0^\text{th}$ eigenvalue of $L_G$ has algebraic multiplicity $1$.
\end{lemma}
\begin{proof}
\end{proof}

\begin{theorem}
    A graph $G$ has $k$ connected components if and only if the $0^\text{th}$ eigenvalue of $L_G$ has algebraic multiplicity $k$.
\end{theorem}

\begin{proof}
    
\end{proof}

\section{Application to Computational Geometry}
\begin{wrapfigure}{r}{0.25\textwidth}
    \centering
    \begin{tikzpicture}
        \node[anchor = south] (mesh) at (0,0.5) {
            \includegraphics[width=0.25\textwidth]{figures/vase_dual.png}
        };
        \node[anchor = north] (dual) at (0,-0.5) {
            \includegraphics[width=0.25\textwidth]{figures/vase_dual.png}
        };
    \draw[-implies,double equal sign distance, very thick] (mesh) -- (dual) node[pos=0.5, anchor=west,right=4mm] {\text{Dual}};
    \end{tikzpicture}
\end{wrapfigure}

In computer graphics, modeling, simulation, etc. we often want a representation of some real world geometry that we can perform computations on. A common way to of doing so is via a mesh. Consider the example mesh of a cat to the right. From some basic observations, the mesh appears to be comprised of points connected by segments/edges which outline faces. We formalize these observations in Definition \ref{def:tri_mesh}.

\begin{definition}
    \label{def:tri_mesh}
    A \textbf{triangular mesh} is a triple $K = (V, E, F)$ such that
    \begin{itemize}
        \item $V \subseteq \R^3$ is a finite set representing the vertices
        \item $E \subseteq [V]^2$ is a set representing non-intersecting edges
        \item $F \subseteq [E]^3$ is the set of faces such that for any $f = \qty{e_1, e_2, e_3} \in F$, we have $e_1 \cap e_2 = \qty{v_1}$, $e_2 \cap e_3 = \qty{v_2}$, and $e_3 \cap e_1 = \qty{v_3}$ for $v_1 \neq v_2 \neq v_3$. 
    \end{itemize}
\end{definition}

Notice that a triangular mesh lends itself to some very natural graph structures. One is simply using the vertices as the graph vertices and the edges as graph edges. The other one which is of use to us is using the faces as graph vertices and faces sharing a common edge as the edge set. This is often referred to as the dual graph of a mesh and appears in computational geometry problems such as mesh based signal processing \cite{taubinDualMeshResampling2002} and refinement of meshed implicit surfaces \cite{10.1145/566282.566308}. 

\begin{definition}
    The \textbf{dual mesh} of a triangular mesh $K = (V_K, E_K, F_K)$ is the graph $G = (V, E)$ such that $V = F_K$ and $f_1 \sim f_2$ if $f_1 \cap f_2 = \qty{e}$ for some in $e \in E_K$.
\end{definition}

\begin{theorem}
    A triangular mesh has $k$ connected components if and only the $0^\text{th}$ eigenvalue of the Laplace matrix of its dual mesh has algebraic multiplicity $k$.
\end{theorem}

\newpage
\nocite{merrisLaplacianMatricesGraphs1994}
\printbibliography

\end{document}
