\documentclass[12pt,titlepage]{extarticle}
% Document Layout and Font
\usepackage{subfiles}
\usepackage[margin=2cm, headheight=15pt]{geometry}
\usepackage{fancyhdr}
\usepackage{enumitem}	
\usepackage{wrapfig}
\usepackage{multicol}
\usepackage{caption, subcaption}

\usepackage[p,osf]{scholax}

\renewcommand*\contentsname{Table of Contents}
\renewcommand{\headrulewidth}{0pt}
\pagestyle{fancy}
\fancyhf{}
\fancyfoot[R]{$\thepage$}
\setlength{\parindent}{0cm}
\setlength{\headheight}{17pt}
\hfuzz=9pt

% Utility Management
\usepackage{color}
\usepackage{colortbl}
\usepackage{xcolor}
\usepackage{xpatch}
\usepackage{xparse}

\definecolor{links}{HTML}{1c73a5}
\definecolor{bar}{HTML}{584AA8}

% Math Packages
\usepackage{mathtools, amsmath, amsthm, thmtools, amssymb, physics}
\usepackage[scaled=1.075,ncf,vvarbb]{newtxmath}

\newcommand\B{\mathbb{B}}
\newcommand\C{\mathbb{C}}
\newcommand\R{\mathbb{R}}
\newcommand\Q{\mathbb{Q}}
\newcommand\N{\mathbb{N}}
\newcommand\Z{\mathbb{Z}}

\newcommand\Prob[1]{\mathbb{P}\qty(#1)}
\newcommand\Var[1]{\text{Var}\qty(#1)}
\newcommand\Exp[1]{\mathbb{E}\qty[#1]}
\newcommand\ball[1]{\B\qty(#1)}
\newcommand\res[1]{\underset{#1}{\operatorname{Res}}\;}
\renewcommand\pv{\mathrm{p.v.}}

\newcommand\conj[1]{\overline{#1}}
\DeclareMathOperator{\Arg}{Arg}
\DeclareMathOperator{\Log}{Log}
\DeclareMathOperator{\cis}{cis}

\DeclareMathOperator{\dom}{dom}
\DeclareMathOperator{\spann}{span}
\DeclareMathOperator{\nullity}{nullity}

\newcommand\st{\text{ s.t. }}

% TIKZ
\usepackage{tikz}
\usepackage{pgfplots}
\usetikzlibrary{arrows.meta}
\usetikzlibrary{math}
\usetikzlibrary{cd}
\usetikzlibrary{patterns}
\usetikzlibrary{decorations.markings}
\usetikzlibrary{calc}

% Boxes and Theorems
\usepackage[most]{tcolorbox}
\tcbuselibrary{skins}
\tcbuselibrary{breakable}
\tcbuselibrary{theorems}

\newtheoremstyle{default}{0pt}{0pt}{}{}{\bfseries}{\normalfont.}{0.5em}{}
\theoremstyle{default}

\renewcommand*{\proofname}{\textit{\textbf{Proof.}}}
\renewcommand*{\qedsymbol}{$\blacksquare$}
\tcolorboxenvironment{proof}{
	breakable,
	coltitle = black,
	colback = white,
	frame hidden,
	boxrule = 0pt,
	boxsep = 0pt,
	borderline west={3pt}{0pt}{bar},
	sharp corners = all,
	enhanced,
}

\newtheorem{theorem}{Theorem}[section]{\bfseries}{}
\tcolorboxenvironment{theorem}{
	breakable,
	enhanced,
	boxrule = 0pt,
	frame hidden,
	coltitle = black,
	colback = blue!7,
	left = 0.5em,
	sharp corners = all,
}

\newtheorem{corollary}{Corollary}[section]{\bfseries}{}
\tcolorboxenvironment{corollary}{
	breakable,
	enhanced,
	boxrule = 0pt,
	frame hidden,
	coltitle = black,
	colback = white!0,
	left = 0.5em,
	sharp corners = all,
}

\newtheorem{lemma}{Lemma}[section]{\bfseries}{}
\tcolorboxenvironment{lemma}{
	breakable,
	enhanced,
	boxrule = 0pt,
	frame hidden,
	coltitle = black,
	colback = green!7,
	left = 0.5em,
	sharp corners = all,
}

\newtheorem{definition}{Definition}[section]{\bfseries}{}
\tcolorboxenvironment{definition}{
	breakable,
	coltitle = black,
	colback = white,
	frame hidden,
	boxsep = 0pt,
	boxrule = 0pt,
	borderline west = {3pt}{0pt}{orange},
	sharp corners = all,
	enhanced,
}

\newtheorem{example}{Example}[section]{\bfseries}{}
\tcolorboxenvironment{example}{
	% title = \textbf{Example},
	% detach title,
	% before upper = {\tcbtitle\quad},
	breakable,
	coltitle = black,
	colback = white,
	frame hidden,
	boxrule = 0pt,
	boxsep = 0pt,
	borderline west={3pt}{0pt}{green!70!black},
	sharp corners = all,
	enhanced,
}

\newtheoremstyle{remark}{0pt}{4pt}{}{}{\bfseries\itshape}{\normalfont.}{0.5em}{}
\theoremstyle{remark}
\newtheorem*{remark}{Remark}


% TColorBoxes
\newtcolorbox{week}{
	colback = black,
	coltext = white,
	fontupper = {\large\bfseries},
	width = 1.2\paperwidth,
	size = fbox,
	halign upper = center,
	center
}

\newcommand{\banner}[2]{
    \pagebreak
    \begin{week}
   		\section*{#1}
    \end{week}
    \addcontentsline{toc}{section}{#1}
    \addtocounter{section}{1}
    \setcounter{subsection}{0}
}

% Hyperref
\usepackage{hyperref}
\hypersetup{
	colorlinks=true,
	linktoc=all,
	linkcolor=links,
	bookmarksopen=true
}


\def\homeworknumber{3}
\usepackage{fancyhdr}
\pagestyle{fancy}
\fancyhead[R]{HW \#\thehwnumber}
\fancyhead[C]{\textbf{Math 130B}}
\fancyhead[L]{Eli Griffiths}


% 7.1 and 7.2 for a,d, 7.3, 7.4, 7.5b 
% 8.1c, 8.1d, 8.2b, 8.2e, 8.4, 8.5a, 8.9a

\begin{document}

\subsection*{7.1/7.2}
\begin{multicols}{2}
\subsubsection*{Part A}
\begin{align*}
    s_1 &= \frac{1}{4} \\
    s_2 &= \frac{1}{7} \\
    s_3 &= \frac{1}{10} \\
    s_4 &= \frac{1}{13} \\
    s_5 &= \frac{1}{16}
\end{align*}

The sequence converges to $0$.

\subsubsection*{Part D}
\begin{align*}
    s_1 &= \frac{\sqrt{2}}{2} \\
    s_2 &= 1 \\
    s_3 &= \frac{\sqrt{2}}{2} \\
    s_4 &= 0 \\
    s_5 &= -\frac{\sqrt{2}}{2} \\
\end{align*}

The sequence does not converge as it cycles.
\end{multicols}

\subsection*{7.3}
\begin{table}[h!]
    \def\arraystretch{1.5}
\begin{tabular}{l|l|l|l|l|l|l|l|l|l|l|l|l|l|l|l|l|l|l|l}
A & B & C & D & E   & F & G        & H   & I & J             & K        & L   & M   & N   & O & P & Q & R & S             & T \\\hline
1 & 1 & 0 & 1 & DNC & 1 & $\infty$ & DNC & 0 & $\frac{7}{2}$ & $\infty$ & DNC & DNC & DNC & 0 & 2 & 0 & 1 & $\frac{4}{3}$ & 0
\end{tabular}
\end{table}

\subsection*{7.4}
\subsubsection*{Part A}
\[
    s_n = \frac{\sqrt{2}}{n} \in \mathbb{I}, \lim_{n\to \infty} s_n = 0 \in \mathbb{Q}
\]

\subsubsection*{Part B}
Let $F_n$ denote the $n$'th fibonacci number with $F_1 = F_2 = 1$.
\[
    s_n = \frac{F_{n+1}}{F_n} \in \mathbb{Q}, \lim_{n\to \infty} s_n = \phi = \frac{1+\sqrt{5}}{2} \in \mathbb{I}
\]

\subsection*{7.5}
\subsubsection*{Part B}
Note that $\sqrt{n^2 + n} - n = \frac{n}{\sqrt{n^2+n} + n} \sim \frac{n}{2n}$. Therefore
\[
    \lim_{n\to\infty} = \sqrt{n^2 + n} - n = \frac{1}{2}
\]
\subsection*{8.1}
\subsubsection*{Part C}
\begin{proof}
    Take $\epsilon > 0$. Let $N \in \mathbb{N} > \frac{3}{5\epsilon}$. Note that
    \[
        \qty|\frac{2n-1}{3n+2} - \frac{2}{3}| = \qty|\frac{7}{3} \cdot \frac{1}{3n+2}| = \frac{7}{3} \cdot \frac{1}{3n+2} \leq \frac{3}{5n}.
    \]
    For $n > N$, $\frac{3}{5n} \leq \frac{3}{5 N} < \epsilon$. Therefore
    \[
        \qty|\frac{2n-1}{3n+2} - \frac{2}{3}| < \epsilon, \forall n > N
    \]
    hence $\lim_{n\to \infty} \frac{2n-1}{3n+2} = \frac{2}{3}$.
\end{proof}

\subsubsection*{Part D}
\begin{proof}
    Take $\epsilon > 0$. Let $N \in \mathbb{N} > \max\qty{6, 6 + \frac{1}{\epsilon}}$. Note that $N > 6$, meaning for all $n > N$
    \[
        \qty|\frac{n+6}{n^2 - 6}| = \frac{n+6}{n^2 - 6}.
    \]
    Since $0 < n^2 - 36 < n^2 - 6$ for all $n > N$,
    \[
        \frac{n+6}{n^2 - 6} \leq \frac{n+6}{n^2-36} = \frac{1}{n-6} < \frac{1}{N - 6} < \epsilon, \forall n > N.
    \]
    Therefore, $\qty|\frac{n+6}{n^2-6}| < \epsilon$ for all $n > N$, hence $\lim_{n \to \infty}\frac{n+6}{n^2-6} = 0$.
\end{proof}

\subsection*{8.2}
\subsubsection*{Part B}
\[
    \lim_{n\to \infty} \frac{7n-19}{3n+7} = \frac{7}{3}
\]
\begin{proof}
    Take $\epsilon > 0$. Let $N \in \mathbb{N} > \frac{5}{9\epsilon}$. Note that
    \[
        \qty|\frac{7n-19}{3n+7} - \frac{7}{3}| = \qty|\frac{-5}{9n+21}| = \frac{5}{9n + 21}.
    \]
    Since $9n \leq 9n + 21$,
    \[
        \frac{5}{9n + 21} \leq \frac{5}{9n} < \frac{5}{9N} < \epsilon, \forall n > N.
    \]
    Therefore
    \[
        \qty|\frac{7n-19}{3n+7} - \frac{7}{3}| < \epsilon, \forall n > N
    \]
    hence $\lim_{n\to \infty}\frac{7n-19}{3n+7} = \frac{7}{3}$.
\end{proof}

\subsubsection*{Part E}
\[
    \lim_{n\to \infty} \frac{1}{n} \sin(n) = 0
\]
\begin{proof}
    Take $\epsilon > 0$. Take $N \in \mathbb{N} > \epsilon$. Note that since $-1 \leq \sin(x) \leq 1$ for all $x \in \mathbb{R}$, $|\sin(n)| \leq 1$ for all $x \in \mathbb{R}$. Therefore for $n \in \mathbb{N}$,
    \[
        \qty|\frac{1}{n} \sin(n)| = \frac{1}{n} |\sin(n)| \leq \frac{1}{n}
    .\]
    For $n > N$, $\frac{1}{n} < \frac{1}{N} < \epsilon$. Therefore
    \[
        \qty|\frac{1}{n} \sin(n)| < \epsilon, \forall n < N
    \]
    hence $\lim_{n\to \infty}\frac{1}{n} \sin(n) = 0$.
\end{proof}

\subsection*{8.4}
\begin{proof}
    Let $t_n$ be a bounded sequence and $s_n$ be a sequence that converges to $0$. Since $t_n$ is bounded, $|t_n| \leq M \in \mathbb{R}$ for all $n \in \mathbb{N}$. Since $s_n$ converges to $0$,
    \[
        \forall\epsilon > 0, \exists N \in \mathbb{N} \text{ s.t. } |s_n| < \frac{\epsilon}{M}, \forall n > N
    \]
    Note that $|s_n \cdot t_n| = |s_n| |t_n| \leq |s_n| M < \epsilon$ for all $n > N$. Therefore $\lim_{n\to \infty} t_n s_n = 0$.
\end{proof}

\subsection*{8.5}
\subsubsection*{Part A}
\begin{proof}
    Let $a_n, b_n, s_n$ be sequences such that $a_n \leq s_n \leq b_n$ for all $n \in \mathbb{N}$ and $\lim_{n \to \infty} a_n = \lim_{n \to \infty} b_n = s \in \mathbb{R}$. Since $a_n$ and $b_n$ converge, for any $\epsilon > 0$,
    \begin{align*}
        \exists N_1 \in \mathbb{N} \text{ s.t. } |a_n - s| < \epsilon, \forall n > N_1 &\implies a_n > s - \epsilon, \forall n > N_1 \\
        \exists N_2 \in \mathbb{N} \text{ s.t. } |b_n - s| < \epsilon, \forall n > N_2 &\implies b_n < s + \epsilon, \forall n > N_2
    \end{align*}
    By taking $N = \max\qty{N_1, N_2}$, it follows that  that $a_n > s - \epsilon$ and $b_n < s + \epsilon$ for all $n > N$. Note then that
    \[
        s - \epsilon < a_n \leq s_n \leq b_n < s + \epsilon \implies |s_n - s| < \epsilon, \forall n > N.
    \]
    Therefore $\lim_{n\to \infty} s_n = s$.
\end{proof}

\subsection*{8.9}
\subsubsection*{Part A}
\begin{proof}
    Let $s_n$ be a convergent sequence such that $s_n \geq a \in \mathbb{R}$ for all but finitely many $n$. Let $s = \lim_{n \to \infty} s_n$ and let $S = \qty{n \in \mathbb{N} : s_n < a}$. Since $S$ is finite, choose $N = \max S$. Note that then for all $n > N$, $s_n \geq a$. Assume towards contradiction that $s < a$. Then $a - s > 0$. Choose $\epsilon > 0$ such that $0 < \epsilon < a - s$. Since $s_n$ converges, $\exists N_0 \in \mathbb{N}$ such that $s_n < s + \epsilon < a$ for all $n > N_0$. This also holds for all $n > \max{N, N_0}$. However, this means that there is an $n > N$ such that $s_n < a$, which contradicts the fact that $N$ is the maximal index such that $s_N < a$. Therefore $s \geq a$.
\end{proof}

\end{document}
