\documentclass[../notes.tex]{subfiles}
\graphicspath{
    {'../figures'}
}

\begin{document}

\banner{Series}

\begin{definition}[Summation]
    Given a sequence $(a_n)$ starting at $m$, then
    \[
        S_n \coloneq \sum_{k = m}^{n} a_k, n \geq m
    \]
    and $(S_n)_{n \geq m}$ is the sequence of partial sums. Then
    \[
        \sum_{k = m}^\infty a_k \text{ converges } \Leftrightarrow (S_n)_{n \geq m} \text{ converges.}
    \]
    Furthermore, if $\lim S_n = s$, then
    \[
        \lim_{n\to \infty} \sum_{k = m}^n a_k = s
    .\]
\end{definition}
\begin{remark}
    Note the following properties for the sequence of partial sums
    \begin{enumerate}[label=\alph*)]
        \item $a_k \geq 0$ for all $k \geq m$, then $(S_n)_{n \geq m}$ is an increasing sequence and either converges or diverges to $\infty$.
        \item As a consequence, $\displaystyle \sum_{k = m}^n a_k$ is always meaningful.
    \end{enumerate}
\end{remark}

The last property motivates defining another form of convergence.

\begin{definition}[Absolute Convergence]
    $\displaystyle \sum_{k = m}^\infty a_k$ converges absolutely if $\displaystyle \sum_{k=m}^\infty |a_k|$ converges.
\end{definition}

\begin{example}
    Note that $(1-r)(1 + r + r^2 + \ldots + r^n) = 1 + r^{n+1}$. Therefore
    \[
        (1 + r + r^2 + \ldots + r^n) = \sum_{k = 0}^n r^k = \frac{1 + r^{n+1}}{1-r}, \forall n \geq 0
    .\]
\end{example}

\begin{example}
    Consider the sum
    \[
        \sum_{n = 1}^{\infty} \frac{1}{n^p}
    \]
    converges if and only if $p > 1$.
\end{example}

\begin{definition}[Cauchy Series]
    A series $\sum a_k$ satisfies the Cauchy criterion if its sequence of partial sums $(S_n)$ is Cauchy.
\end{definition}

\begin{theorem}[Cauchy Criterion]
    \label{thm:cauchycriterion}
    Let $\sum a_k$ be a series.
    \begin{enumerate}
        \item $\sum a_k$ converges iff it satisfies the Cauchy Criterion
        \item If $\sum a_k$ converges, then $\lim a_n = 0$.
    \end{enumerate}
\end{theorem}

\begin{proof}
    Let $\sum a_k$ be a series.
    \begin{enumerate}
        \item
            \begin{enumerate}
                \item[$\Rightarrow)$] Assume that $\sum a_k$ converges. Then the sequence of partial sums is convergent and therefore Cauchy. Hence by the definition of the Cauchy criterion $\sum a_k$ is Cauchy.
                \item[$\Leftarrow)$] Assume that $\sum a_k$ is Cauchy. Therefore the sequence of partial sums are Cauchy and hence converge. Therefore $\sum a_k$ converges.
            \end{enumerate}
    \end{enumerate}
\end{proof}

\begin{remark}
    Note that in the prior theorem that the second statement is not an if or only if. Consider the Harmonic series
    \[
        \sum_{n = 1}^\infty \frac{1}{n}
    .\]
    Since
    \[
        \sum_{k = n}^{2n} \frac{1}{k} \geq n\cdot \frac{1}{2n} = \frac{1}{2} \text{ for all $n$}
    \]
    $(S_n)$ cannot be Cauchy and therefore the partial sums do not converge, meaning the series doesn't converge. But importantly note that $a_n \to 0$, hence why the theorem is not and iff.
\end{remark}

\begin{theorem}
    Let $\sum a_k$ and $\sum b_k$ be series with $a_n \geq 0$ for all $n$. Then
    \begin{enumerate}
        \item If $\sum a_k$ converges and $|b_k| \leq a_k$ for all $k$, then $\sum b_k$ also converges
        \item If $\sum a_k = \infty$ and $b_k \geq a_k$ for all $k$, then $\sum b_k = \infty$
        \item Absolute convergence implies convergence
    \end{enumerate}
\end{theorem}
\begin{proof}
    Let $\sum a_k$ and $\sum b_k$ be series with $a_n \geq 0$ for all $n$. Then
    \begin{enumerate}
        \item%-----------------------------------------------------------------
        Assume that $\sum a_k$ converges and $|b_k| \leq a_k$ for all $k$. Note that
        \[
            0 \leq \qty|\sum_{k = m}^n b_k| \leq \sum_{k = m}^n |b_k| \leq \sum_{k = m}^n a_k
        .\]
        Since $\sum a_k$ converges, for all $\epsilon > 0$ there is an $N \in \mathbb{N}$ such that $\sum_{k = m}^n a_k < \epsilon$ which by the previous inequality implies that $\sum b_k$ satisfies the Cauchy criterion. Hence $\sum b_k$ converges.
        \item%-----------------------------------------------------------------
        % TODO
        \item%-----------------------------------------------------------------
        If $\sum b_k$ absolutely converges, then
        \[
            0 \leq \qty|\sum_{k = m}^n b_k| \leq \sum_{k = m}^n \qty|b_k|
        .\]
        Therefore $\sum b_k$ satisfies the Cauchy criterion and hence converges.
    \end{enumerate}
\end{proof}

\begin{theorem}[Ratio Test]
    Let $\sum a_k$ be a series where $a_n \neq 0$ for all $n$.
    \begin{enumerate}
        \item $\sum a_k$ absolutely converges if $\limsup \qty|\frac{a_{n+1}}{a_n}| < 1$
        \item $\sum a_k$ diverges if $\liminf \qty|\frac{a_{n+1}}{a_n}| > 1$
        \item If $\liminf \qty|\frac{a_{n+1}}{a_n}| \leq 1 \leq \limsup \qty|\frac{a_{n+1}}{a_n}|$, then nothing can be concluded about $\sum a_k$
    \end{enumerate}
\end{theorem}

\begin{theorem}
    For some series $\sum a_k$, define $\alpha \coloneq \limsup |a_n|^{\frac{1}{n}}$. Then
    \begin{enumerate}
        \item If $\alpha < 1$, $\sum a_k$ absolutely converges
        \item If $\alpha > 1$, $\sum a_k$ diverges
        \item If $\alpha = 1$, then nothing can be concluded about $\sum a_k$
    \end{enumerate}
\end{theorem}
\begin{proof}
    Let $\sum a_k$ be a series and define $\alpha \coloneq \limsup |a_n|^{\frac{1}{n}}$.
    \begin{enumerate}
        \item%-----------------------------------------------------------------
        Assume that $\alpha < 1$. Then choose $0 < \epsilon < 1 - \alpha$. Then $\exists N \in \mathbb{N}$ such that
        \[
            \alpha - \epsilon < \sup_{n > N} |a_n|^{\frac{1}{n}} < \alpha + \epsilon
        \]
        meaning that for all $n > N$ that
        \begin{align*}
            |a_n| < (\alpha + \epsilon)^n \tag{$*$}
        \end{align*}
        By the choice of $\epsilon$, $\alpha + \epsilon < 1$ which means that
        \[
            \sum_n (\alpha + \epsilon)^n
        \]
        is convergent. Therefore by $(*)$, $\sum |a_k|$ converges.
        \item%-----------------------------------------------------------------
        Assume that $\alpha > 1$. Then by the definition of the limit superior, $\exists$ a subsequence $(n_k)$ such that $|a_{n_k}|^{\frac{1}{n_k}} > 1$ for all $k$. Then $|a_{n_k}| > 1$ for all $k$. Therefore $a_{n_k} \not\to 0$ which means $\sum a_k$ diverges by the converse of statement 2 in \ref{thm:cauchycriterion}.
        \item%-----------------------------------------------------------------
        Take $a_n = \frac{1}{n}$. Then $\alpha = \limsup \frac{1}{\sqrt[n]{n}} = 1$. Take $a_n = \frac{1}{n^2}$. Then $\alpha = \limsup \frac{1}{n} = 1$. Therefore a series with $\alpha = 1$ may or may not diverge, meaning nothing can be concluded about $\sum a_k$ in general.
    \end{enumerate}
\end{proof}

\begin{remark}
    Let $\sum a_k$ be a series with $a_n \neq 0$ for all $n$. If $\lim \qty|\frac{a_n+1}{a_n}| = 1$ implies that $\limsup |a_n|^{\frac{1}{n}} = 1$.
\end{remark}

\begin{example}
    Consider the series $\sum \frac{n}{n^2 + 3}$. Since
    \[
        \frac{n}{n^2 + 3} \geq \frac{n}{n^2 + 3n^2} \geq \frac{1}{4} \cdot \frac{1}{n}
    \]
    the series diverges. Note that
    \[
        \qty|\frac{a_{n+1}}{a_n}| = \frac{n+1}{n} \cdot \frac{n^2 + 3}{n^2 + 2n + 4} \xrightarrow{n \to \infty} 0
    .\]
\end{example}

\begin{example}
    Consider the series $\sum \frac{n}{3^n}$. Then
    \[
        \qty|\frac{a_{n+1}}{a_n}| = \frac{n+1}{n} \cdot \frac{1}{3} \xrightarrow{n \to \infty} \frac{1}{3} < 1
    .\]
    Therefore the series converges by the ratio test.
\end{example}

\begin{example}
    Consider the series
    \[
        \sum_{n \geq 0} 2^{(-1)^n - n} = 2 + \frac{1}{4} + \frac{1}{2} + \frac{1}{16} + \ldots
    \]
    Since $a_n < \frac{1}{2^{n-1}}$ for all $n$, the series converges by comparison. Additionally,
    \begin{align*}
        \frac{2^{(-1)^{n+1} - n - 1}}{2^{(-1)^n - n}} = \begin{cases}
            \frac{1}{8} & n \text{ even} \\
            2 & n \text{ odd}
        \end{cases}
    \end{align*}
    which means that $\liminf a_n = \frac{1}{8} < 1 < 2 = \limsup a_n$, therefore nothing can be concluded by the ratio test. Attempt to use the root test. Then
    \begin{align*}
        |a_n|^{\frac{1}{n}} = \begin{cases}
            2^{\frac{1}{n} - 1} & n \text{ even} \\
            2^{-\frac{1}{n} - 1} & n \text{ odd}
        \end{cases}
    \end{align*}
    Therefore $\liminf |a_n|^{\frac{1}{n}} = \limsup |a_n|^{\frac{1}{n}} = \lim |a_n|^{\frac{1}{n}} = \frac{1}{2}$ since $2^{\frac{1}{n}} \to 1$ and $2^{-\frac{1}{n}} \to 1$, meaning the series converges.
\end{example}

\begin{example}
    Consider the sereies
    \[
        \sum_{n \geq 1} \frac{(-1)^n}{\sqrt{n}}
    .\]
    For this series, the root test, ratio test, and comparison are useless because of the alternating sign of each term.
\end{example}

\end{document}
