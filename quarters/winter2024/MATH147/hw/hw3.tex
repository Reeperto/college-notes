\documentclass[12pt,titlepage]{extarticle}
% Document Layout and Font
\usepackage{subfiles}
\usepackage[margin=2cm, headheight=15pt]{geometry}
\usepackage{fancyhdr}
\usepackage{enumitem}	
\usepackage{wrapfig}
\usepackage{multicol}
\usepackage{caption, subcaption}

\usepackage[p,osf]{scholax}

\renewcommand*\contentsname{Table of Contents}
\renewcommand{\headrulewidth}{0pt}
\pagestyle{fancy}
\fancyhf{}
\fancyfoot[R]{$\thepage$}
\setlength{\parindent}{0cm}
\setlength{\headheight}{17pt}
\hfuzz=9pt

% Utility Management
\usepackage{color}
\usepackage{colortbl}
\usepackage{xcolor}
\usepackage{xpatch}
\usepackage{xparse}

\definecolor{links}{HTML}{1c73a5}
\definecolor{bar}{HTML}{584AA8}

% Math Packages
\usepackage{mathtools, amsmath, amsthm, thmtools, amssymb, physics}
\usepackage[scaled=1.075,ncf,vvarbb]{newtxmath}

\newcommand\B{\mathbb{B}}
\newcommand\C{\mathbb{C}}
\newcommand\R{\mathbb{R}}
\newcommand\Q{\mathbb{Q}}
\newcommand\N{\mathbb{N}}
\newcommand\Z{\mathbb{Z}}

\newcommand\Prob[1]{\mathbb{P}\qty(#1)}
\newcommand\Var[1]{\text{Var}\qty(#1)}
\newcommand\Exp[1]{\mathbb{E}\qty[#1]}
\newcommand\ball[1]{\B\qty(#1)}
\newcommand\res[1]{\underset{#1}{\operatorname{Res}}\;}
\renewcommand\pv{\mathrm{p.v.}}

\newcommand\conj[1]{\overline{#1}}
\DeclareMathOperator{\Arg}{Arg}
\DeclareMathOperator{\Log}{Log}
\DeclareMathOperator{\cis}{cis}

\DeclareMathOperator{\dom}{dom}
\DeclareMathOperator{\spann}{span}
\DeclareMathOperator{\nullity}{nullity}

\newcommand\st{\text{ s.t. }}

% TIKZ
\usepackage{tikz}
\usepackage{pgfplots}
\usetikzlibrary{arrows.meta}
\usetikzlibrary{math}
\usetikzlibrary{cd}
\usetikzlibrary{patterns}
\usetikzlibrary{decorations.markings}
\usetikzlibrary{calc}

% Boxes and Theorems
\usepackage[most]{tcolorbox}
\tcbuselibrary{skins}
\tcbuselibrary{breakable}
\tcbuselibrary{theorems}

\newtheoremstyle{default}{0pt}{0pt}{}{}{\bfseries}{\normalfont.}{0.5em}{}
\theoremstyle{default}

\renewcommand*{\proofname}{\textit{\textbf{Proof.}}}
\renewcommand*{\qedsymbol}{$\blacksquare$}
\tcolorboxenvironment{proof}{
	breakable,
	coltitle = black,
	colback = white,
	frame hidden,
	boxrule = 0pt,
	boxsep = 0pt,
	borderline west={3pt}{0pt}{bar},
	sharp corners = all,
	enhanced,
}

\newtheorem{theorem}{Theorem}[section]{\bfseries}{}
\tcolorboxenvironment{theorem}{
	breakable,
	enhanced,
	boxrule = 0pt,
	frame hidden,
	coltitle = black,
	colback = blue!7,
	left = 0.5em,
	sharp corners = all,
}

\newtheorem{corollary}{Corollary}[section]{\bfseries}{}
\tcolorboxenvironment{corollary}{
	breakable,
	enhanced,
	boxrule = 0pt,
	frame hidden,
	coltitle = black,
	colback = white!0,
	left = 0.5em,
	sharp corners = all,
}

\newtheorem{lemma}{Lemma}[section]{\bfseries}{}
\tcolorboxenvironment{lemma}{
	breakable,
	enhanced,
	boxrule = 0pt,
	frame hidden,
	coltitle = black,
	colback = green!7,
	left = 0.5em,
	sharp corners = all,
}

\newtheorem{definition}{Definition}[section]{\bfseries}{}
\tcolorboxenvironment{definition}{
	breakable,
	coltitle = black,
	colback = white,
	frame hidden,
	boxsep = 0pt,
	boxrule = 0pt,
	borderline west = {3pt}{0pt}{orange},
	sharp corners = all,
	enhanced,
}

\newtheorem{example}{Example}[section]{\bfseries}{}
\tcolorboxenvironment{example}{
	% title = \textbf{Example},
	% detach title,
	% before upper = {\tcbtitle\quad},
	breakable,
	coltitle = black,
	colback = white,
	frame hidden,
	boxrule = 0pt,
	boxsep = 0pt,
	borderline west={3pt}{0pt}{green!70!black},
	sharp corners = all,
	enhanced,
}

\newtheoremstyle{remark}{0pt}{4pt}{}{}{\bfseries\itshape}{\normalfont.}{0.5em}{}
\theoremstyle{remark}
\newtheorem*{remark}{Remark}


% TColorBoxes
\newtcolorbox{week}{
	colback = black,
	coltext = white,
	fontupper = {\large\bfseries},
	width = 1.2\paperwidth,
	size = fbox,
	halign upper = center,
	center
}

\newcommand{\banner}[2]{
    \pagebreak
    \begin{week}
   		\section*{#1}
    \end{week}
    \addcontentsline{toc}{section}{#1}
    \addtocounter{section}{1}
    \setcounter{subsection}{0}
}

% Hyperref
\usepackage{hyperref}
\hypersetup{
	colorlinks=true,
	linktoc=all,
	linkcolor=links,
	bookmarksopen=true
}


\def\homeworknumber{3}
\usepackage{fancyhdr}
\pagestyle{fancy}
\fancyhead[R]{HW \#\thehwnumber}
\fancyhead[C]{\textbf{Math 130B}}
\fancyhead[L]{Eli Griffiths}


\begin{document}

% Section 33: 2, 4, 11
% Section 34: 1, 2, 5
% Section 36: 2, 5, 9
% Section 38: 11, 14
% Section 42: 2, 3
% Section 43: 5

\subsection*{33.2}
\subsubsection*{Part A}
Since $e = e \cdot e^{2 \pi i n}$ for $n \in \Z$, it follows that
\[
    \log e = \ln e + 2 \pi i n = 1 + 2 \pi i n, n \in \Z
.\]

\subsubsection*{Part B}
Since $i = e^{i \frac{\pi}{2} + 2 \pi n}$ for $n \in \Z$, it follows that
\[
    \log i = \ln 1 + i \qty(\frac{\pi}{2} + 2 \pi n) = \qty(2n + \frac{1}{2}) \pi i, n \in \Z
.\]

\subsubsection*{Part C}
Since $-1 + i\sqrt{3} = 2 \qty(-\frac{1}{2} + i \frac{\sqrt{3}}{2}) = 2 e^{i \qty(\frac{2 \pi}{3} + 2 \pi n)}$ for $n \in \Z$, it follows that
\[
    \log (-1 + i \sqrt{3}) = \ln 2 + i \qty(\frac{2 \pi}{3} + 2 \pi n) = \ln 2 + 2 \pi i \qty(n + \frac{1}{3}), n \in \Z
.\]

\subsection*{33.4}
\begin{proof}
    Since $i^2 = e^{i \pi}$ and $\pi$ is its argument in the branch, $\log(i^2) = \ln 1 + i \pi = i \pi$. Furthermore, $i = e^{i \frac{\pi}{2}}$ which has argument $\frac{5 \pi}{2}$ in the branch meaning $2 \log i = 2(\ln 1 + i \frac{5 \pi}{2}) = 5 \pi i \neq i \pi$
\end{proof}

\subsection*{33.11}
\begin{proof}
    The second partials of $F(x,y) = \ln (x^2 + y^2)$ are
    \begin{align*}
        \pdv[2]{x} \ln(x^2 + y^2) &= \pdv{x} \frac{2x}{x^2 + y^2} = \frac{(x^2 + y^2) \cdot 2 - 2x (2x)}{(x^2 + y^2)^2} = 2\cdot \frac{y^2 - x^2}{(x^2 + y^2)^2} \\
        \pdv[2]{y} \ln(x^2 + y^2) &= \pdv{y} \frac{2y}{x^2 + y^2} = \frac{(x^2 + y^2)\cdot 2 - 2y (2y)}{(x^2 + y^2)^2} = 2 \cdot \frac{x^2 - y^2}{(x^2 + y^2)^2}
    \end{align*}
    Therefore
    \[
        F_{xx} + F_{yy} = 2 \qty(\frac{y^2 - x^2 + x^2 - y^2}{(x^2 + y^2)^2}) = 0
    .\]
    Hence $\ln(x^2 + y^2)$ is harmonic everywhere excluding the origin.
\end{proof}

\begin{proof}
    Note that $\ln(x^2 + y^2)$ is the real component of $2\cdot\log z$ for $z = x + iy$ on any branch and is therefore harmonic on the branches domain. Since the branches
    \[
        \alpha = \qty{-\pi, \frac{\pi}{2}}
    \]
    together cover all $z \neq 0$ and $\ln(x^2 + y^2)$ is then harmonic on both, it follows that $\ln(x^2 + y^2)$ is harmonic everywhere except the origin.
\end{proof}

\subsection*{34.1}
\begin{proof}
    Let $z_1, z_2 \in \C$ with $z_1 = r_1 e^{i \theta_1}$ and $z_2 = r_2 e^{i \theta_2}$ and $- \pi < \theta_1, \theta_2 < \pi$. Note that $- 2 \pi < \theta_1 + \theta_2 < 2 \pi$. Let
    \[
        N = \begin{cases}
            -1 & \theta_1 + \theta_2 > \pi \\
            \;\;1 & \theta_1 + \theta_2 \leq -\pi \\
            \;\;0 & \text{otherwise}
        \end{cases}
    .\]
    and therefore $\Arg{z_1 z_2} = \theta_1 + \theta_2 + 2 \pi N$. It follows that
    \begin{align*}
        \Log(z_1 z_2) = \ln (r_1 r_2) + i\Arg{z_1 z_2} &= \ln r_1 + \ln r_2 + i (\theta_1 + \theta_2 + 2 \pi N) \\
        &= \ln r_1 + i \theta_1 + \ln r_2 + i \theta_2 + 2 \pi N i \\
        &= \Log z_1 + \Log z_2 + 2 \pi N i \qedhere
    \end{align*}
\end{proof}

\subsection*{34.2}
\begin{proof}
    Let $z_1, z_2 \in \C$ with $z_2 \neq 0$. First note that for all $z \neq 0$
    \begin{align*}
        \log \frac{1}{z} &= \log(\frac{1}{|z|} e^{-i \arg z}) \\
        &= \ln \frac{1}{|z|} - i \arg z \\
        &= - \ln |z| - i \arg z \\
        &= - (\ln|z| + i \arg z) = - \log z
    \end{align*}
    Therefore
    \begin{align*}
        \log \frac{z_1}{z_2} &= \log(z_1 \cdot \frac{1}{z_2}) \\
        &= \log(z_1) + \log(\frac{1}{z_2}) \\
        &= \log(z_1) - \log(z_2) \qedhere
    \end{align*}
\end{proof}

\subsection*{34.5}
\begin{proof}
    Let $z = r e^{i \theta}$ where $r > 0$ and $\theta \in (-\pi, \pi]$. Then
    \[
        z^{\frac{1}{n}} = \qty{r^{\frac{1}{n}} e^{i \frac{\theta + 2 k \pi}{n}} : 0 \leq k \leq n-1}
    .\]
    Therefore
    \begin{align*}
        \log z^{\frac{1}{n}} &= \qty{
            \ln(r^{\frac{1}{n}}) + i \qty(\frac{\theta + 2 \pi k}{n} + 2 \pi p) : 0 \leq k \leq n -1, p \in \Z
        } \\
        &= \qty{
        \frac{1}{n} \ln r + i \qty(\frac{\theta + 2 \pi (np + k)}{n}) : 0 \leq k \leq n -1, p \in \Z
        } \\
        &= \frac{1}{n} \qty{
        \ln r + i \qty(\theta + 2 \pi (np + k)) : 0 \leq k \leq n -1, p \in \Z
        }
        \intertext{Note that any integer $q \in \Z$ can be written as $p \equiv k \;\text{(mod $n$)}$ and therefore $q = np + k$ meaning}
        &= \frac{1}{n} \qty{
            \ln r + i \qty(\theta + 2 \pi q) : q \in \Z
        } \\
        &= \frac{1}{n} \log z \qedhere
    \end{align*}
\end{proof}

\subsection*{36.2}
\subsubsection*{Part A}

\[
    (-i)^i = \exp(i \Log(-i)) = \exp(i \Log(e^{- i \frac{\pi}{2}})) = \exp(i \qty(\ln 1 - i \frac{\pi}{2})) = e^{\frac{\pi}{2}}
.\]

\subsubsection*{Part B}
\begin{align*}
    \qty(\frac{e}{2} (-1 - \sqrt{3} i))^{3 \pi i} &= \exp[3 \pi i \cdot \Log\qty(\frac{e}{2} (-1 - i\sqrt{3}))] \\
    &= \exp[
        3 \pi i (\Log\qty(\frac{e}{2}) + \Log(-1 - i\sqrt{3}))
    ] \\
    &= \exp[
    3 \pi i \qty(1 - \ln 2 + \Log\qty(2 \qty(-\frac{1}{2} - i \frac{\sqrt{3}}{2})))
    ] \\
    &= \exp[
    3 \pi i \qty(1 - \ln 2 + \ln 2 + i\qty(- \frac{2 \pi}{3}))
    ] \\
    &= \exp[
    3 \pi i \qty(1 - i\frac{2 \pi}{3})
    ] \\
    &= \exp[
    2 \pi^2 + 3 \pi i
    ] \\
    &= \exp[2 \pi^2] \cdot \exp[3 \pi i] = -\exp(2 \pi^2)
\end{align*}

\subsubsection*{Part C}
\begin{align*}
    (1-i)^{4i} &= \exp[4i \Log(1 - i)] \\
               &= \exp[4i \Log\qty(\sqrt{2} \qty(\frac{1}{\sqrt{2}}) - \frac{1}{\sqrt{2}} i)]\\
               &= \exp[4i \qty(\frac{1}{2} \ln 2 - i\qty(\frac{\pi}{4}))] \\
               &= \exp[\pi + 2 i \ln 2] \\
               &= \exp[\pi] \exp[2 i \ln 2] \\
               &= \exp[\pi] \qty(\cos(2 \ln 2) + i \sin(2 \ln 2))
\end{align*}

\subsection*{36.5}
\begin{proof}
    Let $z_0 = r e^{i \theta} \in \C$ with $\theta \in (-\pi, \pi]$. Then the principal root of $z_0^{\frac{1}{n}}$ from Section 10 is
    \[
        c_0 = r^{\frac{1}{n}} e^{i\frac{\theta}{n}}
    .\]
    Using the complex power function,
    \begin{align*}
        z_0^{\frac{1}{n}} = \exp[\frac{1}{n} \Log z_0] &= \exp[\frac{1}{n} (\ln r + i \theta)] \\
        &= \exp[\ln r^{\frac{1}{n}} + i \frac{\theta}{n}] \\
        &= \exp[\ln r^{\frac{1}{n}}] \cdot \exp[i \frac{\theta}{n}] = r^{\frac{1}{n}} e^{i \frac{\theta}{n}} = c_0 \qedhere
    \end{align*}
\end{proof}

\subsection*{36.9}
\[
    \dv{z} c^{f(z)} = \dv{z} \exp[f(z) \log c] = \exp[f(z) \log c] f'(z) \log c = \boxed{c^{f(z)} \cdot f'(z) \log c}
.\]

\subsection*{38.11}
\begin{proof}
    Let $z = x + iy$. Since
    \begin{align*}
        \sin \conj{z} &= \frac{e^{i \conj{z}} - e^{- i \conj{z}}}{2i} \\
                      &= \frac{e^{i(x - iy)} - e^{-i(x - iy)}}{2i} \\
                      &= \frac{e^{y + ix} - e^{-y - ix}}{2i} \\
                      &= \frac{e^{y} e^{ix} - e^{-y} e^{-ix}}{2i} \\
                      &= \frac{e^{y} (\cos x + i \sin x) - e^{-y} (\cos(-x) + i \sin(-x))}{2i} \\
                      &= \frac{e^{y} (\cos x + i \sin x) - e^{-y} (\cos x - i \sin x)}{2i} \\
                      &= \frac{(e^y - e^{-y})\cos x  + i (e^y + e^{-y}) \sin x }{2i} \\
                      &= \frac{1}{2} (e^y + e^{-y}) \sin x - \frac{i}{2} (e^y - e^{-y}) \cos x
    \end{align*}
    Therefore the partials are
    \begin{alignat*}{3}
        u_x &= \frac{1}{2} (e^y + e^{-y}) \cos x \hspace{1cm} && u_y &&= \frac{1}{2} (e^y - e^{-y}) \sin x \\
        v_x &= \frac{1}{2} (e^y - e^{-y}) \sin x && v_y &&= -\frac{1}{2} (e^y + e^{-y}) \cos x
    \end{alignat*}
    Applying the C.R. equations gives
    \[
        u_x = v_y \implies (e^y + e^{-y}) \cos x = 0 \implies \cos x = 0
    \]
    and
    \[
        u_y = -v_x \implies (e^y - e^{-y}) \sin x = 0 \implies y = 0 \text{ or } \sin x = 0
    \]
    Since there is no $x$ such that $\cos x = \sin x = 0$, the only places where C.R. holds is when $y = 0$ and $\cos x = 0$. However, there are only countably distinct points that satsify this and therefore no neighborhood around them can be differentiable. Hence the function is nowhere analytic.
\end{proof}
\begin{proof}
    Let $z = x + iy$. Since
    \begin{align*}
        \cos \conj{z} &= \frac{e^{i \conj{z}} + e^{- i \conj{z}}}{2} \\
                      &= \frac{e^{i(x - iy)} + e^{-i(x - iy)}}{2} \\
                      &= \frac{e^{y + ix} + e^{-y - ix}}{2} \\
                      &= \frac{e^{y} e^{ix} + e^{-y} e^{-ix}}{2} \\
                      &= \frac{e^{y} (\cos x + i \sin x) + e^{-y} (\cos(-x) + i \sin(-x))}{2} \\
                      &= \frac{e^{y} (\cos x + i \sin x) + e^{-y} (\cos x - i \sin x)}{2} \\
                      &= \frac{1}{2} (e^y + e^{-y}) \cos x + \frac{i}{2} (e^y - e^{-y}) \sin x
    \end{align*}
    Therefore the partials are
    \begin{alignat*}{3}
        u_x &= -\frac{1}{2} (e^y + e^{-y}) \sin x \hspace{1cm} && u_y &&= \frac{1}{2} (e^y - e^{-y}) \cos x \\
        v_x &= \frac{1}{2} (e^{y} + e^{-y}) \cos x && v_y &&= \frac{1}{2} (e^y + e^{-y}) \sin x
    \end{alignat*}
    Applying the C.R. equations gives
    \[
        u_x = v_y \implies (e^y + e^{-y}) \cos x = 0 \implies \cos x = 0
    \]
    and
    \[
        u_y = -v_x \implies (e^y - e^{-y}) \sin x = 0 \implies y = 0 \text{ or } \sin x = 0
    \]
    Since there is no $x$ such that $\cos x = \sin x = 0$, the only places where C.R. holds is when $y = 0$ and $\cos x = 0$. However, there are only countably distinct points that satsify this and therefore no neighborhood around them can be differentiable. Hence the function is nowhere analytic.
\end{proof}

\subsection*{38.14}
\subsubsection*{Part A}
\begin{proof}
    Note that
    \begin{align*}
        \conj{\cos(iz)} &= \conj{\frac{e^{i(iz)} + e^{-i(iz)}}{2}} \\
                        &= \conj{\frac{e^{-z} + e^z}{2}} \\
                        &= \frac{e^{\conj{-z}} + e^{\conj{z}}}{2} \\
                        &= \frac{e^{-\conj{z}} + e^{\conj{z}}}{2}
    \end{align*}
    and
    \begin{align*}
        \cos(i \conj{z}) &= \frac{e^{i (i \conj{z})} + e^{-i (i \conj{z})}}{2} \\
                         &= \frac{e^{-\conj{z}} + e^{\conj{z}}}{2}
    \end{align*}
    Therefore $\conj{\cos(iz)} = \cos(i \conj{z})$.
\end{proof}

\subsubsection*{Part B}
\begin{proof}
    Note that
    \begin{align*}
        \conj{\sin(iz)} &= \conj{\qty(\frac{e^{i(iz)} - e^{-i(iz)}}{2i})} \\
                        &= \conj{\qty(\frac{e^{-z} - e^z}{2i})} \\
                        &= -\frac{e^{\conj{-z}} - e^{\conj{z}}}{2i} \\
                        &= \frac{e^{\conj{z}} - e^{-\conj{z}}}{2i}
    \end{align*}
    and
    \begin{align*}
        \sin(i \conj{z}) &= \frac{e^{i (i \conj{z})} - e^{-i (i \conj{z})}}{2i} \\
                         &= \frac{e^{-\conj{z}} - e^{\conj{z}}}{2i}
    \end{align*}
    Therefore $\conj{\sin(iz)} = \sin(i \conj{z})$ when
    \begin{align*}
        \frac{e^{\conj{z}} - e^{-\conj{z}}}{2i} &= \frac{e^{-\conj{z}} - e^{\conj{z}}}{2i} \\
        2e^{\conj{z}} &= 2e^{-\conj{z}} \\
        2e^{\conj{z}} &= 2e^{-\conj{z}} \\
        e^{\conj{z}} &= e^{-\conj{z}} \\
        e^{z} &= e^{-z}
    \end{align*}
    Let $z = x + iy$. Then
    \begin{align*}
        e^z &= e^{-z} \\
        e^x (\cos y + i \sin y) &= e^{-x} (\cos y - i \sin y)
    \end{align*}
    which is true when $e^x \cos y = e^{-x} \cos y$ and $e^x \sin y = -e^{-x} \sin y$ meaning $x = 0$ and
    \begin{align*}
        \sin y = - \sin y \implies \sin y = 0 \implies y = n \pi i, n \in \Z
    \end{align*}
    Hence $z$ must be $n \pi i$ for $n \in \Z$.
\end{proof}

\subsection*{42.2}
\subsubsection*{Part A}
\begin{align*}
    \int_0^1 (1 + it)^2 \dd t &= \int_0^1 (1 - t^2 + 2it) \dd t \\
    &= \int_0^1 (1-t^2) \dd t + i \int_0^1 2t \dd t \\
    &= \qty[t - \frac{t^3}{3}]_0^1 + i \qty[t^2]_0^1 \\
    &= \frac{2}{3} + i
\end{align*}

\subsubsection*{Part B}
\begin{align*}
    \int_1^2 \qty(\frac{1}{t} - i)^2 \dd t &= \int_1^2 \qty(\frac{1}{t^2} - \frac{2i}{t} - 1) \dd t \\
    &= \int_1^2 \qty(\frac{1}{t^2} - 1) \dd t - 2i \int_1^2 \frac{1}{t} \dd t \\
    &= \qty[-\frac{1}{t} - t]_1^2 - 2i \qty[\ln t]_1^2 \\
    &= -\frac{1}{2} - 2 i \ln 2 = -\frac{1}{2} - i \ln 4
\end{align*}

\subsubsection*{Part C}
\[
    \int_0^{\frac{\pi}{6}} e^{i 2t} \dd t = \frac{1}{2i} e^{i 2 t} \eval_0^{\frac{\pi}{6}} = \frac{1}{2i} e^{i \frac{\pi}{3}} - \frac{1}{2i} e^{0} = \frac{1}{2i} \qty(\frac{1}{2} + i\frac{\sqrt{3}}{2} - 1) = - \frac{i}{2} \qty(- \frac{1}{2} + i \frac{\sqrt{3}}{2}) = \frac{\sqrt{3}}{4} + \frac{i}{4}
.\]

\subsubsection*{Part D}
Since $\Re z > 0$, $z \neq 0$. Then
\[
    \int_0^\infty e^{-z t} = -\frac{1}{z} e^{-z t} \eval_0^\infty = \frac{1}{z}\qty(1 - \lim_{t \to \infty} e^{-zt}) = \frac{1}{z}\qty(1 - 0) = \frac{1}{z}
.\]

\subsection*{42.3}
\begin{proof}
    Let $m,n \in \Z$. Note that $e^{i m \theta} e^{- i n \theta} = e^{i \theta (m - n)}$ Consider two cases
    \begin{enumerate}[leftmargin=2cm]
        \item[$(m = n)$]
            Since $m = n$, $e^{i \theta (m - n)} = e^{0} = 1$. Therefore
            \[
                \int_0^{2 \pi} e^{i m \theta} e^{- i n \theta} \dd \theta = \int_0^{2 \pi} 1 \dd \theta = 2 \pi
            .\]
        \item[$(m \neq n)$]
            Since $m \neq n$, $e^{i \theta (m-n)}$ is non constant and $\frac{1}{m-n}$ is defined. Therefore
            \begin{align*}
                \int_0^{2 \pi} e^{i m \theta} e^{- i n \theta} \dd \theta
                &= \int_0^{2 \pi} e^{i \theta (m-n)} \dd \theta \\ 
                &= -\frac{i}{\theta(m - n)} e^{i \theta (m-n)} \eval_0^{2 \pi} \\
                &= -\frac{i}{\theta(m - n)} \qty(e^{2 \pi i (m-n)} - e^{0})
                \intertext{Since $m - n \in \Z \setminus \qty{0}$, $e^{2 \pi i (m-n)} = e^{2 \pi i}$ which is $1$ it follows that}
                &= -\frac{i}{\theta(m - n)} \qty(1 - 1) = 0
            \end{align*}
    \end{enumerate}
    Hence
    \[
        \int_0^{2 \pi} e^{i m \theta} e^{- i n \theta} \dd \theta = \begin{cases}
            0 & m \neq n \\
            2 \pi & m = n \\
        \end{cases} \qedhere
    \]
\end{proof}

\subsection*{43.5}
\begin{proof}
    Let $f(z) = u(x,y) + i v(x,y)$ and $z(t) = x(t) + i y(t)$ for $a \leq t \leq b$. Then
    \[
        w(t) = f(z(t)) = u(x(t), y(t)) + i v(x(t), y(t))
    .\]
    Therefore by the multivariable chain rule,
    \[
        w'(t) = (u_x x' + u_y y') + i (v_x x' + v_y y')
    .\]
    Since $w(t_0) = f(z(t_0))$ is analytic, then the C.R. equations hold at $t_0$ meaning at $t_0$
    \begin{align*}
        w'(t) &= u_x x' + u_y y' + i (v_x x' + v_y y') \\
        &= u_x x' - v_x y' + i (v_x x' + u_x y') \\
        &= (u_x + iv_x)(x'(t) + i y'(t)) \\
        &= f'(z(t)) z'(t) \qedhere
    \end{align*}
\end{proof}

\end{document}
