\documentclass[12pt,titlepage]{extarticle}
% Document Layout and Font
\usepackage{subfiles}
\usepackage[margin=2cm, headheight=15pt]{geometry}
\usepackage{fancyhdr}
\usepackage{enumitem}	
\usepackage{wrapfig}
\usepackage{multicol}
\usepackage{caption, subcaption}

\usepackage[p,osf]{scholax}

\renewcommand*\contentsname{Table of Contents}
\renewcommand{\headrulewidth}{0pt}
\pagestyle{fancy}
\fancyhf{}
\fancyfoot[R]{$\thepage$}
\setlength{\parindent}{0cm}
\setlength{\headheight}{17pt}
\hfuzz=9pt

% Utility Management
\usepackage{color}
\usepackage{colortbl}
\usepackage{xcolor}
\usepackage{xpatch}
\usepackage{xparse}

\definecolor{links}{HTML}{1c73a5}
\definecolor{bar}{HTML}{584AA8}

% Math Packages
\usepackage{mathtools, amsmath, amsthm, thmtools, amssymb, physics}
\usepackage[scaled=1.075,ncf,vvarbb]{newtxmath}

\newcommand\B{\mathbb{B}}
\newcommand\C{\mathbb{C}}
\newcommand\R{\mathbb{R}}
\newcommand\Q{\mathbb{Q}}
\newcommand\N{\mathbb{N}}
\newcommand\Z{\mathbb{Z}}

\newcommand\Prob[1]{\mathbb{P}\qty(#1)}
\newcommand\Var[1]{\text{Var}\qty(#1)}
\newcommand\Exp[1]{\mathbb{E}\qty[#1]}
\newcommand\ball[1]{\B\qty(#1)}
\newcommand\res[1]{\underset{#1}{\operatorname{Res}}\;}
\renewcommand\pv{\mathrm{p.v.}}

\newcommand\conj[1]{\overline{#1}}
\DeclareMathOperator{\Arg}{Arg}
\DeclareMathOperator{\Log}{Log}
\DeclareMathOperator{\cis}{cis}

\DeclareMathOperator{\dom}{dom}
\DeclareMathOperator{\spann}{span}
\DeclareMathOperator{\nullity}{nullity}

\newcommand\st{\text{ s.t. }}

% TIKZ
\usepackage{tikz}
\usepackage{pgfplots}
\usetikzlibrary{arrows.meta}
\usetikzlibrary{math}
\usetikzlibrary{cd}
\usetikzlibrary{patterns}
\usetikzlibrary{decorations.markings}
\usetikzlibrary{calc}

% Boxes and Theorems
\usepackage[most]{tcolorbox}
\tcbuselibrary{skins}
\tcbuselibrary{breakable}
\tcbuselibrary{theorems}

\newtheoremstyle{default}{0pt}{0pt}{}{}{\bfseries}{\normalfont.}{0.5em}{}
\theoremstyle{default}

\renewcommand*{\proofname}{\textit{\textbf{Proof.}}}
\renewcommand*{\qedsymbol}{$\blacksquare$}
\tcolorboxenvironment{proof}{
	breakable,
	coltitle = black,
	colback = white,
	frame hidden,
	boxrule = 0pt,
	boxsep = 0pt,
	borderline west={3pt}{0pt}{bar},
	sharp corners = all,
	enhanced,
}

\newtheorem{theorem}{Theorem}[section]{\bfseries}{}
\tcolorboxenvironment{theorem}{
	breakable,
	enhanced,
	boxrule = 0pt,
	frame hidden,
	coltitle = black,
	colback = blue!7,
	left = 0.5em,
	sharp corners = all,
}

\newtheorem{corollary}{Corollary}[section]{\bfseries}{}
\tcolorboxenvironment{corollary}{
	breakable,
	enhanced,
	boxrule = 0pt,
	frame hidden,
	coltitle = black,
	colback = white!0,
	left = 0.5em,
	sharp corners = all,
}

\newtheorem{lemma}{Lemma}[section]{\bfseries}{}
\tcolorboxenvironment{lemma}{
	breakable,
	enhanced,
	boxrule = 0pt,
	frame hidden,
	coltitle = black,
	colback = green!7,
	left = 0.5em,
	sharp corners = all,
}

\newtheorem{definition}{Definition}[section]{\bfseries}{}
\tcolorboxenvironment{definition}{
	breakable,
	coltitle = black,
	colback = white,
	frame hidden,
	boxsep = 0pt,
	boxrule = 0pt,
	borderline west = {3pt}{0pt}{orange},
	sharp corners = all,
	enhanced,
}

\newtheorem{example}{Example}[section]{\bfseries}{}
\tcolorboxenvironment{example}{
	% title = \textbf{Example},
	% detach title,
	% before upper = {\tcbtitle\quad},
	breakable,
	coltitle = black,
	colback = white,
	frame hidden,
	boxrule = 0pt,
	boxsep = 0pt,
	borderline west={3pt}{0pt}{green!70!black},
	sharp corners = all,
	enhanced,
}

\newtheoremstyle{remark}{0pt}{4pt}{}{}{\bfseries\itshape}{\normalfont.}{0.5em}{}
\theoremstyle{remark}
\newtheorem*{remark}{Remark}


% TColorBoxes
\newtcolorbox{week}{
	colback = black,
	coltext = white,
	fontupper = {\large\bfseries},
	width = 1.2\paperwidth,
	size = fbox,
	halign upper = center,
	center
}

\newcommand{\banner}[2]{
    \pagebreak
    \begin{week}
   		\section*{#1}
    \end{week}
    \addcontentsline{toc}{section}{#1}
    \addtocounter{section}{1}
    \setcounter{subsection}{0}
}

% Hyperref
\usepackage{hyperref}
\hypersetup{
	colorlinks=true,
	linktoc=all,
	linkcolor=links,
	bookmarksopen=true
}


\def\homeworknumber{1}
\usepackage{fancyhdr}
\pagestyle{fancy}
\fancyhead[R]{HW \#\thehwnumber}
\fancyhead[C]{\textbf{Math 130B}}
\fancyhead[L]{Eli Griffiths}


\begin{document}

% Section 14: 8
% Section 18: 1(c), 7
% Section 20: 8
% Section 24: 1, 3, 8
% Section 26: 5, 7
% Section 27: 2
% Section 30: 1, 3, 10

\subsection*{14.8}
For $z = re^{i \theta}$, it follows that $z^n = r^n e^{ni \theta}$. Hence for each mapping,
\begin{alignat*}{2}
    z^2 &\implies 0 \leq \theta \leq \frac{\pi}{2},&& 0 \leq r \leq 1 \\
    z^3 &\implies 0 \leq \theta \leq \frac{3 \pi}{4}, && 0 \leq r \leq 1 \\
    z^4 &\implies 0 \leq \theta \leq \pi,\hspace{0.5cm} && 0 \leq r \leq 1
\end{alignat*}

\begin{figure}[h!]
    \centering
    \newcommand\sketch[2]{
    \begin{subfigure}[h!]{0.3 \textwidth}
        \begin{tikzpicture}[scale=1.5]
            \pgfmathsetmacro{\ang}{#1}
            \pgfmathsetmacro{\cx}{cos(\ang r)}
            \pgfmathsetmacro{\cy}{sin(\ang r)}
            \draw[pattern=north west lines, pattern color=gray] (0,0) -- (\cx, \cy) arc[start angle={deg(\ang)}, end angle = 0, radius = 1];
            \draw (-1.3, 0) -- (1.3, 0) node[anchor = west, scale = {1 / 1.5}] {$\Re z$};
            \draw (0, -1.3) -- (0, 1.3) node[anchor = south, scale = {1 / 1.5}] {$\Im z$};
        \end{tikzpicture}
        \caption{$w = z^{#2}$}
    \end{subfigure}
    }
    \sketch{(pi / 4)*2}{2}
    \sketch{(pi / 4)*3}{3}
    \sketch{(pi / 4)*4}{4}
\end{figure}

\subsection*{18.1}
\subsubsection*{Part C}
\begin{proof}
    Take $\epsilon > 0$ and let $\delta = \epsilon$. Note then that for $z \in C$ in the $\delta$ deleted neighborhood of $0$ (that is $z \neq 0$)
    \begin{align*}
        |z - 0| < \delta &\implies |z| < \delta \\
                         &\implies \frac{|z|^2}{|z|} < \delta \tag{Since $|z| \neq 0$} \\ 
                      &\implies \frac{|\conj{z}|^2}{|z|} < \delta \\
                      &\implies \qty|\frac{\conj{z}^2}{z}| < \delta \implies \qty|\frac{\conj{z}^2}{z} - 0| < \delta = \epsilon
    \end{align*}
    Therefore by definition $\displaystyle\lim_{z \to 0} \frac{\conj{z}^2}{z} = 0$.
\end{proof}

\subsection*{18.7}
\begin{proof}
    Assume that $\lim_{z\to z_0} f(z) = w_0$. Take $\epsilon > 0$. Then there is some $\delta$ such that $|z - z_0| < \delta \implies |f(z) - w_0| < \epsilon$. Since $||f(z)| - |w_0|| \leq |f(z) - w_0| < \epsilon$. Therefore
    \[
        |z - z_0| < \delta \implies ||f(z)| - |w_0|| < \epsilon
    .\]
    Therefore by definition $\lim_{z \to z_0} |f(z)| = |w_0|$.
\end{proof}

\subsection*{20.8}
\subsubsection*{Part A}
\begin{proof}
    Let $z_0 = x_0 + i y_0$. Let $\Delta z = z - z_0$ and $\Delta f = f(z + \Delta z) - f(z)$. Then
    \[
        \frac{\Delta f}{\Delta z} = \frac{\Re{z + \Delta z} - \Re{z}}{\Delta z} = \frac{\Re{z} + \Re{\Delta z} - \Re{z}}{\Delta z} = \frac{\Re{\Delta z}}{\Delta z}
    .\]
    For $f'(z_0)$ to exist, this quantity must be the same no matter how $\Delta z \to 0$. If $\Delta z$ approaches $0$ along the real axis, then $\Delta z = \Delta x + i 0 = \Delta x$ and $\Re{\Delta z} = \Delta x$. Therefore
    \[
        \frac{\Delta f}{\Delta z} = \frac{\Delta x}{\Delta x} = 1
    .\]
    If $\Delta z$ approachs $0$ along the imaginary axis, then $\Delta z = 0 + i \Delta y = i \Delta y$ and $\Re{\Delta z} = 0$. Therefore
    \[
        \frac{\Delta f}{\Delta z} = \frac{0}{i\Delta y} = 0
    .\]
    Therefore since the value is not the same on every path for $\Delta z \to 0$, the limit cannot exist at $z_0$ which was an arbitaray point in $\C$. Therefore $f$ is not differentiable anywhere on $\C$.
\end{proof}

\subsubsection*{Part B}
\begin{proof}
    Let $z_0 = x_0 + i y_0$. Let $\Delta z = z - z_0$ and $\Delta f = f(z + \Delta z) - f(z)$. Then
    \[
        \frac{\Delta f}{\Delta z} = \frac{\Im{z + \Delta z} - \Im{z}}{\Delta z} = \frac{\Im{z} + \Im{\Delta z} - \Im{z}}{\Delta z} = \frac{\Im{\Delta z}}{\Delta z}
    .\]
    For $f'(z_0)$ to exist, this quantity must be the same no matter how $\Delta z \to 0$. If $\Delta z$ approaches $0$ along the real axis, then $\Delta z = \Delta x + i 0 = \Delta x$ and $\Im{\Delta z} = 0$. Therefore
    \[
        \frac{\Delta f}{\Delta z} = \frac{0}{\Delta x} = 0
    .\]
    If $\Delta z$ approachs $0$ along the imaginary axis, then $\Delta z = 0 + i \Delta y = i \Delta y$ and $\Im{\Delta z} = \Delta y$. Therefore
    \[
        \frac{\Delta f}{\Delta z} = \frac{\Delta y}{i\Delta y} = -i
    .\]
    Therefore since the value is not the same on every path for $\Delta z \to 0$, the limit cannot exist at $z_0$ which was an arbitaray point in $\C$. Therefore $f$ is not differentiable anywhere on $\C$.
\end{proof}

\subsection*{24.1}
\subsubsection*{Part A}
\begin{proof}
    $f(z) = \conj{z}$ will be non-differentiable at all points where $f$ does not satisfy the Cauchy-Riemann equations. For $z = x+iy$, $f(z) = x - iy$. Therefore the partials for $f$ are
    \begin{alignat*}{2}
        u_x &= 1 \;\;\; u_y &&= 0 \\
        v_x &= 0 \;\;\; v_y &&= -1
    \end{alignat*}
    Applying the Cauchy Riemann equations, $1 = -1$ and $0 = 0$. Since $1 = -1$ is never true, it follows that the Cauchy Riemann equations dont hold for any $z \in \C$ and hence $f$ is not differentiable anywhere on $\C$.
\end{proof}

\subsubsection*{Part B}
\begin{proof}
    $f(z) = z - \conj{z}$ will be non-differentiable at all points where $f$ does not satisfy the Cauchy-Riemann equations. For $z = x + iy$, $f(z) = 2iy$. The partials for $f$ therefore are
    \begin{alignat*}{2}
        u_x &= 0 \;\;\; u_y &&= 0 \\
        v_x &= 0 \;\;\; v_y &&= 2i
    \end{alignat*}
    Applying the Cauchy Riemann equations, $0 = 0$ and $0 = 2i$. Since $0 = 2i$ is never true, it follows that the Cauchy Riemann equations dont hold for any $z \in \C$ and hence $f$ is not differentiable anywhere on $\C$.
\end{proof}

\subsubsection*{Part C}
\begin{proof}
    $f(x+iy) = 2x + ixy^2$ will be non-differentiable at all points where $f$ does not satisfy the Cauchy-Riemann equations. The partials for $f$ are
    \begin{alignat*}{3}
        u_x &= 2 \hspace{1cm} && u_y &&= 0 \\
        v_x &= iy^2 && v_y &&= 2ixy
    \end{alignat*}
    Applying the Cauchy Riemann equations, $2 = 2ixy$ and $0 = -iy^2$. Therefore
    \begin{align*}
        0 = -iy^2 &\implies y = 0 \\
        2 = 2ixy &\implies x = 0 \text{ or } y = 0
    \end{align*}
    Therefore the only candidate point for differentiability is $x = y = 0$.
\end{proof}

\subsubsection*{Part D}
\begin{proof}
    $f$ will be potentially differentiable only at points where the Cauchy Riemann equations hold. Rewriting $f$ in terms of its components gives
    \[
        f(z) = e^x e^{-iy} = e^x \qty[\cos(-y) + i \sin(-y)] = e^x \cos y - i e^x \sin y
    .\]
    Therefore the partials are
    \begin{alignat*}{3}
        u_x &= e^x \cos y &\quad& u_y &&= -e^x \sin y \\
        v_x &= -e^x \sin y && v_y &&= -e^x \cos y
    \end{alignat*}
    Applying the Cauchy Riemann equations gives
    \[
        u_x = v_y \implies e^x \cos y = -e^x \cos y \implies \cos y = -\cos y
    \]
    which is true when $y = \frac{\pi}{2} + \pi k$ for $k \in \Z$ and that
    \[
        u_y = -v_x \implies -e^x \sin y = e^x \sin y \implies \sin y = -\sin y
    \]
    which is true when $y = \pi m$ for $m \in \Z$. However, both equations cannot be satisifed simultaneously since there is no $y$ such that both $\sin y = \cos y = 0$. This means that $f$ is not differentiable at any $x + iy$ and therefore cannot be differentiable anywhere.
\end{proof}

\subsection*{24.3}
\subsubsection*{Part A}
\begin{proof}
    The candidate points where $f$ is differentiable are those that satisfy the Cauchy Riemann equations. Since $f(z) = \frac{1}{z} = \frac{\conj{z}}{|z|^2} = \frac{x}{x^2 + y^2} - i \frac{y}{x^2 + y^2}$ for $z \neq 0$, the partials are
    \begin{alignat*}{3}
        u_x &= \frac{y^2 - x^2}{(x^2 + y^2)^2} &\quad& u_y &&= -\frac{2xy}{(x^2 + y^2)^2} \\
        v_x &= \frac{2xy}{(x^2 + y^2)^2} && v_y &&= \frac{y^2 - x^2}{(x^2 + y^2)^2}
    \end{alignat*}
    Applying the Cauchy Riemann equations gives
    \[
        u_x = v_y \implies \frac{y^2 - x^2}{(x^2 + y^2)^2} = \frac{y^2 - x^2}{(x^2 + y^2)^2} \implies 0 = 0
    \]
    and
    \[
        u_y = -v_x \implies -\frac{2xy}{(x^2 + y^2)^2} = -\frac{2xy}{(x^2 + y^2)^2} \implies 0 = 0
    .\]
    Therefore the Cauchy Riemann equations hold for every $z \neq 0$. Since the partials are also conintuous for $x +iy \neq 0$, then $f'$ exists at all $z \neq 0$. Therefore
    \begin{align*}
        f'(x+iy) = u_x + iv_x &= \frac{y^2 - x^2}{(x^2 + y^2)^2} + i \frac{2xy}{(x^2 + y^2)^2} \\
        &= \frac{y^2 - x^2 + i 2xy}{|z|^4} \\
        &= \frac{(x -iy)^2}{|z|^4}  \\
        &= \frac{\conj{z}^2}{|z|^4} \\
        &= \qty(\frac{\conj{z}}{|z|^2})^2 \\
        &= \qty(z^{-1})^2 = \frac{1}{z^2}
    \end{align*}
\end{proof}

\subsubsection*{Part B}
\begin{proof}
    The candidate points where $f$ is differentiable are those that satisfy the Cauchy Riemann equations. The partials are
    \begin{alignat*}{3}
        u_x &= 2x &\quad& u_y &&= 0 \\
        v_x &= 0 && v_y &&= 2y
    \end{alignat*}
    Therefore applying the Cauchy Riemann equations gives $2x = 2y$ and $0 = 0$. Therefore $f$ is not differentiable anywhere off the line $y = x$. Note that for any $\epsilon$-neighborhood of some $z_0$ on this line, the partials exist and are also continuous at $z_0$ since they exist and are continuous everywhere. Therefore $f$ is differentiable on this line and
    \[
        f'(z) = u_x + i v_x = 2x + i 0 = 2x
    .\]
\end{proof}

\subsubsection*{Part C}
\begin{proof}
    The candidate points where $f$ is differentiable are those that satisfy the Cauchy Riemann equations. Note that $f(x+iy) = (x + iy) y = xy + i y^2$. Therefore the partials are
    \begin{alignat*}{3}
        u_x &= y &\quad& u_y &&= x \\
        v_x &= 0 && v_y &&= 2y
    \end{alignat*}
    Therefore applying the Cauchy Riemann equations gives $y = 2y$ and $x = 0$. Therefore $x = y = 0$ is the only possible point that $f$ is differentiable. Since the partials exist and are continuous everywhere, then for any $\epsilon$-neighborhood around $z = 0$ the partials are continuous. Therefore $f'(0) = u_x(0, 0) + iv_x(0,0) = 0$.
\end{proof}

\subsection*{24.8}
\subsubsection*{Part A}
\begin{proof}
    Since $z = x + iy$ and $\conj{z} = x - iy$, then $x = \frac{z + \conj{z}}{2}$ and $y = \frac{z - \conj{z}}{2i}$. $F(x,y)$ can be changed to a function of a single imaginary input $\conj{z}$. Therefore using the multivariable chain rule gives
    \[
        \pdv{F}{\conj{z}} = \pdv{F}{x} \pdv{x}{\conj{z}} + \pdv{F}{y} \pdv{y}{\conj{z}} = \pdv{F}{x} \qty(\frac{1}{2}) + \pdv{F}{y}\qty(-\frac{1}{2i}) = \frac{1}{2} \qty(\pdv{F}{x} + i \pdv{F}{y})
    .\]
\end{proof}

\subsubsection*{Part B}
\begin{proof}
    Let $f(z) = u(x,y) + iv(x,y)$ satisfy the Cauchy Riemann equations. That is $u_x = v_y$ and $u_y = -v_x$. Applying the operator $\pdv{\conj{z}}$ gives
    \begin{align*}
        \pdv{f}{\conj{z}} = \pdv{\conj{z}} f(z) &= \frac{1}{2} \qty(\pdv{x} + i\pdv{y}) (u(x,y) + i v(x,y)) \\
                            &= \frac{1}{2} \qty[
                            u_x - v_y + i u_y + i v_x
                            ] \\
                            &= \frac{1}{2} \qty[
                            u_x - v_y + i (u_y + v_x)
                            ] \\
        \intertext{By substituting $u_x$ for $v_y$ and $u_y$ for $-v_x$,}
                            &= \frac{1}{2} \qty[
                            v_y - v_y + i (-v_x + v_x)
                            ] \\
                            &= 0
    \end{align*}
    Therefore when $f$ satisfies the Cauchy Riemann equations, $\pdv{f}{\conj{z}} = 0$.
\end{proof}

\subsection*{26.5} % TODO
\begin{proof}
    Note that when $|z| > 0$ and $|\Arg{z}| < \frac{\pi}{2}$, $\Re{z} > 0$. Since $g(z)$ is analytic when $|z| > 0$ and $|\Arg{z}| < \frac{\pi}{2}$, $g(z)$ is analytic when $\Re{z} > 1$. Since $f(z) = 2z - 2 + i$ is analytic everywhere in $\C$, $g(f(z))$ will be analytic when $\Re{f(z)} > 0$. That is,
    \[
        \Re f(z) = \Re{2z - 2 + i} = 2x - 2 > 0 \implies x > 1
    .\]
    Therefore $g(f(z))$ will be analytic in the half plane $x > 1$. It follows by the chain rule that
    \[
        G'(x) = \dv{z} g(f(z)) = g'(f(z)) \cdot f'(z) = 2 \cdot \frac{1}{2\cdot g(2z - 2 + i)} = \frac{1}{g(2z - 2 + i)}
    .\]
\end{proof}

\subsection*{26.7}
\begin{proof}
    Since $f(z)$ is real valued, then if $f(z) = u + iv$, it follows $v(x,y) = 0$. Since $f$ is also analytic everywhere in $\mathcal{D}$, it satisfies the Cauchy Riemann equations in its entire domain. Therefore
    \[
        u_x = v_y \implies u_x = 0
    .\]
    Therefore $f'(z) = u_x + i v_x = 0 + i 0 = 0$ on all of $\mathcal{D}$. Therefore $f(z)$ is constant throughout $\mathcal{D}$.
\end{proof}

\subsection*{27.2}
\begin{proof}
    Assume that $z_0 = (x_0, y_0) \in \mathcal{D}$ where $u(x_0, y_0) = v(x_0, y_0)$, that $f'(z_0) \neq 0$, and that $f$ is analytic in $\mathcal{D}$. Since $u(x,y)$ and $v(x,y)$ can be considered as real multivariate functions, the techniques of multivariable calculus can be applied to their level curves $u(x,y) = c_1$ and $v(x,y) = c_2$. The tangent lines on both level curves are perpendicular if their normal vectors are also perpendicular. From multivariable calculus, the gradient of a function is always perpendicular to it's level curves and therefore the tangents are perpendicular when the gradients
    \begin{align*}
        \nabla u &= \langle u_x, u_y \rangle \\
        \nabla v &= \langle v_x, v_y \rangle
    \end{align*}
    are perpendicular. Since $f'(z_0) \neq 0$, then
    \begin{alignat*}{3}
        f'(z_0) = u_x + iv_x &= u_x - i u_y \neq 0 &&\implies \nabla u &&\neq \vec{0} \\
        f'(z_0) = u_x + iv_x &= v_y + i v_x \neq 0 &&\implies \nabla v &&\neq \vec{0}
    \end{alignat*}
    at $z_0$. The gradients are normal when their dot product is zero. Since $f$ is analytic, the Cauchy Riemann equations hold at $z_0$ and therefore
    \[
        \nabla u \cdot \nabla v = u_x v_x + u_y v_y = v_x v_y - v_x v_y = 0
    .\]
    Therefore the normal vectors of each curve at $z_0$ are perpendicular and hence their tangents are perpendicular at $z_0$.
\end{proof}

\subsection*{30.1}
\subsubsection*{Part A}
By splitting the exponential into a real power and complex power
\[
    e^{2 \pm 3 \pi i} = e^2 e^{\pm 3 \pi i}
.\]
Note that
\begin{alignat*}{3}
    e^{3 \pi i} &= \cos(3 \pi) &&+ i \sin(3 \pi) &&= -1 + i(0) = -1 \\
    e^{-3 \pi i} &= \cos(-3 \pi) &&+ i \sin(-3 \pi) &&= -1 + i(0) = -1
\end{alignat*}
Therefore $e^{\pm 3 \pi i} = -1$ meaning $e^{2 \pm 3 \pi i} = -e^2$.

\subsubsection*{Part B}
By splitting the exponential into a real power and complex power
\[
    e^{\frac{2+\pi i}{4}} = e^{\frac{2}{4}} \cdot e^{\frac{\pi i}{4}} = \sqrt{e} e^{\frac{\pi i}{4}}
.\]
Since $e^{\frac{\pi i}{4}} = \cos \frac{\pi}{4} + i \sin \frac{\pi}{4} = \frac{1}{\sqrt{2}} + i \frac{1}{\sqrt{2}}$, it follows
\[
    e^{\frac{2+\pi i}{4}} = \sqrt{e} \qty[\frac{1}{\sqrt{2}} + i \frac{1}{\sqrt{2}}] = \sqrt{\frac{e}{2}} (1 + i)
.\]

\subsubsection*{Part C}
Let $z = x +iy$. Then
\[
    e^{z + \pi i} = e^{x + iy + \pi i} = e^x e^{i(y + \pi)}
.\]
Note that $e^{i(y+\pi)} = \cos(y + \pi) + i \sin(y + \pi) = -\cos y - i \sin y = - e^{iy}$. Therefore
\[
    e^{z + \pi i} = -e^x e^{iy} = -e^{x+ iy} = -e^z
.\]

\subsection*{30.3}
\begin{proof}
    $f$ will be potentially analytic only at points where the Cauchy Riemann equations hold. Rewriting $f$ in terms of its components gives
    \[
        f(z) = e^{\conj{z}} = e^{x - iy} = e^x e^{-iy} = e^x \qty[\cos(-y) + i \sin(-y)] = e^x \cos y - i e^x \sin y
    .\]
    Therefore the partials are
    \begin{alignat*}{3}
        u_x &= e^x \cos y &\quad& u_y &&= -e^x \sin y \\
        v_x &= -e^x \sin y && v_y &&= -e^x \cos y
    \end{alignat*}
    Applying the Cauchy Riemann equations gives
    \[
        u_x = v_y \implies e^x \cos y = -e^x \cos y \implies \cos y = -\cos y
    \]
    which is true when $y = \frac{\pi}{2} + \pi k$ for $k \in \Z$ and that
    \[
        u_y = -v_x \implies -e^x \sin y = e^x \sin y \implies \sin y = -\sin y
    \]
    which is true when $y = \pi m$ for $m \in \Z$. However, both equations cannot be satisifed simultaneously since there is no $y$ such that both $\sin y = \cos y = 0$. This means that $f$ is not analytic at any $x + iy$ and therefore cannot be analytic anywhere.
\end{proof}

\subsection*{30.10}
\subsubsection*{Part A}
\begin{proof}
    Let $z = x + iy$ and assume $e^z$ is purely real. Splitting the exponential gives
    \[
        e^z = e^{x + iy} = e^{x} e^{iy} = e^x \qty[\cos y + i \sin y]
    .\]
    Since $e^z$ is real, then $\Im e^z = 0$ meaning $e^x \sin y = 0$. Since $e^x$ is never $0$, this is true when $\sin y = 0$. Therefore $y = n \pi$ for $n \in \Z$. This means that
    \[
        \Im z = \Im{x + iy} = \Im{x + i(n \pi)} = n \pi, n \in \Z
    .\]
\end{proof}

\subsubsection*{Part B}
The restriction on $z$ is that $\Im z = \frac{\pi}{2} + n \pi$ for some $n \in \Z$
\begin{proof}
    Let $z = x + iy$ and assume $e^z$ is purely imaginary. By part A, $\Re{e^z} = e^x \cos y$. Since $e^z$ is purely imaginary, $\Re{e^z} = 0$. Therefore $e^x \cos y = 0$. Since $e^x$ is never $0$, this is true when $\cos y = 0$. Therefore $y = \frac{\pi}{2} + n \pi$ for $n \in \Z$. This means that
    \[
        \Im z = \Im{x + iy} = \Im{x + i\qty( \frac{\pi}{2} + n \pi)} = \frac{\pi}{2} + n \pi, n \in \Z
    .\]
\end{proof}

\end{document}
