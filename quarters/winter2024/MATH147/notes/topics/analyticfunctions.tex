\documentclass[../notes.tex]{subfiles}
\graphicspath{
    {'../figures'}
}

\begin{document}

\banner{Analytic Functions}

\subsection{Complex Functions}

\begin{definition}[Complex Function]
    A complex function on $S \subset \C$ is a rule that assigns to each $z \in S$ a value $f(z) = w \in \C$, denoted by $f : S \to \C$.
\end{definition}

\begin{example}
    There are (surprise!) many complex functions.
    \begin{enumerate}
        \item The function $f(z) = \frac{1}{z}$ is well defined everywhere except $z = 0$, therefore it's domain of definition is $\C \setminus \qty{0}$.
        \item Any complex polynomial $f(z) = c_n z^n + \ldots + c_1 z + c_0$ with $c_i \in \C$ is a complex function over all of $\C$.
        \item Any rational function $\frac{f(x)}{g(x)}$ where the domain is $\C \setminus \qty{z \in \C : g(z) = 0}$
    \end{enumerate}
\end{example}

A complex function can also often be represented in the form
\[
    f(x+iy) = u(x,y) + i v(x, y)
.\]
Consider the case of $\frac{1}{z}$. Then
\[
    \frac{1}{x+iy} = \frac{x-iy}{x^2+y^2} = \frac{x}{x^2+y^2} - i \cdot \frac{y}{x^2+y^2}
.\]
Therefore in this case $u(x,y) = \frac{x}{x^2+y^2}$ and $v(x,y) = \frac{y}{x^2+y^2}$.

\begin{definition}[Limits in $\C$]
    The limit of a function $f:\dom f \to \C$
    \[
        \lim_{z \to z_0} f(z) = w_0
    \]
    means that for any $\epsilon > 0$, there is some $\delta > 0$ such that
    \[
        |z-z_0| < \delta \implies |f(z) - w_0| < \epsilon
    .\]
    That is, for any $\epsilon$ neighborhood of $w_0$, there is some deleted $\delta$ neighborhood around $z_0$ such that every $z$ in the $\delta$ neighborhood maps into the $\epsilon$ neighborhood.
\end{definition}

\begin{example}
    Consider the function $f(z) = \frac{i}{2} \conj{z}$. One can guess that
    \[
        \lim_{z \to 1} f(z) = \frac{i}{2} 1 = \frac{i}{2}
    .\]
    For this to happen,
    \begin{align*}
        \qty|\frac{i}{2} \conj{z} - \frac{i}{2}| < \epsilon \implies \qty|\frac{i}{2}| |\conj{z} - 1| &< \epsilon \\
        \frac{1}{2} |\conj{z} - 1| &< \epsilon \\
        \frac{1}{2} |z - 1| &< \epsilon \\
        |z - 1| &< 2 \epsilon
    \end{align*}
    Therefore choosing $\delta = 2 \epsilon$ gives the desired result.
\end{example}

\begin{example}
    Consider $f(z) = \frac{\conj{z}}{z}$. Does $f(z)$ have a limit at $z_0 = 0$? Note that along the real axis, $z = x$ and $\conj{z} = x$, hence the limit is $\lim_{x\to 0}\frac{x}{x} = 1$. Along the imaginary axis, $z = y$ and $\conj{z} = -y$, meaning the limit is $\lim_{y\to 0} \frac{-y}{y} = -1$. Therefore there is no limit.
\end{example}

\begin{theorem}[Limit Equivalence]
    \label{thm:limitequivalence}
    If $f(z) = u(z) + iv(z)$ where $u$ and $v$ are real valued functions, then
    \[
        \renewcommand{\arraystretch}{1.5}
        \lim_{z \to z_0} f(z) = u_0 + iv_0 \Longleftrightarrow \begin{array}{rl}
            \displaystyle
            \lim_{z\to z_0} u(z) &= u_0 \\
            \displaystyle
            \lim_{z\to z_0} v(z) &= v_0
        \end{array}
    .\]
\end{theorem}

\end{document}
